%This file makes an abstract for a Ph.D. thesis.
\begin{center}
ABSTRACT
\end{center}
\vspace{0.3in}
\begin{center}
NEUTRAL CURRENT ELASTIC SCATTERING \\ AND THE STRANGE SPIN STRUCTURE OF THE PROTON
\\
BY
\\
KATHERINE WOODRUFF
\end{center}
\vspace{0.3in}
\begin{center}
Doctor of Philosophy

New Mexico State University

Las Cruces, New Mexico, 2018

Dr. Vassili Papavassiliou, Chair
\end{center}
\vspace{0.3in}
%The next line produces an indented paragraph to start the document
 %unit.  The LaTeX defaults start most units without indentations.
\hspace{\parindent}
Neutrinos can be used as a unique and informative probe of the structure within
nucleons. This thesis presents the tools and methodology for studying the
strange quark spin in the nucleon through neutral current elastic
neutrino-proton scattering in the MicroBooNE experiment located at the Fermi
National Accelerator Lab.

An automated boosted decision tree based proton identification algorithm is
used to select proton in liquid argon TPC data with a 70\% efficiency. After a
set of protons are selected, a logistic regression model is used to select
neutral current elastic proton interactions in MicroBooNE. Neutral current
elastic proton interactions are selected with an 11\% efficiency and 30\%
purity.

The number of selected events measured in MicroBooNE data as a function of
$Q^2$ is compared directly to the number of selected events in MicroBooNE
simulation. An event reweighting scheme is used to vary the expected number of
eventsin MicroBooNE simulation based on the values of the physics parameters of
interest. The likelihood for any given value of the physical parameters can
then be easily calculated and the probability distributions of the physics
parameters can be sampled using Markov Chain Monte Carlo.

Future improvements to the detector physics models in MicroBooNE are required
for a precise determination of the strange quark spin structure in the proton.
With the current level of uncertainty on the detector physics models, we are
only able to constrain the net strange quark spin, $\Delta s$ to the range
$-1.8 < \Delta s < 3.8$ with 95\% confidence.

\newpage
%This is the end of the abstract.

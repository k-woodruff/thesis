\section{Simulation and reconstruction}\label{sec:simreco}
In order to understand how different physics models would affect what we would
see in the detector, large and complex Monte Carlo simulations of the beam and
the detector are created.  These simulations also allow us to develop and test
algorithms that reconstruct the underlying neutrino interaction based on what
particles are seen in the detector. The simulation and reconstruction
algorithms used by MicroBooNE are described in this section.


%%%%%%%%%%%%%%%%%%%%%%%%%%%%%%%%%%%%%%%%%%%%%%%%%%%%%%%%%%%
% Monte Carlo Simulation Simulation
%%%%%%%%%%%%%%%%%%%%%%%%%%%%%%%%%%%%%%%%%%%%%%%%%%%%%%%%%%%
\subsection{Monte Carlo simulation}\label{sec:simulation}
  The entire experimental process from the neutrino interactions in and around
  the detector to electronic signal readout to particle identification is
  simulated in software. To interface the different software packages needed to
  simulate each step, a liquid argon software framework (LArSoft) was developed
  at Fermilab. Within the LArSoft framework, the simulation is divided into
  three steps: generation, propagation, and detector simulation.  The output
  from the detector simulation stage is designed to match the real data output
  from the detector as closely as possible. Event reconstruction is also
  handled within LArSoft, and the same algorithms can be applied to both real
  data and simulated data in an identical way. This section describes all three
  stages of simulation and all of the reconstruction stages, including TPC
  particle track and optical flash reconstruction.

  \subsubsection{Cross section model}\label{sec:geniexsec}
    The initial neutrino interactions are simulated using the GENIE Neutrino
    Monte Carlo Generator~\cite{Andreopoulos:2009rq,Andreopoulos:2015wxa}.
    Assuming a given neutrino flux, GENIE simulates the interaction of the
    neutrino with the nucleons inside of the atoms in and around the detector.
    It also simulates the interactions that occur while the nucleons and pions
    from the initial neutrino-nucleon interaction traverse and exit the
    nucleus. Within GENIE the nuclear models and neutrino-nucleon cross
    sections are configurable by the user. This analysis uses the GENIE version
    v2.12.2 default settings with the addition of the empirical MEC cross
    section model~\cite{Alam:2015nkk}. The details of the physics models for
    each of the four processes (the nuclear physics model, the cross section
    model, the neutrino-induced hadron production model, and the
    intranuclear hadron transport model) within GENIE as well as the
    simulated cosmic ray generation are described in this section. The
    uncertainties in the analysis due to the models in this section are
    explored in Sec.~\ref{sec:modeluncertainty}.

    First, the relativistic Fermi gas (RFG) nuclear model~\cite{Smith:1972xh}
    is used for all processes. The mass density for the argon nucleus is taken
    from review articles~\cite{DeJager:1987qc} and the two-parameter
    Woods-Saxon density function is used~\cite{Woods:1954zz}
    \begin{equation}\label{eq:woodssaxon}
      \rho(r) = N_0\frac{1}{1+e^{(r-c)/z}} \,,
    \end{equation}
    where $\rho$ is the density, $r$ is the distance from the center of the
    nucleus, $c$ describes the size of the nucleus (approximately the radius
    where the density falls to half of the central value), and $z$ describes
    the thickness of the surface. For argon, GENIE uses $c=3.53$~fm and
    $z=0.54$~fm as default values.  For elastic nucleon scattering in the
    nucleus, Pauli blocking is applied.
    
    Next, in the case of elastic (and quasi-elastic) neutrino-nucleon
    scattering, the free-nucleon cross section is calculated. The
    neutrino-nucleon cross section and form factor models used are described in
    Secs.~\ref{sec:formfactorforms}~and~\ref{sec:probe}. The methods used to
    evaluate the Monte Carlo given a chosen cross section and form factor
    model are described in section~\ref{sec:reweighting}.

    The hadrons produced in neutrino-nucleon interactions in a nucleus are
    modeled separately from the nuclear model and the free-nucleon cross
    section model. This is because the hadron production models do not match
    the measured neutrino cross sections.  The hadron discrepancy is corrected
    for in GENIE using the AGKY model~\cite{Yang:2009zx} that was developed to
    account for the data seen in the MINOS~\cite{Adamson:2007gu} neutrino
    scattering experiment and tuned using bubble chamber experimental data.

    Finally, the transport of the final state hadrons through the argon nucleus
    is modeled. The hadrons produced in the neutrino-nucleon interactions can
    rescatter as the exit the nucleus which changes the observable final state
    particle distributions. The intranuclear transport model is implemented by
    the GENIE subpackage called INTRANUKE. INTRANUKE uses a cascade model in
    which the hadron sees a nucleus of isolated nucleons. The interaction
    probability is calculated based on the free nucleon cross section and the
    nucleon density in the nucleus~\cite{Andreopoulos:2015wxa}
    \begin{equation}
      \lambda(E,r) = \frac{1}{\sigma_{hN,tot} \rho(r)} \,,
    \end{equation}
    where $\lambda$ is the hadron interaction probability, $E$ is the hadron
    energy, $r$ is the distance from the center of the nucleus,
    $\sigma_{hN,tot}$ is the total \textit{free} nucleon cross section, and
    $\rho$ is the density of the argon nucleus. The free nucleon cross section
    is different for protons, neutrons, and pions. The density of the argon
    nucleus is again determined by equation~\ref{eq:woodssaxon}.

  \subsubsection{Event generation}\label{sec:eventgen}
    The GENIE neutrino events are generated in LArSoft based on the neutrino
    flux described in Sec.~\ref{sec:beam}. The 1.6~$\mu$s long spill containing
    $5\times 10^{12}$ protons on target is simulated.  In addition to neutrino
    events, cosmic ray events are also simulated with the
    CORSIKA~\cite{Heck:1998vt} cosmic ray generator. CORSIKA uses the FLUKA
    interaction and transport simulator~\cite{Battistoni:1997tq} to model
    hadronic interactions in the atmosphere.  To simulate real detector events
    both neutrino and cosmic ray interactions can be generated in the same
    simulation. We can also combine real cosmic data taken when the neutrino
    beam was turned off with simulated neutrino interactions in the same events
    to closely model the real detector data.

    In the standard simulation neutrino interactions are only generated within
    the liquid argon filled cryostat, including interactions generated on the
    cryostat itself. Additional special samples are made to study how secondary
    particles from neutrino interactions outside the cryostat contribute to our
    signal background. One special sample that we generate is a large
    \textit{dirt} sample in which neutrino interactions that happen in the dirt
    and detector hall outside of the cryostat are generated.  The neutrino
    interactions are allowed to happen anywhere in fifty feet of simulated dirt
    outside of the detector hall or anywhere in the detector hall outside of
    the cryostat. These are known as dirt events.

    All of the Monte Carlo samples that are generated and used in this analysis
    are listed here.
    \begin{enumerate}
      \item Simulated neutrino interactions with simulated cosmic ray
      backgrounds. This sample is used to develop the particle identification
      and event selection algorithms.
      \item Simulated neutrino interactions with real cosmic ray data backgrounds overlaid.
      This sample very closely represents the actual detector conditions when
      there is a neutrino interaction in MicroBooNE and is referred to as the
      overlay sample.
      \item Simulated dirt interactions plus simulated cosmic interactions.
      This sample contains all known backgrounds to neutrino events in the TPC
      and is referred to as the dirt sample.
    \end{enumerate}

  \subsubsection{Detector simulation}\label{sec:detsim}

    The final state particles from GENIE are passed to the
    Geant4~\cite{Agostinelli:2002hh} software package to be propagated through
    the simulated geometry.  The entire MicroBooNE detector system, the
    detector hall, and fifty feet of dirt surrounding the detector hall are all
    included in the Geant4 simulation. Figure~\ref{fig:uboonegdml} shows the
    entire simulated geometry including the surrounding dirt. This includes the
    electric field and the detector electronics.  The particles are stepped
    through the geometry and undergo a possible physics process at each step
    with a given probability.  The particles are allowed to interact
    electromagnetically and hadronically with other particles and the detector
    system or decay through one of the physically possible decay modes.
    Additionally, the energy loss through ionization and scintillation is
    simulated for all particles traversing the detector geometry. In the case
    of the ionization of the liquid argon, the resulting electrons are
    propagated through the electric field to the wire readouts.  For the
    scintillation of the argon, a photon library was generated for each
    position in the liquid cryostat. At each step a particle takes through the
    liquid argon, the resulting photons that would interact in the PMTs are
    determined from a look-up table that was generated in a previous full
    optical simulation. The full optical simulation of the scintillation
    photons includes the Rayleigh scattering of the photons in the liquid argon
    as well as the reflection and absorption of the photons at the surfaces in
    the detector.

    \begin{figure}[h]
      \centering
      \includegraphics[angle=0,width=5in]{figures/detector/simreco/uboone_gdml.png}
      \caption{Rendering of the simulated MicroBooNE geometry.}
      \label{fig:uboonegdml}
    \end{figure}

    After the simulated particles interact with the TPC or PMT system, the
    detector response is simulated. The detector-simulation stage includes the
    electronic responses of the sensitive detectors and reproduces the
    electronic signals from the TPC and PMT systems. First, the PMT signal is
    digitized and the PMT software trigger described in
    Sec.~\ref{sec:swtrigger} is fully simulated. Events that do not pass the
    PMT trigger are dropped. The TPC electronics, including the electronic
    noise on the wires and unresponsive wires, are also included at this stage.
    At this point, the simulated data resembles the actual raw detector data as
    closely as possible.

%%%%%%%%%%%%%%%%%%%%%%%%%%%%%%%%%%%%%%%%%%%%%%%%%%%%%%%%%%%
% Event Reconstruction
%%%%%%%%%%%%%%%%%%%%%%%%%%%%%%%%%%%%%%%%%%%%%%%%%%%%%%%%%%%
\subsection{Event reconstruction}\label{sec:reco}
Both the simulated waveforms and the raw detector waveforms need to be
reconstructed into the initial neutrino interactions. A series of
reconstruction algorithms for both PMT and TPC information exist in LArSoft.
These algorithms start by identifying peaks in the waveforms and combine these
peaks in stages to get to a 3-dimensional representation of the physics
interaction.

  \subsubsection{Flash reconstruction}\label{sec:flashreco}
    The optical flash reconstruction algorithm is applied identically to
    detector data and the simulated data that is output from the detector
    simulation stage.  The first step is to find pulses on the electronic
    signals read out from each of the 32 PMTs. This is done using a peak
    finding algorithm on the digitized signal. The time, amplitude, width, and
    area under the pulses are stored per pulse. Next, the flash reconstruction
    algorithm looks for coincident pulses across PMTs. The individual pulses
    are sorted by size, and all of the pulses that are within 8~$\mu$s of the
    largest pulse are collected. If there are at least three pulses in that
    time window and the sum of the pulse areas is at least 6~PE above the noise
    background, it is considered a flash and saved. The peak time, width,
    position, and size are reconstructed and saved along with information about
    the pulses in the individual PMTs that contributed to the flash. This
    process is repeated starting with the next largest remaining pulse until
    there are none left. An individual pulse can only contribute to one flash
    in an event.

  \subsubsection{TPC event reconstruction}\label{sec:tpcreco}
    Reconstructing TPC events involves more steps than the PMTs since there are
    thousands of wires being read out for milliseconds resulting in
    approximately 30~MB of raw data per event. The reconstruction algorithms
    are again applied identically to detector data and the simulated data from
    the detector simulation stage. First, a noise-deconvolution filter is
    passed over each of the digitized wire signals. Then, similar to the
    optical reconstruction, pulses, or \textit{hits}, are found on individual
    wires which are used as base building blocks for reconstructing 3D particle
    tracks across wires and wire planes.

    The 1D hit finding algorithm starts by walking along a wire signal until
    the value is above a given threshold. The point where the signal goes above
    the threshold is considered the start of the pulse, and the end of the
    pulse is defined as the point where the signal goes back down below the
    threshold.  Then, local minima and maxima are found between the start and
    end of the pulse which are used to determined where there are peaks within
    the pulse.  Adjacent pulses are merged if they are close enough in time.
    Once the pulse and the number of peaks are established, the algorithm
    attempts to fit Gaussian peaks to the pulse. The hypothesis signal is
    composed of one Gaussian per peak from the previous step. The mean and
    amplitude of each Gaussian is initially centered at the existing peaks and
    is allowed to float. If the residuals of the fit are sufficiently small,
    each Gaussian peak is saved as a 1D \textit{hit} with an amplitude, width,
    and time given by the fit. The hit finding is repeating along the length of
    the wire for each wire on all three planes.

    The 3D track finding algorithm is separated into two distinct parts. The
    first part attempts to reconstruct and tag as many cosmic-induced tracks.
    The hits associated with these tracks are then removed from the set of hits
    that are available to reconstruct neutrino-induced tracks in the second
    part. 

    The algorithm used to preferentially reconstruct cosmic tracks is called
    \textit{PandoraCosmic}. It is implemented in the Pandora Software
    Development Kit~\cite{Acciarri:2017hat} used by MicroBooNE. In
    PandoraCosmic, 1D hits are first clustered in 2D per wire plane.  All of
    the reconstructed hits output by the hit finding algorithm on a given wire
    plane are used as input. The hits are clustered into unambiguous,
    continuous lines of hits. These initial clusters are meant to have a high
    purity, meaning all of the hits in the cluster were induced by the same
    true particle. The clusters in a 2D plane are then merged pairwise in an
    attempt to improve the completeness of the cluster, meaning most of the
    hits on that wire plane induced by a true particle are included in a single
    merged cluster. The 2D clusters are then matched across the three wire
    planes. The 3D track reconstruction algorithm checks each plane for
    clusters which are likely to have come from the same true particle.  All
    possible combinations of 2D cluster matching are considered by the
    algorithm. The most suitable set of cluster combinations are projected into
    3D and saved as reconstructed track objects.

    Next, the reconstructed tracks from PandoraCosmic are passed to a cosmic
    tagging stage. There are two algorithms used to identify cosmic-induced
    tracks. The first is a geometry tagger that looks for tracks that are not
    fully contained in the TPC during the event, and the second is a
    flash-matching algorithm that looks for tracks that are inconsistent with
    any flashes in the beam spill window. Both of the cosmic tagging algorithms
    try to remove as few neutrino-induced tracks as possible and only remove
    tracks that are very likely cosmic-induced.
    
    The geometry tagger starts by locating any TPC wire hits that are
    reconstructed before or after the 1.6 millisecond readout frame. Any tracks
    that contain these hits are tagged as cosmic. Any tracks whose
    reconstructed start or end points are located within a given distance from
    the TPC boundary are also tagged. If both of the start and end points are
    near a TPC boundary, the track is given a cosmic score of 1. If only one of
    the start or end points is near a boundary, the track is given a cosmic
    score of 0.5.
    
    The flash-matching algorithm creates a hypothesis flash for each
    reconstructed track based on its position, size, and energy deposited. It
    then compares the hypothesis flash to each of the reconstructed flashes in
    the beam spill window. If a hypothesis flash is sufficiently incompatible
    with all true flashes in the window, the track is tagged as cosmic. After
    the cosmic tagging, all reconstructed hits associated with any of the
    tracks tagged as cosmic are removed from the set of possible hits that are
    used to reconstruct neutrino-induced tracks. This is referred to as the
    cosmic hit removal stage.

    The remaining set of hits is used to reconstruct neutrino-induced tracks.
    First, two dimensional clusters of hits are formed on each of the three
    wire planes using the TrajCluster algorithm in LArSoft. TrajCluster creates
    line-like collections of hits and adds new hits to the cluster based on the
    2D trajectory of the current set of hits.  The algorithm stops when there
    are no additional hits along the trajectory of the cluster and is followed
    by an additional stage that combines clusters which start and end near each
    other. Next, 3D tracks are formed using the projection matching algorithm
    (PMA)~\cite{Antonello:2012hu} which was developed for the ICARUS experiment
    and implemented in LArSoft. PMA works by proposing nodes and lines
    connecting the nodes in 3D, projecting the 3D lines onto the 2D planes, and
    determining the most likely positions of the nodes in 3D based on the fit
    of the 2D projections to the existing 2D reconstructed clusters from the
    previous stage. The algorithm starts with a two-node hypothesis and adds
    nodes to the 3D line until a maximum number of nodes, which is based on the
    number of hits on the wires, has been reached.

    Calorimetric information is extracted when the reconstructed track objects
    are created. The calorimetric information that we calculate and use is
    related to the energy loss of the particle that created the track along its
    trajectory. At each point along the track, the difference in the total
    charge between the current point and the previous trajectory point is
    calculated. This gives us the change in charge as a function of distance
    along the track, which we label $dQ/dx$. The $dQ/dx$ values along each
    track are found for each of the three wire planes. The $dQ/dx$ values can
    be converted to energy loss per unit distance, $dE/dx$, by multiplying the
    $dQ/dx$ by a measured conversion factor. This conversion factor depends on
    the strength of the electric field and the gain of the readout electronics
    and is determined empirically~\cite{uBCalibrationNote}.


%This is the end of detector section

\section{Neutrino-Nucleon Cross Sections} \label{sec:theory}
%The next line produces an indented paragraph to start the document
 %unit.  The LaTeX defaults start most units without indentations.
\hspace{\parindent}

%%%%%%%%%%%%%%%%%%%%%%%%%%%%%%%%%%%%%%%%%%%%%%%%%%%%%%%%%%%
% Neutrinos as Nucleon Probes
%%%%%%%%%%%%%%%%%%%%%%%%%%%%%%%%%%%%%%%%%%%%%%%%%%%%%%%%%%%
\subsection{Neutrino Cross Sections}\label{sec:probe}
  \subsubsection{Weak interactions and the Axial Form Factor}
    Why are neutrinos better probes than electrons or photons?
  \subsubsection{Electroweak Interactions}
    Go from electroweak lagrangean to weak current matrix elements to actual
    cross sections. Show both neutral-current and charged-current version of
    elastic scattering cross section.

  %%% starting text %%%
  From Alberico, starting with 2.3: \\
  The NC lagrangean is (interaction of leptons and quarks with $Z^0$:
  \[
      \mathcal{L}^{NC} = -\frac{g}{2\textrm{cos}\theta_W}j_{\alpha}^{NC}Z^{\alpha},
  \]
  where $j_{\alpha}^{NC}$ is \textit{the} neutral current. Since unification of
  weak and electromagnetic is required, we get
  \[
    j_{\alpha}^{NC} = 2j_{\alpha}^3 - 2\textrm{sin}^2\theta_W j_{\alpha}^{em},
  \]
  where $j_{\alpha}^{em}$ is electromagnetic current and $j_{\alpha}^3$ is some
  sort of isovector component?
  \[
      j_{\alpha}^{em} = \sum_{l = e,\mu,\tau} (-1)\bar{l}\gamma_{\alpha}l + \sum_{q=u,d,...}e_q\bar{q}\gamma_{\alpha}q
  \]
  and
  \[
    j_{\alpha}^i = \sum_a \bar{\psi}_{aL} \gamma_{\alpha}\frac{1}{2} \tau^i \psi_{aL} \,,
  \]
  and $\psi_{aL} = \frac{1}{2}(1-\gamma_5)\psi_a$ are the left-handed doublets
  of the SU(2)xU(1) gauge group of the standard model, and I don't know what $\tau$ is.
  So, we get
  \begin{align*}
      j_{\alpha}^{NC} &= \sum_{q=u,c,t}\bar{q}\gamma_{\alpha}(1-\gamma_5)\frac{1}{2}q
                       + \sum_{q=d,s,b}\bar{q}\gamma_{\alpha}(1-\gamma_5)\left(-\frac{1}{2}\right)q \\
                      &+ \sum_{l=e,\mu,\tau}\bar{\nu_l}\gamma_{\alpha}(1-\gamma_5)\frac{1}{2}\nu_l
                       + \sum_{l=e,\mu,\tau}\bar{l}\gamma_{\alpha}(1-\gamma_5)\left(-\frac{1}{2}\right)l \\
                      &- 2\mathrm{sin}^2\theta_W j_{\alpha}^{em} \,.
  \end{align*}
  It's convenient to separate the contributions from the light ($u,d$) and
  heavy ($s,c,...$) quarks. We get
  \[
    \j_{\alpha}^{NC;q} = v_{\alpha}^3 - a_{\alpha}^3 - \frac{1}{2}(v_{\alpha}^s - a_{\alpha}^s) - 2\mathrm{sin}^2\theta_W j_{\alpha}^{em} \,,
  \]
  where we define
  \begin{align*}
      v_{\alpha}^3 &= \bar{u}\gamma_{\alpha}\frac{1}{2}u - \bar{d}\gamma_{\alpha}\frac{1}{2}d \equiv \bar{N} \gamma_{\alpha}\frac{1}{2}\tau_3 N \\
      a_{\alpha}^3 &= \bar{u}\gamma_{\alpha}\gamma_5\frac{1}{2}u - \bar{d}\gamma_{\alpha}\gamma_5\frac{1}{2}d \equiv \bar{N} \gamma_{\alpha}\gamma_5\frac{1}{2}\tau_3 N \,.
  \end{align*}
  where $N = \left(\begin{matrix}{u}\\{d}\end{matrix} \right)$ is the isotopic
  SU(2) group doublet and the currents $v_{\alpha}^3$ and $a_{\alpha}^3$
  are the third components of the isovectors
  \begin{align*}
      v_{\alpha}^i &= \bar{N} \gamma_{\alpha}\frac{1}{2}\tau^i N \\
      a_{\alpha}^i &= \bar{N} \gamma_{\alpha}\gamma_5\frac{1}{2}\tau^i N \,.
  \end{align*}
  On the other hand, $v_{\alpha}^s$ and $a_{\alpha}^s$ are isoscalars. They
  represent the heavier quark contirbution to $j_{\alpha}^{NC;q}$. If we only
  include the strange quark, we have
  \begin{align*}
      v_{\alpha}^s &= \bar{s}\gamma_{\alpha}s \\
      a_{\alpha}^s &= \bar{s}\gamma_{\alpha}\gamma_5 s \,.
  \end{align*}
  The quark part of the electromagnetic current is
  \[
    j_{\alpha}^{em;q} = \sum_{q=u,d,...} e_q\bar{q} \gamma_{\alpha}q \,,
  \]
  which we can also separate between light and heavy contributions
  \[
    j_{\alpha}^{em;q} = v_{\alpha}^3 + v_{\alpha}^0 \,.
  \]
  where $v_{\alpha}^0$ is the isoscalar current which is given by
  \[
    v_{\alpha}^0 = \frac{1}{6}\bar{N}\gamma_{\alpha}N + \left(-\frac{1}{3}\right)\bar{s}\gamma_{\alpha}s \,,
  \]
  if we ignore charm and up.

  Time for one-nucleon matrix elements! 

  %%% ending text %%%

  \subsubsection{Previous measurements}
    E734

%This is the end of nu-n cross sections section

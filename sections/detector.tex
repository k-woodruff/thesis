\section{The MicroBooNE experiment}\label{microboone}

%%%%%%%%%%%%%%%%%%%%%%%%%%%%%%%%%%%%%%%%%%%%%%%%%%%%%%%%%%%
% MicroBooNE and Neutrino Beam
%%%%%%%%%%%%%%%%%%%%%%%%%%%%%%%%%%%%%%%%%%%%%%%%%%%%%%%%%%%
\subsection{The Booster Neutrino Beam and the MicroBooNE detector}\label{beam}
  \subsubsection{The neutrino beam}
    Where do the proton come from, how do they get accelerated, what is their final energy distribution.
    Bunches and spills.
    The target/horn, it's shape and materials.
    The decay chain of particles coming out of the target (mention dirt).
    The final neutrino flux at MicroBooNE (also mention uncertainty ==> hence ratio).
    Expected number of neutrino interactions, expected NCE interactions.
  \subsubsection{Dirt neutrons}
  \subsubsection{MicroBooNE LArTPC}
    Liquid argon target and TPC.
    LAr properties (density, ionization, scintillation).
    MicroBooNE electric field, drift time/distance.
    MicroBooNE specs (dimensions, wire counts).
  \subsubsection{MicroBooNE PMT system}
    PMT layout, timing, TPB.

%%%%%%%%%%%%%%%%%%%%%%%%%%%%%%%%%%%%%%%%%%%%%%%%%%%%%%%%%%%
% DAQ and Trigger
%%%%%%%%%%%%%%%%%%%%%%%%%%%%%%%%%%%%%%%%%%%%%%%%%%%%%%%%%%%
\subsection{Data Acquisition and Trigger}\label{daq}
  \subsubsection{PMT Readout Electronics and Trigger}
    Consist of signal shaper boards, an FEM modified from TPC version, PMT
    feedthrough, HV/signal splitters, and a trigger board. Write about optical
    flash reconstruction.
  \subsubsection{TPC Readout Electronics}
    Describe cold electronics (front end ASIC preamplifies and shapes), warm
    interface electronics (intermediate amplifier for transmission over 20 m.
    long cable), and digitizing electronics (TPC readout board in crate
    continuously samples received signals and passing from ADC to FPGA for
    processing, reducing, and storage). Maybe also talk about cabling and
    signal feedthrough. Also discuss specs like sampling rate.
  \subsubsection{DAQ System}
    Receives and buffers data from TPC and PMT readout system, then builds and
    records event based on trigger decision. Write about SEBs and EVB and
    passing data between. Should go into detail here about trigger algorithm.

%%%%%%%%%%%%%%%%%%%%%%%%%%%%%%%%%%%%%%%%%%%%%%%%%%%%%%%%%%%
% Simulation and Reconstruction
%%%%%%%%%%%%%%%%%%%%%%%%%%%%%%%%%%%%%%%%%%%%%%%%%%%%%%%%%%%
\subsection{Simulation and Reconstruction}\label{reco}
  The entire experimental process from the neutrino interactions in and around
  the detector to electronic signal readout to particle identification is
  simulated in software. To interface the software packages needed to simulate
  each step, a liquid argon software framework (LArSoft) was developed at
  Fermilab.
  \subsubsection{Simulation}
    The initial neutrino interactions are simulated using the GENIE Neutrino
    Monte Carlo Generator~\cite{Andreopoulos:2009rq,Andreopoulos:2015wxa}.
    Describe all 3 stages of simulations: Genie (generation), Geant4
    (propagation), and detector simulation.
  \subsubsection{Flash reconstruction}
    Describe simpleFlash algorithm.
  \subsubsection{Noise Filters and Hits}
    Write about filtering raw signal and hit finding algorithm used.
    I think give efficiencies at this stage.
  \subsubsection{Clusters and Tracks}
    Explain how hits are clustered and then combined into tracks or showers.
    Explain algorithm used and give efficiencies at this stage. Also include
    cosmic tagging and calorimetry.

%%%%%%%%%%%%%%%%%%%%%%%%%%%%%%%%%%%%%%%%%%%%%%%%%%%%%%%%%%%
% Particle Identification
%%%%%%%%%%%%%%%%%%%%%%%%%%%%%%%%%%%%%%%%%%%%%%%%%%%%%%%%%%%
\subsection{Particle Identification}
  Classification goals (cosmic rejection and neutrino-induced particle type).
  \subsubsection{Reconstructed track features}
    Itemize and discuss every feature used as inuot to classifier.
    Separate into cosmic rejection and particle ID.
  \subsubsection{Boosted decision trees}
    Decision trees, tree boosting, xgboost.
  \subsubsection{Performance}
    Show efficiency, accuracy, itemize backgrounds.
    Discuss reasons for different backgrounds.


%This is the end of detector section

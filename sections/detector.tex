\section{The MicroBooNE experiment}\label{microboone}
MicroBooNE is an accelerator neutrino experiment at Fermilab. This means that
the experiment measures neutrino physics properties by studying neutrino
interactions from neutrinos produced by Fermilab accelerators. The MicroBooNE
detector is a liquid argon time projection chamber (LArTPC) with an additional
light collection system. LArTPC technology is relatively new, and allows for
high-resolution imaging of the neutrino interactions in the liquid argon. In
order to understand how different physics models affect what we would see in
the detector, a lot of work goes in to Monte Carlo simulations of the beam and
the detector. These simulations also allow us to develop and test algorithms
that reconstruct the underlying neutrino interaction based on what particles
are seen in the detector. The production of the neutrino beam, the MicroBooNE
detector, and the simulation and reconstruction algorithms are described in
this section.

%%%%%%%%%%%%%%%%%%%%%%%%%%%%%%%%%%%%%%%%%%%%%%%%%%%%%%%%%%%
% MicroBooNE and Neutrino Beam
%%%%%%%%%%%%%%%%%%%%%%%%%%%%%%%%%%%%%%%%%%%%%%%%%%%%%%%%%%%
\subsection{The Booster Neutrino Beam and the MicroBooNE detector}\label{sec:beam}
  The neutrino beam used by the MicroBooNE experiment is produced using proton
  accelerators that were already in use. In fact, the proton beam that is
  extracted to go to the proton target used by MicroBooNE to produce neutrinos,
  is the same initial proton beam used by every other accelerator experiment at
  Fermilab. The Booster Neutrino experiments, which include MicroBooNE, are the
  first set of experiments on the proton accelerator line.

  \subsubsection{The neutrino beam}
    The first step in creating the Booster Neutrino Beamline (BNB) is to
    accelerate negatively ionized hydrogen through a linear accelerator, or
    Linac. The Linac is 500~ft long and accelerates the hydrogen ions to
    400~MeV using radio frequency (RF) cavities. The hydrogen ions are then
    injected into the Booster accelerator. During injection, the ions pass
    through a foil that strips both electrons leaving only the positive proton.
    The Booster~\cite{MinibooneBeam} is a 474-meter-circumference, 15~Hz
    synchrotron that accelerates the protons from 400~MeV to 8~GeV. The proton
    beam leaving the Booster has a bunched structure. Each turn contains 81
    proton bunches that are 2~ns wide and 19~ns apart. At this stage, a
    fraction of the 8~GeV protons are extracted to be sent to the Booster
    neutrino target. A kicker magnet is used to extract all 81 proton bunches
    in a turn which we refer to as a spill. One spill is 1.6~$\mu$s long and
    contains $5\times 10^{12}$ protons.

    The proton target is a beryllium cylinder 71.1~cm long and 0.51~cm thick
    that is aligned with the proton beam. The interaction of the protons in the
    beryllium produce charged pions that are focused by an pulsed toroidal
    electromagnet. The magnet can operate at a maximum frequency of 5~Hz which
    is the limiting factor in the beam spill frequency. In neutrino mode, the
    magnet focuses positive pions and defocuses negative pions. The positive
    pions then pass through an air-filled decay pipe where they quickly decay
    into positive muons (anti-muons) and neutrinos. In anti-neutrino mode,
    which is not used in this analysis, negative pions are focused which decay
    into negative muons and anti-neutrinos. The anti-muons and neutrinos next
    pass through a beam stop made of steel and concrete to filter out remaining
    protons, pions, and kaons. Finally, the anti-muons and neutrinos pass
    through almost 500~meters of dirt where the anti-muons decay and a (mostly)
    pure neutrino beam remains. The predicted neutrino flux at MicroBooNE is
    shown in Fig.~\ref{fig:beamflux}.

    \begin{figure}[h]
      \centering
      \includegraphics[angle=0,width=4in]{figures/detector/beam/FluxPrediction.pdf}
      \caption{Predicted neutrino beam flux at MicroBooNE.}
      \label{fig:beamflux}
    \end{figure}

    Neutrinos from the BNB can interact in the dirt upstream of the MicroBooNE
    detector. Most of these interactions have no effect in the detector because
    the secondary particles do not travel far enough to pass through the dirt
    and enter the detector. However, secondary neutrons that are produced in
    the dirt near the detector can enter the detector at a significant rate.
    Figure~\ref{fig:dirtstart} shows the position of all of simulated neutrino
    interactions in the dirt surrounding MicroBooNE in gray and the subset of
    those interactions that produce a neutron which enters the MicroBooNE TPC
    in red from two different angles. The details of the simulation are
    described in Sec.~\ref{sec:simreco}. If one of these neutrons interacts in
    the MicroBooNE detector and scatters a single proton, the signal could look
    indistinguishable from a neutral-current elastic neutrino interaction that
    occurred in the detector.

    \begin{figure}[h]
      \centering
      \begin{subfigure}{2.5in}
        \includegraphics[angle=0,width=2.5in]{figures/detector/beam/DirtStart_ZY.png}
        \caption{Side view of dirt interaction positions.}
      \end{subfigure}
      \hspace{2pt}
      \begin{subfigure}{2.5in}
        \includegraphics[angle=0,width=2.5in]{figures/detector/beam/DirtStart_ZX.png}
        \caption{Top down view of dirt interaction positions.}
      \end{subfigure}
      \caption{Gray is all simulated dirt interactions. Red is simulated
      interaction positions where neutron enters the TPC.}
      \label{fig:dirtstart}
    \end{figure}

  \subsubsection{MicroBooNE LArTPC}\label{sec:lartpc}
    The MicroBooNE LArTPC~\cite{detector} acts as both a target for the
    neutrino beam and a detector for the charged particles produced in the
    neutrino-argon interactions. The MicroBooNE TPC is submerged in 170~tons of
    liquid argon, of which 87~tons is actually contained inside the TPC. All of
    this is contained within a cylindrical cryostat.
    
    Liquid argon is used as the detector material for several reasons. Like all
    noble liquids, argon produces both ionization charge and scintillation
    light when stimulated.  The ionization electrons do not recombine easily,
    so they are able to pass through the liquid argon to be detected.
    Additionally, argon is transparent to its own scintillation light, again
    making it detectable. Noble liquids also have good dielectric properties
    allowing them to withstand high voltages without breaking down. Liquid
    argon is also relatively dense at 1.4~g/cm$^3$, making it a good target for
    neutrinos which interact extremely rarely. Lastly, argon is very abundant.
    It makes up 1\% of Earth's atmosphere and is therefore an affordable option
    compared to heaver noble elements.

    \begin{figure}[h]
      \centering
      \includegraphics[angle=0,width=5in]{figures/detector/tpc/LArTPC_Concept.png}
      \caption{Representation of the operational principle of the MicroBooNE
      LArTPC. The process is described in the text.}
      \label{fig:tpccartoon}
    \end{figure}

    The MicroBooNE time projection chamber (TPC) uses a uniform electric field
    to guide the electrons that are produced when the liquid argon is ionized
    to an anode. The electrons are drifted horizontally, perpendicular to the
    beam direction. To produce this uniform, horizontal electric field, a
    10~m$\times$2.6~m cathode plane makes up one face of the TPC. This face is
    vertical and parallel to the beam. A voltage of -70~kV is applied to the
    anode plane which results in a 273~V/cm electric field across the 2.3~m TPC
    width.  At the anode are three planes, perpendicular to the electron drift
    direction and parallel to the cathode plane, that are strung with thousands
    of wires spaced 3~mm apart.  The three planes are all parallel to each
    other and are also spaced 3~mm apart.  The wires on the first plane have a
    small negative voltage (-200~V), the middle plane has no voltage, and the
    last plane has a small positive voltage (+440~V). When the electrons
    approach the first wire plane, they induce a signal on the closest wires,
    and are then attracted to the next plane because of its slightly less
    negative voltage. This is repeated at the second wire plane, and the
    electrons are terminated at the third plane that we refer to as the
    collection plane. The electronic signals from all 8256 wires are read out
    and saved. The wires on each plane are at different angles to allow for an
    accurate 2-dimensional reconstruction of the event in the detector. The
    collection plane wires are vertical, and the wires on the two induction
    planes are at positive and negative 60~degrees. The third dimension,
    perpendicular to the anode planes, is reconstructed based on the time that
    the electrons arrived at the wires. A cartoon representation of this
    process is shown in Fig.~\ref{fig:tpccartoon}. The wires are read out for
    1.6~ms per frame to ensure that the electrons have time to traverse the
    entire width of the detector.

    \begin{figure}[h]
      \centering
      \begin{subfigure}[t]{2.5in}
        \includegraphics[angle=0,width=2.5in]{figures/detector/tpc/Interactions_MicroBooNE.eps}
        \caption{The number of interactions is broken down by interaction type for CC interactions.}
        \label{fig:interactionsal}
      \end{subfigure}
      \hspace{2pt}
      \begin{subfigure}[t]{2.5in}
        \includegraphics[angle=0,width=2.5in]{figures/detector/tpc/NC_Interactions_MicroBooNE.eps}
        \caption{The number of interactions is broken down by interaction type
        for NC interactions. NC elastic scattering off of both protons and
        neutrons is shown in red.}
        \label{fig:interactionsnc}
      \end{subfigure}
      \caption{Expected number of muon neutrino interactions in the
      MicroBooNE TPC based on Monte Carlo simulation as a function of POT.
      MicroBooNE is approved to receive $6.6\times 20$ POT.}
      \label{fig:interactions}
    \end{figure}

    The expected number of neutrino interactions in MicroBooNE as a function of
    protons on target (POT) is shown in Fig.~\ref{fig:interactions}. These were
    determined using s Monte Carlo simulation which will be described in
    Sec.~\ref{sec:simreco}. The plots show the expected events up to $1\times
    10^{21}$ POT. MicroBooNE is approved to run for a total of $6.6\times
    10^{20}$ POT. Figure~\ref{fig:interactionsnc} shows the expected number for
    a subset of neutral current interactions with NC elastic in red. This line
    includes interactions on both proton and neutrons, so we expect the number
    of NC elastic proton events to be about half this. The integrated number of
    all expected NC elastic neutrino-proton events from $6.6\times 10^{20}$ POT
    is approximately 16,000.

    The MicroBooNE detector is positioned at near the surface of the Earth (6~m
    underground). The rate of cosmic rays interactions in the detector is
    therefore large. Based on Fermilab's elevation (226~m above sea level) and
    MicroBooNE's geometry within the detector hall, the expected cosmic muon
    flux in MicroBooNE is $160$~m$^{-2}$s$^{-1}$ based on Monte Carlo
    simulation~\cite{cosmicnote}. This gives a rate of about 9 muons per 1.6~ms
    readout frame. We only expect one detectable neutrino interaction in the
    TPC for every 500 neutrino beam spills (about one every 10~s), which is
    much smaller than the cosmic ray rate.

  \subsubsection{MicroBooNE light collection system and event trigger}\label{sec:swtrigger}
    Liquid argon also produces scintillation light. About 24,000 photons are
    produced per MeV of deposited energy ~\cite{detectorpaper}. The
    scintillation photons have a wavelength of 128 nm and can not re-interact
    with the argon. The light collection system in MicroBooNE consists of
    32 photomultiplier tubes (PMTs) outside of the anode side of the
    TPC. Each PMT is shielded by an acrylic plate coated in
    tetraphenyl-butadiene (TPB) that shifts the ultraviolet photons to the
    visible spectrum before interacting with the PMT.

    The light collection system in MicroBooNE is particularly useful for timing
    information. Since each TPC event is readout over several milliseconds, it
    is difficult to determine which events have activity inside the neutrino
    beam spill time window and which consist of only cosmic background activity
    based on the TPC information alone. The PMTs have nanosecond timing
    resolution which allows us to only consider events that have optical
    activity during a time window surrounding the 1.6~$\mu$s neutrino beam
    spill. Figure~\ref{fig:bnbtiming} shows the fractional increase in flashes
    during the neutrino beam spill window.

    \begin{figure}[h]
      \centering
      \includegraphics[angle=0,width=5in]{figures/detector/beam/BNB_Timing.pdf}
      \caption{Fractional rate of flashes surrounding BNB spill window. The
      rate of flashes in the light collection system increases during the
      1.6~$\mu$s beam spill window.}
      \label{fig:bnbtiming}
    \end{figure}

    A PMT trigger is implemented in the data acquisition (DAQ) software that
    determines whether an event is saved. Events are only read into the DAQ
    when a neutrino beam-spill signal is received. Each event includes a
    23~$\mu$s window of PMT data that includes the time of the beam spill.
    This 23~$\mu$s of data is used to form a PMT software trigger. First,
    pulses are found on the individual PMT signals. The pulses are found using
    constant-fraction discriminators that open a 100~ns discriminator window.
    The discriminator difference required to open a discriminator window is 10
    ADC counts which corresponds to approximately a 0.5 photoelectron (PE)
    signal in the PMT. The largest discriminator difference during the
    discriminator window is saved as the pulse height. A new discriminator
    window cannot be initiated until the current one has close and there has
    been a 15~ns period without the discriminator firing.  Next, coincident
    pulses across the different PMTs are found and combined.  Pulse windows
    that are coincident in time are combined and their pulse heights are
    summed. If the sum of the coincident pulse heights is above 20 ADC counts,
    corresponding to approximately 6.5 PE, and the time of the coincident
    pulses is within the neutrino beam-spill window, the event is saved.


%%%%%%%%%%%%%%%%%%%%%%%%%%%%%%%%%%%%%%%%%%%%%%%%%%%%%%%%%%%
% Simulation and Reconstruction
%%%%%%%%%%%%%%%%%%%%%%%%%%%%%%%%%%%%%%%%%%%%%%%%%%%%%%%%%%%
\subsection{Simulation and Reconstruction}\label{sec:simreco}
  The entire experimental process from the neutrino interactions in and around
  the detector to electronic signal readout to particle identification is
  simulated in software. To interface different the software packages needed to
  simulate each step, a liquid argon software framework (LArSoft)~\cite{larsoft}
  was developed at Fermilab. Within the LArSoft framework, the simulation is
  divided into three steps: generation, propagation, and detector simulation.
  The simulated data output from the detector simulation stage is designed to
  match the real data output from the detector as closely as possible. Event
  reconstruction is also handled within LArSoft, and the same algorithms can be
  applied to both real data and data simulated through the detector simulation
  stage in an identical way. This section describes all three stages of
  simulation and reconstruction, including TPC particle track and optical flash
  reconstruction.

  \subsubsection{Cross section model}\label{sec:geniexsec}
    The initial neutrino interactions are simulated using the GENIE Neutrino
    Monte Carlo Generator~\cite{Andreopoulos:2009rq,Andreopoulos:2015wxa}.
    Assuming a given neutrino flux, genie simulates the interaction of the
    neutrino with the nucleons inside of an Argon atom. It also simulates the
    intranuclear interactions that occur while the nucleons and pions from the
    initial neutrino-nucleon interaction traverse the Argon nucleus.  Within
    GENIE the nuclear models and neutrino-nucleon cross sections are
    configurable by the user.  MicroBooNE uses the GENIE version v2\_12\_2
    default settings with the addition of the empirical MEC cross section
    model. The details of the physics models for each of the four processes
    (the nuclear physics model, the cross section model, the neutrino-induced
    hadron production model, and the intranuclear hadron transport model)
    within GENIE as well as the simulated cosmic ray generation are described
    in this section.

    First, the relativistic Fermi gas (RFG) nuclear model is used for all
    processes. It has been modified in GENIE to incorporate short-range
    nucleon-nucleon correlations using the model by Bodek and
    Ritchie~\cite{BodekrRitchie}. The mass density for the argon nucleus is
    taken from review articles~\cite{nucdensity} and the two-parameter
    Woods-Saxon density function is used~\cite{WoodsSaxon}
    \begin{equation}\label{eq:woodssaxon}
      \rho(r) = N_0\frac{1}{1+e^{(r-c)/z}} \,,
    \end{equation}
    where $\rho$ is the density, $r$ is the distance from the center of the
    nucleus, $c$ describes the size of the nucleus, and $z$ describes the width
    of the surface. For argon, GENIE uses $c=3.53$ and $z=0.54$.  For elastic
    nucleon scattering in the nucleus, Pauli blocking is
    applied~\cite{PauliBlock}.
    
    Next, in the case of elastic (and quasi-elastic) neutrino-nucleon
    scattering, the free-nucleon cross section is calculated. For
    neutral-current elastic scattering, the differential cross section model
    described by Ahrens et al.~\cite{Ahrens} is used. The vector form
    factors, $F_1$ and $F_2$ are given by
    \begin{equation}
      F_1(Q^2) = \frac{1}{2}\frac{(1+\sin^2\theta_W)(1+\frac{Q^2}{4M_p^2}(1+\mu_p)) 
        - (1-3\sin^2\theta_W)\mu_n}
        {(1+\frac{Q^2}{4M_p^2})(1+\frac{Q^2}{M_V^2})^2} \,,
    \end{equation}
    \begin{equation}
      F_2(Q^2) = \frac{1}{2}\frac{(1-3\sin^2\theta_W)\mu_p - (1+\sin^2\theta_W)\mu_n}
         {(1+\frac{Q^2}{4M_p^2})(1+\frac{Q^2}{M_V^2})^2} \,,
    \end{equation}
    where $\theta_W$ is the Weinberg angle, $Q^2$ is the momentum transfer,
    $M_p$ is the mass of the proton, $\mu_p$ and $\mu_n$ are the nucleon
    anomalous magnetic moments, and $M_V$ is the vector mass, all described in
    Sec.~\ref{sec:theory}. The dipole representation is used for the axial
    form factor, as well:
    \begin{equation}
      G_A(Q^2) = \frac{1}{2}\frac{g_A - \Delta s}{(1+\frac{Q^2}{M_A^2})^2} \,,
    \end{equation}
    where $g_A$ is the isovector axial charge and $M_A$ is the axial mass, both
    also described in Sec.~\ref{sec:theory}. For charged-current quasi-elastic
    scattering, the differential cross section model described by
    Llewellyn-Smith~\cite{Llewellyn} is used. The BBA2005~\cite{BBA05} vector
    form factors are used, and the same dipole axial form factor used in
    neutral-current elastic scattering is used here. Other neutrino-nucleon
    interactions are simulated and contribute to the elastic scattering
    backgrounds. They are described in detail by Andreopoulos et
    al.~\cite{Andreopoulos:2015wxa}.

    The hadrons produced in neutrino-nucleon interactions in a nucleus are
    modeled separately from the nuclear model and the free-nucleon cross
    section model. This is because hadron production in empirical neutrino
    scattering measurements does not match nuclear and neutrino-nucleon cross
    section models. The hadron discrepancy is corrected for in GENIE using the
    AGKY model~\cite{AGKY} that was developed to account for the data seen in
    the MINOS~\cite{MINOS} neutrino scattering experiment and tuned using
    bubble chamber experimental data.

    Finally, the transport of the final state hadrons through the argon atom is
    modeled. The hadrons produced in the neutrino-nucleon interactions can
    rescatter as the exit the nucleus which changes the observable final state
    particle distributions. The intranuclear transport model is implemented by
    the GENIE subpackage called INTRANUKE. INTRANUKE uses a cascade model in
    which the hadron sees a nucleus of isolated nucleons. The interaction
    probability is calculated based on the free nucleon cross section and the
    nucleon density~\cite{Andreopoulos:2015wxa}
    \begin{equation}
      \lambda(E,r) = \frac{1}{\sigma_{hN,tot} \rho(r)} \,,
    \end{equation}
    where $\lambda$ is the hadron interaction probability, $E$ is the hadron
    energy, $r$ is the distance from the center of the nucleus,
    $\sigma_{hN,tot}$ is the total \textit{free} nucleon cross section, and
    $\rho$ is the density of the argon nucleus. The free nucleon cross section
    is different for protons, neutrons, and pions. The density of the argon
    nucleus is again determined by equation~\ref{eq:woodssaxon}.

  \subsubsection{Event generation}
    The GENIE neutrino events are generated in LArSoft based on the neutrino
    flux described in Sec.~\ref{sec:beam}. The spill structure with 1.6~$\mu$s
    long spill containing $5\times 10^{12}$ protons on target is simulated.  In
    addition to simulating neutrino events, cosmic ray events are also
    simulated. The CORSIKA~\cite{corsika} cosmic ray generator is used to
    generate these interactions. CORSIKA uses the FLUKA cosmic ray model and
    models the flux of protons, He, N, Mg, and Fe on the atmosphere.  To
    simulate real detector events as closely as possible, both neutrino and
    cosmic ray interactions can be generated in the same simulation. They can
    also be generated in separate events in order to isolate effects.

    Only neutrino interactions in the liquid argon filled cryostat are
    generated in the standard simulation. Special samples can be and are made
    to study how secondary particles from neutrino interactions outside the
    cryostat contribute to our signal background. One special sample that we
    generate is a large \textit{dirt} sample where neutrino interactions that
    happen in the dirt and detector hall outside of the cryostat are generated.
    The neutrino interactions are allowed to happen anywhere in the fifty feet
    of simulated dirt outside of the detector hall or anywhere in the detector
    hall outside of the cryostat. These are known as beam-induced, TPC-external
    (BITE) events.

    All of the Monte Carlo samples that are generated are listed here.
    \begin{enumerate}
      \item Neutrino-only interactions in the cryostat. This sample is to study
      exactly what neutrino signals would look like if there were no cosmic
      background.
      \item Neutrino plus cosmic interactions in the cryostat. This sample
      very closely represents the actual detector conditions when there is a
      neutrino interaction in MicroBooNE.
      \item BITE plus cosmic interactions. This sample contains all known
      backgrounds to neutrino events in the TPC.
      \item In-time cosmic only. This sample simulates events where cosmic
      interactions happen in time with the beam. This is the most problematic
      background because both the TPC and PMT signals looks very similar to
      neutrino events. It also occurs at a rate about ten times as large as the
      triggered neutrino event rate in MicroBooNE.
    \end{enumerate}

  \subsubsection{Detector simulation}
    The simulated final state particles from GENIE are passed to the
    GEANT4~\cite{geant4} software package to be propagated through the
    simulated geometry geometry. The entire MicroBooNE detector system, the
    detector hall, and fifty feet of dirt surrounding the detector hall are all
    included in the GEANT4 simulation. This includes the electric field and the
    detector electronics. The particles are stepped through the geometry with
    the possible physics processes simulated at each step. The particles are
    allowed to interact electromagnetically and hadronically with other
    particles and the detector system or decay through one of the physically
    possible decay modes. Additionally, the energy loss through ionization and
    scintillation is simulated for all particles traversing the detector
    geometry. In the case of the ionization of the liquid argon, the resulting
    electrons are propagated through the electric field to the wire readouts.
    For the scintillation of the argon, a photon library was generated for each
    position in the liquid cryostat. At each step particle takes through the
    liquid argon, the resulting photons that would interact in the PMTs are
    determined from a look-up table that was generated in a previous full
    optical simulation. 

    After the simulated particles interact with the TPC or PMT system, the
    detector response is simulated. The detector-simulation stage includes the
    electronic responses of the sensitive detectors and reproduces the
    electronic signals from the TPC and PMT systems. First, the PMT signal is
    digitized and the PMT software trigger described in
    Sec.~\ref{sec:swtrigger} is fully simulated. Events that do not pass the
    PMT trigger are dropped. The TPC electronics, including the electronic
    noise on the wires and unresponsive wires, are also included at this stage.
    At this point, the simulated data resembles the actual raw detector data as
    closely as possible.

  \subsubsection{Flash reconstruction}
    The optical flash reconstruction algorithm is applied identically to
    detector data and the simulated data that is output from the detector
    simulation stage.  The first step is to find pulses on the electronic
    signals read out from each of the 32 PMTs. This is done using a peak
    finding algorithm on the digitized signal. The time, amplitude, width, and
    area under the pulses are stored per pulse. Next, the flash reconstruction
    algorithm looks for coincident pulses across PMTs. The individual pulses
    are sorted by size, and all of the pulses that are within 30 nanoseconds of
    the largest pulse are collected. If there are at least three pulses in that
    time window and the sum of the pulse areas is at least 6~PE, it is considered a
    flash and saved. The peak time, width, position, and size are reconstructed
    and saved along with information about the pulses in the individual PMTs
    that contributed to the flash. This process is repeated starting with the
    next largest remaining pulse until there are none left. An individual pulse
    can only contribute to one flash in an event.

  \subsubsection{TPC event reconstruction}\label{sec:tpcreco}
    Reconstructing TPC events involves more steps than the PMTs since there are
    thousands of wires being read out for milliseconds resulting in
    approximately 30~MB of raw data per event. The reconstruction algorithms
    are again applied identically to detector data and the simulated data that
    is output from the detector simulated stage. First, a noise-deconvultion
    filter is passed over the digitized wire signals. Then, similar to the
    optical reconstruction, pulses, or \textit{hits}, are found on individual
    wires which are used as base building blocks for reconstructing 3D particle
    tracks across wires and wire planes.

    The 1D hit finding algorithm starts by walking along a wire signal until
    the value is above a given threshold. The point where the signal goes above
    the threshold is considered the start of the pulse, and the end of the
    pulse is defined as the point where the signal goes back down below the
    threshold.  Then, local minima and maxima are found between the start and
    end of the pulse which are used to determined where there are peaks within
    the pulse.  Adjacent pulses are merged if they are close enough in time.
    Once the pulse and the number of peaks is established, the algorithm
    attempts to fit Gaussian peaks to the pulse. The hypothesis signal is
    composed of one Gaussian per peak from the previous step. The mean and
    amplitude of each Gaussian is initially centered at the existing peaks and
    is allowed to float. If the residuals of the fit are sufficiently small,
    each Gaussian peak is saved as a 1D \textit{hit} with an amplitude, width,
    and time given by the fit. The hit finding is repeating along the length of
    the wire for each wire on all three planes.

    The 3D track finding algorithm is separated into two distinct parts. The
    first part attempts to reconstruct and tag as many cosmic-induced tracks.
    The hits associated with these tracks are then removed from the set of hits
    that are available to reconstruct neutrino-induced tracks in the second
    part. 

    The algorithm used to preferentially reconstruct cosmic tracks is called
    \textit{PandoraCosmic}. It is implemented in the Pandora Software
    Development Kit~\cite{Pandora} used by MicroBooNE. In PandoraCosmic, 1D
    hits are first clustered in 2D per wire readout plane. All of the
    reconstructed hits output by the hit finding algorithm on a given wire
    plane are used as input. The hits are clustered into unambiguous,
    continuous lines of hits. These initial clusters are meant to have a high
    purity, meaning all of the hits in the cluster were induced by the same
    true particle. The clusters in a 2D plane are then merged pairwise in an
    attempt to improve the completeness of the cluster, meaning most of the
    hits on that wire plane induced by a true particle are included in a single
    merged cluster. The 2D clusters are then matched across the three wire
    planes. The 3D track reconstruction algorithm attempts to find a cluster on
    each plane that each corresponds to the same true particle. All possible
    combinations of 2D cluster matching are considered by the algorithm. The
    most suitable set of cluster combinations are projected into 3D and saved
    as reconstructed track objects.

    Next, the reconstructed tracks from PandoraCosmic are passed to a cosmic
    tagging stage. There are two algorithms used to identify cosmic-induced
    tracks. The first is a geometry tagger that looks for tracks that are not
    fully contained in the TPC during the event, and the second is a
    flash-matching algorithm that looks for tracks that are inconsistent with
    any flashes in the beam spill window. Both of the cosmic tagging algorithms
    try to remove as few neutrino-induced tracks as possible and only remove
    tracks that are very likely cosmic-induced.
    
    The geometry tagger starts by locating any TPC wire hits that are
    reconstructed before of after the 1.6 millisecond readout frame. Any tracks
    that contain these hits are tagged as cosmic. Any tracks whose
    reconstructed start or end points are located within a given distance from
    the TPC boundary are also tagged. If both of the start and end points are
    near a TPC boundary, the track is given a cosmic score of 1. If only one of
    the start or end points is near a boundary, the track is given a cosmic
    score of 0.5.
    
    The flash-matching algorithm creates a hypothesis flash for each
    reconstructed track based on its position, size, and energy deposited. It
    then compares the hypothesis flash to each of the reconstructed flashes in
    the beam spill window. If a hypothesis flash is sufficiently incompatible
    with all true flashes in the window, the track is tagged as cosmic. After
    the cosmic tagging, all reconstructed hits associated with any of the
    tracks tagged as cosmic are removed from the set of possible hits that are
    used to reconstruct neutrino-induced tracks. This is refered to as the
    cosmic hit removal stage.

    Here describe the cluster and track finding algorithms that you end up
    using. Need to wait for MCC8 to finish to decide. Should give a lot of
    information about the final efficiency on MC data for the tracking
    algorithm. Give efficiency for protons and muons as a function of length,
    position, angle.

    Calorimetric information is extracted when the reconstructed track objects
    are created. The calorimetric information that we calculate and use is
    related to the energy loss of the particle that created the track along its
    trajectory. At each point along the track, the difference in the total
    charge between the current point and the previous trajectory point is
    calculated. This gives us the change in charge as a function of distance
    along the track, which we label dQ/dx. The change in charge is determined
    by adding the total charge of each of the reconstructed wire hits between
    the two points on the track. The charge difference and the dQ/dx values are
    found for each of the three wire planes. The dQ/dx values can be converted
    to energy loss per unit distance, dE/dx, by multiplying the dQ/dx by a
    measured conversion factor. This conversion factor depends on the strength
    of the electric field and the gain of the readout electronics and is
    determined empirically.

    There are also two different algorithms that parameterize the dE/dx curves
    into single parameters to easily evaluate how closely the dE/dx curve of a
    reconstructed track resembles the expected dE/dx curve for a given particle
    type. The first is PIDA which represents the magnitude of the dE/dx curve
    near the end of the track, and the second is PID $\chi^2$ which represents
    the difference between the dE/dx curve and the expected template. The PIDA
    algorithm uses an idealized power law parameterization, $dE/dx = AR^{-b}$,
    where $R$ is the residual range (distance to the end of the track) and $A$
    and $b$ are fit parameters. In the algorithm, $b$ is set to $-0.42$ and $A$
    is calculated at every step along the reconstructed track. All of the
    calculated $A$-values for a given track are fit with a Gaussian hypothesis,
    and the mean of the fit is the output PIDA parameter. The PID $\chi^2$
    algorithm calculates the $\chi^2$ residuals between the dE/dx curve of the
    reconstructed track and each of four template dE/dx curves. There is a
    template curve for a proton, kaon, pion, and muon hypothesis. The template
    curve with the lowest $\chi^2$ value is the most similar to the
    reconstructed track. All four template $\chi^2$ values are saved for each
    reconstructed track.


%This is the end of detector section

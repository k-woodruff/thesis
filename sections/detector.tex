\section{The MicroBooNE experiment}\label{microboone}

%%%%%%%%%%%%%%%%%%%%%%%%%%%%%%%%%%%%%%%%%%%%%%%%%%%%%%%%%%%
% MicroBooNE and Neutrino Beam
%%%%%%%%%%%%%%%%%%%%%%%%%%%%%%%%%%%%%%%%%%%%%%%%%%%%%%%%%%%
\subsection{The Booster Neutrino Beam and the MicroBooNE detector}\label{sec:beam}
  \subsubsection{The neutrino beam}
    Where do the proton come from, how do they get accelerated, what is their
    final energy distribution.
    Bunches and spills.
    The target/horn, it's shape and materials.
    The decay chain of particles coming out of the target (mention dirt).
    The final neutrino flux at MicroBooNE (also mention uncertainty ==> hence ratio).
    Expected number of neutrino interactions, expected NCE interactions.

  \subsubsection{Dirt neutrons}

  \subsubsection{MicroBooNE LArTPC}\label{sec:lartpc}
    Liquid argon target and TPC.
    LAr properties (density, ionization, scintillation).
    MicroBooNE electric field, drift time/distance.
    MicroBooNE specs (dimensions, wire counts).

  \subsubsection{MicroBooNE light collection system and event trigger}\label{sec:swtrigger}
    Liquid argon also produces scintillation light. About 24,000 photons are
    produced per MeV of deposited energy ~\cite{detectorpaper}. The
    scintillation photons have a wavelength of 128 nm and can not re-interact
    with the argon. The light collection system in MicroBooNE consists of
    thirty-two photomultiplier tubes (PMTs) outside of the anode side of the
    TPC. Each PMT is shielded by an acrylic plate coated in
    tetraphenyl-butadiene (TPB) that shifts the ultraviolet photons to the
    visible spectrum before interacting with the PMT.

    The light collection system in MicroBooNE is particularly useful for timing
    information. Since each TPC event is readout over several milliseconds, it
    is difficult to determine which events have activity inside the neutrino
    beam time window. The PMTs have nanosecond timing resolution which allows
    us to only consider events that have optical activity during a time window
    surrounding the 1.6 microsecond neutrino beam spill. 

    A PMT trigger is implemented in the data acquisition (DAQ) software that
    determines whether an event is saved. Events are only read into the DAQ
    when a neutrino beam-spill signal is received. Each event includes a
    23~microsecond window of PMT data that includes the time of the beam spill.
    This 23~microseconds of data is used to form a PMT software trigger. First,
    pulses are found on the individual PMT signals. The pulses are found using
    constant-fraction discriminators that open a 100~nanoseconds discriminator
    window. The discriminator difference required to open a discriminator
    window is 10 ADC counts which corresponds to approximately a 0.5
    photoelectron (PE) signal in the PMT. The largest discriminator difference
    during the discriminator window is saved as the pulse height. A new
    discriminator window cannot be initiated until the current one has close
    and there has been a 15~nanoseconds period without the discriminator
    firing.  Next, coincident pulses across the different PMTs are found and
    combined.  Pulse windows that are coincident in time are combined and their
    pulse heights are summed. If the sum of the coincident pulse heights is
    above 20 ADC counts, corresponding to approximately 6.5 PE, and the time of
    the coincident pulses is within the neutrino beam-spill window, the event
    is saved.

    


%%%%%%%%%%%%%%%%%%%%%%%%%%%%%%%%%%%%%%%%%%%%%%%%%%%%%%%%%%%
% Simulation and Reconstruction
%%%%%%%%%%%%%%%%%%%%%%%%%%%%%%%%%%%%%%%%%%%%%%%%%%%%%%%%%%%
\subsection{Simulation and Reconstruction}\label{sec:simreco}
  The entire experimental process from the neutrino interactions in and around
  the detector to electronic signal readout to particle identification is
  simulated in software. To interface different the software packages needed to
  simulate each step, a liquid argon software framework (LArSoft)~\cite{larsoft}
  was developed at Fermilab. Within the LArSoft framework, the simulation is
  divided into three steps: generation, propagation, and detector simulation.
  The simulated data output from the detector simulation stage is designed to
  match the real data output from the detector as closely as possible. Event
  reconstruction is also handled within LArSoft, and the same algorithms can be
  applied to both real data and data simulated through the detector simulation
  stage in an identical way. This section describes all three stages of
  simulation and reconstruction, including TPC particle track and optical flash
  reconstruction.

  \subsubsection{Event generation and cross section model}
    The initial neutrino interactions are simulated using the GENIE Neutrino
    Monte Carlo Generator~\cite{Andreopoulos:2009rq,Andreopoulos:2015wxa}.
    Assuming a given neutrino flux, genie simulates the interaction of the
    neutrino with the nucleons inside of an Argon atom. It also simulates the
    intranuclear interactions that occur while the nucleons and pions from the
    initial neutrino-nucleon interaction traverse the Argon nucleus.  Within
    GENIE the nuclear models and neutrino-nucleon cross sections are
    configurable by the user.  MicroBooNE uses the GENIE version v2\_12\_2
    default settings with the addition of the empirical MEC cross section model.

    Need to add: Corsika, cryostat volume, 5e12 POT, cross section and nuclear
    model, streams (BNB,BNB+cosmic,NuMI,intime-cosmic)

  \subsubsection{Detector simulation}
    The simulated final state particles from GENIE are passed to the
    GEANT4~\cite{geant4} software package to be propagated through the
    simulated geometry geometry. The entire MicroBooNE detector system, the
    detector hall, and fifty feet of dirt surrounding the detector hall are all
    included in the GEANT4 simulation. This includes the electric field and the
    detector electronics. The particles are stepped through the geometry with
    the possible physics processes simulated at each step. The particles are
    allowed to interact electromagnetically and hadronically with other
    particles and the detector system or decay through one of the physically
    possible decay modes. Additionally, the energy loss through ionization and
    scintillation is simulated for all particles traversing the detector
    geometry. In the case of the ionization of the liquid argon, the resulting
    electrons are propagated through the electric field to the wire readouts.
    For the scintillation of the argon, a photon library was generated for each
    position in the liquid cryostat. At each step particle takes through the
    liquid argon, the resulting photons that would interact in the PMTs are
    determined from a look-up table that was generated in a previous full
    optical simulation. 

    After the simulated particles interact with the TPC or PMT system, the
    detector response is simulated. The detector-simulation stage includes the
    electronic responses of the sensitive detectors and reproduces the
    electronic signals from the TPC and PMT systems. First, the PMT signal is
    digitized and the PMT software trigger described in
    Sec.~\ref{sec:swtrigger} is fully simulated. Events that do not pass the
    PMT trigger are dropped. The TPC electronics, including the electronic
    noise on the wires and unresponsive wires, are also included at this stage.
    At this point, the simulated data resembles the actual raw detector data as
    closely as possible.

  \subsubsection{Flash reconstruction}
    The optical flash reconstruction algorithm is applied identically to
    detector data and the simulated data that is output from the detector
    simulation stage.  The first step is to find pulses on the electronic
    signals read out from each of the 32 PMTs. This is done using a peak
    finding algorithm on the digitized signal. The time, amplitude, width, and
    area under the pulses are stored per pulse. Next, the flash reconstruction
    algorithm looks for coincident pulses across PMTs. The individual pulses
    are sorted by size, and all of the pulses that are within 30 nanoseconds of
    the largest pulse are collected. If there are at least three pulses in that
    time window and the sum of the pulse areas is at least 6~PE, it is considered a
    flash and saved. The peak time, width, position, and size are reconstructed
    and saved along with information about the pulses in the individual PMTs
    that contributed to the flash. This process is repeated starting with the
    next largest remaining pulse until there are none left. An individual pulse
    can only contribute to one flash in an event.

  \subsubsection{TPC event reconstruction}
    Reconstructing TPC events involves more steps than the PMTs since there are
    thousands of wires being read out for milliseconds resulting in
    approximately 30~MB of raw data per event. The reconstruction algorithms
    are again applied identically to detector data and the simulated data that
    is output from the detector simulated stage. First, a noise-deconvultion
    filter is passed over the digitized wire signals. Then, similar to the
    optical reconstruction, pulses, or \textit{hits}, are found on individual
    wires which are used as base building blocks for reconstructing 3D particle
    tracks across wires and wire planes.

    The 1D hit finding algorithm starts by walking along a wire signal until
    the value is above a given threshold. The point where the signal goes above
    the threshold is considered the start of the pulse, and the end of the
    pulse is defined as the point where the signal goes back down below the
    threshold.  Then, local minima and maxima are found between the start and
    end of the pulse which are used to determined where there are peaks within
    the pulse.  Adjacent pulses are merged if they are close enough in time.
    Once the pulse and the number of peaks is established, the algorithm
    attempts to fit Gaussians to the pulse. The hypothesis signal is composed
    of one Gaussian per peak from the previous step. The mean and amplitude of
    each Gaussian is initially centered at the existing peaks and is allowed to
    float. If the residuals of the fit are sufficiently small, each Gaussian
    peak is saved as a 1D \textit{hit} with an amplitude, width, and time given
    by the fit. The hit finding is repeating along the length of the wire for
    each wire on all three planes.

    Here describe the cluster and track finding algorithms that you end up
    using. Need to wait for MCC8 to finish to decide. Should give a lot of
    information about the final efficiency on MC data for the tracking
    algorithm. Give efficiency for protons and muons as a function of length,
    position, angle.

    Need to add: pandoraCosmic and flash matching/cosmic hit removal

    Calorimetric information is extracted when the reconstructed track objects
    are created. The calorimetric information that we calculate and use is
    related to the energy loss of the particle that created the track along its
    trajectory. At each point along the track, the difference in the total
    charge between the current point and the previous trajectory point is
    calculated. This gives us the change in charge as a function of distance
    along the track, which we label dQ/dx. The change in charge is determined
    by adding the total charge of each of the reconstructed wire hits between
    the two points on the track. The charge difference and the dQ/dx values are
    found for each of the three wire planes. The dQ/dx values can be converted
    to energy loss per unit distance, dE/dx, by multiplying the dQ/dx by a
    measured conversion factor. This conversion factor depends on the strength
    of the electric field and the gain of the readout electronics and is
    determined empirically. 


%This is the end of detector section

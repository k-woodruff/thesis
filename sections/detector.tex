\section{The MicroBooNE experiment}\label{microboone}

%%%%%%%%%%%%%%%%%%%%%%%%%%%%%%%%%%%%%%%%%%%%%%%%%%%%%%%%%%%
% MicroBooNE and Neutrino Beam
%%%%%%%%%%%%%%%%%%%%%%%%%%%%%%%%%%%%%%%%%%%%%%%%%%%%%%%%%%%
\subsection{The Booster Neutrino Beam and the MicroBooNE detector}\label{beam}
  \subsubsection{The neutrino beam}
    Where do the proton come from, how do they get accelerated, what is their final energy distribution.
    Bunches and spills.
    The target/horn, it's shape and materials.
    The decay chain of particles coming out of the target (mention dirt).
    The final neutrino flux at MicroBooNE (also mention uncertainty ==> hence ratio).
    Expected number of neutrino interactions, expected NCE interactions.
  \subsubsection{Dirt neutrons}
  \subsubsection{MicroBooNE LArTPC}
    Liquid argon target and TPC.
    LAr properties (density, ionization, scintillation).
    MicroBooNE electric field, drift time/distance.
    MicroBooNE specs (dimensions, wire counts).
  \subsubsection{MicroBooNE PMT system}
    PMT layout, timing, TPB.

%%%%%%%%%%%%%%%%%%%%%%%%%%%%%%%%%%%%%%%%%%%%%%%%%%%%%%%%%%%
% DAQ and Trigger
%%%%%%%%%%%%%%%%%%%%%%%%%%%%%%%%%%%%%%%%%%%%%%%%%%%%%%%%%%%
\subsection{Data Acquisition and Trigger}\label{daq}
  \subsubsection{PMT Readout Electronics and Trigger}
    Consist of signal shaper boards, an FEM modified from TPC version, PMT
    feedthrough, HV/signal splitters, and a trigger board. Write about optical
    flash reconstruction.
  \subsubsection{TPC Readout Electronics}
    Describe cold electronics (front end ASIC preamplifies and shapes), warm
    interface electronics (intermediate amplifier for transmission over 20 m.
    long cable), and digitizing electronics (TPC readout board in crate
    continuously samples received signals and passing from ADC to FPGA for
    processing, reducing, and storage). Maybe also talk about cabling and
    signal feedthrough. Also discuss specs like sampling rate.
  \subsubsection{DAQ System}
    Receives and buffers data from TPC and PMT readout system, then builds and
    records event based on trigger decision. Write about SEBs and EVB and
    passing data between. Should go into detail here about trigger algorithm.

%%%%%%%%%%%%%%%%%%%%%%%%%%%%%%%%%%%%%%%%%%%%%%%%%%%%%%%%%%%
% Simulation and Reconstruction
%%%%%%%%%%%%%%%%%%%%%%%%%%%%%%%%%%%%%%%%%%%%%%%%%%%%%%%%%%%
\subsection{Simulation and Reconstruction}\label{reco}
  The entire experimental process from the neutrino interactions in and around
  the detector to electronic signal readout to particle identification is
  simulated in software. To interface different the software packages needed to
  simulate each step, a liquid argon software framework (LArSoft)~\cite{larsoft}
  was developed at Fermilab. Within the LArSoft framework, the simulation is
  divided into three steps: generation, propagation, and detector simulation.
  The simulated data output from the detector simulation stage is designed to
  match the real data output from the detector as closely as possible. Event
  reconstruction is also handled within LArSoft, and the same algorithms can be
  applied to both real data and data simulated through the detector simulation
  stage in an identical way. This section describes all three stages of
  simulation and reconstruction, including TPC particle track and optical flash
  reconstruction.
  \subsubsection{Simulation}
    The initial neutrino interactions are simulated using the GENIE Neutrino
    Monte Carlo Generator~\cite{Andreopoulos:2009rq,Andreopoulos:2015wxa}.
    Assuming a given neutrino flux, genie simulates the interaction of the
    neutrino with the nucleons inside of an Argon atom. It also simulates the
    intranuclear interactions that occur while the nucleons and pions from the
    initial neutrino-nucleon interaction traverse the Argon nucleus.  Within
    GENIE the nuclear models and neutrino-nucleon cross sections are
    configurable by the user.  MicroBooNE uses the GENIE version v2\_12\_2
    default settings with the addition of the empirical MEC cross section model
    to be described in section~\ref{ratios}.

    The simulated final state particles from GENIE are passed to the
    GEANT4~\cite{geant4} software package to be propagated through the
    simulated geometry geometry. The Entire MicroBooNE detector system, the
    surrounding detector hall, and fifty feet of dirt are all included in the
    GEANT4 simulation. Both the ionization and scintillation of the liquid
    argon is simulated, and the resulting photons and electrons are propagated
    to the sensitive detectors.

    After the simulated particles interact with the TPC or PMT system, the
    detector response is simulated. The detector simulation section includes
    the analog to digital conversion of the PMT and TPC signals and the
    simulation and application of the beam and optical trigger. The detector
    noise and unresponsive TPC wires are also included at this stage of the
    simulation. At this point, the simulated data resembles the actual detector
    data as closely as possible.

  \subsubsection{TPC event reconstruction}
    The event reconstruction steps are applied identically to detector data and
    the simulated data that is output from the detector simulation stage.
    First, software filters are applied to reduce the electronic noise on the
    wire signal. 

    Write about filtering raw signal and hit finding algorithm used.
    I think give efficiencies at this stage.
    Explain how hits are clustered and then combined into tracks or showers.
    Explain algorithm used and give efficiencies at this stage. Also include
    cosmic tagging and calorimetry.
  \subsubsection{Flash reconstruction}
    Describe simpleFlash algorithm.


%%%%%%%%%%%%%%%%%%%%%%%%%%%%%%%%%%%%%%%%%%%%%%%%%%%%%%%%%%%
% Particle Identification
%%%%%%%%%%%%%%%%%%%%%%%%%%%%%%%%%%%%%%%%%%%%%%%%%%%%%%%%%%%
\subsection{Particle Identification}
  Classification goals (cosmic rejection and neutrino-induced particle type).
  \subsubsection{Reconstructed track features}
    Itemize and discuss every feature used as inuot to classifier.
    Separate into cosmic rejection and particle ID.
  \subsubsection{Boosted decision trees}
    Decision trees, tree boosting, xgboost.
  \subsubsection{Performance}
    Show efficiency, accuracy, itemize backgrounds.
    Discuss reasons for different backgrounds.


%This is the end of detector section

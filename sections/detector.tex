\section{The MicroBooNE experiment}\label{microboone}

%%%%%%%%%%%%%%%%%%%%%%%%%%%%%%%%%%%%%%%%%%%%%%%%%%%%%%%%%%%
% MicroBooNE and Neutrino Beam
%%%%%%%%%%%%%%%%%%%%%%%%%%%%%%%%%%%%%%%%%%%%%%%%%%%%%%%%%%%
\subsection{The Booster Neutrino Beam and the MicroBooNE detector}\label{beam}
  \subsubsection{The neutrino beam}
    Where do the proton come from, how do they get accelerated, what is their
    final energy distribution.
    Bunches and spills.
    The target/horn, it's shape and materials.
    The decay chain of particles coming out of the target (mention dirt).
    The final neutrino flux at MicroBooNE (also mention uncertainty ==> hence ratio).
    Expected number of neutrino interactions, expected NCE interactions.
  \subsubsection{Dirt neutrons}
  \subsubsection{MicroBooNE LArTPC}
    Liquid argon target and TPC.
    LAr properties (density, ionization, scintillation).
    MicroBooNE electric field, drift time/distance.
    MicroBooNE specs (dimensions, wire counts).
  \subsubsection{MicroBooNE light collection system and event trigger}
    Liquid argon also produces scintillation light. About 24,000 photons are
    produced per MeV of deposited energy ~\cite{detectorpaper}. The
    scintillation photons have a wavelength of 128 nm and can not re-interact
    with the argon. The light collection system in MicroBooNE consists of
    thirty-two photomultiplier tubes (PMTs) outside of the anode side of the
    TPC. Each PMT is shielded by an acrylic plate coated in
    tetraphenyl-butadiene (TPB) that shifts the ultraviolet photons to the
    visible spectrum before interacting with the PMT.

    The light collection system in MicroBooNE is particularly useful for timing
    information. Since each TPC event is readout over several milliseconds, it
    is difficult to determine which events have activity inside the neutrino
    beam time window. The PMTs have nanosecond timing resolution which allows
    us to only consider events that have optical activity during a time window
    surrounding the 1.6 microsecond neutrino beam spill. 

    A PMT trigger is implemented in the data acquisition (DAQ) software that
    determines whether an event is saved. Events are only read into the DAQ
    when a neutrino beam-spill signal is received. Each event includes a
    23~microsecond window of PMT data that includes the time of the beam spill.
    This 23~microseconds of data is used to form a PMT software trigger. First,
    pulses are found on the individual PMT signals. The pulses are found using
    constant-fraction discriminators that open a 100~nanoseconds discriminator
    window. The discriminator difference required to open a discriminator
    window is 10 ADC counts which corresponds to approximately a 0.5
    photoelectron (PE) signal in the PMT. The largest discriminator difference
    during the discriminator window is saved as the pulse height. A new
    discriminator window cannot be initiated until the current one has close
    and there has been a 15~nanoseconds period without the discriminator
    firing.  Next, coincident pulses across the different PMTs are found and
    combined.  Pulse windows that are coincident in time are combined and their
    pulse heights are summed. If the sum of the coincident pulse heights is
    above 20 ADC counts, corresponding to approximately 6.5 PE, and the time of
    the coincident pulses is within the neutrino beam-spill window, the event
    is saved.

    


%%%%%%%%%%%%%%%%%%%%%%%%%%%%%%%%%%%%%%%%%%%%%%%%%%%%%%%%%%%
% Simulation and Reconstruction
%%%%%%%%%%%%%%%%%%%%%%%%%%%%%%%%%%%%%%%%%%%%%%%%%%%%%%%%%%%
\subsection{Simulation and Reconstruction}\label{reco}
  The entire experimental process from the neutrino interactions in and around
  the detector to electronic signal readout to particle identification is
  simulated in software. To interface different the software packages needed to
  simulate each step, a liquid argon software framework (LArSoft)~\cite{larsoft}
  was developed at Fermilab. Within the LArSoft framework, the simulation is
  divided into three steps: generation, propagation, and detector simulation.
  The simulated data output from the detector simulation stage is designed to
  match the real data output from the detector as closely as possible. Event
  reconstruction is also handled within LArSoft, and the same algorithms can be
  applied to both real data and data simulated through the detector simulation
  stage in an identical way. This section describes all three stages of
  simulation and reconstruction, including TPC particle track and optical flash
  reconstruction.
  \subsubsection{Simulation}
    The initial neutrino interactions are simulated using the GENIE Neutrino
    Monte Carlo Generator~\cite{Andreopoulos:2009rq,Andreopoulos:2015wxa}.
    Assuming a given neutrino flux, genie simulates the interaction of the
    neutrino with the nucleons inside of an Argon atom. It also simulates the
    intranuclear interactions that occur while the nucleons and pions from the
    initial neutrino-nucleon interaction traverse the Argon nucleus.  Within
    GENIE the nuclear models and neutrino-nucleon cross sections are
    configurable by the user.  MicroBooNE uses the GENIE version v2\_12\_2
    default settings with the addition of the empirical MEC cross section model
    to be described in section~\ref{ratios}.

    The simulated final state particles from GENIE are passed to the
    GEANT4~\cite{geant4} software package to be propagated through the
    simulated geometry geometry. The Entire MicroBooNE detector system, the
    surrounding detector hall, and fifty feet of dirt are all included in the
    GEANT4 simulation. Both the ionization and scintillation of the liquid
    argon is simulated, and the resulting photons and electrons are propagated
    to the sensitive detectors.

    After the simulated particles interact with the TPC or PMT system, the
    detector response is simulated. The detector simulation section includes
    the analog to digital conversion of the PMT and TPC signals and the
    simulation and application of the beam and optical trigger. The detector
    noise and unresponsive TPC wires are also included at this stage of the
    simulation. At this point, the simulated data resembles the actual detector
    data as closely as possible.

  \subsubsection{TPC event reconstruction}
    The event reconstruction steps are applied identically to detector data and
    the simulated data that is output from the detector simulation stage.
    First, software filters are applied to reduce the electronic noise on the
    wire signal.

    Here describe noise deconvolution as it is in MCC8 and the hit finding
    algorithm. The hit-finding will be gaushit, need to read about how it
    works.

    Here describe the cluster and track finding algorithms that you end up
    using. Need to wait for MCC8 to finish to decide. Should give a lot of
    information about the final efficiency on MC data for the tracking
    algorithm. Give efficiency for protons and muons as a function of length,
    position, angle.

    Calorimetric information is extracted when the reconstructed track objects
    are created. The calorimetric information that we calculate and use is
    related to the energy loss of the particle that created the track along its
    trajectory. At each point along the track, the difference in the total
    charge between the current point and the previous trajectory point is
    calculated. This gives us the change in charge as a function of distance
    along the track, which we label dQ/dx. The change in charge is determined
    by adding the total charge of each of the reconstructed wire hits between
    the two points on the track. The charge difference and the dQ/dx values are
    found for each of the three wire planes. The dQ/dx values can be converted
    to energy loss per unit distance, dE/dx, by multiplying the dQ/dx by a
    measured conversion factor. This conversion factor depends on the strength
    of the electric field and the gain of the readout electronics and is
    determined empirically. 

  \subsubsection{Flash reconstruction}

    In additional to the TPC signals, the optical information needs to be
    reconstructed. Again, the optical flash reconstruction is handled in the
    same way for both simulation and detector data.

    Similar to TPC reconstruction, the first step in optical reconstruction is
    to find peaks in the electronic signals read out from the PMTs. We call the
    algorithm used to find these peak \textit{OpHit}. 

  \subsubsection{Cosmic tagging}


%This is the end of detector section

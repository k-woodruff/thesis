\section{The MicroBooNE experiment}\label{microboone}
MicroBooNE is an accelerator neutrino experiment at Fermilab. The experiment
measures neutrino physics properties by studying the interactions of neutrinos
produced by Fermilab accelerators. The MicroBooNE detector is a liquid argon
time projection chamber (LArTPC) with an additional light collection system.
LArTPC technology is relatively new, and allows for high-resolution imaging of
the neutrino interactions in the liquid argon. The production of the neutrino
beam, the MicroBooNE LArTPC, and the light collection system are described in
this section.

%%%%%%%%%%%%%%%%%%%%%%%%%%%%%%%%%%%%%%%%%%%%%%%%%%%%%%%%%%%
% The Neutrino Beam
%%%%%%%%%%%%%%%%%%%%%%%%%%%%%%%%%%%%%%%%%%%%%%%%%%%%%%%%%%%
\subsection{The Booster Neutrino Beam}\label{sec:beam}
  The neutrino beam used by the MicroBooNE experiment is produced using proton
  accelerators that were already in use at Fermilab. In fact, the proton beam
  that is extracted to go to the proton target used by MicroBooNE to produce
  neutrinos, is the same initial proton beam used by every other accelerator
  experiment at the lab. The Booster Neutrino experiments, which include
  MicroBooNE, are the first set of experiments on the proton accelerator line.

  The first step in creating the Booster Neutrino Beamline (BNB) is to
  accelerate negatively ionized hydrogen through a linear accelerator, or
  Linac. The Linac is 500~ft long and accelerates the hydrogen ions to
  400~MeV using radio frequency (RF) cavities. The hydrogen ions are then
  injected into the Booster accelerator. During injection, the ions pass
  through a foil that strips both electrons leaving only the positive proton.
  The Booster~\cite{AguilarArevalo:2008yp} is a 474-meter-circumference, 15~Hz
  synchrotron that accelerates the protons from 400~MeV to 8~GeV. The proton
  beam leaving the Booster has a bunched structure. Each turn contains 81
  proton bunches that are 2~ns wide and 19~ns apart. At this stage, a
  fraction of the 8~GeV protons are extracted to be sent to the Booster
  neutrino target. A kicker magnet is used to extract all 81 proton bunches
  in a turn which we refer to as a spill. One spill is 1.6~$\mu$s long and
  contains $5\times 10^{12}$ protons.

  The proton target is a beryllium cylinder 71.1~cm long and 0.51~cm thick that
  is aligned with the proton beam. The interactions of the protons in the
  beryllium produce secondary hadrons including charged and neutral pion,
  kaons, and nucleons. The decay of the charged pions, charged and neutral
  kaons, and muons from pion and kaon decays all contribute to the neutrino
  flux at the MicroBooNE detector. The charged pions and kaons are focused by a
  pulsed toroidal electromagnet operating at a maximum frequency of 5~Hz which
  is the limiting factor in the beam spill frequency. In neutrino mode, the
  magnet focuses positive particles and defocuses negative particles.  The
  positive pions and kaons then pass through an air-filled decay pipe where
  they quickly decay into positive muons (anti-muons) and neutrinos. In
  anti-neutrino mode, which is not used in this analysis, negative pions and
  kaons are focused which decay into negative muons and anti-neutrinos. The
  anti-muons and neutrinos next pass through a beam stop made of steel and
  concrete to filter out remaining protons, pions, and kaons.  Finally, the
  anti-muons and neutrinos pass through almost 500~meters of dirt where the
  anti-muons stop and decay, and a (mostly) pure neutrino beam remains.  The
  predicted neutrino flux at MicroBooNE is shown in Fig.~\ref{fig:beamflux} per
  $1\times 10^6$ protons on the target (POT). The neutrino flux prediction is
  based on Geant4~\cite{Agostinelli:2002hh} (describe in Sec.~\ref{sec:detsim})
  based Monte Carlo simulation of protons interacting with the beryllium target
  and the magnetic focusing horn. The production rate of the secondary hadrons
  are determined from external proton-beryllium experimental
  data~\cite{Catanesi:2007ab}.

  \begin{figure}[h]
    \centering
    \includegraphics[angle=0,width=4in]{figures/detector/beam/FluxPrediction.pdf}
    \caption{Predicted neutrino beam flux at MicroBooNE.}
    \label{fig:beamflux}
  \end{figure}

  Neutrinos from the BNB can interact in the dirt upstream of the MicroBooNE
  detector. Most of these interactions have no effect in the detector because
  the secondary particles do not travel far enough to pass through the dirt and
  enter the detector. However, secondary neutrons that are produced in the dirt
  near the detector can enter the detector at a significant rate.
  Figures~\ref{fig:dirtstarttop}~and~\ref{fig:dirtstartside} show the position
  of all of simulated neutrino interactions in the dirt surrounding MicroBooNE
  in gray and the subset of those interactions that produce a neutron which
  enters the MicroBooNE TPC in red from two different angles. The details of
  the simulation are described in Sec.~\ref{sec:simreco}. If one of these
  neutrons interacts in the MicroBooNE detector and scatters a single proton,
  the signal could look indistinguishable from a neutral-current elastic
  neutrino interaction that occurred in the detector.

  \begin{figure}[h]
    \includegraphics[angle=0,width=5in]{figures/detector/beam/DirtStart_ZX.png}
    \caption{Top view of all simulated neutrino interaction positions.}
    \label{fig:dirtstarttop}
  \end{figure}
  \begin{figure}[h]
    \includegraphics[angle=0,width=5in]{figures/detector/beam/DirtStart_ZY.png}
    \caption{Side view of all simulated neutrino interaction positions.}
    \label{fig:dirtstartside}
  \end{figure}

%%%%%%%%%%%%%%%%%%%%%%%%%%%%%%%%%%%%%%%%%%%%%%%%%%%%%%%%%%%
% The LArTPC
%%%%%%%%%%%%%%%%%%%%%%%%%%%%%%%%%%%%%%%%%%%%%%%%%%%%%%%%%%%
\subsection{MicroBooNE LArTPC}\label{sec:lartpc}
  The MicroBooNE LArTPC~\cite{Acciarri:2016smi} acts as both a target for the neutrino
  beam and a detector for the charged particles produced in the neutrino-argon
  interactions. The MicroBooNE TPC is submerged in 170~tons of liquid argon, of
  which 87~tons is contained inside the TPC. All of this is contained within a
  cylindrical cryostat.
  
  Liquid argon is used as the detector material for several reasons. Like all
  noble liquids, argon produces both ionization charge and scintillation light
  when stimulated.  The ionization electrons do not easily recombine, so they
  are able to pass through the liquid argon to be collected.  Additionally,
  argon is transparent to its own scintillation light, making it detectable.
  Noble liquids also have good dielectric properties allowing them to withstand
  high voltages without breaking down. Liquid argon is also relatively dense at
  1.4~g/cm$^3$, making it a good target for neutrinos which interact extremely
  rarely. Lastly, argon is very abundant.  It makes up 1\% of Earth's
  atmosphere and is therefore an affordable option compared to heaver noble
  elements.

  \begin{figure}[h]
    \centering
    \includegraphics[angle=0,width=5in]{figures/detector/tpc/LArTPC_Concept.png}
    \caption{Representation of the operational principle of the MicroBooNE
    LArTPC.}
    \label{fig:tpccartoon}
  \end{figure}

  The MicroBooNE time projection chamber (TPC) uses a uniform electric field to
  guide the ionization electrons to an anode. The electrons are drifted
  horizontally, perpendicular to the beam direction. To produce this uniform,
  horizontal electric field, a 10~m$\times$2.6~m cathode plane makes up one
  face of the TPC. This face is vertical and parallel to the beam. A voltage of
  -70~kV is applied to the anode plane which results in a 273~V/cm electric
  field across the 2.3~m TPC width.  There are three wire planes at the anode,
  perpendicular to the electron drift direction and parallel to the cathode
  plane, that are strung with thousands of wires spaced 3~mm apart.  The three
  planes are all parallel to each other and are also spaced 3~mm apart.  The
  wires on the first plane have a small negative voltage (-200~V), the middle
  plane has no voltage, and the last plane has a small positive voltage
  (+440~V). When the electrons approach the first wire plane, they induce a
  signal on the closest wires, and are then attracted to the next plane because
  of its slightly less negative voltage. This is repeated at the second wire
  plane, and the electrons are terminated at the third plane that we refer to
  as the collection plane. The electronic signals from all 8256 wires are read
  out and saved. The wires on each plane are at different angles to allow for
  an accurate 2-dimensional reconstruction of the event in the detector. The
  collection plane wires are vertical, and the wires on the two induction
  planes are at positive and negative 60~degrees from the beam direction. The
  third dimension, perpendicular to the anode planes, is reconstructed based on
  the time that the electrons arrived at the wires. A cartoon representation of
  this process is shown in Fig.~\ref{fig:tpccartoon}. The wires are read out
  for 1.6~ms per event to ensure that the ionization electrons have time to
  traverse the entire width of the detector. One 1.6~ms readout window is
  referred to as a frame.

  \begin{figure}[h]
    \centering
    \begin{subfigure}[t]{2.5in}
      \includegraphics[angle=0,width=2.5in]{figures/detector/tpc/Interactions_MicroBooNE.eps}
      \caption{The number of interactions is broken down by interaction type
      for CC interactions.}
      \label{fig:interactionsal}
    \end{subfigure}
    \hspace{2pt}
    \begin{subfigure}[t]{2.5in}
      \includegraphics[angle=0,width=2.5in]{figures/detector/tpc/NC_Interactions_MicroBooNE.eps}
      \caption{The number of interactions is broken down by interaction type
      for NC interactions. NC elastic scattering off of both protons and
      neutrons is shown in red.}
      \label{fig:interactionsnc}
    \end{subfigure}
    \caption{Expected number of muon neutrino interactions in the
    MicroBooNE TPC based on Monte Carlo simulation as a function of POT.}
    \label{fig:interactions}
  \end{figure}

  The expected number of neutrino interactions in MicroBooNE as a function of
  protons on target (POT) is shown in Fig.~\ref{fig:interactions}. These were
  determined using a Monte Carlo simulation described in
  Sec.~\ref{sec:simulation}. The plots shows the expected number of events up
  to $1\times 10^{21}$ POT. MicroBooNE is approved to run for a total of
  $1.3\times 10^{21}$ POT. Figure~\ref{fig:interactionsnc} shows the expected
  number of the subset of neutral current interactions with NC elastic
  interactions in red. This line includes interactions on both proton and
  neutrons, so we expect the number of NC elastic proton events to be about
  half this. The integrated number of all expected NC elastic neutrino-proton
  events from $1.3\times 10^{21}$ POT is approximately 30,000.

  The MicroBooNE detector is positioned just below the surface of the Earth,
  with no substantial overburden for shielding cosmic rays. The rate of cosmic
  ray interactions in the detector is therefore large. Based on Fermilab's
  elevation (226~m above sea level) and MicroBooNE's geometry within the
  detector hall, the expected cosmic muon flux in MicroBooNE is
  $160$~m$^{-2}$s$^{-1}$ based on MicroBooNE simulation~\cite{uBCosmicNote}
  using the CORSIKA Monte Carlo cosmic ray generator~\cite{Heck:1998vt}
  (described in Sec.~\ref{sec:eventgen}). This gives a rate of about 9 muons
  per 1.6~ms readout frame. We only expect one detectable neutrino interaction
  in the TPC for every 500 neutrino beam spills (about one every 10~s), which
  is much smaller than the cosmic ray rate.

%%%%%%%%%%%%%%%%%%%%%%%%%%%%%%%%%%%%%%%%%%%%%%%%%%%%%%%%%%%
% The Light Collection System
%%%%%%%%%%%%%%%%%%%%%%%%%%%%%%%%%%%%%%%%%%%%%%%%%%%%%%%%%%%
\subsection{MicroBooNE light collection system and event trigger}\label{sec:swtrigger}
  Liquid argon also produces scintillation light. Tens of thousands of photons
  are produced per MeV of deposited energy ~\cite{Acciarri:2016smi}. The
  scintillation photons have a vacuum wavelength of 128~nm and do not
  re-interact with the argon assuming there is no significant amount of
  contaminants. In MicroBooNE, the electron drift lifetime is greater than 6
  ms, which is twice as long as what is required to perform
  analyses~\cite{uBPurityNote}. The light collection system in MicroBooNE
  consists of 32 photomultiplier tubes (PMTs) behind the TPC anode wire planes.
  Each PMT is shielded by an acrylic plate coated in tetraphenyl-butadiene
  (TPB) that shifts the ultraviolet photons to the visible spectrum before they
  interact with the PMT.

  The light collection system in MicroBooNE is particularly useful for timing
  information. Since each TPC event is read out over several milliseconds, it
  is difficult to determine which events have activity coinciding with the
  neutrino beam spill and which consist of only cosmic background activity
  based on the TPC information alone. The PMTs have nanosecond timing
  resolution which allows us to only save events that have optical activity
  during a time window surrounding the 1.6~$\mu$s neutrino beam spill.
  Figure~\ref{fig:bnbtiming} shows the fractional increase in flashes during
  the neutrino beam spill window.

  \begin{figure}[h]
    \centering
    \includegraphics[angle=0,width=5in]{figures/detector/beam/BNB_Timing.pdf}
    \caption{Fractional rate of flashes surrounding BNB spill window.}
    \label{fig:bnbtiming}
  \end{figure}

  An event is saved when there is a coincidence between the accelerator beam
  signal and the PMT trigger implemented in the MicroBooNE data acquisition
  (DAQ) software. The PMT trigger is based on a $23 \mu$s window of PMT data
  which include the time of the neutrino beam spill. To form a trigger, first
  pulses are found on the thirty-two individual PMT signals using
  constant-fraction discriminators that open a 100~ns discriminator window, and
  a new window cannot be opened until the previous window has closed and there
  has been a 15~ns time period without the signal going above the discriminator
  threshold.  The discriminator threshold is 10~ADC counts which corresponds to
  approximately 0.5 photoelectrons (PE) in the PMT. The maximum of the PMT
  signal during the discriminator window is saved as the pulse height.  Next,
  coincident pulses across the different PMTs are found and combined.  Pulse
  windows that occur at the same time are combined and their pulse heights are
  summed. If the sum of the coincident pulse heights is above 20 ADC counts,
  corresponding to approximately 6.5 PE, and the time of the coincident pulses
  is within the neutrino beam-spill window, the event is saved.


%This is the end of detector section

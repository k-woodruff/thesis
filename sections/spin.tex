\section{The Spin Structure of the Nucleon} \label{sec:nuctheory}
%The next line produces an indented paragraph to start the document
 %unit.  The LaTeX defaults start most units without indentations.
\hspace{\parindent}

%%%%%%%%%%%%%%%%%%%%%%%%%%%%%%%%%%%%%%%%%%%%%%%%%%%%%%%%%%%
% Spin Structure of Nucleons
%%%%%%%%%%%%%%%%%%%%%%%%%%%%%%%%%%%%%%%%%%%%%%%%%%%%%%%%%%%
\subsection{The Spin Structure of Nucleons} \label{sec:nuctheory}
  Proton spin: \\
  --- spin vector $s_{\mu}$ from forward matrix element of axial vector current \\
  --- derive axial charges \\
  From Bass 1.1 (need to fill in with Peskin): \\
  Forward matrix element of the axial current vector (derive from peskin):
  \[
      2Ms_{\mu} = <p,s|\bar{\psi}\gamma_{\mu} \gamma_{5} \psi|p,s>
  \]
  where $s_{\mu}$ is the proton's spin vector, $\psi$ is the proton field
  vector and $M$ is the proton mass. The quark axial charges measure
  information about the quark ``spin content".
  \[
    2Ms_{\mu}\Delta q = <p,s| \bar{q}\gamma_{\mu}\gamma_{5}q|p,s> \,,
  \]
  where $q$ is the quark field operator and $\Delta q$ is the quark
  flavor-dependent axial charge, $\Delta u$, $\Delta d$, or $\Delta s$. The
  isovector, SU(3) octet, and flavor-singlet axial charges, $g_A^{(3)}$,
  $g_A^{(8)}$, and $g_A^{(0)}$, respectively, can be written as linear
  combinations of the quark axial charges (derive from Peskin)
  \begin{align}
      g_A^{(3)} &= \Delta u - \Delta d \\
      g_A^{(8)} &= \Delta u + \Delta d - 2\Delta s \\
      g_A^{(0)} &= \Delta u + \Delta d + \Delta s \,.
  \end{align}
  Can be interpreted semi-classically as amount of spin carried by quarks and
  antiquarks of flavor $q$.


  --- expectation of axial charges from non-relativistic quark model
  --- scaling and Regge theory?
  \subsubsection{Proton spin puzzle}
    --- parton model \\
    --- spin structure functions (g1 and g2) \\
    --- spin sum rules: Bjorken and Ellis-Jaffe \\
  \subsubsection{Nucleon Form Factors}
    Talk about how to parameterize the nucleon in terms of the form factors
    Discuss Dirac, Pauli, electric, and magnetic.
    %%% starting text %%%
    From Thomas, we derive elastic lepton-nucleon scattering starting with
    conservation of energy and momentum. \\
    Section 2.1: \\
    Elastic scattering means that the nucleon remains in it's ground state so
    that the energy transfer is
    \[
        \nu = \epsilon - \epsilon' = E' - E,
     \]
    where $\epsilon$ is energy of incident electron, E is energy of the nucleon.
    Also, three-momentum transfer is
    \[
        \bar{q} = \bar{p} - \bar{p}' = \bar{P}' - \bar{P} \,.
    \]
    So, squared four-momentum is
    \[
        q^2 = \nu^2 - \bar{q}^2 = -Q^2 < 0
    \]
    which is Lorentz invariant.
    From energy and momentum conservation, we get
    \[
        \epsilon' = \frac{\epsilon}{1+\frac{2\epsilon}{M}\mathrm{sin}^2\frac{\theta}{2}}
    \]
    and
    \[
        Q^2 = 4\epsilon \epsilon' \mathrm{sin}^2 \frac{\theta}{2} \,.
    \]
    This leads to the Mott differential cross section (pointlike, spinless)
    \[
        \left(\frac{d\sigma}{d\Omega} \right)_\textrm{Mott} = \frac{\alpha^2}{4\epsilon^2 \textrm{sin}^4\frac{\theta}{2}}\frac{\epsilon'}{\epsilon}\textrm{cos}^2\frac{\theta}{2} \,.
    \]
    If we make it spin 1/2 with a normal (Dirac) magnetic moment (still pointlike), we get an additional term (increases backwards angles):
    \[
        \frac{d\sigma}{d\Omega} = \left(\frac{d\sigma}{d\Omega} \right)_\textrm{Mott} \left[ 1 + \frac{Q^2}{4M^2}2\mathrm{tan}^2\frac{\theta}{2} \right] \,.
    \]
    Now we just need to add finite structure and anomolous magnetic moment,
    \[
        \frac{d\sigma}{d\Omega} = \left(\frac{d\sigma}{d\Omega} \right)_\textrm{Mott} 
        \left(F_1^2(Q^2) +\frac{Q^2}{4M}\left[F_2^2(Q^2)+2(F_1(Q^2)+F_2(Q^2))^2\,\textrm{tan}^2\frac{\theta}{2} \right] \right) \,.
    \]
    The Dirac and Pauli form factors, $F_1$ and $F_2$ respectively, can be parameterized as the electric and magnetic, or Sachs, form factors
    \begin{align}
        G_E(Q^2) &= F_1(Q^2) - \frac{Q^2}{4M}F_2(Q^2) \\
        G_M(Q^2) &= F_1(Q^2) + F_2(Q^2) \,.
    \end{align}
    Low $Q^2$ data. Sachs form factors when $Q^2$ goes to zero. Three dimensional fourier integral and slopes go to mean squared radii.

  \subsubsection{The Axial Form Factor}
    Derive the axial form factor. Discuss different axial form factor models
    (dipole...). Show that you can determine $\Delta s$. Make explicit that
    this is the same $\Delta s$ from first subsubsection.
  \subsubsection{Previous measurements}
    Polarized lepton-nucleon inclusive DIS \\
    SIDIS
    

%This is the end of the nucleon spin section

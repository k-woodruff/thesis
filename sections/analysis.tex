\section{Analysis}\label{sec:analysis}
%The next line produces an indented paragraph to start the document
 %unit.  The LaTeX defaults start most units without indentations.
\hspace{\parindent}
This section describes the analysis tools used to extract the strange axial
form factor parameters from the number of events in data and simulation that
pass the NC elastic proton selection. First, the method for determining the
expected number of events in data given different model and systematic
parameter values is discussed. Next, the formula for calculating the likelihood
of the observed data given a set of parameter values, and the technique used to
sample the probability distribution of the parameters is described. Last, the
results of the measured probability distributions of the strange axial form
factor parameters are shown.

%%%%%%%%%%%%%%%%%%%%%%%%%%%%%%%%%%%%%%%%%%%%%%%%%%%%%%%%%%%
% Comparison to Simulation
%%%%%%%%%%%%%%%%%%%%%%%%%%%%%%%%%%%%%%%%%%%%%%%%%%%%%%%%%%%
\subsection{Comparison of Data to Simulation}
  To determine what underlying true physics values cause the data that we
  measure, we compare the data to Monte Carlo simulation. The physics model of
  the data is a combination of many different models including nuclear physics
  models, neutrino cross section models, nucleon structure models, cosmic ray
  models, which would be very complicated to calculate directly.  Instead, we
  simulate the data we expect to see in the detector given a set of physical
  parameters and models and compare that directly to the actual data we
  measured. We do this for many possible values of the parameters and calculate
  the likelihood for each set of parameter values. In addition to varying the
  parameters that we want to measure in the simulation, we vary the physical
  parameters whose true values aren't well constrained which might have a large
  effect on the final data. This allows us to quantify the uncertainty due to
  the unknown quantities.
  \subsubsection{Event Reweighting}\label{sec:reweighting}
    Full Monte Carlo simulations are both time and compute intensive. Instead
    of re-running the simulation for each possible parameter value that we are
    interested in, we can calculate the ratio of the probabilities of each
    interaction occurring given the new parameter values to the probabilities
    of each interaction occurring for the original simulated parameter values.
    We refer to this ratio as an event weight:
    \begin{equation}
      w = \frac{P(\textrm{event}|\theta')}{P(\textrm{event}|\theta)} \,,
    \end{equation}
    where $w$ is the event weight for a given event, $P($event$|\theta)$ is the
    probability of the simulated event given the set of original parameters
    used in the simulation, $\theta$, and $P($event$|\theta')$ is the
    probability of the simulated event given a new set of parameters,
    $\theta'$.

    The four contributions to the GENIE cross section model (the nuclear
    physics model, the neutrino-nucleon cross section model, the hadron
    production model, and the intranuclear hadron transport model from
    Sec.~\ref{sec:geniexsec}) are treated independently in NC elastic
    events~\cite{Andreopoulos:2009rq}. This allows us to factor the event
    weight,
    \begin{equation}\label{eq:weights}
      w = w_{\textrm{nuclear}}\times w_{\textrm{neutrino-nucleon}}\times 
          w_{\textrm{hadron prod.}}\times w_{\textrm{intranuclear}} \,,
    \end{equation}
    where $w_{\textrm{nuclear}}$ is the nuclear physics model weight,
    $w_{\textrm{neutrino-nucleon}}$ is the neutrino-nucleon cross section model
    weight, $w_{\textrm{hadron prod.}}$ is the hadron production model weight,
    and $w_{\textrm{intranuclear}}$ is the intranuclear hadron transport model
    weight. Only the NC elastic cross section probability ratio needs to be
    calculated to see the effect of $\Delta s$ or another cross section
    parameter of interest on reconstructed $Q^2$ of selected events.

    The probability of a neutrino interaction is proportional to the
    interaction cross section, so the weight, $w_{\textrm{neutrino-nucleon}}$
    is simply a ratio of the cross sections
    \begin{equation}
      w_{\textrm{neutrino-nucleon}} = w_{\sigma} 
        = \frac{d^n\sigma'_{\nu}/dK^n}{d^n\sigma_{\nu}/dK^n} \,,
    \end{equation}
    where $d^n\sigma/dK^n$ is the differential cross section for the initially
    simulated neutrino-nucleon interaction, and $d^n\sigma'/dK^n$ is the
    differential cross section with the modified parameters evaluated at the
    kinematical phase space $\{K^n\}^3$. The differential cross section is a
    function of the neutrino energy in the rest frame of the scattered nucleon,
    $E_{\nu}^{(NRF)}$, the interaction four-momentum transfer, $Q^2$, and the
    physics model, including the model parameters.

    To determine the effect of the strange axial form factor parameters $\Delta
    s$ and $M_A^s$ (or $a_0^s$, $a_1^s$, and $a_2^s$) on the data, we calculate
    the NC elastic cross section given these new parameters and the cross
    section given the initial simulation parameters for each NC elastic event
    in the Monte Carlo simulation. To get an accurate weight, the denominator
    needs to be calculated exactly as the cross section was calculated in the
    initial GENIE simulation. However, there is no reason that a different
    model can't be used for the numerator. For the numerator in elastic
    interactions, we use the Llewellyn-Smith neutrino-nucleon elastic cross
    section parameterization described in Sec.~\ref{sec:probe} with the $z$
    expansion vector and strange axial form factors described in
    Sec.~\ref{sec:zexpansion}.

    To determine the effect of a set of NC elastic cross section parameters
    given our parameterization, we calculate the neutrino-nucleon weight for
    each NC elastic event
    \begin{equation}\label{eq:xsecweight}
      w_\sigma = \frac{\left(\frac{d\sigma}{dQ^2}(a_0^s,a_1^s,a_2^s)\right)_{\textrm{L.S.}}}
                      {\left(\frac{d\sigma}{dQ^2}\right)_{\textrm{GENIE}}} \,,
    \end{equation}
    where the L.S. subscript represents the cross section calculated using the
    Llewellyn-Smith parameterization with the $z$ expansion form factors, and
    the GENIE subscript represents the default cross section in the simulation.

    We also calculate weights to determine the effect on the simulated data of
    each of the sources of systematic uncertainty described in
    Sec.~\ref{sec:systematics}.  We can sample the probability space of each of
    these ``nuisance" parameters and calculate a weight based on that value.
    If the nuisance parameters are independent of each other, multiplying the
    nuisance parameter weights together is equivalent to sampling the combined,
    $N$-dimensional probability space of the systematic parameters, where $N$
    is the number of parameters. Sampling each of the relevant systematic
    parameters from Sec.~\ref{sec:systematics} gives
    \begin{equation}\label{eq:systweight}
      w_{syst} = w_{\textrm{flux}} \times w_{\textrm{D.I.C.}}\times w_{\textrm{S.P.E.}}\times
                 w_{\textrm{MEC}} \times w_{\textrm{P.B.}} \times w_{\textrm{dirt}}\,.
    \end{equation}
    Combining Eqns.~\ref{eq:weights},~\ref{eq:xsecweight},
    and~\ref{eq:systweight} gives an event weight which combines the effect
    due to a sample from the model parameter distribution and the values of the
    strange axial form factor parameters that we want to measure
    \begin{equation}
      w = w_{\sigma}\times w_{syst} \,.
    \end{equation}

  \subsubsection{Likelihood calculation}\label{sec:likelihood}
    To compare the simulation and the weight calculations directly to the data,
    we sum the weights for each reconstructed $Q^2$ bin in the distribution of
    events passing the NC elastic proton selection
    \begin{equation}\label{eq:expected}
      N_{NCE}(Q^2_i) = \sum\limits_{j\in NC_i} w_j \,,
    \end{equation}
    where $i$ is the $Q^2$ bin, $NC_i$ is the set of events selected as NC
    elastic neutrino-proton events in the $i^{th}$ $Q^2$ bin, and $w = w_j$ is the
    calculated weight of the event. If the event is not a true simulated
    elastic event, then $w_\sigma = 1$ giving $w_j = w_{syst}$. At this point,
    the Monte Carlo simulated data is directly comparable to the detector data.

    To evaluate how well the model represents the data, we calculate the
    probability of the observed data given a model and set of parameters.  This
    probability is called the likelihood. We assume that the measured data is
    normally distributed, and the likelihood is
    \begin{equation}\label{eq:likelihood}
      \mathcal{L} = P(D^{obs}|\theta) = \prod_{i\in I_{Q^2}} P(D^{obs}_i|\theta) = \frac{1}{\sqrt{2\pi \sigma_i^2}}
             e^{-\frac{1}{2}(D^{obs}_i - D^{exp}_i(\theta))^2/\sigma_i^2} \,,
    \end{equation}
    where $D^{obs}$ is the measured NC elastic neutrino-proton event selection
    distribution in data, $I_{Q^2}$ is the set of $Q^2$ bins, $D^{obs}_i$ is
    the measured number of selected events of that bin, and $\sigma_i$ is the
    statistical uncertainty of the difference between the observed and expected
    events in the measured bin value. The expected value of the event selection
    distribution, $D^{exp}_i(\theta)$, is a function of the model and the set
    of model parameters, $\theta$. It is equal to $N_{NCE}(Q^2_i)$ in
    Eqn.~\ref{eq:expected}. The set of model parameters, $\theta$, contains the
    systematic model parameters described in Sec.~\ref{sec:systematics} and the
    strange axial form factor parameters, $a_0^s$, $a_1^s$, and $a_2^s$
    described in Sec.~\ref{sec:ncaxial}. In this analysis, only the statistical
    uncertainty is contained in $\sigma_i$, and the systematic uncertainty is
    handled by sampling the systematic parameter values contained in $\theta$
    as described in Sec.~\ref{sec:mcmc}.

    When computing and comparing the likelihood values, the  natural log of the
    likelihood (log-likelihood) is actually used to avoid rounding errors in
    the computation. Taking th e log of Eq.~\ref{eq:likelihood} gives
    \begin{equation}
      log\mathcal{L} = -\frac{N_{\textrm{bins}}}{2}\, log(2\pi) \sum_{i\in I_{Q^2}}\Big[ log(\sigma_i) \frac{1}{2}(D_i^{obs} - D_i^{exp}(\theta))^2/\sigma_i^2 \Big]
    \end{equation}

    The likelihood calculation depends on nine different parameters, the three
    parameters of the strange axial form factor and the six systematic
    parameters described in Sec.~\ref{sec:systematics}. Calculating likelihood
    distributions for nine parameters would be computationally very time
    consuming. Figure~\ref{fig:likelihoods} shows the likelihood distributions
    calculated on a grid over the three strange axial form factor parameters,
    $a_0^s$, $a_1^s$, and $a_2^s$, while setting all of the systematic
    uncertainties to zero. Figures~\ref{fig:lka0a1m20}~and~\ref{fig:lka0a1p10}
    show the grids of likelihood values for $a_0^s$ and $a_1^s$ with $a_2^s$
    held at $-20$ and $10$, respectively.
    Figures~\ref{fig:lka0a2m3}~and~\ref{fig:lka0a2p2} show the grids of
    likelihood values for $a_0^s$ and $a_2^s$ with $a_1^s$ held to $-3$ and
    $2$, respectively.  Figures~\ref{fig:lka1a2m0}~and~\ref{fig:lka1a2p1} show
    the grids of likelihood values for $a_1^s$ and $a_2^s$ with $a_0^s$ held to
    $0$ and $1$, respectively. Even ignoring the systematic parameters and just
    varying three parameters is computationally difficult and can only be done
    for discrete values of the parameters. The sampling method described in
    Sec.~\ref{sec:mcmc} makes calculating the distributions in larger
    dimensions achievable.
    \begin{figure}[h]
      \centering
      \begin{subfigure}[t]{2.8in}
        \includegraphics[angle=0,width=2.8in]{figures/analysis/results/lnlike_a0a1_a2-20_tempo.pdf}
        \caption{With $a_2^s = -20$.}
        \label{fig:lka0a1m20}
      \end{subfigure}
      \hspace{2pt}
      \begin{subfigure}[t]{2.8in}
        \includegraphics[angle=0,width=2.8in]{figures/analysis/results/lnlike_a0a1_a210_tempo.pdf}
        \caption{With $a_2^s = 10$.}
        \label{fig:lka0a1p10}
      \end{subfigure}
      \begin{subfigure}[t]{2.8in}
        \includegraphics[angle=0,width=2.8in]{figures/analysis/results/lnlike_a0a2_a1-3_tempo.pdf}
        \caption{With parameter $a_1^s = -3$.}
        \label{fig:lka0a2m3}
      \end{subfigure}
      \hspace{2pt}
      \begin{subfigure}[t]{2.8in}
        \includegraphics[angle=0,width=2.8in]{figures/analysis/results/lnlike_a0a2_a12_tempo.pdf}
        \caption{With $a_1^s = 2$.}
        \label{fig:lka0a2p2}
      \end{subfigure}
      \begin{subfigure}[t]{2.8in}
        \includegraphics[angle=0,width=2.8in]{figures/analysis/results/lnlike_a1a2_a00_tempo.pdf}
        \caption{With parameter $a_0^s = 0$.}
        \label{fig:lka1a2m0}
      \end{subfigure}
      \hspace{2pt}
      \begin{subfigure}[t]{2.8in}
        \includegraphics[angle=0,width=2.8in]{figures/analysis/results/lnlike_a1a2_a01_tempo.pdf}
        \caption{With $a_0^s = 1$.}
        \label{fig:lka1a2p1}
      \end{subfigure}
      \caption{Calculated likelihood values at discrete $a_0^s$, $a_1^s$, and
      $a_2^s$ with zero systematic uncertainty.}
      \label{fig:likelihoods}
    \end{figure}

    \FloatBarrier
    

%%%%%%%%%%%%%%%%%%%%%%%%%%%%%%%%%%%%%%%%%%%%%%%%%%%%%%%%%%%
% Joint Estimation
%%%%%%%%%%%%%%%%%%%%%%%%%%%%%%%%%%%%%%%%%%%%%%%%%%%%%%%%%%%
\subsection{Strange axial form factor parameter estimation}\label{sec:deltas}
  The likelihood gives the probability of the observed data given a model and a
  set of parameters. What we are really interested in is the probability of the
  parameters $a_0^2$, $a_1^s$, and $a_2^s$ given the observed data and our
  model. We can determine this probability distribution using Bayesian
  inference and probability sampling methods.

  \subsubsection{Bayesian inference}
    Bayes' theorem is used to convert between the likelihood and the
    probability of the model given the data
    \begin{equation}
      P(\theta|D^{obs}) = \frac{P(D^{obs}|\theta)P(\theta)}{P(D^{obs})} \,.
    \end{equation}
    The likelihood, $P(D^{obs}|\theta)$, was described in detail in
    Sec.~\ref{sec:likelihood} and is defined in Eqn.~\ref{eq:likelihood}. The
    other three factors in Bayes' theorem deserve some explanation.
    
    First, the probability of $\theta$ given the observed data,
    $P(\theta|D^{obs})$, is the probability distribution that we ultimately
    want to determine. It is referred to as the posterior distribution.
    Implicit in the notation is the model. If we wanted to write it explicitly,
    it would be:
    \begin{equation}
      P(\theta|D^{obs})\equiv P(\theta|D^{obs},\mathcal{M}) \,,
    \end{equation}
    where $\mathcal{M}$ represents the physics model. The parameter set
    $\theta$ is still the set containing the strange axial form factor
    parameters, and the systematic parameters from Sec.~\ref{sec:reweighting}.

    Next, the probability of the parameters, $P(\theta)$, is referred to as the
    prior distribution. It is also implicitly conditional on the model,
    $\mathcal{M}$. This is where we include prior information that we know to
    be true. It is impossible not to include some prior information in
    inference. For example, using a uniform prior on $\Delta s$ is the same as
    saying that $\Delta s$ has the same probability of being zero as it does of
    being infinite. The prior should be used to exclude unphysical parameter
    values, like negative mass. In a good model with adequate data the
    posterior distribution should be robust to the choice of a prior. It is
    always necessary to evaluate the effect of the choice of priors on the
    posterior. In Sec.~\ref{sec:results} we show the effect of different
    priors, including a uniform prior, on the posterior distribution.

    Last, the marginal probability of the observed data $P(D^{obs})$ is
    integrated over all $\theta$ values.  It too is implicitly conditional on
    $\mathcal{M}$ and is referred to as the evidence of the model. Explicitly,
    it can be written as
    \begin{equation}
      P(R^{obs}) \equiv P(D^{obs}|\mathcal{M}) 
          = \int_{\theta}P(D^{obs}|\theta)P(\theta|\mathcal{M})d\theta \,.
    \end{equation}

  \subsubsection{Markov Chain Monte Carlo}\label{sec:mcmc}
    There is no way to calculate an exact solution to our posterior
    distribution analytically, but it can be estimated numerically by sampling.
    Most sampling techniques would be computationally impossible due to the
    fact that our posterior distribution is nine-dimensional ($a_0^a$, $a_1^s$,
    $a_2^s$, and the six systematic nuisance parameters), and each likelihood
    calculation requires thousands of event weights to be calculated.  Markov
    chain Monte Carlo (MCMC) is a class of methods for sampling
    multi-dimensional posterior distributions. Two of the most common MCMC
    algorithms are the Metropolis algorithm~\cite{Metropolis:1953am} and Gibbs
    sampling~\cite{Geman:1987}. Both are used in this analysis.

    The Metropolis algorithm is a random walk in the $N$-dimensional parameter
    space with a rule to either accept or reject each step in the walk. Each
    proposed step is drawn from a proposal distribution.  In this analysis, we
    use a multivariate normal proposal distribution centered at the current
    position.  The step is accepted if the value of the posterior distribution
    at the proposed position is greater than at the current position. If the
    value at the proposed distribution is smaller than the current value, the
    proposed step is accepted with a probability equal to the ratio of the
    value at the proposed position to the value at the current position. The
    decision to accept or reject a proposed step can be determined entirely by
    calculating the \textit{ratio} of the posterior values at the proposed and
    current positions. Since the evidence, $P(D^{obs})$, doesn't depend on
    $\theta$, it never needs to be calculated. At every proposed step only the
    likelihood, $P(D^{obs}|\theta)$, and the prior, $P(\theta)$, need to be
    calculated.

    Gibbs sampling as used in this analysis is also a random walk in the
    $N$-dimensional parameter space, but it samples the posterior directly. We
    cannot sample the posterior directly for the strange axial form factor
    parameters that we want to infer from the data, but we can make an
    approximation that allows us to use Gibbs sampling for our nuisance
    parameters. At each Gibbs sampling iteration, subvectors of parameters,
    $\theta_i \in \theta$, are cycled through and sampled conditionally on the
    position of the previous subvector
    $P(\theta_i^t|\theta_{-i}^{t-1},D^{obs})$, where $t$ is the current
    iteration, $\theta_{-i}^{t-1}$ represents the current position all of the
    other subvectors (excluding $\theta_i$), $\theta_{-i}^{t-1} =
    (\theta_0^t,...,\theta_{i-1}^t,\theta_{i+1}^{t-1},...,\theta_d^{t-1})$, and
    $d$ is the number of subvectors in $\theta$.  Since the posterior is being
    sampled directly, the sample is also conditional on the data, $D^{obs}$.
    This is where the approximation comes in. If we assume that the nuisance
    parameters are independent of the data, then
    \begin{equation}\label{eq:gibbsstep}
      P(\theta_i|\theta_{-i},D^{obs}) \approx P(\theta_i|\theta_{-i}) \,.
    \end{equation}
    The assumption being made here is that these parameter distributions, the
    position and variance of the systematic nuisance parameters, would not
    change very much based on the data we are comparing to, $D^{obs}$. In other
    words we would not be able to tell from the observed data events passing
    the NC elastic proton selection what the true mean and variance of the
    systematic nuisance parameters are. For example, we don't believe that we
    can measure the single PE background rate and its uncertainty by looking at
    the NC elastic event selection in the data. If that assumption is true,
    then the posterior distributions of the systematic parameters are equal to
    the prior distributions, which were estimated in
    Sec.~\ref{sec:systematics}.

    The Metropolis and Gibbs sampling methods are combined for the parameter
    estimation.  First, a Metropolis step is proposed for new $a_0^s$, $a_1^s$
    and $a_2^s$ values. The likelihood value at the proposed position is
    evaluated conditionally on the current position of the systematic nuisance
    parameters.  Then, a Gibbs step is taken in all of the systematic
    parameters at once ($\theta_i$ is all of the nuisance parameters), and the
    likelihood is evaluated conditionally on the current position of $a_0^s$,
    $a_1^s$, and $a_2^s$. The proposed positions of each of the six parameters
    in the step are independent of each other and of the current parameter
    values.  This is repeated iteratively until the target distribution has
    been reached. Metrics for determining how many samples are required are
    discussed in Sec.~\ref{sec:mcmcsyst}.


%%%%%%%%%%%%%%%%%%%%%%%%%%%%%%%%%%%%%%%%%%%%%%%%%%%%%%%%%%%
% Joint Estimation
%%%%%%%%%%%%%%%%%%%%%%%%%%%%%%%%%%%%%%%%%%%%%%%%%%%%%%%%%%%
\subsection{Results}\label{sec:results}
  Using the reweighting method, the likelihood calculation, and the Metropolis
  and Gibbs MCMC sampling, the strange axial form factor parameters can now be
  estimated. First, the tools are tested by sampling the strange axial form
  factor parameter posterior distributions without including the systematic
  nuisance parameters. Then, the systematic parameters are included, the
  convergence of the MCMC sampling is evaluated, and the posterior
  distributions in terms of $\Delta s$ and $M_A^s$ are shown.

  \subsubsection{MCMC Sampling without Systematic Nuisance Parameters}\label{sec:mcmcnosyst}
    The three strange axial form factor parameters are sampled first assuming no
    systematic uncertainty. The main reason for doing this is to see if we get
    something reasonable as a proof of concept. The extracted parameters don't
    inform us about the true strange axial form factor when the systematic
    uncertainty isn't accounted for.
    
    To sample without the systematic uncertainty, the proposal distribution for
    each systematic nuisance parameter is set to zero. The proposal
    distribution for the strange axial form factors, described below, was
    determined empirically to maximize the coverage of the allowed parameter
    space. The final proposal distribution used for the strange axial form
    factor parameters is a multivariate normal distribution with a small amount
    of covariance between each of the parameters. The covariance is included
    because of the strong anti-correlation in the parameter likelihood
    distributions in Sec.~\ref{sec:likelihood}. If the covariance were not
    included, the final sampled distributions would eventually be the same, but
    it would take longer the to cover the parameter space. The metric used to
    determine the sampling performance is the acceptance rate. If the proposed
    step is too large, most moves will be rejected (a low acceptance rate), and
    it will take a long time to cover the parameter space. If the proposed stop
    is too small, most moves will be accepted (a high acceptance rate), and it
    will take a long time to cover more than just a small portion of the
    parameter space~\cite{Brooks:2011}.
    
    The proposal distribution used for the strange axial form factor parameters
    as each MCMC iteration is a multivariate normal centered at the current
    position of the MCMC chain at that iteration with covariance:
    \begin{equation}\label{eq:axproposal}
      \Sigma = 
      \begin{bmatrix}
        0.075 & -0.1 & -0.2 \\
        -0.1  &  1.5 & -2.0 \\
        -0.2  & -2.0 &  30  
      \end{bmatrix} \,,
    \end{equation}
    which gives a acceptance rate when ignoring systematic parameters of 0.55.
    Figure~\ref{fig:gasproposals} shows the covariance between the three
    parameters in the proposal distribution. The initial position of the MCMC
    chain is sampled from the proposal distribution centered at the zero for
    each of the parameters.
    \begin{figure}[h]
      \includegraphics[angle=0,width=2.8in]{figures/analysis/results/proposals_nosyst_a0a1.pdf}
      \hspace{2pt}
      \includegraphics[angle=0,width=2.8in]{figures/analysis/results/proposals_nosyst_a0a2.pdf}
      \begin{flushright}
      \includegraphics[angle=0,width=2.8in]{figures/analysis/results/proposals_nosyst_a1a2.pdf}
      \end{flushright} 
      \caption{Two-dimensional views of the multivariate normal proposal
      distribution of the strange axial form factor parameters.}
      \label{fig:gasproposals}
    \end{figure}

    The prior distributions on the strange axial form factor parameters are
    used to encode bounds on the coefficients $a_k^s$ for $k=0,\ldots,6$
    introduced in Sec.~\ref{sec:ncaxial}. Following the $z$ expansion form
    factor analyses in Refs.~\cite{Hill:2010yb,Bhattacharya:2011ah}, we enforce
    $|a_k^s| \le 10$. Recall that $a_3^s,\ldots,a_6^s$ are linear combinations
    of $a_0^s$, $a_1^s$, and $a_2^s$, making the coefficient bounds:
    \begin{align}\label{eq:axprior}
      |a_0^s| \le 10 \,, \hspace{5mm}
      |a_1^s| \le 10 \,, \hspace{5mm}
      |a_2^s| &\le 10 \,, \\
      |-20 a_0^s - 10a_1^s - 4a_2^s| &\le 10 \,, \\
      | 45 a_0^s + 20a_1^s + 6a_2^s| &\le 10 \,, \\
      |-36 a_0^s - 15a_1^s - 4a_2^s| &\le 10 \,, \\
      | 10 a_0^s + 4a_1^s + a_2^s| &\le 10 \,.
    \end{align}
    The optimal value of $t_0$ from Eq.~\ref{eq:topt} is used for the mapping
    of $Q^2$ to $z(Q^2)$,
    \begin{equation}
      \begin{aligned}
        t_0^{\textrm{optimal}}(Q^2) &= t_{cut}\Big(1 - \sqrt{1+Q^2_{\textrm{max}}/t_{cut}}\Big) \\
                                    &= 9m_{\pi}^2 \Big(1-\sqrt{1+1.0\,\textrm{GeV}^2/(9m_{\pi}^2)}\Big) \\
                                    &= -0.28\, \textrm{GeV}\,.
      \end{aligned}
    \end{equation}
    
    Figures~\ref{fig:chainns}-\ref{fig:samplesws} show the results of 50,000
    MCMC steps in $a_0^s$, $a_1^s$, and $a_2^s$ with the systematic parameters
    held at zero. Figure~\ref{fig:chainns} shows the chains of MCMC samples for
    $a_0^s$, $a_1^s$, and $a_2^s$. Figure~\ref{fig:samplesns} shows the 50,000
    samples of the posterior distributions for each of the two-parameter pairs
    of the strange axial form factor parameters. The correlation between the
    $a_0^s$ and $a_1^s$ posterior distributions and the correlation between the
    $a_1^s$ and $a_2^s$ posterior distributions are similar to the correlations
    in the corresponding likelihood distributions even after introducing the
    coefficient bounds and allowing all three parameters to vary
    simultaneously. The calculated Pearson correlation coefficient between
    $a_0^s$ and $a_1^s$ is $r_{01} = -0.77$, and the coefficient between
    $a_1^s$ and $a_2^s$ is $r_{12} = -0.93$. There doesn't appear to be a
    negative correlation between the $a_0^s$ and $a_2^s$ posterior
    distributions, however, even though one exists between the likelihood
    distributions. In fact, the correlation coefficient between $a_0^s$ and
    $a_2^s$ is positive, $r_{02} = 0.51$.
    \begin{figure}[h]
      \centering
      \includegraphics[angle=0,width=4.5in]{figures/analysis/results/samples_chain_nosyst_a0.pdf} \\
      \includegraphics[angle=0,width=4.5in]{figures/analysis/results/samples_chain_nosyst_a1.pdf} \\
      \includegraphics[angle=0,width=4.5in]{figures/analysis/results/samples_chain_nosyst_a2.pdf}
      \caption{MCMC chains of the strange axial form factor parameters after
      50,000 steps with the systematic parameters held to zero.}
      \label{fig:chainns}
    \end{figure}
    \begin{figure}[h]
      \includegraphics[angle=0,width=2.8in]{figures/analysis/results/samples_nosyst_a1a0.png}
      \hspace{2pt}
      \includegraphics[angle=0,width=2.8in]{figures/analysis/results/samples_nosyst_a2a0.png}
      \begin{flushright}
      \includegraphics[angle=0,width=2.8in]{figures/analysis/results/samples_nosyst_a2a1.png}
      \end{flushright} 
      \caption{Two-dimensional views of the 50,000 MCMC samples of the
      posterior distributions of the strange axial form factor parameters with
      the systematic parameters held to zero.}
      \label{fig:samplesns}
    \end{figure}
  

  \subsubsection{MCMC Sampling with Systematic Nuisance Parameters}\label{sec:mcmcsyst}
    The prior distributions of the systematic nuisance parameters are described
    in Sec.~\ref{sec:systematics}. Since we don't expect the NC elastic proton
    selection in data to increase our knowledge of the mean and variance of the
    systematic parameters significantly, the posterior distributions of the
    systematic parameters should look similar to the prior distributions. The
    best proposal distributions of the parameters are therefore their prior
    distributions since they are our best estimate of their posteriors which we
    are not actually trying to measure.

    To account for the systematic uncertainty from the sources described in
    Sec.~\ref{sec:systematics}, we sample directly from the distributions
    described in that section in the Gibbs sampling step.  For each of the
    systematic proposals, we sample a standard Gaussian with a mean of zero and
    a variance of one and translate the sampled value to the effect of that
    parameter on the number of simulated events passing the NC elastic proton
    selection. In the case of the dynamic induced charge the negative of the
    absolute value is used because the uncertainty is only in one direction.
    Each simulated event passing the NC elastic proton selection is multiplied
    by a weight equal to the sampled parameter value times the effect of the
    parameter on the event for each parameter.  These weights are samples from
    the uncertainty distributions that were described for each systematic
    parameter in Sec.~\ref{sec:systematics}. We assume there is no correlation
    between the parameters and that there is a correlation of one between bins
    for a given parameter. For the single PE background rate the effect is the
      same for every event, and the weight is 
    \begin{equation}
      w_{\textrm{S.P.E}} = 1 + 0.2\cdot r_{\textrm{S.P.E}} \,,
    \end{equation}
    where $r_{\textrm{S.P.E.}}$ is the randomly sampled single PE value, and
    the factor of 0.2 is the 20\% uncertainty on the number of simulated events
    passing the NC elastic proton selection due to the uncertainty on the
    single PE rate.  The effect of the dirt normalization uncertainty is the
    same for all dirt events and zero for all non-dirt events,
    \begin{equation}
      w_{\textrm{dirt}} = 
      \begin{cases}
        1 &\textrm{non-dirt events} \\
        1 + 0.5\cdot r_{\textrm{dirt}}  &\textrm{dirt events}
      \end{cases} \,,
    \end{equation}
    where $r_{\textrm{dirt}}$ is the randomly sampled dirt parameter value, and
    the factor of 0.5 is the 50\% uncertainty on the normalization of  the
    simulated dirt events. The effect of the dynamic induced charge is applied
    to all simulated events and depending on the reconstructed $Q^2$ value of
    the event,
    \begin{equation}
      w_{\textrm{D.I.C.}} = 
      \begin{cases}
        1 - 0.4\cdot |r_{\textrm{D.I.C.}}|  &Q^2_{reco} < 0.175 \,\textrm{GeV}^2 \\
        1 - 0.25\cdot |r_{\textrm{D.I.C.}}|  &Q^2_{reco} \ge 0.175 \,\textrm{GeV}^2
      \end{cases} \,,
    \end{equation}
    where $r_{\textrm{D.I.C.}}$ is the randomly sampled induced charged
    parameter value, and the factors of 0.4 and 0.25 are the 40\% uncertainty
    on the lowest reconstructed $Q^2$ simulated events and the 25\% uncertainty
    on all other simulated events, respectively. The effect of the neutrino
    flux uncertainty is applied to all simulated events depending on the true
    neutrino energy of the event as shown in Fig.~\ref{fig:ubflux}
    \begin{equation}
      w_{\textrm{flux}} = 1 + f_{\textrm{flux}}(E_{\nu})\cdot r_{\textrm{flux}} \,,
    \end{equation}
    where $f_{\textrm{flux}}(E_{\nu})$ is the fractional uncertainty on the
    simulated flux for the given true neutrino energy, and $r_{\textrm{flux}}$
    is the randomly sampled flux parameter value. The MEC uncertainty weight is
    applied only to simulated true MEC events and is dependent on the true
    $Q^2$ value of the event as shown in Fig.~\ref{fig:ccmecscale},
    \begin{equation}
      w_{\textrm{MEC}} = 
      \begin{cases}
        1 &\textrm{non-MEC events} \\
        1 + f_{\textrm{MEC}}(Q^2_{true})\cdot r_{\textrm{MEC}}  &\textrm{MEC events}
      \end{cases} \,,
    \end{equation}
    where $f_{\textrm{MEC}}(Q^2_{true})$  is the fractional uncertainty on the
    number of simulated MEC events for the given true $Q^2$, and
    $r_{\textrm{MEC}}$ is the randomly sampled MEC parameter value. Finally,
    the Pauli blocking uncertainty weight is applied only to all simulated
    events depending on the true $Q^2$ value of the event as shown in
    Fig~\ref{fig:pauliblock},
    \begin{equation}
      w_{\textrm{P.B.}} = 1 + f_{\textrm{P.B.}}(Q^2_{true})\cdot r_{\textrm{P.B.}} \,,
    \end{equation}
    where $f_{\textrm{P.B.}}(Q^2_{true})$ is the fractional uncertainty on the
    simulated events for the given true $Q^2$, and $r_{\textrm{P.B.}}$ is the
    randomly sampled Pauli blocking parameter value.  The asymmetry in the
    uncertainty is ignored to make it easier to evaluate the probability
    distribution. This is expected to be inconsequential because the asymmetry
    is small, the overall uncertainty is small relative to the other sources of
    uncertainty, and the larger of the negative or positive uncertainties is
    used at each true $Q^2$ bin which is the conservative choice.

    The strange axial form factor parameters are treated the same as when the
    systematic parameters were held to zero. The proposal distribution is the
    same multivariate normal in Eq.~\ref{eq:axproposal} centered at the current
    position, and the prior distributions are the same in Eq.~\ref{eq:axprior}
    used to enforce the coefficient bounds.

    \begin{figure}[h]
      \centering
      \includegraphics[angle=0,width=4.5in]{figures/analysis/results/samples_chain_wsyst_a0_200000.pdf} \\
      \includegraphics[angle=0,width=4.5in]{figures/analysis/results/samples_chain_wsyst_a1_200000.pdf} \\
      \includegraphics[angle=0,width=4.5in]{figures/analysis/results/samples_chain_wsyst_a2_200000.pdf}
      \caption{MCMC chains of the strange axial form factor parameters after
      200,000 steps with the systematic parameters included.}
      \label{fig:chainws}
    \end{figure}
    Figures~\ref{fig:chainws}-\ref{fig:samplesws} show the results of 200,000
    MCMC steps in $a_0^s$, $a_1^s$, and $a_2^s$ with the systematic parameters
    included.  Figure~\ref{fig:chainws} shows the chains of MCMC samples for
    $a_0^s$, $a_1^s$, and $a_2^s$. Figure~\ref{fig:samplesws} shows the 200,000
    samples of the posterior distributions for each of the two-parameter pairs
    of the strange axial form factor parameters. The correlations between the
    parameters are similar to the case with no systematic uncertainty, but with
    larger variance. The correlations between the parameters are similar to the
    correlations when the systematic parameters are not included. The Pearson
    correlation coefficient between $a0^s$ and $a_1^s$ is $r_{01} = -0.84$, the
    coefficient between $a_1^s$ and $a_2^s$ is $r_{12} = -0.80$, and the
    coefficient between $a_0^s$ and $a_2^s$ is again positive at $r_{02} =
    0.36$.
    \begin{figure}[h]
      \includegraphics[angle=0,width=2.8in]{figures/analysis/results/samples_wsyst_a1a0_200000.png}
      \hspace{2pt}
      \includegraphics[angle=0,width=2.8in]{figures/analysis/results/samples_wsyst_a2a0_200000.png}
      \begin{flushright}
      \includegraphics[angle=0,width=2.8in]{figures/analysis/results/samples_wsyst_a2a1_200000.png}
      \end{flushright} 
      \caption{Two-dimensional views of the 200,000 MCMC samples of the
      posterior distributions of the strange axial form factor parameters with
      the systematic parameters included.}
      \label{fig:samplesws}
    \end{figure}

    A bimodal shape in the $a_0^s$ and $a_1^s$ posterior distributions is
    slightly visible in Figs.~\ref{fig:chainws}~and~\ref{fig:samplesws} and
    becomes much more apparent when we look at the two-dimensional histogram of
    the samples in $a_1^s$ and $a_0^s$ shown in Fig.~\ref{fig:hist2da1a0}.
    Figure~\ref{fig:hist1dws} shows the one-dimensional histograms of the
    samples of $a_0^s$, $a_1^s$, and $a_2^s$. The bimodal nature of the $a_0^s$
    posterior distribution is very obvious in one dimension with one clear mode
    at $a_0^s = -0.1$ and another at $a_0^s = 1.3$. The $a_1^s$ posterior
    distribution also shows a slight bimodal shape in one dimension, but it is
    not nearly as pronounced as in $a_0^s$.
    \begin{figure}[h]
      \centering
      \includegraphics[angle=0,width=4.5in]{figures/analysis/results/hist2d_a1a0_wsyst.pdf} \\
      \caption{Two-dimension histogram of the MCMC samples of the posterior
      distribution for the $a_0^s$ and $a_1^s$ parameters.}
      \label{fig:hist2da1a0}
    \end{figure}
    \begin{figure}[h]
      \centering
      \includegraphics[angle=0,width=1.87in]{figures/analysis/results/hist1d_a0_wsyst.pdf} 
      \includegraphics[angle=0,width=1.87in]{figures/analysis/results/hist1d_a1_wsyst.pdf} 
      \includegraphics[angle=0,width=1.87in]{figures/analysis/results/hist1d_a2_wsyst.pdf}
      \caption{One-dimensional histograms of the strange axial form factor
      parameter samples.}
      \label{fig:hist1dws}
    \end{figure}

    Figure~\ref{fig:postpred} shows the posterior predictive distribution of
    the number of expected events passing the NC elastic proton selection in
    Run~I. The posterior predictive distribution is a prediction of what the
    data would look like based on the posterior distribution of the parameters.
    It is used to check how well the model fits the data. If the posterior
    predictive distribution is very different than the observed data, it shows
    that the model is unable to fit the data. In Fig.~\ref{fig:postpred}, the
    blue distributions at each value of reconstructed $Q^2$ are the posterior
    predictive distributions, and the black points are the observed
    neutrino-beam data events in Run~I that pass the NC elastic proton
    selection and the corresponding statistical uncertainty.
    \begin{figure}[h]
      \centering
      \includegraphics[angle=0,width=5.in]{figures/analysis/results/posterior_predict_modes.pdf} 
      \caption{The posterior predictive distribution of the number of selected
      events compared to Run~I neutrino-beam data as a function of
      reconstructed $Q^2$.}
      \label{fig:postpred}
    \end{figure}

    To understand the different $a_0^s$ and $a_1^s$ modes better, the posterior
    was divided into two samples: one with $a_0^s$ sample values less than 0.5
    and one with $a_0^s$ sample values greater than 0.5. The posterior
    predictive distributions were determined for each sample and the median of
    each sample is plotted in Fig.~\ref{fig:postpred} in red and yellow. Mode
    1, in red, is the sample with $a_0^s$ sample values less than 0.5, and Mode
    2, in yellow, is the sample with $a_0^s$ values greater than 0.5. The
    results are what would be expected. The Mode 1 sample, with low $a_0^s$
    values and high $a_1^s$ corresponds to a larger number of selected events
    at low $Q^2$ and a steeper slope in that region. The strange axial form
    factor contributes negatively to the NC elastic neutrino-proton cross
    section, as shown in Sec.~\ref{sec:strangeness}, so a large negative value
    for the $a_0^s$ term in the $z$ expansion corresponds to a larger cross
    section at low $Q^2$. The Mode 2 sample, with more positive values of
    $a_0^s$, corresponds to a smaller number of selected events at low $Q^2$
    and a shallower slope in that region. It isn't clear from the posterior
    predictive distribution why there are two distinct modes, however.

  \subsubsection{MCMC Test Metrics}
    There is no perfect test of whether an MCMC chain has sampled to entire
    target distribution~\cite{Brooks:2011}, but there are some metrics to check
    to avoid common problems. The obvious problems to check for are whether the
    step size is too small and the MCMC chain is simply performing a random
    walk, whether the step size is too large and the MCMC chain isn't moving,
    whether the MCMC chain is stuck in a local maximum, called
    \text{pseudo-convergence}, or if the MCMC chain simply hasn't taken enough
    samples.

    To test whether the step size in the proposal distribution is too large or
    small, the simplest and most effective metrics are the acceptance rate and
    plotting the trace of the MCMC chain itself. As discussed briefly in
    Sec.~\ref{sec:mcmcnosyst}, if the acceptance rate is close to one, the MCMC
    chain is performing a random walk, and if the acceptance rate is close to
    zero, the MCMC chain is stationary. It has been shown that the ideal
    acceptance rate for efficient Metropolis MCMC sampling with one parameter
    is 0.5 decreasing to 0.23 for many parameters~\cite{Roberts:1997}. The
    acceptance rate of the MCMC sampling chain was consistently 0.55, which is
    slightly higher that the theoretical ideal, but still far from one and
    zero. The trace of the MCMC chains are shown in Fig.~\ref{fig:chainws}. The
    chains do not look to be taking random walks and do not have large changes
    over time.

    Another straight-forward test is to change the prior and proposal
    distributions and check that the posterior distribution is as expected. The
    MCMC sampling was run for 50,000 iterations without any priors on the
    strange axial form factor parameters and compared to the results with the
    priors. Recall that the priors on the strange axial form factor parameters
    are used to enforce the coefficient bounds described in
    Sec.~\ref{sec:mcmcnosyst} and are otherwise uniform. The mean values of the
    samples of the $a_0^s$, $a_1^s$, $a_2^s$ did not change significantly when
    the coefficient bounds were removed. In fact, the mean values of each
    parameter were within a standard deviation of the mean values when the
    coefficient bounds were enforced. However, the variance of the samples
    greatly increased. To test the validity of the proposal distribution the
    MCMC was run for 50,000 iterations with a proposal distribution with twice
    the variance on each parameter half of the correlation between each pair of
    parameters. The means and variances of the MCMC samples did not change.
    These simple checks show that the sampling is fairly robust to the choice
    of prior and proposal distributions.

    The Gelman-Rubin test~\cite{Gelman:1992zz} compares the variance between
    separate MCMC chains to the variance within an MCMC chain to test the
    potential reduction in variance if the chain were to be run longer. The
    $\hat{R}$ value is the measure of the potential scale reduction, and an
    ideal value is close to one. A result of $\hat{R} < 1.1$ is a sign that
    sampling variability is negligible~\cite{Brooks:1997}. Ten separate MCMC
    chains were run for 10,000 iterations each to calculate $\hat{R}$ for each
    strange axial form factor parameter
    \begin{equation}
      \begin{aligned}
        \hat{R}_{a_0^s} &= 1.02 \,, \\
        \hat{R}_{a_1^s} &= 1.02 \,, \\
        \hat{R}_{a_2^s} &= 1.01 \,.
      \end{aligned}
    \end{equation}
    These Gelman-Rubin test scores represent the potential scale reduction
    after 10,000 iterations. Since the MCMC sampling used for the final results
    here was run for 200,000 iterations, we can be reasonably assured that the
    final posterior distributions would not change drastically with further
    running.

  \subsubsection{Distributions of $\Delta s$ and $M_A^s$}
    To transform the strange axial form factor $z$ expansion coefficients into
    $\Delta s$, we use the relationship derived in Sec.~\ref{sec:ncaxial}
    between $\Delta s$ and the coefficients,
    \begin{equation}
      \Delta s = a_0^s + a_1^s z_0 + a_2^s z_0^2 
        + a_3^s z_0^3 + a_4^s z_0^4 + a_5^s z_0^5 + a_6^s z_0^6 \,.
    \end{equation}
    Recall that the coefficients $a_3^s,\ldots,a_6^s$ are linear combinations
    of the first three coefficients as shown in Eq.~\ref{eq:coefficients}.
    Additionally, the strange axial mass can be redefined in terms of the slope
    of the strange axial form factor at $Q^2=0$, \begin{equation}
    \begin{aligned} M_A^s &= \sqrt{2\frac{G_A^s(Q^2=0)}{G_A^{s\prime}(Q^2=0)}}
    \\ &= \sqrt{2 \Delta s/(a_1^s + 2a_2^s z_0 + 3a_3^s z_0^2 + 4a_4^s z_0^3 +
    5a_5^s z_0^4 + 6a_6^s z_0^5)} \,.  \end{aligned} \end{equation}

    The dependence of $M_A^s$ on the square root of $\Delta s$ leads to an
    interesting looking relationship in the posterior distribution between the
    two parameters. Figure~\ref{fig:hist2dmads} shows the two-dimensional
    histogram of the samples from the posterior distribution in terms of
    $M_A^s$ and $\Delta s$. There is a clear difference in the shape of the
    distribution for $\Delta s$ above or below zero. When $\Delta s$ is
    positive, there is a positive correlation between $\Delta s$ and $M_A^s$,
    and when $\Delta s$ is negative, there is a negative correlation between
    the two.  \begin{figure}[h] \centering
    \includegraphics[angle=0,width=4.5in]{figures/analysis/results/hist2d_dsMA.pdf}
    \caption{Two-dimensional histogram of the MCMC samples transformed to
    $\Delta s$ and $M_A^s$.} \label{fig:hist2dmads} \end{figure}

    Figures~\ref{fig:hist1dds}~and~\ref{fig:hist1dma} show the one-dimensional
    histograms of the samples MCMC of the posterior distribution in terms of
    $\Delta s$, in Fig.~\ref{fig:hist1dds}, and $M_A^s$, in
    Fig.~\ref{fig:hist1dma}. The one-dimensional distributions make it clear
    how large the variance of the parameters is. The mode, and 68\% and 95\%
    credible intervals are shown for both parameters. The credible intervals
    used here are the highest posterior density intervals (HPDs). The HPD is
    defined as the narrowest interval that contains the given percentage of the
    data.  \begin{figure}[h] \centering
    \includegraphics[angle=0,width=4.5in]{figures/analysis/results/hist1d_ds.pdf}
    \caption{One-dimensional histogram of the MCMC samples transformed to
    $\Delta s$ with 68\% and 95\% credible intervals.} \label{fig:hist1dds}
    \end{figure}

    For $\Delta s$, the mode of the samples is at $\Delta s = -0.4$ with a 95\%
    credible interval of $-1.8 < \Delta s < 3.8$ and a 68\% credible interval
    of $-1.0 < \Delta s < 2.6$, as shown in Fig.~\ref{fig:hist1dds}. For
    $M_A^s$, the mode of the samples is at $M_A^s = 0.8$~GeV with a 95\%
    credible interval of $0\, \textrm{GeV} < M_A^s < 2.1\, \textrm{GeV}$ and a
    68\% credible interval of $0.4\, \textrm{GeV} < M_A^s < 1.0\,
    \textrm{GeV}$.
    \begin{figure}[h]
      \centering
      \includegraphics[angle=0,width=4.5in]{figures/analysis/results/hist1d_MA.pdf} 
      \caption{One-dimensional histogram of the MCMC samples transformed to
      $M_A^s$ with 68\% and 95\% credible intervals.}
      \label{fig:hist1dma}
    \end{figure}
    
\subsection{Conclusions}
  The measurement of the strange quark spin structure in the nucleon through NC
  elastic neutrino-proton scattering remains an intriguing and important cross
  check to measurements performed through deep inelastic scattering of
  polarized charged-leptons and nucleons. MicroBooNE remains a good prospect
  for carrying out this measurement because of its high resolution and neutrino
  energy range. The machinery to identify protons in liquid argon TPCs and
  select NC elastic events has been developed and is in a promising state. The
  limiting factor of the measurement of $\Delta s$ in MicroBooNE is currently
  the understanding of the detector physics that effect both the energy
  reconstruction of TPC objects and the reconstruction of optical flashes in
  the PMTs.

  \subsubsection{Discussion of Results and Prospects for a Future MicroBooNE $\Delta s$ Measurement}
    The 68\% credible interval around $\Delta s$ is larger than the previous
    two NC elastic neutrino measurements of $\Delta s$ described in
    Sec.~\ref{sec:neutrinos}. If we ignore the second mode in the $\Delta s$
    posterior distribution, the 68\% interval ranges from less than negative
    one to greater than positive one. However, the potential to drastically
    reduce the uncertainty on the $\Delta s$ measurement exists. Due to the
    fact that liquid argon TPCs are such a new technology, the ability to
    identify protons, select low-$Q^2$ events with a reasonable efficiency, and
    understand the relationship between simulated events in MicroBooNE and
    measured data events are all hindered by the understanding of the detector
    physics.

    The largest source of systematic uncertainty on the expected number of
    events is the dynamic induced charge, described in Sec.~\ref{sec:detvar}. A
    full, two-dimensional simulation of the induced charge on each of the wire
    is currently being incorporated into the MicroBooNE simulation, which
    should have a huge effect on the uncertainty in the simulation. The dynamic
    induced charge and other sources of uncertainty on the reconstruction of
    TPC objects effect the use of calorimetry to identify particles, as well. A
    lot of work has gone in to calibrating the energy deposited in the detector
    and the charge collected on the wires, but the track reconstruction needs
    to be precise enough to precisely determine the shape of the energy
    deposited along the distance of the track. Improvements in our
    understanding of the TPC physics will lead to reduced systematic
    uncertainty on the NC elastic proton selection, as well as a higher
    selection efficiency because we will be able to make more precise cuts on
    reconstructed variables.
    
    The next largest effect is due to the optical model. The uncertainty on the
    simulation of the light in MicroBooNE effects several aspects of this
    analysis. As discussed in Sec.~\ref{sec:detvar}, the rate of a single PE
    background shifts the center of the reconstructed flashes causing an
    increase or reduction in the number of events selected.  Additionally,
    because the light is not well modeled in MicroBooNE, the PE threshold for
    events to be reconstructed was raised to reduce the disagreement between
    the measured data and the expectation from simulation.  Since single, low
    energy protons produce flashes near the threshold, this has a large effect
    on the efficiency of the selection. Future iterations of MicroBooNE
    simulation will have improvements to the optical model including a
    treatment of Cerenkov light production and the time dependence of the
    single PE rate leading to an increase in the selection efficiency and
    reduction of systematic uncertainty.

    The large variance in the posterior distribution of $\Delta s$ is mainly
    due to a disagreement between the observed data and the expectation from
    simulation that cannot be accounted for easily in the model. Once the
    detector model has improved, a more precise measurement of $\Delta s$ will
    be achievable. Several improvements are currently being made, and a better
    measurement should be possible in the next iteration of MicroBooNE the
    simulation model.


%This is the end of analysis section

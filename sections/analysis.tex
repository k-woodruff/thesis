\section{Analysis}\label{analysis}
%The next line produces an indented paragraph to start the document
 %unit.  The LaTeX defaults start most units without indentations.
\hspace{\parindent}

%%%%%%%%%%%%%%%%%%%%%%%%%%%%%%%%%%%%%%%%%%%%%%%%%%%%%%%%%%%
% Background Estimation
%%%%%%%%%%%%%%%%%%%%%%%%%%%%%%%%%%%%%%%%%%%%%%%%%%%%%%%%%%%
\subsection{Background Estimation}\label{background}
  \subsubsection{Beam Induced Dirt Background}
    Discuss dirt neutrons, how they happen and estimated rates and energy
    distributions.  Show how well we can seperate or understand them. Show any
    sort of data-driven correction we did to dirt neutron background and how it
    affects our uncertainty. Talk about how well we can tag cryostat neutrons
    with the PMTs.
  \subsubsection{Beam Induced TPC Background}
    Talk about neutral-current elastic neutrons that are produced in the TPC
    and how their distributions differ from NCEp ones.
  \subsubsection{Cosmic Background}
    Discuss the difference between cosmic tracks and beam proton tracks. How do
    we separate them? What is the rate?

%%%%%%%%%%%%%%%%%%%%%%%%%%%%%%%%%%%%%%%%%%%%%%%%%%%%%%%%%%%
% Sources of Measurement Uncertainty
%%%%%%%%%%%%%%%%%%%%%%%%%%%%%%%%%%%%%%%%%%%%%%%%%%%%%%%%%%%
\subsection{Sources of Measurement Uncertainty}\label{uncertainties}
  \subsubsection{TPC Efficiency}
    The actual physical efficiency. If ionization electrons actually reach the
    TPC and leave a signal.
  \subsubsection{PMT Efficiency}
    Talk here about efficiency of PMT software trigger. Show studies that have
    been done for single proton efficiency. Show efficiency of protons in data.
  \subsubsection{Reconstruction Efficiency}
    Here show reconstruction efficiency.  Efficiency of track algorithms
    finding protons at all and then talk about using BDTs or some other cuts.
  \subsubsection{Nuclear Effects and Final State Interactions}
    Describe what Nuclear effects and fsi are. Emphasize the parts that will
    actually remain after taking the ratio (any sort of energy shift or
        something).

%%%%%%%%%%%%%%%%%%%%%%%%%%%%%%%%%%%%%%%%%%%%%%%%%%%%%%%%%%%
% Ratio of Cross Sections
%%%%%%%%%%%%%%%%%%%%%%%%%%%%%%%%%%%%%%%%%%%%%%%%%%%%%%%%%%%
\subsection{Ratio of NCEp to CCQEn Cross Sections}\label{ratios}
  Show how the ratio gets rid of a lot of measurement uncertainty like beam
  flux and efficiencies. Give exact equation that we will be using for
  analysis. Show how $\Delta s$ is still large at low $Q^2$.

%%%%%%%%%%%%%%%%%%%%%%%%%%%%%%%%%%%%%%%%%%%%%%%%%%%%%%%%%%%
% Quantifying Error on Ratio
%%%%%%%%%%%%%%%%%%%%%%%%%%%%%%%%%%%%%%%%%%%%%%%%%%%%%%%%%%%
\subsection{Quantifying Uncertainty on Ratio}\label{errorcalc}
  Calculate exact uncertainty and show it here.

%%%%%%%%%%%%%%%%%%%%%%%%%%%%%%%%%%%%%%%%%%%%%%%%%%%%%%%%%%%
% Joint Estimation
%%%%%%%%%%%%%%%%%%%%%%%%%%%%%%%%%%%%%%%%%%%%%%%%%%%%%%%%%%%
\subsection{Joint Estimation of $M_A$ and $\Delta s$}\label{deltas}
  Here is where we do the actual MCMC.
  \subsubsection{Probability of $\theta$ given D}
    Walk through the Bayesian math. Start with wanting the probability of
    $\theta$ given the data and the model and get final equation. Then show how
    to combine with previous experiments. Should give detail for each factor in
    equation (prior, likelihood, posterior, evidence). Should especially give
    detail on how we choose priors. Maybe talk about model selection when
    talking about the evidence.
  \subsubsection{Markov Chain Monte Carlo}
    Go through detail on our implementation of MCMC. Discuss how Markov Chains
    work in general. Then describe affine invariance and parallel tempering.
    Explain that parallel tempering would allow us to do model selection.
  \subsubsection{Results}
    Lots of plots! $\Delta s$!


%This is the end of analysis section

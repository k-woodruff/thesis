\section{Neutrino-Nucleon Cross Sections} \label{theory}
%The next line produces an indented paragraph to start the document
 %unit.  The LaTeX defaults start most units without indentations.
\hspace{\parindent}

%%%%%%%%%%%%%%%%%%%%%%%%%%%%%%%%%%%%%%%%%%%%%%%%%%%%%%%%%%%
% Neutrinos as Nucleon Probes
%%%%%%%%%%%%%%%%%%%%%%%%%%%%%%%%%%%%%%%%%%%%%%%%%%%%%%%%%%%
\subsection{Neutrino Cross Sections}\label{probe}
  \subsubsection{Neutrino Cross Sections}
    Go from electroweak lagrangean to actual cross sections. Show both
    neutral-current and charged-current version of elastic scattering cross
    section.
  \subsubsection{Form Factors}
    Show how cross section changes from point-like to finite structure.
  \subsubsection{$\Delta s$ in Axial Form Factor}
    Discuss where $\Delta s$ shows up and how to maximise it.

%This is the end of introduction

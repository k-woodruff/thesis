\section{Neutrino-Nucleon Cross Sections} \label{theory}
%The next line produces an indented paragraph to start the document
 %unit.  The LaTeX defaults start most units without indentations.
\hspace{\parindent}

%%%%%%%%%%%%%%%%%%%%%%%%%%%%%%%%%%%%%%%%%%%%%%%%%%%%%%%%%%%
% Spin Structure of Nucleons
%%%%%%%%%%%%%%%%%%%%%%%%%%%%%%%%%%%%%%%%%%%%%%%%%%%%%%%%%%%
\subsection{The Spin Structure of Nucleons}
  \subsubsection{Proton spin puzzle}
    Discuss spin structure functions, spin sum rules, singlet axial charge.
  \subsubsection{Nucleon Form Factors}
    Talk about how to parameterize the nucleon in terms of the form factors
    Discuss Dirac, Pauli, electric, and magnetic.
  \subsubsection{The Axial Form Factor}
    Derive the axial form factor. Discuss different axial form factor models
    (dipole...). Show that you can determine $\Delta s$. Make explicit that
    this is the same $\Delta s$ from first subsubsection.
  

%%%%%%%%%%%%%%%%%%%%%%%%%%%%%%%%%%%%%%%%%%%%%%%%%%%%%%%%%%%
% Neutrinos as Nucleon Probes
%%%%%%%%%%%%%%%%%%%%%%%%%%%%%%%%%%%%%%%%%%%%%%%%%%%%%%%%%%%
\subsection{Neutrino Cross Sections}\label{probe}
  \subsubsection{Neutrino Cross Sections}
    Go from electroweak lagrangean to actual cross sections. Show both
    neutral-current and charged-current version of elastic scattering cross
    section.
  \subsubsection{Weak interactions and the Axial Form Factor}
    Why are neutrinos better probes than electrons or photons?

%This is the end of nu-n cross sections section

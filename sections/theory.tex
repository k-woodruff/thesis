\section{Neutrino-Nucleon interactions} \label{sec:theory}
%The next line produces an indented paragraph to start the document
 %unit.  The LaTeX defaults start most units without indentations.
\hspace{\parindent}
This section describes the mathematical foundation for the analysis. It gives a
derivation of the neutral current elastic neutrino-proton cross section, how
the cross section depends on the vector and axial form factors, and the
relationship of the neutral current form factors to the ones measured in
charged current scattering. The section ends with a discussion of the shape of
the form factors that are used in the analysis.

%%%%%%%%%%%%%%%%%%%%%%%%%%%%%%%%%%%%%%%%%%%%%%%%%%%%%%%%%%%
% Particle interactions
%%%%%%%%%%%%%%%%%%%%%%%%%%%%%%%%%%%%%%%%%%%%%%%%%%%%%%%%%%%
\subsection{Two-particle interactions}
  The differential cross section for two-fermion scattering, shown in
  Fig.~\ref{fig:feynmantwofermion}, is given by Fermi's Golden
  Rule~\cite{Aitchison:2004cs}
  \begin{equation}\label{eq:twobodyxsec}
    \sigma = \int \frac{(2\pi)^4 |\mathcal{M}|^2}{4 \sqrt{(p_1\cdot p_2)^2 - m_1^2m_2^2}}
      \cross d\Phi_2(p_1+p_2;p_1',p_2') \,,
  \end{equation}
  where $d\Phi_2(p_1+p_2;p_1',p_2')$ is an element of two-body phase space
  given by
  \begin{equation}\label{eq:twobodyphase}
      d\Phi_2(p_1+p_2;p_1',p_2') = \delta^4(p_1+p_2 - p_1'-p_2')
        \frac{d^3\mathbf{p}_1'}{(2\pi)^3 2E_1'}\frac{d^3\mathbf{p}_2'}{(2\pi)^3 2E_2'} \,,
  \end{equation}
  $p_1$ and $p_2$ are the incoming four-momenta of the fermions $f_1$ and
  $f_2$, respectively, $p_1'$ and $p_2'$ are the outgoing fermion momenta,
  $m_1$ and $m_2$ are the fermion masses, $E_1'$ and $E_2'$ are the outgoing
  fermion energies, and $\mathcal{M}$ is the scattering amplitude.
  Combining Eqns.~\ref{eq:twobodyxsec}~and~\ref{eq:twobodyphase} gives
  \begin{equation}\label{eq:genxsec}
       \sigma = \int \frac{|\mathcal{M}|^2}{64\pi^2}
        \frac{\delta^4(p_1+p_2-p_1'-p_2')}{E_1'E_2'\sqrt{(p_1\cdot p_2)^2 - m_1^2m_2^2}}
        \, d\mathbf{p}_1'd\mathbf{p}_2' \,.
  \end{equation}
  \begin{figure}[ht]
    \centering
    \includegraphics[angle=0,width=4in]{figures/theory/f1f2_feynman.pdf}
    \caption{Feynman diagram of two-fermion scattering.}
    \label{fig:feynmantwofermion}
  \end{figure}

  The scattering amplitude is given by the matrix element of the scattering
  matrix, $S$, between the final and initial states ($\mathcal{M} =
  \phys*{f}{S}{i}$).  The general form of $S$ is~\cite{Aitchison:2004cs}
  \begin{equation}
    S = \sum_{n=0}^{\infty} \frac{(-i)^n}{n!}\int\dots\int d^4x_1 \, d^4x_2 \dots d^4x_n\,
      T\{\hat{\mathcal{H}}'_I(x_1)\hat{\mathcal{H}}'_I(x_2)\dots\hat{\mathcal{H}}'_I(x_n)\} \,,
  \end{equation}
  where $\hat{\mathcal{H}}'_I(x_i)$ is the interaction Hamiltonian density.
  
  The matrix element can more easily be determined using Feynman calculus.  For
  a massive, vector-boson propagator the matrix element for this interaction is
  given by~\cite{Aitchison:2004cs}
  \begin{equation}\label{eq:genmatel}
    \mathcal{M} = \phys*{f}{S}{i} = \phys*{p_1'}{J^{\mu}(0)}{p_1} \frac{i}{q^2 - M_V^2}(-g_{\mu\nu} 
            + q_{\mu}q_{\nu}/M_V^2) \phys*{p_2'}{J^{\mu}(0)}{p_2} \,,
  \end{equation}
  where $q$ is the four-momentum carried by the vector-boson propagator, $M_V$
  is the mass of the propagator, $J^{\mu}(0)$ is the probability current
  operator, and $g_{\mu\nu}$ is the metric tensor.


%%%%%%%%%%%%%%%%%%%%%%%%%%%%%%%%%%%%%%%%%%%%%%%%%%%%%%%%%%%
% Electroweak Interactions
%%%%%%%%%%%%%%%%%%%%%%%%%%%%%%%%%%%%%%%%%%%%%%%%%%%%%%%%%%%
\subsection{Electroweak interactions}

  The charged current, $j^{\mu}_{CC}$, which corresponds to the exchange of a
  $W^{\pm}$ boson, and the neutral current, $j^{\mu}_{NC}$, which corresponds
  to the exchange of the $Z^0$ boson are given by~\cite{Paschos:2007pi}
  \begin{align}\label{eq:ccurrent}
      j^{\mu}_{CC} &= \sum_f \bar{\psi}_f \gamma^{\mu} (1-\gamma_5) \frac{1}{2}(\tau_1 + i\tau_2) \psi_f \\
        \label{eq:ncurrent}
      j^{\mu}_{NC} &= \sum_f \bar{\psi}_f \gamma^{\mu} (1-\gamma_5) \frac{1}{2}(\tau_3) \psi_f 
       - 2\sin^2(\theta_W) j^{\mu}_{em}
  \end{align}
  where $j^{\mu}_{em}$ is the electromagnetic current, $\psi_{f}$ are the weak
  isospin doublets, and $\tau_i$ are the Pauli matrices
  \begin{equation}
      \tau_1 = 
      \begin{pmatrix}
        0 & 1 \\
        1 & 0
      \end{pmatrix} \,,
      \hspace{5mm}
      \tau_2 = 
      \begin{pmatrix}
        0 & -i \\
        i & 0
      \end{pmatrix} \,,
      \hspace{5mm}
      \tau_3 = 
      \begin{pmatrix}
        1 & 0 \\
        0 & -1
      \end{pmatrix} \,.
  \end{equation}
  The lepton weak isospin doublets are~\cite{Paschos:2007pi}
  \begin{equation}
    \psi_{e} = 
    \begin{pmatrix}
        \hat{\nu}_{e} \\
        \hat{e}^-
    \end{pmatrix} \,,
      \hspace{5mm}
    \psi_{\mu} = 
    \begin{pmatrix}
        \hat{\nu}_{\mu} \\
        \hat{\mu}^-
    \end{pmatrix} \,,
      \hspace{5mm}
    \psi_{\tau} = 
    \begin{pmatrix}
        \hat{\nu}_{\tau} \\
        \hat{\tau}^-
    \end{pmatrix} \,,
  \end{equation}
  and the quark weak isospin doublets are
  \begin{equation}
    \psi_{1} = 
    \begin{pmatrix}
        \hat{u} \\
        \hat{d'}
    \end{pmatrix} \,,
      \hspace{5mm}
    \psi_{2} = 
    \begin{pmatrix}
        \hat{c} \\
        \hat{s'}
    \end{pmatrix} \,,
      \hspace{5mm}
    \psi_{3} = 
    \begin{pmatrix}
        \hat{t} \\
        \hat{b'}
    \end{pmatrix} \,,
  \end{equation}
  where $d'$, $s'$, and $b'$ represent the ``mixed" states
  \begin{equation}
      \begin{pmatrix}
        \hat{d}' \\
        \hat{s}' \\
        \hat{b}'
      \end{pmatrix}
      =
      \begin{pmatrix}
          V_{ud} & V_{us} & V_{ub} \\
          V_{cd} & V_{cs} & V_{cb} \\
          V_{td} & V_{ts} & V_{tb}
      \end{pmatrix}
      \begin{pmatrix}
        \hat{d} \\
        \hat{s} \\
        \hat{b}
      \end{pmatrix} \,.
  \end{equation}
  The matrix, $V$, is the Cabibbo-Kobayashi-Maskawa matrix. These doublets
  contain the allowed weak transitions where $V_{ab}$ is the probability of a
  transition from a quark with flavor $a$ to a quark with flavor $b$.

  \subsubsection{The charged and neutral currents}
  The linear combination of Pauli matrices in the charged current
  \begin{equation}
      \frac{1}{2}\tau^+ = \frac{1}{2}(\tau_1 + \tau_2) \,,
  \end{equation}
  acts as an ``isospin raising matrix" and corresponds to the exchange of a
  $W^+$ boson.
  For the leptons, this gives~\cite{Paschos:2007pi}
  \begin{equation}
    \begin{aligned}
      j^{\mu}_{CC}(\textrm{leptons}) &= \sum_{l=e,\mu,\tau}
      \begin{pmatrix}
          \bar{\hat{\nu}}_l \\
          \bar{\hat{l}}
      \end{pmatrix}
      \gamma^{\mu}(1-\gamma_5)\frac{1}{2}
      \begin{pmatrix}
        0 & 1 \\
        0 & 0
      \end{pmatrix}
      \begin{pmatrix}
        \hat{\nu}_l \\
          \hat{l}
      \end{pmatrix} \\
      &= \sum_{l=e,\mu,\tau} \bar{\hat{\nu}}_{l} \gamma^{\mu}(1-\gamma_5)\frac{1}{2}\, \hat{l} \,,
    \end{aligned}
  \end{equation}
  and, similarly, for the quarks we get
  \begin{equation}
      j^{\mu}_{CC}(\textrm{quarks}) =
       \bar{\hat{u}}\gamma^{\mu}(1-\gamma_5)\frac{1}{2}\,\hat{d}'
       +\bar{\hat{c}}\gamma^{\mu}(1-\gamma_5)\frac{1}{2}\,\hat{s}'
       +\bar{\hat{t}}\gamma^{\mu}(1-\gamma_5)\frac{1}{2}\,\hat{b}' \,,
  \end{equation}
  with the total charged current being $j^{\mu}_{CC} =
  j^{\mu}_{CC}(\textrm{leptons}) + j^{\mu}_{CC}(\textrm{quarks})$.

  In neutral current scattering, $\frac{1}{2}\tau_3$ gives the weak isospin
  which acts as a ``weak charge". The electromagnetic current is given by
  \begin{equation}
      j^{\mu}_{em} = \sum_f Q_f \bar{\hat{f}} \gamma^{\mu} \hat{f}
  \end{equation}
  where the sum is over the fermion flavors, and $Q_f$ is the electric charge
  of $f$.  So, the total neutral current is
  \begin{equation}
    \begin{aligned}
        j^{\mu}_{NC} &= \sum_{l=e,\mu,\tau} \left(\bar{\hat{\nu}}_{l}
        \gamma^{\mu}(1-\gamma_5) \frac{1}{2}\, \hat{\nu}_{l} - \bar{\hat{l}}
        \gamma^{\mu}(1-\gamma_5) \frac{1}{2}\, \hat{l} 
        +\sin^2\theta_W \bar{\hat{l}}\gamma^{\mu}\hat{l} \right) \\
        &+ \sum_{q=u,c,t} \left(\bar{\hat{q}} \gamma^{\mu}(1-\gamma_5)\frac{1}{2}\hat{q} 
        - \sin^2\theta_W \frac{2}{3} \bar{\hat{q}}\gamma^{\mu}\hat{q} \right) \\
        &+ \sum_{q=d,s,b} \left(- \bar{\hat{q}} \gamma^{\mu}(1-\gamma_5)\frac{1}{2}\hat{q} 
        + \sin^2\theta_W \frac{1}{3} \bar{\hat{q}}\gamma^{\mu}\hat{q} \right) \,.
     \end{aligned}
  \end{equation}
 
  \subsubsection{V$-$A structure}

  We can separate the currents into their vector and pseudovector, or axial
  vector, components. The terms that contain just $\gamma^{\mu}$ behave like
  vectors under a parity transformation~\cite{Alberico:2001sd}
  \begin{equation}
    \hat{\textrm{\textbf{P}}}\hat{\psi}(\textbf{x},t)\hat{\textrm{\textbf{P}}}^{-1} \, 
      \gamma^{\mu} \, \hat{\textrm{\textbf{P}}}\hat{\psi(\textbf{x},t)}\hat{\textrm{\textbf{P}}}^{-1} 
      = - \hat{\textrm{\textbf{P}}}\hat{\psi}(-\textbf{x},t)\hat{\textrm{\textbf{P}}}^{-1} \, 
      \gamma^{\mu} \, \hat{\textrm{\textbf{P}}}\hat{\psi(-\textbf{x},t)}\hat{\textrm{\textbf{P}}}^{-1}
  \end{equation}
  where $\hat{\textrm{\textbf{P}}}$ is the parity operator
  \begin{equation}
    \textrm{\textbf{P}}: \textbf{x} \rightarrow -\textbf{x}, t \rightarrow t \,.
  \end{equation}
  The terms that contain $\gamma^{\mu}\gamma_{5}$ behave like axial vectors
  under a parity transformation
  \begin{equation}
    \hat{\textrm{\textbf{P}}}\hat{\psi}(\textbf{x},t)\hat{\textrm{\textbf{P}}}^{-1} \, 
      \gamma^{\mu}\gamma_5 \, \hat{\textrm{\textbf{P}}}\hat{\psi(\textbf{x},t)}\hat{\textrm{\textbf{P}}}^{-1} 
      = + \hat{\textrm{\textbf{P}}}\hat{\psi}(-\textbf{x},t)\hat{\textrm{\textbf{P}}}^{-1} \, 
      \gamma^{\mu}\gamma_5 \, \hat{\textrm{\textbf{P}}}\hat{\psi(-\textbf{x},t)}\hat{\textrm{\textbf{P}}}^{-1}
      \,.
  \end{equation}
  The charged current is
  \begin{equation}
    \begin{aligned}
    j^{\mu}_{CC} = \frac{g}{\sqrt{2}}\Bigg[&\sum_{l=e,\mu,\tau} \bar{\hat{\nu}}_l (\gamma^{\mu} 
          - \gamma^{\mu}\gamma_5)\frac{1}{2}\,\hat{l} \\
      &+ \bar{\hat{u}}(\gamma^{\mu} - \gamma^{\mu}\gamma_5)\frac{1}{2}\, \hat{d}'
      + \bar{\hat{c}}(\gamma^{\mu} - \gamma^{\mu}\gamma_5)\frac{1}{2}\, \hat{s}'
      + \bar{\hat{t}}(\gamma^{\mu} - \gamma^{\mu}\gamma_5)\frac{1}{2}\, \hat{b}' \Bigg] \,,
    \end{aligned}
  \end{equation}
  and the neutral current becomes
  \begin{equation}
    \begin{aligned}
        j^{\mu}_{NC} = \frac{g}{2\cos\theta_W} \Bigg[&\sum_{l=e,\mu,\tau} 
        \left(\bar{\hat{\nu}}_{l}(g_V^{l} \gamma^{\mu}- g_A^{l} \gamma^{\mu}\gamma_5) \hat{\nu}_{l} 
         + \bar{\hat{l}}(g_V^{l} \gamma^{\mu}- g_A^{l} \gamma^{\mu}\gamma_5)\, \hat{l} \right) \\
        &+ \sum_{q=u,d,c,s,t,b} 
         \left(- \bar{\hat{q}} (g_V^{q} \gamma^{\mu}-g_A^{q} \gamma^{\mu}\gamma_5)\hat{q} \right)\Bigg] \,.
     \end{aligned}
  \end{equation}
  where 
  \begin{equation}
    g_V^{f} = \frac{1}{2}\tau_3^{f} - 2\sin^2\theta_W Q_f \,,
    \hspace{3mm}
    g_A^{f} = \frac{1}{2}\tau_3^{f} \,.
  \end{equation}

%%%%%%%%%%%%%%%%%%%%%%%%%%%%%%%%%%%%%%%%%%%%%%%%%%%%%%%%%%%
% Electroweak Scattering Matrix Elements
%%%%%%%%%%%%%%%%%%%%%%%%%%%%%%%%%%%%%%%%%%%%%%%%%%%%%%%%%%%
\subsection{Electroweak Scattering Matrix Elements}

  \begin{figure}[h]
    \centering
    \begin{subfigure}{2.5in}
      \includegraphics[angle=0,width=2.5in]{figures/theory/nun_feynman.pdf}
      \caption{Feynman diagram of charged-current elastic lepton-nucleon
      scattering.}
      \label{fig:ccqefeynman}
    \end{subfigure}
    \hspace{2pt}
    \begin{subfigure}{2.5in}
      \includegraphics[angle=0,width=2.5in]{figures/theory/nup_feynman.pdf}
      \caption{Feynman diagram of neutral-current elastic lepton-nucleon
      scattering.}
      \label{fig:ncefeynman}
    \end{subfigure}
  \end{figure}

  To calculate the charged-current neutrino-nucleon scattering matrix element,
  as shown in Fig.~\ref{fig:ccqefeynman}, we need to determine the lepton CC
  matrix element, $_{l}{\phys*{k'}{J^{\mu}_{CC}(0)}{k}}_{\nu_l}$, and the nucleon
  CC matrix element, $_{\textrm{p}}{\phys*{p'}{J^{\mu}_{CC}(0)}{p}}_{\textrm{n}}$,
  and for neutral-current, as shown in Fig.~\ref{fig:ncefeynman}, we need the
  lepton NC matrix element, $_{\nu_l}\phys*{k'}{J^{\mu}_{NC}(0)}{k}_{\nu_l}$, and
  the nucleon NC matrix element,
  $_{\textrm{p}}\phys*{p'}{J^{\mu}_{NC}(0)}{p}_{\textrm{p}}$. Leptons are
  point-like particles, so their matrix elements are straight-forward
  \begin{equation}\label{eq:leptonmatel}
    \begin{aligned}
      _l\phys*{k'}{J^{\mu}_{CC}(0)}{k}_{\nu_l}
        &= -i\frac{G_F}{\sqrt{2}}\bar{u}(k')(\gamma^{\mu} - \gamma^{\mu}\gamma_5)u(k) \,, \\
        _{\nu_l}\phys*{k'}{J^{\mu}_{NC}(0)}{k}_{\nu_l}
        &= -\frac{G_F}{\sqrt{2}}\bar{u}(k')(\gamma^{\mu} - \gamma^{\mu}\gamma_5)u(k) \,.
    \end{aligned}
  \end{equation}

  \subsubsection{Nucleon matrix elements}
  Since nucleons have a finite structure, the nucleon matrix elements have a
  more complicated form. The current is corrected by form factors which encode
  the longitudinal nucleon structure.

  If we define the vector and axial parts of the nucleon currents by
  \begin{equation}
    \begin{aligned}
    v^{\mu}_i &= \bar{\hat{\psi}}_f\, \gamma^{\mu}\frac{1}{2}\tau_i\, \hat{\psi}_f \,, \\
    a^{\mu}_i &= \bar{\hat{\psi}}_f\, \gamma^{\mu}\gamma_5\frac{1}{2}\tau_i\, \hat{\psi}_f \,,
    \end{aligned}
  \end{equation}
  where $\hat{\psi}$ are the quark doublets and $\tau_i$ are the Pauli matrices
  still, then the nucleon charged and neutral current become
  \begin{equation}
    \begin{aligned}
      j^{\mu}_{CC} &= v^{\mu}_+ - a^{\mu}_+ \,, \\
      j^{\mu}_{NC} &= v^{\mu}_3 - a^{\mu}_3 - 2\sin^2\theta_W j^{\mu}_{em;q} \,,
    \end{aligned}
  \end{equation}
  where $v^{\mu}_+ = v^{\mu}_1 + iv^{\mu}_2$ and $a^{\mu}_+ = a^{\mu}_1 +
  ia^{\mu}_2$. The nucleon matrix elements become
  \begin{equation}
    \begin{aligned}
      _{\textrm{p}}\phys*{p'}{J^{\mu}_{CC}}{p}_{\textrm{n}}
          &= \prescript{}{\textrm{p}}{\phys*{p'}{V^{\mu}_{+}}{p}}_{\textrm{n}}
            - \prescript{}{\textrm{p}}{\phys*{p'}{A^{\mu}_{+}}{p}}_{\textrm{n}} \,, \\
      _{\textrm{p}}\phys*{p'}{J^{\mu}_{NC}}{p}_{\textrm{p}}
          &= \prescript{}{\textrm{p}}{\phys*{p'}{V^{\mu}_{3}}{p}}_{\textrm{p}} 
            - \prescript{}{\textrm{p}}{\phys*{p'}{A^{\mu}_{3}}{p}}_{\textrm{p}} 
            - \prescript{}{\textrm{p}}{\phys*{p'}{J^{\mu}_{em}}{p}}_{\textrm{p}} \,,
    \end{aligned}
  \end{equation}
  where $J^{\mu}$, $V^{\mu}$, and $A^{\mu}$ are current operators.

  \subsubsection{Electromagnetic Scattering}

  It is easiest to first find an equation for $j^{\mu}_{em;q}$ in terms of the
  electric and magnetic form factors. The nucleon electromagnetic current
  should have the same vector form as the electromagnetic current for
  point-like particles. The available physical variables to construct the
  nucleon EM current are $p$, $p'$, $\gamma^{\mu}$. The most general form
  is~\cite{Alberico:2001sd}
  \begin{equation}
    \Gamma^{\mu} = \gamma^{\mu}\cdot A + (p'^{\mu} + p^{\mu})\cdot B + (p'^{\mu} - p^{\mu})\cdot C \,,
  \end{equation}
  where $\Gamma^{\mu}$ is given by the equation
  $_{\textrm{p}}\phys*{p'}{J^{\mu}_{em}(0)}{p}_{\textrm{p}} =
  \bar{u}(p')\Gamma^{\mu}u(p)$, and $A$, $B$, and $C$ are arbitrary form
  factors.  $\Gamma^{\mu}$ is constrained further by the Ward
  identity~\cite{Ward:1950xp}, $q_{\mu}\Gamma^{\mu} = 0$, which guarantees
  current conservation. The $\gamma^{\mu}$ and $(p'^{\mu} + p^{\mu})$ terms in
  $q_{\mu}\Gamma^{\mu}$ go to zero, but the $(p'^{\mu} - p^{\mu})$ term does
  not, so $C=0$. Using the Gordon identity~\cite{Gordon:1928}, the general form
  for the nucleon EM current matrix element is
  \begin{equation}\label{eq:FFem}
    _{\textrm{p}}\phys*{p'}{J^{\mu}_{em}(0)}{p}_{\textrm{p}} =
      \bar{u}(p')\left[ \gamma^{\mu}F_1(Q^2) + \frac{i\sigma^{\mu\nu}q_{\mu}}{2M}F_2(Q^2)  \right]u(p) \,.
  \end{equation}
  The form factors $F_1$ and $F_2$ are the Dirac and Pauli form factors,
  respectively, and they are functions of $Q^2 = -q^2$, the negative
  four-momentum transfer. They can be transformed into the Sachs form factors,
  $G_E$ and $G_M$ by the relationships
  \begin{equation}
    G_E(Q^2) = F_1(Q^2) - \frac{Q^2}{4M^2}F_2(Q^2)\,, \hspace{5mm} G_M(Q^2) = F_1(Q^2) + F_2(Q^2) \,.
  \end{equation}
  The electric form factor, $G_E$ encodes the longitudinal electric charge
  structure of the nucleon and the magnetic form factor, $G_M$, encodes the
  longitudinal magnetic structure. At the limit when $Q^2$ goes to zero, the
  Sachs form factors become the net charge and magnetic moment of the nucleon.
  \begin{equation}
    \begin{aligned}
      G_{E;p}(Q^2=0) = 1 \,,& \hspace{5mm} G_{M;p}(Q^2=0) = \mu_p \,, \\
      G_{E;n}(Q^2=0) = 0 \,,& \hspace{5mm} G_{M;n}(Q^2=0) = \mu_n \,,
    \end{aligned}
  \end{equation}
  where $\mu_p$ and $\mu_n$ are the proton and neutron magnetic moments.
 
  \subsubsection{Vector current}

    The quark part of the electromagnetic current, $j^{\mu}_{em;q}$, can be
    written as~\cite{Alberico:2001sd}
    \begin{equation}\label{eq:emcurrent30}
      \begin{aligned}
        j^{\mu}_{em;q} &= \bar{\hat{\psi}}_f \, Q_f \gamma^{\mu}\, \hat{\psi}_f \\
                       &= \bar{\hat{\psi}}_f \, (\tau_3 + \frac{1}{6})\gamma^{\mu} \, \hat{\psi}_f \\
                       &= v^{\mu}_3 + v^{\mu}_0 \,,
      \end{aligned}
    \end{equation}
    where $v^{\mu}_0 = \frac{1}{6}\bar{\hat{\psi}}_f \, \gamma^{\mu}
    \hat{\psi}_f$. Now we can write the nucleon electromagnetic current matrix
    element in terms of the vector and isoscalar parts
    \begin{equation}
      \begin{aligned}
        _{\textrm{p(n)}}\phys*{p'}{J^{\mu}_{em}(0)}{p}_{\textrm{p(n)}} 
            &= \prescript{}{\textrm{p(n)}}{\phys*{p'}{V^{\mu}_3(0) + V^{\mu}_0(0))}{p}}_{\textrm{p(n)}} \\
            &= \prescript{}{\textrm{p(n)}}{\phys*{p'}{V^{\mu}_3(0)}{p}}_{\textrm{p(n)}} 
             + \prescript{}{\textrm{p(n)}}{\phys*{p'}{V^{\mu}_0(0)}{p}}_{\textrm{p(n)}} \,,
      \end{aligned}
    \end{equation}
    Since $V^{\mu}_3$ behaves as a vector under the charge symmetry operator
    and $V^{\mu}_0$ behaves as a scalar, the following equations are true
    \begin{equation}
      \begin{aligned}
        \prescript{}{\textrm{p}}{\phys*{p'}{V^{\mu}_3(0)}{p}}_{\textrm{p}} 
          &= - \prescript{}{\textrm{n}}{\phys*{p'}{V^{\mu}_3(0)}{p}}_{\textrm{n}} \,, \\
        \prescript{}{\textrm{p}}{\phys*{p'}{V^{\mu}_0(0)}{p}}_{\textrm{p}} 
          &= + \prescript{}{\textrm{n}}{\phys*{p'}{V^{\mu}_0(0)}{p}}_{\textrm{n}} \,. \\
      \end{aligned}
    \end{equation}
    Then
    \begin{align}\label{eq:V3em}
      \prescript{}{\textrm{p}}{\phys*{p'}{V^{\mu}_3(0)}{p}}_{\textrm{p}} 
        &= \frac{1}{2} \left[\prescript{}{\textrm{p}}{\phys*{p'}{J^{\mu}_{em}(0)}{p}}_{\textrm{p}} 
         - \prescript{}{\textrm{n}}{\phys*{p'}{J^{\mu}_{em}(0)}{p}}_{\textrm{n}} \right] \,, \\
      \prescript{}{\textrm{p}}{\phys*{p'}{V^{\mu}_0(0)}{p}}_{\textrm{p}} 
        &= \frac{1}{2} \left[\prescript{}{\textrm{p}}{\phys*{p'}{J^{\mu}_{em}(0)}{p}}_{\textrm{p}} 
         + \prescript{}{\textrm{n}}{\phys*{p'}{J^{\mu}_{em}(0)}{p}}_{\textrm{n}} \right] \,.
    \end{align}
    Combining equations~\ref{eq:FFem}~and~\ref{eq:V3em} gives
    \begin{equation}
      \prescript{}{\textrm{p}}{\phys*{p'}{V^{\mu}_3(0)}{p}}_{\textrm{p}} 
        = \bar{u}(p') \left[\gamma^{\mu}F_1^V(Q^2) 
          + \frac{i\sigma^{\mu\nu}q_{\mu}}{2M}F_2^V(Q^2)\right] u(p) \,,
    \end{equation}
    where the vector form factors, $F_1^V$ and $F_2^V$ are defined as
    \begin{equation}
      \begin{aligned}
        F_1^V(Q^2) &= \frac{1}{2}\left( F_{1,\textrm{p}}(Q^2) - F_{1,\textrm{n}}(Q^2)\right) \,, \\
        F_2^V(Q^2) &= \frac{1}{2}\left( F_{2,\textrm{p}}(Q^2) - F_{2,\textrm{n}}(Q^2)\right) \,.
      \end{aligned}
    \end{equation}

    The isoscalar nucleon matrix element is then
    \begin{equation}
      {\phys*{p'}{V^{\mu}_0(0)}{p}}
      = \bar{u}(p')\left[\gamma^{\mu}\frac{1}{2}\left(F_{1,p}(Q^2) + F_{1,n}(Q^2)\right) 
        + \frac{i\sigma^{\mu\nu}q_{\mu}}{2M}\frac{1}{2}\left(F_{2,p}(Q^2) + F_{2,n}(Q^2)\right)\right]u(p)
    \end{equation}
 
    Under the conserved vector current (CVC)
  hypothesis~\cite{Gerstein:1956,Feynman:1958ty} the ``raising" and ``lowering"
  vector currents, $v^{\mu}_{\pm}$, in the charged current interactions are the
  same as the vector part of the electromagnetic current, $v^{\mu}_{3}$.

  \subsubsection{Axial current}

    The axial current nucleon matrix elements can also be parameterized in
    terms of form factors. Starting with the charged axial current, the most
    general form contains an axial, a pseudoscalar, and a tensor
    term,~\cite{Alberico:2001sd}
    \begin{equation}
      \prescript{}{\textrm{p}}{\phys*{p'}{A^{\mu}_+(0)}{p}}_{\textrm{n}} 
        = \bar{u}(p') \left[\gamma^{\mu}\gamma^{5} G_A(Q^2) 
          + \frac{q^{\mu}\gamma^{5}}{2M}G_P^{CC}(Q^2) 
          + \frac{i\sigma^{\mu\nu}q_{\mu}\gamma^{5}}{2M}G_T^{CC}(Q^2)\right] u(p) \,,
    \end{equation}
    where $G_A$, $G_P^{CC}$, and $G_T^{CC}$ are the axial form factors.  From
    isospin symmetry, we have
    \begin{equation}
      \prescript{}{\textrm{p}}{\phys*{p'}{A^{\mu}_+(0)}{p}}_{\textrm{n}} 
      = \prescript{}{\textrm{n}}{\phys*{p'}{A^{\mu}_-(0)}{p}}_{\textrm{p}} 
      \equiv \prescript{}{\textrm{p}}{\phys*{p}{A^{\mu}_+(0)}{p'}}_{\textrm{n}}^* \,,
    \end{equation}
    which implies that $G_T^{CC}(Q^2) = 0$. In quasi-elastic scattering the
    pseudoscalar term containing $G_P^{CC}(Q^2)$ is proportional to the lepton
    mass and can be ignored in neutral current scattering since $m_{\nu}\approx
    0$. The charged current axial matrix element is then just
    \begin{equation}
      \prescript{}{\textrm{p}}{\phys*{p'}{A^{\mu}_+(0)}{p}}_{\textrm{n}} 
        = \bar{u}(p') \,\gamma^{\mu}\gamma^{5} G_A(Q^2)\, u(p) \,.
    \end{equation}
    The form factor $G_A$ encodes the longitudinal spin structure of the
    nucleon due to the spin of the up and down quarks and is referred to as the
    charged current axial form factor, or just the axial form factor.

    The most general form for the neutral axial current nucleon matrix element
    is similarly
    \begin{equation}
      \prescript{}{\textrm{p}}{\phys*{p'}{A^{\mu}_3(0)}{p}}_{\textrm{n}} 
        = \bar{u}(p') \left[\gamma^{\mu}\gamma^{5} G_A^{NC}(Q^2) 
          + \frac{q^{\mu}\gamma^{5}}{2M}G_P^{NC}(Q^2) 
          + \frac{i\sigma^{\mu\nu}q_{\mu}\gamma^{5}}{2M}G_T^{NC}(Q^2)\right] u(p) \,.
    \end{equation}
    Just as in the charged-current case, the tensor part is zero, $G_T^{NC}(Q^2) =
    0$, and we can again neglect $G_P^{NC}$ which is proportional to the lepton
    mass. The neutral axial current matrix element is
    \begin{equation}
      \prescript{}{\textrm{p}}{\phys*{p'}{A^{\mu}_3(0)}{p}}_{\textrm{n}} 
        = \bar{u}(p') \,\gamma^{\mu}\gamma^{5} G_A^{NC}(Q^2)\, u(p) \,.
    \end{equation}
    which can be related to the charged current axial form factor through
    isospin symmetry.  The relationship between the neutral and charged axial
    current operators is
    \begin{equation}
      [I^k,A_j^{\mu}] = i\epsilon^{kjl}A^{\mu}_l \,,
    \end{equation}
    where $I^k$ is the total isospin operator, and $\epsilon^{kjl}$ is the
    antisymmetric tensor. From this relationship, we get
    \begin{equation}
      \prescript{}{\textrm{p}}{\phys*{p'}{A^{\mu}_3(0)}{p}}_{\textrm{p}}
       = \frac{1}{2}\prescript{}{\textrm{p}}{\phys*{p'}{A^{\mu}_+(0)}{p}}_{\textrm{n}}\,,
    \end{equation}
    which implies
    \begin{equation}
      G_A^{NC}(Q^2) = \frac{1}{2}G_A(Q^2) \,,
    \end{equation}
    assuming only contributions from up and down quarks.

%%%%%%%%%%%%%%%%%%%%%%%%%%%%%%%%%%%%%%%%%%%%%%%%%%%%%%%%%%%
% Strangeness in the Nucleon
%%%%%%%%%%%%%%%%%%%%%%%%%%%%%%%%%%%%%%%%%%%%%%%%%%%%%%%%%%%
\subsection{Strangeness in the nucleon} \label{sec:strangeness}

  If contributions to the nucleon from quarks heavier than the strange are
  neglected, the quark part of the charged and neutral currents can be
  separated between the light quarks and the strange
  quark,~\cite{Alberico:2001sd}
  \begin{equation}\label{eq:chargedcurrent}
    \begin{aligned}
    j^{\mu}_{CC}(\textrm{quarks}) &= \bar{\hat{u}}(\gamma^{\mu} 
                - \gamma^{\mu}\gamma_5)\frac{1}{2}\tau_{\pm}\hat{u}
                - \bar{\hat{d}}(\gamma^{\mu} - \gamma^{\mu}\gamma_5)\frac{1}{2}\tau_{\pm}\hat{d} \\
                 &\equiv \bar{\hat{N}}(\gamma^{\mu} 
                - \gamma^{\mu}\gamma_5)\frac{1}{2}\tau_{\pm}\hat{N} \,,
    \end{aligned}
  \end{equation}
  \begin{equation}\label{eq:neutralcurrent}
    \begin{aligned}
    j^{\mu}_{NC}(\textrm{quarks}) = \bar{\hat{N}}(\gamma^{\mu} 
                - \gamma^{\mu}\gamma_5)\frac{1}{2}\tau_3\hat{N} 
                &- \bar{\hat{s}}(\gamma^{\mu} - \gamma^{\mu}\gamma_5)\frac{1}{2}\tau_3\hat{s}  \\
                  &- 2\sin^2\theta_W j^{\mu}_{em;q} \,.
    \end{aligned}
  \end{equation}

  \subsubsection{Strange currents}
    If we redefine the neutral vector currents as
    \begin{equation}\label{eq:updowncurrents}
      \begin{aligned}
        v^{\mu}_3 &\equiv \bar{\hat{N}} \gamma^{\mu}\frac{1}{2}\tau_3 \hat{N} \\
        a^{\mu}_3 &\equiv \bar{\hat{N}} \gamma^{\mu}\gamma_5\frac{1}{2}\tau_3 \hat{N} \,,
      \end{aligned}
    \end{equation}
    and define the strange part of the currents as
    \begin{equation}\label{eq:strangecurrents}
      \begin{aligned}
        v^{\mu}_s &\equiv \bar{\hat{s}} \gamma^{\mu} \hat{s} \\
        a^{\mu}_s &\equiv \bar{\hat{s}} \gamma^{\mu}\gamma_5 \hat{s} \,,
      \end{aligned}
    \end{equation}
    we can combine
    Eqs.~\ref{eq:chargedcurrent},~\ref{eq:neutralcurrent},~\ref{eq:updowncurrents},~and~\ref{eq:strangecurrents}
    to get
    \begin{align}
      j^{\mu}_{CC}(\textrm{quarks}) &= (v^{\mu}_3 - a^{\mu}_3) \,, \\
      j^{\mu}_{NC}(\textrm{quarks}) &= (v^{\mu}_3 - a^{\mu}_3) 
        - \frac{1}{2}(v^{\mu}_s - a^{\mu}_s) - 2\sin^2\theta_W j^{\mu}_{em;q} \,.
    \end{align}
    
    The quark part of the electromagnetic current can also be separated into
    light and heavy quark components. Separating the vector and isoscalar terms
    from Eq.~\ref{eq:emcurrent30}, into quark components gives
    \begin{equation}
      j^{\mu}_{em;q} = (v^{\mu}_3 + v^{\mu}_0) - \frac{1}{2}(v^{\mu}_s + v^{\mu}_{0s}) \,,
    \end{equation}
    where $v^{\mu}_3$ and $v^{\mu}_s$ are as defined in
    Eqs.~\ref{eq:updowncurrents}~and~\ref{eq:strangecurrents}, and $v^{\mu}_0$
    and $v^{\mu}_{0s}$ are defined as
    \begin{equation}
      \begin{aligned}
        v^{\mu}_0 &\equiv \frac{1}{6}\bar{\hat{N}}\gamma^{\mu}\hat{N} \,, \\
        v^{\mu}_{0s} &\equiv -\frac{1}{3}\bar{\hat{s}}\gamma^{\mu}\hat{s} \,.
      \end{aligned}
    \end{equation}
    The quark part of the neutral current becomes
    \begin{equation}
      j^{\mu}_{NC}(\textrm{quarks}) = (1-2\sin^2\theta_W)(v^{\mu}_3 - \frac{1}{2}v^{\mu}_s)
                                    - (a^{\mu}_3 - \frac{1}{2}a^{\mu}_s)
                                    - 2\sin^2\theta_W(v^{\mu}_0 - \frac{1}{2}v^{\mu}_{0s}) \,.
    \end{equation}

  \subsubsection{Strange nucleon matrix elements}

  The single-nucleon matrix elements for the charged current becomes
  \begin{equation}
     \prescript{}{\textrm{p}}{\phys*{p'}{J^{\mu}_{CC}(0)}{p}}_{\textrm{n}} 
       = \prescript{}{\textrm{p}}{\phys*{p'}{V^{\mu}_3(0)}{p}}_{\textrm{n}}
       - \prescript{}{\textrm{p}}{\phys*{p'}{A^{\mu}_3(0)}{p}}_{\textrm{n}} \,.
  \end{equation}
  In terms of the vector and axial form factors, the matrix element is
  \begin{equation}
     \prescript{}{\textrm{p}}{\phys*{p'}{J^{\mu}_{CC}(0)}{p}}_{\textrm{n}}
       = \bar{u}(p')\left[\gamma^{\mu} F_1^V(Q^2)
          + \frac{i\sigma^{\mu\nu}q_{\mu}}{2M} F_2^V(Q^2)
          - \gamma^{\mu}\gamma_5G_A(Q^2) \right] u(p) \,.
  \end{equation}
  The single-nucleon matrix elements for the neutral current becomes
  \begin{equation}
    \begin{aligned}
      \prescript{}{\textrm{p}}{\phys*{p'}{J^{\mu}_{NC}(0)}{p}}_{\textrm{p}} 
       = (1-2\sin^2\theta_w)&(\prescript{}{\textrm{p}}{\phys*{p'}{V^{\mu}_3(0)}{p}}_{\textrm{p}}
        - \frac{1}{2}\prescript{}{\textrm{p}}{\phys*{p}{V^{\mu}_s(0)}{p}}_{\textrm{p}}) \\
       - &(\prescript{}{\textrm{p}}{\phys*{p'}{A^{\mu}_3(0)}{p}}_{\textrm{p}}
        - \frac{1}{2}\prescript{}{\textrm{p}}{\phys*{p}{A^{\mu}_s(0)}{p}}_{\textrm{p}}) \\
       - 2\sin^2\theta_W&(\prescript{}{\textrm{p}}{\phys*{p'}{V^{\mu}_0(0)}{p}}_{\textrm{p}}
        - \frac{1}{2}\prescript{}{\textrm{p}}{\phys*{p}{V^{\mu}_{0s}(0)}{p}}_{\textrm{p}}) \,.
    \end{aligned}
  \end{equation}
  If we ignore the strange components of the Dirac and Pauli form factors
  ($F_1$ and $F_2$), the matrix element is
  \begin{equation}\label{eq:nucmatel}
    \begin{aligned}
     \prescript{}{\textrm{p}}{\phys*{p'}{J^{\mu}_{NC}(0)}{p}}_{\textrm{p}}
       = \bar{u}(p')&\bigg[(1-\sin^2\theta_W)\{\gamma^{\mu}F_1^{NC}(Q^2)
          + \frac{i\sigma^{\mu\nu}q_{\mu}}{2M}F_2^{NC}(Q^2)\}  \\
          - \gamma^{\mu}&\gamma_5 G_A^{NC}(Q^2)
          - 2\sin^2\theta_W\{\gamma^{\mu}F_1^p(Q^2) 
          + \frac{i\sigma^{\mu\nu}q_{\mu}}{2M}F_2^p(Q^2)\} \bigg]u(p) \,,
    \end{aligned}
  \end{equation}
  where
  \begin{equation}
    F_{1,2}^{NC}(Q^2) = \frac{1}{2}F_{1,2}^V(Q^2) \,,
  \end{equation}
  and
  \begin{equation}
    G_A^{NC} = \frac{1}{2}G_A(Q^2) - \frac{1}{2}G_A^s(Q^2) \,.
  \end{equation}
  The neutral current axial form factor, $G_A^{NC}$, represents the
  longitudinal spin structure of the nucleon due to all three of the quark
  flavors (up, down, and strange).
  

%%%%%%%%%%%%%%%%%%%%%%%%%%%%%%%%%%%%%%%%%%%%%%%%%%%%%%%%%%%
% Neutrino-proton elastic cross section
%%%%%%%%%%%%%%%%%%%%%%%%%%%%%%%%%%%%%%%%%%%%%%%%%%%%%%%%%%%
\subsection{Neutrino-proton neutral current elastic cross section}\label{sec:probe}

  We know have everything to calculate the neutrino-proton neutral current
  elastic cross section. Figure~\ref{fig:feynmannup} shows the Feynman diagram
  of the interaction. The neutrino is represented by the letter $\nu$ with
  incoming four-momentum $k$ and outgoing four-momentum $k'$. The proton is
  represented by the letter p with incoming four-momentum $p$ and outgoing
  four-momentum $p'$. The four-momentum transferred by the $Z^0$ boson is $q$
  with $q = k-k' = p' - p$.
  \begin{figure}[ht]
    \centering
    \includegraphics[angle=0,width=4in]{figures/theory/nup_feynman.pdf}
    \caption{Feynman diagram of neutrino-proton neutral current elastic
    scattering.}
    \label{fig:feynmannup}
  \end{figure}

  Combining
  Eqs.~\ref{eq:genxsec},~\ref{eq:genmatel},~\ref{eq:leptonmatel},~and~\ref{eq:nucmatel},
  averaging over the spin states, and setting the outgoing neutrino mass to
  zero gives (in the LLewellyn-Smith formalism)~\cite{LLewellynSmith:1971uhs}
  \begin{equation}
    (\frac{d\sigma}{dQ^2})^{NC} = \frac{G_F^2 M_p^2}{8\pi E_{\nu}^2} 
      \bigg[A - \frac{(4M_p E_{\nu} - Q^2)}{M_P^2} B + \frac{(4M_p E_{\nu} - Q^2)^2}{M_P^4} C \bigg]
  \end{equation}
  with
  \begin{align}
    A &= 4\tau\big[(1+\tau)(G_A^{NC})^2 - (1-\tau)(F_2^{NC})^2 + \tau(1 - \tau)(F_2^{NC})^2 + 4\tau F_1^{NC} F_2^{NC}\big] \,, \\
    B &= 4\tau \big[G_A^{NC}(F_1^{NC} + F_2^{NC}) \big] \,, \\
    C &= \frac{1}{4}\big[ (G_A^{NC})^2 + (F_1^{NC})^2 + (F_2^{NC})^2 \big] \,,
  \end{align}
  where $E_{\nu}$ is the incoming neutrino energy, $M_P$ is the proton mass,
  $\tau = \frac{Q^2}{4M_P}$, and the form factors are functions of $Q^2$.


%%%%%%%%%%%%%%%%%%%%%%%%%%%%%%%%%%%%%%%%%%%%%%%%%%%%%%%%%%%
% Determination of the Form Factors
%%%%%%%%%%%%%%%%%%%%%%%%%%%%%%%%%%%%%%%%%%%%%%%%%%%%%%%%%%%
\subsection{Determination of the Nucleon Form Factors} \label{sec:formfactorforms}

  Up to this point, we have only parameterized the electromagnetic and weak
  currents in terms of the vector and axial form factors, but have said nothing
  about their form. The actual form factors are determined empirically from
  experiment. To determine the form factors from experimental data some
  parameterization must be chosen.

  There are many existing parameterizations of the electromagnetic form factors
  which have been fit to electromagnetic scattering data. Since there are much
  more data available from electromagnetic scattering than from weak
  scattering, the parameterizations of the electromagnetic form factors can be
  more precise with more parameters.  See~\cite{Perdrisat:2006hj} for an
  extensive review of the nucleon electromagnetic form factors.
  
  The most common parameterization of the axial form factor, the dipole form,
  has only one free parameter known as the axial mass, $M_A$,
  \begin{equation}
    G_A^{(dipole)}(Q^2) = \frac{G_A(0)}{(1+Q^2/M_A^2)^2} \,.
  \end{equation}
  The dipole form of the axial form factor was motivated by early models of the
  electromagnetic form factors which are no longer used. There hasn't been
  strong physical motivation for new axial form factor parameterizations since
  not as much data exist as for the electromagnetic form factors.

  Constraining the shape of the axial form factor to one parameter can lead to
  both overconfidence in the uncertainty of the measurement of the form factor
  and disagreement of the parameters between different experimental
  measurements at different ranges of negative four-momentum transfer
  squared~\cite{Bhattacharya:2011ah}.

  \subsubsection{Model-Independent Form Factor Parameterization}\label{sec:zexpansion}

  Both the electromagnetic and axial form factors can be determined using a
  model-independent parameterization referred to as $z$
  expansion~\cite{Boyd:1997qw}.

  The $z$ expansion parameterization is made by mapping $Q^2$ onto a domain
  where the form factor, $F$ is analytic
  \begin{equation}\label{eq:zdef}
    z(Q^2,t_{cut},t_0) = 
    \frac{\sqrt{t_{cut}+Q^2} - \sqrt{t_{cut} - t_0}}{\sqrt{t_{cut}+Q^2} + \sqrt{t_{cut} - t_0}} \,,
  \end{equation}
  where $t_{cut}$ is the leading threshold for states that can be produced by
  the vector or axial current. The form factor is analytic everywhere where
  $Q^2 \ge -t_{cut}$.  In the case of the electromagnetic (vector) form factors,
  $t_{cut} = (2m_{\pi})^2$ (two-pion threshold). In the axial form factor case,
  $t_{cut} = (3m_{\pi})^2$ (three-pion threshold)~\cite{Federbush:1958zz}. The
  parameter $t_0$ is an arbitrary number that can be chosen to minimize the
  absolute value of $z$~\cite{Meyer:2016oeg}.

  Since $F(z(Q^2))$ is analytic by definition of $z$, and $z$ can be
  constrained to be less than unity by choice of $t_0$ and the finite range of
  $Q^2$ for a given experiment, the Taylor series of $F(z(Q^2))$ around zero
  will converge to the true $F(Q^2)$
  \begin{equation}
    F(Q^2) = \sum_{k=0}^{\infty}a_{k}z(Q^2)^k \,,
  \end{equation}
  where $a_k$ are dimensionless parameters that encode the nucleon structure.

  Physical quantities can still be extracted from the general form of the form
  factors. For example, the electric charge radius is still defined by the
  slope of the electric form factor at $Q^2 = 0$ and the axial mass can be
  redefined by the slope of the axial form factor at $Q^2 = 0$.  Importantly,
  the net contribution to the proton spin from the individual quark spins,
  $\Delta q$, is the value of the axial form factor at $Q^2 = 0$, and
  specifically, the net contribution to the spin of the proton from the spin of
  the strange quarks in the nucleon is equal to the value of the strange axial
  form factor at $Q^2 = 0$.

  \subsubsection{Fits of $z$ Expansion Form Factors to Data}

  An approximation to the general form of the form factors can be made by
  fitting the coefficients, $a_k$, out to a value $k_{max}$
  \begin{equation}
    F(Q^2) \approx \sum_{k=0}^{k_{max}} a_{k}z(Q^2)^k \,.
  \end{equation}
  Obviously, the larger $k_{max}$ is, the better the approximation will be. The
  limit of $k_{max}$ is generally determined by the experimental data.

  According to asymptotic scaling predictions in QCD~\cite{Lepage:1980fj}, the
  vector and axial form factors should have a $1/Q^4$ behavior at large values
  of $Q^2$. This can be encoded in the $z$ expansion parameterization by
  enforcing four sum rules~\cite{Lee:2015jqa}
  \begin{align}\label{eq:zsumrules}
    \frac{d^n}{dz^n} F\Big\rvert_{z=1} = 0\,, \hspace{5mm} n=0,1,2,3 \,.
  \end{align}
  Enforcing these sum rules means that there are fewer than $k_{max}$ free
  parameters in the z expansion fit.

  The optimal value for $t_0$ can be chosen to minimize the maximum size of
  $|z|$~\cite{Meyer:2016oeg}
  \begin{equation}\label{eq:topt}
    t_0^{\textrm{optimal}}(Q^2) = t_{cut}\Big(1 - \sqrt{1+Q^2_{\textrm{max}}/t_{cut}}\Big) \,,
  \end{equation}
  where $Q^2_{\textrm{max}}$ is the maximum $Q^2$ in the data being fit to.

  Fits of the electric and magnetic form factor $z$ expansion coefficients to
  data have been performed
  in~\cite{Hill:2010yb,Epstein:2014zua,Lee:2015jqa,Ye:2017gyb}.  We use the
  recent fit to electron scattering data performed in~\cite{Ye:2017gyb}. In it,
  the proton electric and magnetic form factors, $G_E^p(Q^2)$ and $G_M^p(Q^2)$,
  are fit simultaneously up to $k_{max} = 12$ (seven free parameters) using a
  previous unpolarized electron-proton scattering data and $G_E^p/G_m^p$ ratios
  extracted from polarized electron-proton data. The neutron electric and
  magnetic form factors, $G_E^n(Q^2)$ and $G_M^n(Q^2)$, are fit separately to
  up $k_{max} = 10$ (five free parameters) using polarized and unpolarized
  electron-deuterium and electron-helium-3 scattering data.

  Fits to the charged current axial form factor $z$ expansion coefficients to
  neutrino data have been performed
  in~\cite{Bhattacharya:2011ah,Bhattacharya:2015mpa,Meyer:2016oeg} We use the
  fit to deuterium bubble chamber neutrino data performed
  in~\cite{Meyer:2016oeg}. The form factor, $G_A(Q^2)$, is fit up to $k_{max} =
  8$ (four free parameters) using accelerator neutrino-deuterium data from the
  deuterium bubble chamber experiments at Argonne National Lab
  (ANL)~\cite{Mann:1973pr,Barish:1977qk,Miller:1982qi},
  BNL~\cite{Baker:1981su}, and FNAL~\cite{Kitagaki:1983px}. They found results
  similar to a dipole form but with larger, more realistic uncertainty
  estimates.


  \subsubsection{$z$ Expansion Fit to the Neutral Current Axial Form Factor}\label{sec:ncaxial}

  There are no existing fits of the neutral current axial form factor to data
  using the $z$ expansion parameterization. In this analysis, we do a
  three-parameter fit of the strange part of the NC axial form factor to
  MicroBooNE neutral current elastic neutrino-proton scattering data and use
  the previous fit to the CC axial form factor in~\cite{Meyer:2016oeg} for the
  up and down quark spin contributions.
  \begin{equation}
    \begin{aligned}
    G_A^{NC}(Q^2) = G_A(Q^2) \hspace{2mm} &(\textrm{previous fit}) \\
                + G_A^s(Q^2) \hspace{2mm} &(\textrm{fit to MicroBooNE data}) \,.
    \end{aligned}
  \end{equation}

  The $Q^2$ range of the NC elastic data that we will use for the fit in
  MicroBooNE is from $0.1$~GeV$^2$ to $1.0$~GeV$^2$.  Because the MicroBooNE NC
  elastic proton data set has relatively low statistics (see
  Sec.~\ref{sec:effbg}), we do a fit to the data with only three free
  parameters. This corresponds to $k_{max} = 6$ with $a_i^s$ for $i=3,4,5,6$
  determined by $a_i^s$ for $i=1,2,3$, and the sum rules in
  Eq.~\ref{eq:zsumrules}. The strange axial form factor is written as
  \begin{equation}
    G_A^s(z) = a_0^s + a_1^s z + a_2^s z^2 
      + a_3^s z^3 + a_4^s z^4 + a_5^s z^5 + a_6^s z^6 \,,
  \end{equation}
  with $a_i^s$ ($i=0,1,2$) free parameters to fit to data, and
  \begin{align}\label{eq:coefficients}
    a_3^s &= - 20 a_0^s - 10 a_1^s - 4 a_2^s \,,\\
    a_4^s &= + 45 a_0^s + 20 a_1^s + 6 a_2^s \,, \\
    a_5^s &= - 36 a_0^s - 15 a_1^s - 4 a_2^s \,, \\
    a_6^s &= + 10 a_0^s + 4 a_1^s + a_2^s \,.
  \end{align}
  where $a_i^s$ are the coefficients of the strange axial form factor that we
  are fitting to data, and $z \equiv z(Q^2)$ from Eq.~\ref{eq:zdef}.

  The net contribution of the strange quark spin to the spin of the proton,
  $\Delta s$, is defined as the value of the strange axial form factor at $Q^2
  = 0$,
  \begin{equation}
    \Delta s = a_0^s + a_1^s z_0 + a_2^s z_0^2 
      + a_3^s z_0^3 + a_4^s z_0^4 + a_5^s z_0^5 + a_6^s z_0^6 \,,
  \end{equation}
  where
  \begin{equation}
    z_0 = z(Q^2 = 0,t_{cut},t_0) = \frac{\sqrt{t_{cut}} - \sqrt{t_{cut} - t_0}}{\sqrt{t_{cut}} + \sqrt{t_{cut} - t_0}} \,.
  \end{equation}

  The specific fitting procedure and parameters are described in
  Sec.~\ref{sec:analysis}.


%This is the end of nu-N cross sections section

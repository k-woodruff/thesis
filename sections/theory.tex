\section{Neutrino-Nucleon interactions} \label{sec:theory}
%The next line produces an indented paragraph to start the document
 %unit.  The LaTeX defaults start most units without indentations.
\hspace{\parindent}

%%%%%%%%%%%%%%%%%%%%%%%%%%%%%%%%%%%%%%%%%%%%%%%%%%%%%%%%%%%
% Particle interactions
%%%%%%%%%%%%%%%%%%%%%%%%%%%%%%%%%%%%%%%%%%%%%%%%%%%%%%%%%%%
\subsection{Particle interactions}
  The differential cross section for two-particle scattering is given by
  \begin{equation}\label{eq:twobodyxsec}
    d\sigma = \frac{(2\pi)^4 |\mathcal{M}|^2}{4 \sqrt{(p_1\cdot p_2)^2 - m_1^2m_2^2}}
      \cross d\Phi_2(p_1+p_2;p_3,p_4) \,,
  \end{equation}
  where $d\Phi_2(p_1+p_2;p_3,p_4)$ is an element of two-body phase space given by
  \begin{equation}\label{eq:twobodyphase}
      d\Phi_2(p_1+p_2;p_3,p_3) = \delta^4(p_1+p_2 - p_3-p_4)
        \frac{d^3\mathbf{p}_3}{(2\pi)^3 2E_3}\frac{d^3\mathbf{p}_4}{(2\pi)^3 2E_4} \,,
  \end{equation}
  and $\mathcal{M}$ is the scattering amplitude.
  Combining Eqns.~\ref{eq:twobodyxsec}~and~\ref{eq:twobodyphase} gives
  \begin{equation}
      d\sigma = \frac{|\mathcal{M}|^2}{64\pi^2}
        \frac{\delta^4(p_1+p_2-p_3-p_4)}{E_3E_4\sqrt{(p_1\cdot p_2)^2 - m_1^2m_2^2}}
        \, d\mathbf{p}_3d\mathbf{p}_4 \,.
  \end{equation}

  The transition amplitude is given by the matrix element of the interaction
  Hamiltonian between final and initial states: $M_{fi} = \mel{f}{H_{int}}{i}$
    \\
  What are currents?

%%%%%%%%%%%%%%%%%%%%%%%%%%%%%%%%%%%%%%%%%%%%%%%%%%%%%%%%%%%
% Electroweak Interactions
%%%%%%%%%%%%%%%%%%%%%%%%%%%%%%%%%%%%%%%%%%%%%%%%%%%%%%%%%%%
\subsection{Electroweak Interactions}

  The charged current (CC) and neutral current (NC) interaction Lagrangians are given by
  \begin{align}
      \mathcal{L}_{CC} &= - \frac{g}{2\sqrt{2}}\, j^{\mu}_{CC}\, W_{\mu} + (\textrm{Hermition conjugate}) \\
      \mathcal{L}_{NC} &= - \frac{g}{2cos\theta_W}\, j^{\mu}_{NC}\, Z_{\mu}
  \end{align}
  The charged current, $j^{\mu}_{CC}$, which corresponds to the exchange of a
  $W^{\pm}$ boson, and the neutral current, $j^{\mu}_{NC}$, which corresponds
  to the exchange of the $Z^0$ boson are given by
  \begin{align}\label{eq:ccurrent}
      j^{\mu}_{CC} &= \sum_f \bar{\psi}_f \gamma^{\mu} (1-\gamma_5) \frac{1}{2}(\tau_1 + i\tau_2) \psi_f \\
        \label{eq:ncurrent}
      j^{\mu}_{NC} &= \sum_f \bar{\psi}_f \gamma^{\mu} (1-\gamma_5) \frac{1}{2}(\tau_3) \psi_f 
       - 2\textrm{sin}^2(\theta_W) j^{\mu}_{em}
  \end{align}
  where $j^{\mu}_{em}$ is the electromagnetic current, $\psi_{f}$ are the weak
  isospin doublets, and $\tau_i$ are the Pauli matrices
  \begin{equation}
      \tau_1 = 
      \begin{pmatrix}
        0 & 1 \\
        1 & 0
      \end{pmatrix} \,,
      \hspace{5mm}
      \tau_2 = 
      \begin{pmatrix}
        0 & -i \\
        i & 0
      \end{pmatrix} \,,
      \hspace{5mm}
      \tau_3 = 
      \begin{pmatrix}
        1 & 0 \\
        0 & -1
      \end{pmatrix} \,.
  \end{equation}
  The lepton weak isospin doublets are
  \begin{equation}
    \psi_{e} = 
    \begin{pmatrix}
        \hat{\nu}_{e} \\
        \hat{e}^-
    \end{pmatrix} \,,
      \hspace{5mm}
    \psi_{\mu} = 
    \begin{pmatrix}
        \hat{\nu}_{\mu} \\
        \hat{\mu}^-
    \end{pmatrix} \,,
      \hspace{5mm}
    \psi_{\tau} = 
    \begin{pmatrix}
        \hat{\nu}_{\tau} \\
        \hat{\tau}^-
    \end{pmatrix} \,,
  \end{equation}
  and the quark weak isospin doublets are
  \begin{equation}
    \psi_{1} = 
    \begin{pmatrix}
        \hat{u} \\
        \hat{d'}
    \end{pmatrix} \,,
      \hspace{5mm}
    \psi_{2} = 
    \begin{pmatrix}
        \hat{c} \\
        \hat{s'}
    \end{pmatrix} \,,
      \hspace{5mm}
    \psi_{3} = 
    \begin{pmatrix}
        \hat{t} \\
        \hat{b'}
    \end{pmatrix} \,,
  \end{equation}
  where $d'$, $s'$, and $b'$ represent the ``mixed" states
  \begin{equation}
      \begin{pmatrix}
        \hat{d}' \\
        \hat{s}' \\
        \hat{b}'
      \end{pmatrix}
      =
      \begin{pmatrix}
          V_{ud} & V_{us} & V_{ub} \\
          V_{cd} & V_{cs} & V_{cb} \\
          V_{td} & V_{ts} & V_{tb}
      \end{pmatrix}
      \begin{pmatrix}
        \hat{d} \\
        \hat{s} \\
        \hat{b}
      \end{pmatrix}
  \end{equation}
  where $V$ is the Cabibbo-Kobayashi-Maskawa matrix. These doublets contain the
  allowed weak transitions.

  \subsubsection{The charged current}
  The combination of Pauli matrices in the charged current
  \begin{equation}
      \frac{1}{2}\tau_+ = \frac{1}{2}(\tau_1 + \tau_2) \,,
  \end{equation}
  acts as an ``isospin raising matrix" and corresponds to the exchange of a
  $W^+$ boson.
  For the leptons, this gives
  \begin{equation}
      j^{\mu}_{CC}(\textrm{leptons}) = \sum_{l=e,\mu,\tau}
      \begin{pmatrix}
          \bar{\hat{\nu}}_l \\
          \bar{\hat{l}}
      \end{pmatrix}
      \gamma^{\mu}(1-\gamma_5)\frac{1}{2}
      \begin{pmatrix}
        0 & 1 \\
        0 & 0
      \end{pmatrix}
      \begin{pmatrix}
        \hat{\nu}_l \\
          \hat{l}
      \end{pmatrix}
      = \sum_{l=e,\mu,\tau} \bar{\hat{\nu}}_{l} \gamma^{\mu}\, \hat{l} \,,
  \end{equation}
  and, similarly, for the quarks we get
  \begin{equation}
      j^{\mu}_{CC}(\textrm{quarks}) =
       \bar{\hat{u}}\gamma^{\mu}(1-\gamma_5)\frac{1}{2}\,\hat{d}'
       +\bar{\hat{c}}\gamma^{\mu}(1-\gamma_5)\frac{1}{2}\,\hat{s}'
       +\bar{\hat{t}}\gamma^{\mu}(1-\gamma_5)\frac{1}{2}\,\hat{b}' \,,
  \end{equation}
  with the total charged current being $j^{\mu}_{CC} =
  j^{\mu}_{CC}(\textrm{leptons}) + j^{\mu}_{CC}(\textrm{quarks})$.

  \subsubsection{The neutral current}
  In neutral current scattering, $\frac{1}{2}\tau_3$ gives the weak isospin
  which acts as a ``weak charge". The electromagnetic current is given by
  \begin{equation}
      j^{\mu}_{em} = \sum_f e_f \bar{\hat{f}} \gamma^{\mu} \hat{f}
  \end{equation}
  where $f$ is a fermions, and $e_f$ is the electric charge of $f$.
  So, the total neutral current is
  \begin{equation}
    \begin{aligned}
        j^{\mu}_{NC} &= \sum_{l=e,\mu,\tau} \left(\bar{\hat{\nu}}_{l}
        \gamma^{\mu}(1-\gamma_5) \frac{1}{2}\, \hat{\nu}_{l} - \bar{\hat{l}}
        \gamma^{\mu}(1-\gamma_5) \frac{1}{2}\, \hat{l} 
        +\textrm{sin}^2\theta_W \bar{\hat{l}}\gamma^{\mu}\hat{l} \right) \\
        &+ \sum_{q=u,c,t} \left(\bar{\hat{q}} \gamma^{\mu}(1-\gamma_5)\frac{1}{2}\hat{q} 
        - \textrm{sin}^2\theta_W \frac{2}{3} \bar{\hat{q}}\gamma^{mu}\hat{q} \right) \\
        &+ \sum_{q=d,s,b} \left(- \bar{\hat{q}} \gamma^{\mu}(1-\gamma_5)\frac{1}{2}\hat{q} 
        + \textrm{sin}^2\theta_W \frac{1}{3} \bar{\hat{q}}\gamma^{mu}\hat{q} \right) \,.
     \end{aligned}
  \end{equation}
 
  \subsubsection{V-A structure}

  %%% starting text %%%
  From Alberico, starting with 2.3: \\
  It's convenient to separate the contributions from the light ($u,d$) and
  heavy ($s,c,...$) quarks. We get
  \[
    \j_{\alpha}^{NC;q} = v_{\alpha}^3 - a_{\alpha}^3 - \frac{1}{2}(v_{\alpha}^s - a_{\alpha}^s) - 2\mathrm{sin}^2\theta_W j_{\alpha}^{em} \,,
  \]
  where we define
  \begin{align*}
      v_{\alpha}^3 &= \bar{u}\gamma_{\alpha}\frac{1}{2}u - \bar{d}\gamma_{\alpha}\frac{1}{2}d \equiv \bar{N} \gamma_{\alpha}\frac{1}{2}\tau_3 N \\
      a_{\alpha}^3 &= \bar{u}\gamma_{\alpha}\gamma_5\frac{1}{2}u - \bar{d}\gamma_{\alpha}\gamma_5\frac{1}{2}d \equiv \bar{N} \gamma_{\alpha}\gamma_5\frac{1}{2}\tau_3 N \,.
  \end{align*}
  where $N = \left(\begin{matrix}{u}\\{d}\end{matrix} \right)$ is the isotopic
  SU(2) group doublet and the currents $v_{\alpha}^3$ and $a_{\alpha}^3$
  are the third components of the isovectors
  \begin{align*}
      v_{\alpha}^i &= \bar{N} \gamma_{\alpha}\frac{1}{2}\tau^i N \\
      a_{\alpha}^i &= \bar{N} \gamma_{\alpha}\gamma_5\frac{1}{2}\tau^i N \,.
  \end{align*}
  On the other hand, $v_{\alpha}^s$ and $a_{\alpha}^s$ are isoscalars. They
  represent the heavier quark contirbution to $j_{\alpha}^{NC;q}$. If we only
  include the strange quark, we have
  \begin{align*}
      v_{\alpha}^s &= \bar{s}\gamma_{\alpha}s \\
      a_{\alpha}^s &= \bar{s}\gamma_{\alpha}\gamma_5 s \,.
  \end{align*}
  The quark part of the electromagnetic current is
  \[
    j_{\alpha}^{em;q} = \sum_{q=u,d,...} e_q\bar{q} \gamma_{\alpha}q \,,
  \]
  which we can also separate between light and heavy contributions
  \[
    j_{\alpha}^{em;q} = v_{\alpha}^3 + v_{\alpha}^0 \,.
  \]
  where $v_{\alpha}^0$ is the isoscalar current which is given by
  \[
    v_{\alpha}^0 = \frac{1}{6}\bar{N}\gamma_{\alpha}N + \left(-\frac{1}{3}\right)\bar{s}\gamma_{\alpha}s \,,
  \]
  if we ignore charm and up.

  Time for one-nucleon matrix elements! 

  %%% ending text %%%

%%%%%%%%%%%%%%%%%%%%%%%%%%%%%%%%%%%%%%%%%%%%%%%%%%%%%%%%%%%
% Nucleon Form Factors
%%%%%%%%%%%%%%%%%%%%%%%%%%%%%%%%%%%%%%%%%%%%%%%%%%%%%%%%%%%
\subsection{Nucleon Form Factors}
  Talk about how to parameterize the nucleon in terms of the form factors
  Discuss Dirac, Pauli, electric, and magnetic.
  %%% starting text %%%
  From Thomas, we derive elastic lepton-nucleon scattering starting with
  conservation of energy and momentum. \\
  Section 2.1: \\
  Elastic scattering means that the nucleon remains in it's ground state so
  that the energy transfer is
  \[
      \nu = \epsilon - \epsilon' = E' - E,
   \]
  where $\epsilon$ is energy of incident electron, E is energy of the nucleon.
  Also, three-momentum transfer is
  \[
      \bar{q} = \bar{p} - \bar{p}' = \bar{P}' - \bar{P} \,.
  \]
  So, squared four-momentum is
  \[
      q^2 = \nu^2 - \bar{q}^2 = -Q^2 < 0
  \]
  which is Lorentz invariant.
  From energy and momentum conservation, we get
  \[
      \epsilon' = \frac{\epsilon}{1+\frac{2\epsilon}{M}\mathrm{sin}^2\frac{\theta}{2}}
  \]
  and
  \[
      Q^2 = 4\epsilon \epsilon' \mathrm{sin}^2 \frac{\theta}{2} \,.
  \]
  This leads to the Mott differential cross section (pointlike, spinless)
  \[
      \left(\frac{d\sigma}{d\Omega} \right)_\textrm{Mott} = \frac{\alpha^2}{4\epsilon^2 \textrm{sin}^4\frac{\theta}{2}}\frac{\epsilon'}{\epsilon}\textrm{cos}^2\frac{\theta}{2} \,.
  \]
  If we make it spin 1/2 with a normal (Dirac) magnetic moment (still pointlike), we get an additional term (increases backwards angles):
  \[
      \frac{d\sigma}{d\Omega} = \left(\frac{d\sigma}{d\Omega} \right)_\textrm{Mott} \left[ 1 + \frac{Q^2}{4M^2}2\mathrm{tan}^2\frac{\theta}{2} \right] \,.
  \]
  Now we just need to add finite structure and anomolous magnetic moment,
  \[
      \frac{d\sigma}{d\Omega} = \left(\frac{d\sigma}{d\Omega} \right)_\textrm{Mott} 
      \left(F_1^2(Q^2) +\frac{Q^2}{4M}\left[F_2^2(Q^2)+2(F_1(Q^2)+F_2(Q^2))^2\,\textrm{tan}^2\frac{\theta}{2} \right] \right) \,.
  \]
  The Dirac and Pauli form factors, $F_1$ and $F_2$ respectively, can be parameterized as the electric and magnetic, or Sachs, form factors
  \begin{align}
      G_E(Q^2) &= F_1(Q^2) - \frac{Q^2}{4M}F_2(Q^2) \\
      G_M(Q^2) &= F_1(Q^2) + F_2(Q^2) \,.
  \end{align}
  Low $Q^2$ data. Sachs form factors when $Q^2$ goes to zero. Three dimensional fourier integral and slopes go to mean squared radii.

  \subsubsection{The Axial Form Factor}
    Derive the axial form factor. Discuss different axial form factor models
    (dipole...). Show that you can determine $\Delta s$. Make explicit that
    this is the same $\Delta s$ from first subsubsection.

%%%%%%%%%%%%%%%%%%%%%%%%%%%%%%%%%%%%%%%%%%%%%%%%%%%%%%%%%%%
% Spin Structure of Nucleons
%%%%%%%%%%%%%%%%%%%%%%%%%%%%%%%%%%%%%%%%%%%%%%%%%%%%%%%%%%%
\subsection{The Spin Structure of Nucleons} \label{sec:nuctheory}
  Proton spin: \\
  --- spin vector $s_{\mu}$ from forward matrix element of axial vector current \\
  --- derive axial charges \\
  From Bass 1.1 (need to fill in with Peskin): \\
  Forward matrix element of the axial current vector (derive from peskin):
  \[
      2Ms_{\mu} = <p,s|\bar{\psi}\gamma_{\mu} \gamma_{5} \psi|p,s>
  \]
  where $s_{\mu}$ is the proton's spin vector, $\psi$ is the proton field
  vector and $M$ is the proton mass. The quark axial charges measure
  information about the quark ``spin content".
  \[
    2Ms_{\mu}\Delta q = <p,s| \bar{q}\gamma_{\mu}\gamma_{5}q|p,s> \,,
  \]
  where $q$ is the quark field operator and $\Delta q$ is the quark
  flavor-dependent axial charge, $\Delta u$, $\Delta d$, or $\Delta s$. The
  isovector, SU(3) octet, and flavor-singlet axial charges, $g_A^{(3)}$,
  $g_A^{(8)}$, and $g_A^{(0)}$, respectively, can be written as linear
  combinations of the quark axial charges (derive from Peskin)
  \begin{align}
      g_A^{(3)} &= \Delta u - \Delta d \\
      g_A^{(8)} &= \Delta u + \Delta d - 2\Delta s \\
      g_A^{(0)} &= \Delta u + \Delta d + \Delta s \,.
  \end{align}
  Can be interpreted semi-classically as amount of spin carried by quarks and
  antiquarks of flavor $q$.

  --- expectation of axial charges from non-relativistic quark model

%%%%%%%%%%%%%%%%%%%%%%%%%%%%%%%%%%%%%%%%%%%%%%%%%%%%%%%%%%%
% Neutrino-proton elastic cross section
%%%%%%%%%%%%%%%%%%%%%%%%%%%%%%%%%%%%%%%%%%%%%%%%%%%%%%%%%%%
\subsection{Neutrino-proton elastic cross section}\label{sec:probe}
  \subsubsection{Cross section model}

  \subsubsection{Previous measurements}
    E734

%This is the end of nu-N cross sections section

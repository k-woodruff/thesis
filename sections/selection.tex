\section{Particle Identification and Event Selection}\label{protonid}
%The next line produces an indented paragraph to start the document
 %unit.  The LaTeX defaults start most units without indentations.
\hspace{\parindent}
Proton ID and event selection.

%%%%%%%%%%%%%%%%%%%%%%%%%%%%%%%%%%%%%%%%%%%%%%%%%%%%%%%%%%%
% Particle Identification and Event Selection
%%%%%%%%%%%%%%%%%%%%%%%%%%%%%%%%%%%%%%%%%%%%%%%%%%%%%%%%%%%
\subsection{Particle Identification}
  After the particle tracks are reconstructed, we use a predictive model to
  classify proton tracks. The inputs to the model are the reconstructed
  physical variables, and the output is the probability that the track is from
  a proton vs. some other particle. There are many predictive models that we
  can use, each with advantages and disadvantages. We chose gradient-boosted
  decision trees for a few main reasons: they are easily interpretable, the
  inputs can be a mix of numeric and categorical variables, and boosted
  decision trees perform well at identifying a small signal in a large
  background.  Each tree is essentially a series of cuts based on physical
  variables which have been fine-tuned to increase the efficiency and purity of
  the final selected sample.
  \subsubsection{Reconstructed track features}\label{sec:features}
    The reconstructed features that are used as input to the classifier are
    listed below. Most of the features come directly from the track object, but
    some are created for this classifier. Each of the features used to identify
    protons either helps to separate neutrino-induced tracks from
    cosmic-induced tracks or to separate neutrino-induced proton tracks from
    other neutrino-induced particle types. For colorimetric information we only
    use information from the collection plane.

    Below is a list and description of the features designed to separate
    neutrino-induced protons from other neutrino-induced particle types.
    \begin{itemize}
      \item \textbf{Number of hits:} This is the total number of hits on the
      collection plane that are associated with track. When used in combination
      with track length and average energy deposited, this feature can be used
      to determine the hit and energy density of the track.
      \item \textbf{Straightness:} This is the ratio of distance between
      reconstructed end points (displacement) to reconstructed path length. It
      represents the amount of scattering a track undergoes. The value is
      always between zero and one with one being perfectly straight.
      \item \textbf{Cosmic score:} This is the geometry tagging cosmic score
      from Sec.~\ref{sec:tpcreco}. Tracks with a cosmic score of 1 have already
      been removed in the cosmic hit removal stage. So, this value is either 0
      (fully contained within the TPC) or 0.5 (entering or exiting the TPC).
      \item \textbf{Length:} This is the reconstructed 3D track length found by
      stepping along the trajectory points.
      \item \textbf{Start dE/dx:} This is the total energy deposited on the
      collection plane in the first six non-zero hits along the track divided
      by the distance between hits to account for the angle with respect to the
      wire plane.
      \item \textbf{End to start dE/dx ratio:} This is the ratio of the total
      $dE/dx$ from the last six non-zero hits along the track on the collection
      plane to the total $dE/dx$ from the first six non-zero hits along the
      track on the collection plane.
      \item \textbf{Truncated total dE/dx:} This is the sum of the $dE/dx$ of a
      truncated set of hits on the collection plane associated with track. The
      truncated set includes all hits along the track with a $dE/dx$ value
      within one standard deviation of the median $dE/dx$ value of all hits
      along the track on the collection plane.
      \item \textbf{Truncated average dE/dx:} This is the truncated total
      $dE/dx$ divided by the number of hits in the truncated hit set associated
      with track.
    \end{itemize}

    Next is the list and description of the features designed to separate
    neutrino-induced tracks from cosmic-induced tracks.
    \begin{itemize}
      \item \textbf{Start and end positions:} These are the reconstructed x, y,
      and z positions of start and end of the track. Tracks that start closer
      to a TPC boundary are more likely to be cosmic-induced.
      \item \textbf{$\theta$ and $\phi$:} These are the reconstructed polar and
      azimuthal angles with respect to the beam direction. Vertical tracks are
      much more likely to be cosmic-induced, while forward-going tracks are
      more likely to be from the neutrino beam.
    \end{itemize}
   
    Determining which end of a track is the beginning is difficult when a
    vertex is not observable. Since we are particularly interested in
    neutral-current elastic events with only a single proton, the direction of
    the track is a concern. A proton will deposit much more energy at the end
    of its track than at the beginning which can be used to determine the true
    direction. Since this correction is not currently implemented within the
    reconstruction, we take all reconstructed tracks that have a higher
    deposited energy at the beginning of the track than at the end of the track
    and flip them. The deposited energy at the beginning (ending) of the track
    is defined as the total $dE/dx$ of the first (last) six non-zero hits along
    the track on the collection plane.  This includes changing the saved start
    positions, end positions, $\theta$, $\phi$, start $dE/dx$, end $dE/dx$, and
    the end to start $dE/dx$ ratio.
    
  \subsubsection{Boosted decision trees}\label{sec:decisiontrees}
    A decision tree can be thought of as a series of if/else statements that
    separate a data set into two or more classes as illustrated in
    Fig.~\ref{fig:dtree}. At each node of the tree, a split is chosen to
    maximize information gain until a set level of separation is reached.  At
    the terminus of the series of splits, called a leaf, a class is assigned.
    The usual parameters that can be set when creating a decision tree are: the
    maximum depth of the tree (how many layers of nodes you will allow), the
    minimum split size (how many data points do you require to keep splitting),
    and minimum leaf size (how small does a leaf have to be before you stop). 
    \begin{figure}[ht]
      \centering
      \includegraphics[angle=0,width=4in]{figures/analysis/protonid/trees_diagram.pdf}
      \caption{Graphical example of a decision tree.}
      \label{fig:dtree}
    \end{figure}
    
    A single tree can easily overfit a data set if it is at all complex, and
    its output is just a class label. Gradient-boosting addresses both of these
    issues by combining many weak classifiers into a strong one. Each weak
    classifier is built based on the error of the previous one. For a given
    training set, whenever a sample is classified incorrectly by a tree, that
    sample is given a higher importance when the next tree is being created.
    Mathematically, each tree is training on the gradient of the loss function.
    After all of the trees have been created, each tree is given a weight based
    on its ability to classify the training set, and the output of the
    gradient-boosted decision tree classifier is the probability that a sample
    is in a given class.
    
    The gradient-boosted decision tree software package we use is
    XGBoost~\cite{Chen:2016btl}. There are two types of classifiers we can use
    to separate protons from other tracks: binary and multiclass. Both
    classifiers are trained on all types of reconstructed tracks. A binary
    classifier classifies each track as either a proton or not a proton, and a
    multiclass classifier classifies a track as one of many types including a
    proton. We choose to use multiclass because the information about
    non-proton tracks is useful for selecting neutral current events. The five
    classes that we train the decision trees to classify are protons (both
    neutrino-induced and cosmic), neutrino-induced muons, neutrino-induced
    pions, neutrino-induced electrons/photons, and all non-proton cosmics.
    
  \subsubsection{Training}
    \begin{table}[ht]
      \caption{Breakdown by simulated particle type reconstructed tracks in the
        gradient-boosted decision tree training set.
      \label{tab:mctrain}}
      \begin{tabularx}{\textwidth}{ l r r r r r }
        \hline
        & Protons & Muons & Pions & EM Showers & Non-proton Cosmics \\
        \hline
        No. of tracks  & 90,922 & 57,583 & 12,848 & 473,323 & 2,586,527 \\
        Fraction of set & 0.028 & 0.018 & 0.004 & 0.147 & 0.803 \\
        Class weight  & 0.141 & 0.223 & 1.000 & 0.027 & 0.005 \\
        \hline
      \end{tabularx}
    \end{table}

    The gradient-boosted decision tree model was trained on 95,600 events with
    both simulated GENIE neutrino interactions and simulated CORSIKA cosmic
    interactions. Each track in every event was treated as a separate training
    sample. Table~\ref{tab:mctrain} shows the number of each type of track that
    was used for training. There are were a total of 3,221,203 simulated
    training tracks.

    Because the training set has unbalanced classes (there are different
    numbers of each particle type) each training sample is initially weighted
    so that the sum of weights is equal to the size of the smallest class, in
    this case pions.
    \begin{equation*}
      N_s = \sum_{i=1}^{N_{n}}w^n_i \,,
    \end{equation*}
    where $N_s$ is the number of samples in the smallest class, $N_{n}$ is the
    number samples in the $n^{th}$ class, and $w^n_i$ is the weight given to
    the $ith$ sample in that class. The same weight is used for each sample in
    a class, so the value of each positive weight is $w^n=\frac{N_s}{N_{n}}$.
    Balancing the training set prevents the classifier from only learning the
    most frequent classes. In our case, the classifier could achieve a high
    accuracy by classifying everything as a cosmic in the unbalanced set
    because over 80\% of the tracks are cosmic-induced. One of our main goals
    is to have a proton ID efficiency, and since protons only make up 3\% of
    the training set, giving them a higher weight makes it more important to
    the classifier that they are correctly classified.

    The parameters used for training were chosen to both maximize
    classification accuracy and minimize overfitting to the training set.
    Overfitting occurs when the performance on the training set is more
    accurate than the performance on an external test set. The final training
    parameter settings are:
    \begin{itemize}
      \item \textbf{Objective: multiclass: softprob} \\
      The learning objective. We want to classify five different track types
      and get a probability of each class.  
      \item \textbf{Learning rate: 0.045}  \\
      The factor each incorrectly classified sample gets re-weighted by for the
      next tree.  A smaller learning rate requires more trees but prevents
      overfitting.
      \item \textbf{Number of trees: 500} \\
      The total number of trees in classifier.
      \item \textbf{Maximum depth: 10} \\ 
      The maximum number of layers of nodes each tree can have.
      \item \textbf{Maximum sampled features: 0.8} \\
      The fraction of total features that each tree can use to train. These are
      randomly sampled.
      \item \textbf{Maximum sampled observations: 0.85} \\
      The fraction of total samples that each tree can use to train. These are
      randomly sampled.
    \end{itemize}

  \subsubsection{Performance on a Test Set}\label{sec:protonidtest}
    The performance of the gradient-boosted decision tree classifier was tested
    on a set of 3,200,000 reconstructed tracks from 96,200 events with
    simulated GENIE neutrino interactions and simulated CORSIKA cosmic
    interactions. This set of tracks was generated in the exact same way as the
    training set.

    \begin{figure}[ht]
      \centering
      \includegraphics[angle=0,width=5in]{figures/analysis/protonid/protonid_mc_output_norm.pdf}
      \caption{Area-normalized histograms of decision tree proton
      identification scores for simulated protons and other simulated proton
      tracks.}
      \label{fig:pidmcout}
    \end{figure}
    \begin{figure}[ht]
      \centering
      \includegraphics[angle=0,width=5in]{figures/analysis/protonid/protonid_mc_output_norm_ncelastic.pdf}
      \caption{Area-normalized histogram of decision tree proton identification
      scores for simulated proton tracks from NC elastic proton interactions.}
      \label{fig:pidmcoutNCE}
    \end{figure}
    Figure~\ref{fig:pidmcout} shows normalized histograms of the output proton
    score for every track in the test set. The proton score ranges from zero to
    one with zero being the least proton-like and one being the most. The blue
    histogram shows all simulated neutrino-induced and cosmic induced proton
    tracks normalized so that the area under the histogram is one. The orange
    histogram shows every other simulated track type, also normalized so that
    the area under it is equal to one. Figure~\ref{fig:pidmcoutNCE} shows the
    area-normalized histogram of proton scores for simulated proton tracks that
    were produced in neutral current elastic proton events.

    \begin{figure}[ht]
      \centering
      \includegraphics[angle=0,width=5.5in]{figures/analysis/protonid/heatmap_mcc87_pmtrack_final.pdf}
      \caption{Heatmap showing the fraction of each class that is made up of a
      given particle type.}
      \label{fig:heatmap}
    \end{figure}
    Figure~\ref{fig:heatmap} shows the overall classification performance of
    the gradient-boosted decision tree model on the test set for each class.
    The x axis shows the true particle type and the y axis shows the particle
    classes. The numbers in the boxes are the fraction of the class that is
    made up of the given true particle type. The fraction of true protons in
    the set of tracks classified as protons is 0.71, the fraction of true muons
    in that set is 0.04, the fraction of true pions is 0.09, the fraction of
    electromagnetic shower particles is 0.04, and the fraction of non-proton
    cosmics in the proton-classified set is 0.05. A track is labelled as a
    given class type in this plot if the particle's decision tree score for
    that class is higher than its score for any of the other four classes.  The
    numbers in this plot were calculated using equal numbers of each true
    particle type. In reality, there are far more non-proton cosmic tracks than
    there are true protons, and the fraction of true protons in the set
    classified as protons will be smaller.

    \begin{figure}[h]
      \centering
      \begin{subfigure}[t]{2.5in}
        \includegraphics[angle=0,width=2.5in]{figures/analysis/protonid/PID_efficiency_allmcke.pdf}
        \caption{The full simulated kinetic energy range.}
        \label{fig:pideffkeall}
      \end{subfigure}
      \hspace{2pt}
      \begin{subfigure}[t]{2.5in}
        \includegraphics[angle=0,width=2.5in]{figures/analysis/protonid/PID_efficiency_mcke.pdf}
        \caption{The kinetic energy range used in this analysis.}
        \label{fig:pideffkerng}
      \end{subfigure}
      \caption{The efficiency of simulated neutrino-induced proton tracks
        correctly classified as protons as a function of true proton kinetic energy.}
      \label{fig:pideffke}
    \end{figure}
    Figure~\ref{fig:pideffke} shows the efficiency of the decision tree proton
    identification on simulated neutrino-induced protons as a function of true
    proton kinetic energy. The left plot (\ref{fig:pideffkeall}) shows the full
    simulated range of true proton kinetic energy, and the right
    (\ref{fig:pideffkerng}) shows the range of interest to this analysis. In
    the interesting range of kinetic energies, the proton identification
    efficiency stays relatively flat between 0.8 and 0.8 efficiency, with an
    average efficiency of 0.71. A track is considered positively identified as
    a proton in these plots if its decision tree proton score is higher than
    0.5, meaning it is more likely than not to be a proton.

    \begin{figure}[h]
      \centering
      \begin{subfigure}[t]{2.5in}
        \includegraphics[angle=0,width=2.5in]{figures/analysis/protonid/PID_efficiency_mctheta_kerange.pdf}
        \caption{Efficiency as a function of the cosine of the true proton
          angle from the beam direction.}
        \label{fig:pideffangletheta}
      \end{subfigure}
      \hspace{2pt}
      \begin{subfigure}[t]{2.5in}
        \includegraphics[angle=0,width=2.5in]{figures/analysis/protonid/PID_efficiency_mcphi_kerange.pdf}
        \caption{Efficiency as a function of the true proton angle around the beam direction.}
        \label{fig:pideffanglephi}
      \end{subfigure}
      \caption{The efficiency of simulated neutrino-induced proton tracks
        correctly classified as protons as a function of true proton angle.}
      \label{fig:pideffangle}
    \end{figure}
    Figure~\ref{fig:pideffangle} shows the efficiency of the decision tree proton
    identification on simulated neutrino-induced protons as a function of true
    proton angle. The efficiencies in these plots are calculated using only the
    simulated protons within the kinetic energy range of interest ($0.05$ GeV
    $\le T_p \le 0.5$ GeV) used in Figure~\ref{fig:pideffkerng}. The left plot
    (\ref{fig:pideffangletheta}) shows the efficiency as a function of
    $\cos(\theta_p)$, where $\theta_p$ is the angle of the proton from the
    neutrino beam direction. At $\cos(theta_p) = 1$ the proton is parallel to
    the beam, at $\cos(\theta_p) = -1$ the proton is anti-parallel to the beam,
    and at $\cos(\theta_p) = 0$ the proton is perpendicular to the beam. When
    the proton is perpendicular to the beam, it is aligned with the anode
    collection plane, and will not traverse more than one collection plane
    wire. A large contribution to the decrease in efficiency at $\cos(\theta_p)
    = 0$ is the fact that the decision tree classifier only uses calorimetry
    information from the collection plane. The right
    plot(\ref{fig:pideffanglephi}) shows the efficiency as a function of
    $\phi_p$ which is the angle around the neutrino beam direction. The flat
    efficiency is due to the fact that the neutrino-induced proton angle should
    be isotropic in $\phi_p$, and the angle around the beam direction has no
    effect on the angle with respect to the angle of the anode wires. Again, a
    track is considered positively identified as a proton in these plots if its
    decision tree proton score is higher than 0.5.
    \begin{figure}[ht]
      \centering
      \includegraphics[angle=0,width=5.0in]{figures/analysis/protonid/PID_efficiency_mcthetavmcke.pdf}
      \caption{Two-dimensional efficiency for true proton $\cos(\theta_p)$
      versus true proton kinetic energy.}
      \label{fig:pideffthetake}
    \end{figure}
    Figure~\ref{fig:pideffthetake} shows the two-dimensional efficiency for
    true proton $\cos(\theta_p)$ versus true proton kinetic energy. The kinetic
    energy range of interest to this analysis goes up to $0.5$ GeV (the bottom
    half of the plot).

  \subsubsection{Performance on a Neutrino Data Subset}\label{sec:datamcpid}
    The gradient-boosted decision tree classifier was tested on a subset of
    MicroBooNE neutrino data corresponding to 5e19 POT ($< 5\%$ of the fill
    MicroBooNE approved POT). The data set is taken entirely from MicroBooNE's
    first year of running (Run I). The results of the classifier on the
    neutrino data subset was compared to the results on a combination of
    neutrino and cosmic simulation and off-beam data. Each of the samples is
    scaled to 5e19 POT. The samples used in these comparisons and the scaling
    factors are listed below. 
    \begin{enumerate}
      \item Run I 5e19 POT neutrino data subset
      \begin{itemize}
        \item Number of events: 171,603
        \item POT (\texttt{tor860\_wcut}): $4.41e19$
        \item Number of triggers (\texttt{E1DCNT\_wcut}): 9,779,224
        \item Normalization factor: 1
      \end{itemize}
      \item Run I off-beam data subset
      \begin{itemize}
        \item Number of events: 189,226
        \item Number of triggers (\texttt{EXT}): 14,579,406
        \item Normalization factor: 0.7063
      \end{itemize}
      \item Neutrino Monte Carlo with cosmic data overlay set
      \begin{itemize}
        \item Number of events: 200,160
        \item POT: $2.08e20$
        \item Flux correction factor: 1.029
        \item Normalization factor: 0.2429
      \end{itemize}
    \item Neutrino dirt Monte Carlo with simulated cosmic data set
      \begin{itemize}
        \item Number of events: 105214
        \item POT: $4.66e20$
        \item POT normalization factor: 0.0947
        \item Data driven normalization factor: 0.5 (described in Sec.~\ref{sec:dirt})
        \item Normalization factor: 0.0474
      \end{itemize}
    \end{enumerate}
    The number of triggers listed is the total number of beam spill triggers
    issued by the accelerator division and does not include the optical
    software trigger implemented in MicroBooNE.  This is how we scale the
    off-beam data to the neutrino beam data. The number of events is the number
    of events left after the optical pre-selection described in
    section~\ref{sec:optpresel}. The off-beam data is a direct measurement of
    the background to the neutrino beam data that is due to a cosmic
    interaction occurring in-time with the beam. The neutrino Monte Carlo
    simulates neutrino beam interactions that occur within the liquid argon
    cryostat, and the neutrino dirt Monte Carlo simulates the background
    neutrino beam interactions that occur outside of the cryostat.
    
    Figure~\ref{fig:pidscores} shows the comparison of the decision tree proton
    score between the subset of MicroBooNE neutrino data and the MicroBooNE
    simulation. The top left plot is in linear scale and the top right plot is
    in log scale. The bottom plots are the same, and they show the ratio
    between the on-beam neutrino data and the combination of neutrino
    simulation and off-beam data.
    \begin{figure}[h]
      \centering
      \begin{subfigure}[t]{2.5in}
        \includegraphics[angle=0,width=2.5in]{figures/analysis/protonid/datamcpid/pscore_overlay.pdf}
        \includegraphics[angle=0,width=2.5in]{figures/analysis/protonid/datamcpid/pscore_overlay-ratio.pdf}
        \caption{All reconstructed tracks.}
      \end{subfigure}
      \hspace{2pt}
      \begin{subfigure}[t]{2.5in}
        \includegraphics[angle=0,width=2.5in]{figures/analysis/protonid/datamcpid/pscore_overlay_log.pdf}
        \includegraphics[angle=0,width=2.5in]{figures/analysis/protonid/datamcpid/pscore_overlay-ratio.pdf}
        \caption{All reconstructed tracks that are classified as protons.}
      \end{subfigure}
      \caption{Comparison of the decision tree proton scores between a subset
      of MicroBooNE neutrino data and a combination of MicroBooNE neutrino
      simulation and off-beam data.}
      \label{fig:pidscores}
    \end{figure}

    In all of the figures in this section, the black points in the top plots
    show the subset of neutrino data.  The horizontal bars represent the bin
    width, and the vertical bars represent the statistical uncertainty. The
    light gray filled histogram includes tracks from the off-beam data. These
    tracks represent the background of events where a cosmic interaction in the
    detector coincident with the beam time window triggered the event, and
    there was no actual neutrino interaction. The dark gray filled histograms
    include cosmic tracks that are in the background of events with actual
    neutrino interactions that triggered the event. For simulated neutrino
    interactions inside the detector, real data cosmic tracks are overlaid on
    the simulated event, and for simulated neutrino interaction outside the
    detector, the background cosmic tracks are from simulation. The color
    filled histograms include tracks from simulated neutrino interactions. The
    peach colored histograms include simulated neutrino-induced proton tracks,
    the dark green includes simulated neutrino-induced pion tracks, the light
    green includes simulated neutrino-induced muon tracks, and the purple
    includes simulated neutrino-induced electromagnetic shower tracks. The
    fraction of proton tracks in the right plots (the tracks classified as
    protons) is much larger than in the left plots (all tracks), which is the
    goal of the classifier.  The bottom plots in all of the figures show the
    ratio of the neutrino data points to the sum of all of the stacked, filled
    histograms. A ratio of one means perfect data to simulation agreement.

    Figures~\ref{fig:pidnhits}-\ref{fig:pidphi} show comparisons between the
    subset of MicroBooNE neutrino data and the MicroBooNE simulation for each
    of the input variables being used in the decision tree classifier. A
    description of each of these reconstructed track features is given in
    Sec.~\ref{sec:features}. The left plots show the histogram of the given
    variable for all tracks being input to the classifier, and the right plots
    show the histograms of the given variable for the tracks that were
    classified as protons. A track is considered classified as a proton if the
    decision tree proton score is greater than 0.5 for that track.
    
    \begin{figure}[h]
      \centering
      \begin{subfigure}[t]{2.5in}
        \includegraphics[angle=0,width=2.5in]{figures/analysis/protonid/datamcpid/pid_nhits_all_truncated.pdf}
        \includegraphics[angle=0,width=2.5in]{figures/analysis/protonid/datamcpid/pid_nhits_all_truncated-ratio.pdf}
        \caption{All reconstructed tracks.}
      \end{subfigure}
      \hspace{2pt}
      \begin{subfigure}[t]{2.5in}
        \includegraphics[angle=0,width=2.5in]{figures/analysis/protonid/datamcpid/pid_nhits_pass_truncated.pdf}
        \includegraphics[angle=0,width=2.5in]{figures/analysis/protonid/datamcpid/pid_nhits_pass_truncated-ratio.pdf}
        \caption{All reconstructed tracks that are classified as protons.}
      \end{subfigure}
      \caption{Breakdown of the different particle track types in neutrino data
      and simulation as a function of the number of hits on the collection
      plane.}
      \label{fig:pidnhits}
    \end{figure}
    \begin{figure}[h]
      \centering
      \begin{subfigure}[t]{2.5in}
        \includegraphics[angle=0,width=2.5in]{figures/analysis/protonid/datamcpid/pid_distlenratio_all_truncated.pdf}
        \includegraphics[angle=0,width=2.5in]{figures/analysis/protonid/datamcpid/pid_distlenratio_all_truncated-ratio.pdf}
        \caption{All reconstructed tracks.}
      \end{subfigure}
      \hspace{2pt}
      \begin{subfigure}[t]{2.5in}
        \includegraphics[angle=0,width=2.5in]{figures/analysis/protonid/datamcpid/pid_distlenratio_pass_truncated.pdf}
        \includegraphics[angle=0,width=2.5in]{figures/analysis/protonid/datamcpid/pid_distlenratio_pass_truncated-ratio.pdf}
        \caption{All reconstructed tracks that are classified as protons.}
      \end{subfigure}
      \caption{Breakdown of the different particle track types in neutrino data
      and simulation as a function of the track straightness.}
      \label{fig:piddistlenratio}
    \end{figure}
    \begin{figure}[h]
      \centering
      \begin{subfigure}[t]{2.5in}
        \includegraphics[angle=0,width=2.5in]{figures/analysis/protonid/datamcpid/pid_length_all_truncated.pdf}
        \includegraphics[angle=0,width=2.5in]{figures/analysis/protonid/datamcpid/pid_length_all_truncated-ratio.pdf}
        \caption{All reconstructed tracks.}
      \end{subfigure}
      \hspace{2pt}
      \begin{subfigure}[t]{2.5in}
        \includegraphics[angle=0,width=2.5in]{figures/analysis/protonid/datamcpid/pid_length_pass_truncated.pdf}
        \includegraphics[angle=0,width=2.5in]{figures/analysis/protonid/datamcpid/pid_length_pass_truncated-ratio.pdf}
        \caption{All reconstructed tracks that are classified as protons.}
      \end{subfigure}
      \caption{Breakdown of the different particle track types in neutrino data
      and simulation as a function of the track length.}
      \label{fig:pidlength}
    \end{figure}
    \begin{figure}[h]
      \centering
      \begin{subfigure}[t]{2.5in}
        \includegraphics[angle=0,width=2.5in]{figures/analysis/protonid/datamcpid/pid_startdedx_all_truncated.pdf}
        \includegraphics[angle=0,width=2.5in]{figures/analysis/protonid/datamcpid/pid_startdedx_all_truncated-ratio.pdf}
        \caption{All reconstructed tracks.}
      \end{subfigure}
      \hspace{2pt}
      \begin{subfigure}[t]{2.5in}
        \includegraphics[angle=0,width=2.5in]{figures/analysis/protonid/datamcpid/pid_startdedx_pass_truncated.pdf}
        \includegraphics[angle=0,width=2.5in]{figures/analysis/protonid/datamcpid/pid_startdedx_pass_truncated-ratio.pdf}
        \caption{All reconstructed tracks that are classified as protons.}
      \end{subfigure}
      \caption{Breakdown of the different particle track types in neutrino data
      and simulation as a function of the track start $dE/dx$.}
      \label{fig:piddedx}
    \end{figure}
    \begin{figure}[h]
      \centering
      \begin{subfigure}[t]{2.5in}
        \includegraphics[angle=0,width=2.5in]{figures/analysis/protonid/datamcpid/pid_dedxratio_all_truncated.pdf}
        \includegraphics[angle=0,width=2.5in]{figures/analysis/protonid/datamcpid/pid_dedxratio_all_truncated-ratio.pdf}
        \caption{All reconstructed tracks.}
      \end{subfigure}
      \hspace{2pt}
      \begin{subfigure}[t]{2.5in}
        \includegraphics[angle=0,width=2.5in]{figures/analysis/protonid/datamcpid/pid_dedxratio_pass_truncated.pdf}
        \includegraphics[angle=0,width=2.5in]{figures/analysis/protonid/datamcpid/pid_dedxratio_pass_truncated-ratio.pdf}
        \caption{All reconstructed tracks that are classified as protons.}
      \end{subfigure}
      \caption{Breakdown of the different particle track types in neutrino data
      and simulation as a function of the end to start $dE/dx$ ratio.}
      \label{fig:piddedxratio}
    \end{figure}
    \begin{figure}[h]
      \centering
      \begin{subfigure}[t]{2.5in}
        \includegraphics[angle=0,width=2.5in]{figures/analysis/protonid/datamcpid/pid_trtotaldedx_all_truncated.pdf}
        \includegraphics[angle=0,width=2.5in]{figures/analysis/protonid/datamcpid/pid_trtotaldedx_all_truncated-ratio.pdf}
        \caption{All reconstructed tracks.}
      \end{subfigure}
      \hspace{2pt}
      \begin{subfigure}[t]{2.5in}
        \includegraphics[angle=0,width=2.5in]{figures/analysis/protonid/datamcpid/pid_trtotaldedx_pass_truncated.pdf}
        \includegraphics[angle=0,width=2.5in]{figures/analysis/protonid/datamcpid/pid_trtotaldedx_pass_truncated-ratio.pdf}
        \caption{All reconstructed tracks that are classified as protons.}
      \end{subfigure}
      \caption{Breakdown of the different particle track types in neutrino data
      and simulation as a function of the track truncated total $dE/dx$.}
      \label{fig:pidtrtotaldedx}
    \end{figure}
    \begin{figure}[h]
      \centering
      \begin{subfigure}[t]{2.5in}
        \includegraphics[angle=0,width=2.5in]{figures/analysis/protonid/datamcpid/pid_traveragededx_all_truncated.pdf}
        \includegraphics[angle=0,width=2.5in]{figures/analysis/protonid/datamcpid/pid_traveragededx_all_truncated-ratio.pdf}
        \caption{All reconstructed tracks.}
      \end{subfigure}
      \hspace{2pt}
      \begin{subfigure}[t]{2.5in}
        \includegraphics[angle=0,width=2.5in]{figures/analysis/protonid/datamcpid/pid_traveragededx_pass_truncated.pdf}
        \includegraphics[angle=0,width=2.5in]{figures/analysis/protonid/datamcpid/pid_traveragededx_pass_truncated-ratio.pdf}
        \caption{All reconstructed tracks that are classified as protons.}
      \end{subfigure}
      \caption{Breakdown of the different particle track types in neutrino data
      and simulation as a function of the truncated average $dE/dx$.}
      \label{fig:pidtraveragededx}
    \end{figure}
    \begin{figure}[h]
      \centering
      \begin{subfigure}[t]{2.5in}
        \includegraphics[angle=0,width=2.5in]{figures/analysis/protonid/datamcpid/pid_starty_all_truncated.pdf}
        \includegraphics[angle=0,width=2.5in]{figures/analysis/protonid/datamcpid/pid_starty_all_truncated-ratio.pdf}
        \caption{All reconstructed tracks.}
      \end{subfigure}
      \hspace{2pt}
      \begin{subfigure}[t]{2.5in}
        \includegraphics[angle=0,width=2.5in]{figures/analysis/protonid/datamcpid/pid_starty_pass_truncated.pdf}
        \includegraphics[angle=0,width=2.5in]{figures/analysis/protonid/datamcpid/pid_starty_pass_truncated-ratio.pdf}
        \caption{All reconstructed tracks that are classified as protons.}
      \end{subfigure}
      \caption{Breakdown of the different particle track types in neutrino data
      and simulation as a function of the track starting $y$ position.}
      \label{fig:pidstarty}
    \end{figure}
    \begin{figure}[h]
      \centering
      \begin{subfigure}[t]{2.5in}
        \includegraphics[angle=0,width=2.5in]{figures/analysis/protonid/datamcpid/pid_endy_all_truncated.pdf}
        \includegraphics[angle=0,width=2.5in]{figures/analysis/protonid/datamcpid/pid_endy_all_truncated-ratio.pdf}
        \caption{All reconstructed tracks.}
      \end{subfigure}
      \hspace{2pt}
      \begin{subfigure}[t]{2.5in}
        \includegraphics[angle=0,width=2.5in]{figures/analysis/protonid/datamcpid/pid_endy_pass_truncated.pdf}
        \includegraphics[angle=0,width=2.5in]{figures/analysis/protonid/datamcpid/pid_endy_pass_truncated-ratio.pdf}
        \caption{All reconstructed tracks that are classified as protons.}
      \end{subfigure}
      \caption{Breakdown of the different particle track types in neutrino data
      and simulation as a function of the track ending $y$ position.}
      \label{fig:pidendy}
    \end{figure}
    \begin{figure}[h]
      \centering
      \begin{subfigure}[t]{2.5in}
        \includegraphics[angle=0,width=2.5in]{figures/analysis/protonid/datamcpid/pid_startz_all_truncated.pdf}
        \includegraphics[angle=0,width=2.5in]{figures/analysis/protonid/datamcpid/pid_startz_all_truncated-ratio.pdf}
        \caption{All reconstructed tracks.}
      \end{subfigure}
      \hspace{2pt}
      \begin{subfigure}[t]{2.5in}
        \includegraphics[angle=0,width=2.5in]{figures/analysis/protonid/datamcpid/pid_startz_pass_truncated.pdf}
        \includegraphics[angle=0,width=2.5in]{figures/analysis/protonid/datamcpid/pid_startz_pass_truncated-ratio.pdf}
        \caption{All reconstructed tracks that are classified as protons.}
      \end{subfigure}
      \caption{Breakdown of the different particle track types in neutrino data
      and simulation as a function of the track starting $z$ position.}
      \label{fig:pidstartz}
    \end{figure}
    \begin{figure}[h]
      \centering
      \begin{subfigure}[t]{2.5in}
        \includegraphics[angle=0,width=2.5in]{figures/analysis/protonid/datamcpid/pid_endz_all_truncated.pdf}
        \includegraphics[angle=0,width=2.5in]{figures/analysis/protonid/datamcpid/pid_endz_all_truncated-ratio.pdf}
        \caption{All reconstructed tracks.}
      \end{subfigure}
      \hspace{2pt}
      \begin{subfigure}[t]{2.5in}
        \includegraphics[angle=0,width=2.5in]{figures/analysis/protonid/datamcpid/pid_endz_pass_truncated.pdf}
        \includegraphics[angle=0,width=2.5in]{figures/analysis/protonid/datamcpid/pid_endz_pass_truncated-ratio.pdf}
        \caption{All reconstructed tracks that are classified as protons.}
      \end{subfigure}
      \caption{Breakdown of the different particle track types in neutrino data
      and simulation as a function of the track ending $z$ position.}
      \label{fig:pidendz}
    \end{figure}
    \begin{figure}[h]
      \centering
      \begin{subfigure}[t]{2.5in}
        \includegraphics[angle=0,width=2.5in]{figures/analysis/protonid/datamcpid/pid_costheta_all_truncated.pdf}
        \includegraphics[angle=0,width=2.5in]{figures/analysis/protonid/datamcpid/pid_costheta_all_truncated-ratio.pdf}
        \caption{All reconstructed tracks.}
      \end{subfigure}
      \hspace{2pt}
      \begin{subfigure}[t]{2.5in}
        \includegraphics[angle=0,width=2.5in]{figures/analysis/protonid/datamcpid/pid_costheta_pass_truncated.pdf}
        \includegraphics[angle=0,width=2.5in]{figures/analysis/protonid/datamcpid/pid_costheta_pass_truncated-ratio.pdf}
        \caption{All reconstructed tracks that are classified as protons.}
      \end{subfigure}
      \caption{Breakdown of the different particle track types in neutrino data
      and simulation as a function of the track $\cos(\theta)$ angle.}
      \label{fig:pidcostheta}
    \end{figure}
    \begin{figure}[h]
      \centering
      \begin{subfigure}[t]{2.5in}
        \includegraphics[angle=0,width=2.5in]{figures/analysis/protonid/datamcpid/pid_phi_all_truncated.pdf}
        \includegraphics[angle=0,width=2.5in]{figures/analysis/protonid/datamcpid/pid_phi_all_truncated-ratio.pdf}
        \caption{All reconstructed tracks.}
      \end{subfigure}
      \hspace{2pt}
      \begin{subfigure}[t]{2.5in}
        \includegraphics[angle=0,width=2.5in]{figures/analysis/protonid/datamcpid/pid_phi_pass_truncated.pdf}
        \includegraphics[angle=0,width=2.5in]{figures/analysis/protonid/datamcpid/pid_phi_pass_truncated-ratio.pdf}
        \caption{All reconstructed tracks that are classified as protons.}
      \end{subfigure}
      \caption{Breakdown of the different particle track types in neutrino data
      and simulation as a function of the track $\phi$ angle.}
      \label{fig:pidphi}
    \end{figure}

    \FloatBarrier


%%%%%%%%%%%%%%%%%%%%%%%%%%%%%%%%%%%%%%%%%%%%%%%%%%%%%%%%%%%
% Neutral Current Elastic Proton Event Selection
%%%%%%%%%%%%%%%%%%%%%%%%%%%%%%%%%%%%%%%%%%%%%%%%%%%%%%%%%%%
\subsection{Event Selection}\label{sec:selection}
  The neutral current elastic proton event selection consists of some simple
  pre-selection cuts to remove events that are very unlike the signal and a
  final event selection using a logistic regression model based on event
  details. The proton identification in the previous section was designed to
  identify any proton induced track from any type of event. To select protons
  from NC elastic neutrino-proton interactions, we use optical timing and
  position information, information about activity surrounding the proton
  candidate, and information about unrelated tracks in the event that may be
  neutrino induced.

  \subsubsection{Optical Pre-selection}\label{sec:optpresel}
    A common optical pre-filter is run over MicroBooNE data before any of the
    events are reconstructed. The common optical filter requires both that
    there is a sufficient optical flash within the 2~$\mu s$ beam time window
    and that there is no such flash in a 2~$\mu s$ time window immediately
    preceding the beam. This 2~$\mu s$ window before the beam is called the
    veto window.

    First, each of the 2~$\mu s$ windows is sliced into 339 bins that are
    94~ns wide. Then the total number of PE of any optical pulses that
    occur within a given bin are added to that time bin. If any of the time
    bins within the larger 2~$\mu s$ window add to more than 20 PE, it is
    considered a sufficient flash to the optical filter. If any one of these
    flashes occur within the beam window and none of the flashes occur within
    the veto window the event is accepted. Otherwise, it is rejected.
    \begin{figure}[ht]
      \centering
      \includegraphics[angle=0,width=4.5in]{figures/analysis/selection/optfilter_efficiency.pdf}
      \caption{Efficiency of optical pre-selection on simulated NC elastic
        proton events.}
      \label{fig:opfiltereff}
    \end{figure}

    Figure~\ref{fig:opfiltereff} shows the efficiency of the optical
    pre-selection on simulated NC elastic proton events as a function of true
    negative four-momentum squared. The overall efficiency in range between
    $Q^2 = 0.1$~GeV$^2$ and $Q^2 = 1.0$~GeV$^2$ is 81\%.

  \subsubsection{NC Elastic Pre-selection}
    Before making a final neutral current elastic event selection, some simple
    cuts are made to reject a large number of background events that are very
    unlikely to be NC elastic interactions. The result of each of these
    pre-cuts on the scaled data and simulation samples described in
    Sec.~\ref{sec:datamcpid} are shown in Tab.~\ref{tab:preseldatamc}. 
    \begin{table}
      \caption{Number of remaining events in 5e19 POT data set after each of
        the NC elastic pre-selection cuts.
      \label{tab:preseldatamc}}
      \begin{tabularx}{\textwidth}{l r r | r r}
        \hline
        Cut & Simulation & Off-Beam Data & Sim.+Off-Beam & Neutrino Data \\
        \hline
        Beam Flash & 33743 & 71208 & 104951 & 108064 \\
        Containment & 33742 & 71207 & 104949 & 108058 \\
        Length & 33523 & 70426 & 103949 & 107123 \\
        Proton Score & 16805 & 23491 & 40296 & 41867 \\
        \hline
      \end{tabularx}
    \end{table}
    The first column shows the total number of simulated neutrino events that
    pass each cut scaled to 5e19 POT, and the second column shows the total
    number of off-beam data events that pass each cut scaled to 5e19 POT. The
    third column is the sum of the first two columns which is the expected
    number of on-beam events in 5e19 POT. The last column shows the total
    number of on-beam events in the 5e19 POT data after each cut. If the
    simulation were a perfect representation of reality, the last two columns
    would be the same. At each cut in the pre-selection the measured number of
    event in the on-beam neutrino data is within 5\% of the expected number of
    events from simulation and off-beam data.

    The first pre-selection requirement is that there is at least one
    reconstructed optical flash inside the 1.6~$\mu s$ neutrino beam time
    window. This cut is to reduce the amount of cosmic background. The next two
    pre-selection requirements are that there is at least one reconstructed
    track fuller contained within a fiducial TPC volume that is at least 2.5~cm
    long. The fiducial volume is defined as being at least 10~cm away from
    either $y$ boundary of the TPC active volume and at least 5~cm away from
    any of the $x$ and $z$ boundaries of the TPC active volume. The containment
    and length requirement don't reduce the number of events by a lot, but they
    reduce the number of tracks within each event that are considered when
    trying to select NC elastic proton tracks. The last pre-selection
    requirement is that there is at least one track in the event that has a
    proton score from the gradient-boosted decision tree classifier greater
    than 0.5. To summarize all of the cuts, the set of events after the
    pre-selection had an interaction in-time with the neutrino beam and at
    least one track that is likely to be a proton contained in the TPC fiducial
    volume.  The efficiency and purity of simulated NC elastic proton events
    after each of the cuts is shown in Tab.~\ref{tab:preseleff}.
    \begin{table}
      \caption{Efficiency and purity of simulated NC elastic proton events
        after each of the NC elastic pre-selection cuts.
      \label{tab:preseleff}}
      \begin{tabularx}{\textwidth}{l r r r}
        \hline
        Cut & Efficiency & Relative Efficiency & Purity \\
        \hline
        Optical Pre-Filter & 0.81 & 0.81 & 0.004 \\
        Beam Flash & 0.78 & 0.97 & 0.005 \\
        Reconstruction & 0.62 & 0.79 & 0.005 \\
        Containment & 0.53 & 0.86 & 0.005 \\
        Length & 0.46 & 0.87 & 0.005 \\
        Proton Score & 0.39 & 0.84 & 0.009 \\
        \hline
      \end{tabularx}
    \end{table}
    Figure~\ref{fig:preseleff} shows the efficiency of the pre-selection on
    simulated neutrino events as a function of true negative four-momentum
    squared.
    \begin{figure}[ht]
      \centering
      \includegraphics[angle=0,width=4.5in]{figures/analysis/selection/presel_efficiency.pdf}
      \caption{Efficiency of simulated NC elastic proton events as a function
        of true $Q^2$ after the pre-selection cuts.}
      \label{fig:preseleff}
    \end{figure}
    \begin{figure}[ht]
      \centering
      \includegraphics[angle=0,width=4.5in]{figures/analysis/selection/presel_length.pdf} \\
      \includegraphics[angle=0,width=4.5in]{figures/analysis/selection/presel_length_ratio.pdf}
      \caption{Reconstructed track length of the longest proton candidate track
      in the events after all pre-selection cuts.}
      \label{fig:presellength}
    \end{figure}

    Figure~\ref{fig:presellength} shows the reconstructed track lengths of the
    longest remaining proton candidate track in the events remaining after the
    preselection. The black points in the top plot include the events in the
    5e19 POT neutrino data set with statistical uncertainty only. The gray
    filled histogram include the off-beam data events scaled to 5e19 POT, and
    the color filled histograms include the simulated neutrino events scaled to
    5e19 POT. The simulated NC elastic proton events in the TPC are in peach,
    the simulated charged current events in the TPC are in blue, the simulated
    neutral current background TPC events are in purple, the simulated events
    in which the neutrino interaction occurred in the liquid argon but outside
    of the TPC are in green, and the simulated events in which the neutrino
    interaction occurred outside of the liquid argon cryostat are in yellow.
    Each of the remaining background type is described in more detail in
    section~\ref{sec:effbg}.

  \subsubsection{Selection Variables}\label{sec:selectionvars}
    To select NC elastic proton interactions we look for events with a track
    that is very likely to be a proton, the track is near the reconstructed
    beam flash, there are no other tracks near the proton candidate track,
    there are no tracks likely to be from charged current interactions near the
    reconstructed beam flash, and the proton candidate track is often in the
    direction of the neutrino beam. The following seven variables are used to
    select these events,
    \begin{enumerate}
      \item the decision tree proton ID score,
      \item the shortest distance between either reconstructed endpoint of the
      candidate track and the next closest endpoint of a different
      reconstructed track,
      \item the distance from the center of the reconstructed track to the
      center of the reconstructed beam flash in the $z$ direction,
      \item the distance from the center of the reconstructed track to the
      center of the reconstructed beam flash in the $y$ direction,
      \item whether of not the reconstructed track is in the neutrino beam
      direction,
      \item the distance between any reconstructed tracks with a decision tree
      muon ID score greater than 0.5 and the reconstructed beam flash in the
      $z$ direction,
      \item the distance between any reconstructed tracks with a decision tree
      pion ID score greater than 0.5 and the reconstructed beam flash in the
      $z$ direction.
    \end{enumerate}
    The center of the reconstructed track is defined as the halfway point
    between the reconstructed track endpoints in the dimension of interest. The
    center of the reconstructed flash is the PE-weighted reconstructed center
    of the flash. The beam flash is defined as a flash whose peak amplitude
    occurs within the neutrino beam time window. A track is defined as in the
    beam direction if its reconstructed endpoint is downstream (higher in $z$)
    than its reconstructed start point. If there are no tracks in the event
    identified as a muon (in item 6.) or a pion (in item 7.), the value is set
    to 999~cm, which is close to the maximum distance a track can be from a
    flash in the TPC. A comparison of each of these variables between data and
    simulation after the pre-selection is shown in
    Figs.~\ref{fig:lrpid}-\ref{fig:lrpidist}.
    \begin{figure}[ht]
      \centering
      \includegraphics[angle=0,width=4in]{figures/analysis/selection/LR_input_all_pscore.pdf} \\
      \includegraphics[angle=0,width=4in]{figures/analysis/selection/LR_input_all_pscore-ratio.pdf}
      \caption{Decision tree proton score after the pre-selection cuts.}
      \label{fig:lrpid}
    \end{figure}
    \begin{figure}[ht]
      \centering
      \includegraphics[angle=0,width=4in]{figures/analysis/selection/LR_input_all_vtxdist.pdf} \\
      \includegraphics[angle=0,width=4in]{figures/analysis/selection/LR_input_all_vtxdist-ratio.pdf}
      \caption{Distance to the next closest track after the pre-selection cuts.}
      \label{fig:lrvtxdist}
    \end{figure}
    \begin{figure}[ht]
      \centering
      \includegraphics[angle=0,width=4in]{figures/analysis/selection/LR_input_all_fzdist.pdf} \\
      \includegraphics[angle=0,width=4in]{figures/analysis/selection/LR_input_all_fzdist-ratio.pdf}
      \caption{Distance to the beam flash in $z$ after the
        pre-selection cuts.}
      \label{fig:lrfzdist}
    \end{figure}
    \begin{figure}[ht]
      \centering
      \includegraphics[angle=0,width=4in]{figures/analysis/selection/LR_input_all_fydist.pdf} \\
      \includegraphics[angle=0,width=4in]{figures/analysis/selection/LR_input_all_fydist-ratio.pdf}
      \caption{Distance to the beam flash in $y$ after the
        pre-selection cuts.}
      \label{fig:lrfydist}
    \end{figure}
    \begin{figure}[ht]
      \centering
      \includegraphics[angle=0,width=4in]{figures/analysis/selection/LR_input_all_forward.pdf} \\
      \includegraphics[angle=0,width=4in]{figures/analysis/selection/LR_input_all_forward-ratio.pdf}
      \caption{Whether or not the track is forward going after the pre-selection
        cuts.}
      \label{fig:lrforward}
    \end{figure}
    \begin{figure}[ht]
      \centering
      \includegraphics[angle=0,width=4in]{figures/analysis/selection/LR_input_all_mudist.pdf} \\
      \includegraphics[angle=0,width=4in]{figures/analysis/selection/LR_input_all_mudist-ratio.pdf}
      \caption{Distance between muon track and flash after the pre-selection
        cuts.}
      \label{fig:lrmudist}
    \end{figure}
    \begin{figure}[ht]
      \centering
      \includegraphics[angle=0,width=4in]{figures/analysis/selection/LR_input_all_pidist.pdf} \\
      \includegraphics[angle=0,width=4in]{figures/analysis/selection/LR_input_all_pidist-ratio.pdf}
      \caption{Distance between pion track and flash after the pre-selection
        cuts.}
      \label{fig:lrpidist}
    \end{figure}

    \FloatBarrier

  \subsubsection{Logistic Regression}
    To determine which events are NC elastic like based on these seven
    variables, we use them as input to a logistic regression
    model~\cite{Hosmer2005}. The difference between a logistic regression model
    and a linear regression model is that the outcome is binary (or logistic).
    Otherwise, the methods used to fit logistic and linear regression models
    follow the same principles. In our logistic regression a multi-dimensional
    sigmoid function is fit to the signal and background data. The output is a
    score that can be used to determine how signal-like a data point is.
    \begin{equation*}
      S(g({\bf x})) = \frac{e^{g({\bf x})}}{1 + e^{g({\bf x})}} \,,
    \end{equation*}
    where $g({\bf x})$ is a linear combination of the selection variables, ${\bf x}$,
    \begin{equation*}
      g({\bf x}) = w_0 + w_1 x_1 + w_2 x_2 + ... + w_7 x_7 \,.
    \end{equation*}
    Here $x_1$ is item 1 from the list in Sec.~\ref{sec:selectionvars} (the
    proton ID score), $x_2$ is item 2 (the distance to the next closest track),
    etc. The set of weights, $w_0,...,w_7$ are determined from a fit to the
    data. We determined these weights using the StatsModels
    module~\cite{pystats} in Python to fit the model to a subset of the
    simulated neutrino events described in Sec.~\ref{sec:protonidtest} along
    with a set of simulated cosmic events that produce an optical flash in-time
    with the neutrino beam. The simulated set of in-time cosmics should match
    the off-beam data in a perfect simulation. For the fit, 998 simulated NC
    elastic proton events, 2000 simulated background neutrino events, and 1806
    simulated background in-time cosmic events were used. The final set of
    weights used are
    \begin{equation}
      \begin{aligned}
        w_0 = -5.943956,& \\
        w_1 =  5.388985,& \\
        w_2 =  0.021189,& \\
        w_3 = -0.016710,& \\
        w_4 = -0.017510,& \\
        w_5 =  0.592240,& \\
        w_6 =  0.001084,& \\
        w_7 =  0.000989.&
      \end{aligned}
    \end{equation}
    The $\chi^2/DoF$ of the fit was $8161/4804 = 1.70$. The degrees of freedom
    in the fit are the 4804 simulated events used.
    
    \begin{figure}[ht]
      \centering
      \includegraphics[angle=0,width=4.5in]{figures/analysis/selection/LR_output_overlay_stacked.pdf} \\
      \includegraphics[angle=0,width=4.5in]{figures/analysis/selection/LR_output_overlay-ratio.pdf}
      \caption{Logistic regression NC elastic events selection score.}
      \label{fig:lroutput}
    \end{figure}
    Figure~\ref{fig:lroutput} shows the output logistic regression score on the
    simulated neutrino events plus the off-beam data compared to the output on
    the 5e19 POT neutrino beam data. Again, the black points in the top plot
    include the 5e19 POT subset of neutrino data with statistical uncertainty
    only, the gray filled histogram includes off-beam data scaled to 5e19 POT,
    and the color filled histograms include the simulated neutrino interactions
    overlaid with cosmic data backgrounds scaled to 5e19 POT. The peach color
    is NC elastic proton events in the TPC, blue is charged current background
    events in the TPC, purple is neutral current background events in the TPC,
    green is background from neutrino interactions in the liquid argon outside
    of the TPC, and yellow is background from neutrino interactions outside of
    the liquid argon cryostat. The higher the logistic regression score, the
    more likely the event is NC elastic proton. The bottom plot shows the ratio
    of the 5e19 POT neutrino beam data to the sum of the simulated neutrino
    events and off-beam data.
    
    Figure~\ref{fig:lreffpur} shows the efficiency and purity of the NC elastic
    signal for cutting on given logistic regression scores.
    Figure~\ref{fig:lreffnpur} shows both the efficiency curve and the purity
    curve of the NC elastic event selection on the simulated neutrino events
    and off-beam data as a function of the logistic regression score cut. This
    cut is imposed after the pre-selection cuts, which is why the efficiency
    only goes as high as 0.41. Figure~\ref{fig:lreffvpur} shows the purity on
    the $y$-axis as a function of efficiency on the $x$-axis for different
    logistic regression score cuts. Some score cut values are labelled along
    the curve.
    \begin{figure}[h]
      \centering
      \begin{subfigure}[t]{2.5in}
        \includegraphics[angle=0,width=2.5in]{figures/analysis/selection/LR_effandpur.pdf}\hspace{2pc}%
        \caption{Efficiency and purity curves overlaid.}
        \label{fig:lreffnpur}
      \end{subfigure}
      \hspace{2pt}
      \begin{subfigure}[t]{2.5in}
        \includegraphics[angle=0,width=2.5in]{figures/analysis/selection/LR_effvpur.pdf}\hspace{2pc}%
        \caption{Purity as a function of efficiency.}
        \label{fig:lreffvpur}
      \end{subfigure}
      \caption{Efficiency and purity of the NC elastic proton event selection
      given several different cut values on the logistic regression score.
      \label{fig:lreffpur}}
    \end{figure}
   
    In this analysis we choose a logistic regression score cut of 0.9 to
    minimize the backgrounds as much as reasonably possible. The gives us an
    overall NC elastic proton events selection efficiency of 0.11 and a purity
    of 0.30 based on simulated neutrino events and off-beam data. The
    efficiency of this selection on simulated NC elastic proton events is shown
    in Fig.~\ref{fig:nceeff} as a function of true $Q^2$. The overall shape of
    the NC elastic proton selection efficiency remains relatively flat across
    the $Q^2$ range of interest.
    \begin{figure}[ht]
      \centering
      \includegraphics[angle=0,width=4.5in]{figures/analysis/selection/ncsel_efficiency.pdf}
      \caption{Efficiency of the NC elastic proton event selection on simulated
      NC elastic proton events.}
      \label{fig:nceeff}
    \end{figure}

  \subsubsection{Comparison of Event Selection on Neutrino Beam Data to
  Expectation from Simulation and Off-Beam Data}
    Figures~\ref{fig:ncetheta}-{fig:ncestartz} show comparisons between the
    final selected events in the neutrino beam data and the neutrino simulation
    and off-beam data. The full Run I and simulated data sets are used in these
    comparisons. Each of the data sets is scaled to the size of the Run I data
    (1.7e20 POT). The data sets and the scaling factors are listed below.
    \begin{enumerate}
      \item Run I 1.7e19 POT neutrino data set
      \begin{itemize}
        \item Number of events: 
        \item POT (\texttt{tor860\_wcut}): 
        \item Number of triggers (\texttt{E1DCNT\_wcut}):
        \item Normalization factor: 1
      \end{itemize}
      \item Run I off-beam data subset
      \begin{itemize}
        \item Number of events: 941,584
        \item Number of triggers (\texttt{EXT}): 73,761,274
        \item Normalization factor: 
      \end{itemize}
      \item Neutrino Monte Carlo with cosmic data overlay set
      \begin{itemize}
        \item Number of events: 750,629
        \item POT: 
        \item Flux correction factor: 1.029
        \item Normalization factor: 
      \end{itemize}
    \item Neutrino dirt Monte Carlo with simulated cosmic data set
      \begin{itemize}
        \item Number of events: 105214
        \item POT: $4.66e20$
        \item POT normalization factor: 0.0947
        \item Data driven normalization factor: 0.5 (described in Sec.~\ref{sec:dirt})
        \item Normalization factor: 0.0474
      \end{itemize}
    \end{enumerate}

    Figure~\ref{fig:nceq2} shows the comparison between neutrino beam data and
    the expectation from simulation and off-beam data as a function of
    reconstructed negative four-momentum transfer squared, which is calculated
    from the proton kinetic energy, $T_p$, as derived in
    Sec.~\ref{sec:neutrinos},
    \begin{equation*}
      Q_p^2 = 2 T_p M_p,
    \end{equation*}
    where $M_p$ is the proton mass.  This is the comparison that will be used
    to extract the strange axial form factor parameters (described in
    Sec.~\ref{sec:analysis}). In this and all of the following figures in this
    section, the black points in the top plot include the neutrino beam data
    with statistical uncertainty only, the gray filled histogram includes
    off-beam data, and the color filled histograms include simulated neutrino
    events with overlaid cosmic data backgrounds. The peach are NC elastic
    proton events, the blue are CC background events in the TPC, the purple are
    NC background events in the TPC, the green are background events from
    neutrino interactions in the liquid argon outside of the TPC, and the
    yellow are backgrounds from neutrino interactions outside of the liquid
    argon cryostat. Each of these backgrounds are described in more detail in
    Sec.~\ref{sec:effbg}. The bottom plot shows the ratio of the selected
    neutrino beam data events to the combination of selected simulated neutrino
    events and selected off-beam data events. If the simulation and data were
    exactly the same, the ratio would be equal to one.
    
    \begin{figure}[ht]
      \centering
      \includegraphics[angle=0,width=4.5in]{figures/analysis/selection/ncselection_bacgkrounds_q2.pdf} \\
      \includegraphics[angle=0,width=4.5in]{figures/analysis/selection/ncselection_bacgkrounds_q2-ratio.pdf} \\
      \caption{Selected NC elastic proton events as a function of reconstructed $Q_p^2$.}
      \label{fig:nceq2}
    \end{figure}

    Figures~\ref{fig:ncetheta}~and~\ref{fig:ncephi} show the comparison between
    neutrino beam data and the expectation from neutrino simulation and
    off-beam data as a function of the reconstructed proton angle.
    Figure~\ref{fig:ncetheta} shows this as a function of the cosine of the
    proton angle from the beam direction, $\cos(\theta_p)$, and
    Fig.~\ref{fig:ncephi} shows it as a function of the proton angle around the
    beam direction, $\phi_p$. The $\cos(\theta_p)$ distribution appears to have
    a peak near 0.4 and another peak near 1.0. These peaks in the off-beam data
    are a combination of angle dependence of the cosmic ray flux and the
    reconstruction efficiency of the MicroBooNE TPC. The simulated
    neutrino-induced tracks tend to be more in the direction of the neutrino
    beam. The $\phi_p$ distribution has clear peaks in the off-beam data at
    $\phi_p = \pm \pi/2$. The $-\pi/2$ peak corresponds to tracks that are
    reconstructed as vertically down-going in the TPC and the $+\pi/2$ peak
    corresponds to tracks that are reconstructed as vertically up-going. Both
    peaks are due to down-going cosmic tracks, but some of these tracks are
    mis-reconstructed as going up through the detector. The simulated
    neutrino-induced tracks are relatively isotropic in $\phi$.
    \begin{figure}[ht]
      \centering
      \includegraphics[angle=0,width=4.5in]{figures/analysis/selection/ncselection_bacgkrounds_theta.pdf} \\
      \includegraphics[angle=0,width=4.5in]{figures/analysis/selection/ncselection_bacgkrounds_theta-ratio.pdf} \\
      \caption{Selected NC elastic proton events as a function of reconstructed $\cos(\theta_p)$.}
      \label{fig:ncetheta}
    \end{figure}
    \begin{figure}[ht]
      \centering
      \includegraphics[angle=0,width=4.5in]{figures/analysis/selection/ncselection_bacgkrounds_phi.pdf} \\
      \includegraphics[angle=0,width=4.5in]{figures/analysis/selection/ncselection_bacgkrounds_phi-ratio.pdf} \\
      \caption{Selected NC elastic proton events as a function of reconstructed $\phi_p$.}
      \label{fig:ncephi}
    \end{figure}

    Figures~\ref{fig:ncestartx},~\ref{fig:ncestarty}~and~\ref{fig:ncestartz}
    show the comparison between neutrino beam data and the expectation from
    neutrino simulation and off-beam data as a function of the start points of
    the reconstructed tracks in $x$, $y$, and $z$, respectively. In the
    reference frame used by MicroBooNE, the $x$ dimension is the dimension
    across the width of the detector with the anode near $x = 0$~cm and the
    cathode near $x = 250$~cm, the $y$ dimension is the vertical dimension
    across the height of the detector with the bottom of the TPC near $y =
    -100$~cm and the top near y = $+100$~cm, and the $z$ dimension is the
    dimension in the beam direction across the length of the detector with the
    upstream (front) end of the detector near $z = 0$~cm and the downstream
    (back) end of the detector near $z = 1000$~cm.
    
    In Fig.~\ref{fig:ncestartx}, the higher number of selected events at low
    $x$ is due to the greater charge and light collection efficiency for
    particles that are near the anode since both the TPC readout wires and the
    PMTs are on the anode side of the detector at $x = 0$.
    \begin{figure}[ht]
      \centering
      \includegraphics[angle=0,width=4.5in]{figures/analysis/selection/ncselection_bacgkrounds_startx.pdf} \\
      \includegraphics[angle=0,width=4.5in]{figures/analysis/selection/ncselection_bacgkrounds_startx-ratio.pdf} \\
      \caption{Selected NC elastic proton events as a function of reconstructed $x$ position.}
      \label{fig:ncestartx}
    \end{figure}

    In Fig~\ref{fig:ncestarty}, the increase in the number of selected off-beam
    tracks from the bottom ($y = -100$~cm) to the top ($y = +100$~cm) of the
    TPC is due to the fact that cosmic tracks enter through the top of the
    detector, and not all tracks make it all of the way through. The greater
    number of selected simulated neutrino events near the center of the
    detector ($y = 0$) might be due to the fact that tracks at an angle are
    more likely to fail the fiducial TPC containment cut when they are closer
    to the edges of the detector. This may be more visible in the $y$ dimension
    than the $x$ because there are no other competing effects.
    \begin{figure}[ht]
      \centering
      \includegraphics[angle=0,width=4.5in]{figures/analysis/selection/ncselection_bacgkrounds_starty.pdf} \\
      \includegraphics[angle=0,width=4.5in]{figures/analysis/selection/ncselection_bacgkrounds_starty-ratio.pdf} \\
      \caption{Selected NC elastic proton events as a function of reconstructed $y$ position.}
      \label{fig:ncestarty}
    \end{figure}

    In Fig.~\ref{fig:ncestartz}, the selected off-beam and the selected
    neutrino events are fairly uniform in the $z$ dimension. The small decrease
    in selected events near the very back of the TPC ($z \approx 1000$~cm) is
    due to the fact that tracks are much less likely to be contained in the TPC
    if they are produced so close to the back edge.
    \begin{figure}[ht]
      \centering
      \includegraphics[angle=0,width=4.5in]{figures/analysis/selection/ncselection_bacgkrounds_startz.pdf} \\
      \includegraphics[angle=0,width=4.5in]{figures/analysis/selection/ncselection_bacgkrounds_startz-ratio.pdf} \\
      \caption{Selected NC elastic proton events as a function of reconstructed $z$ position.}
      \label{fig:ncestartz}
    \end{figure}

    \FloatBarrier

%%%%%%%%%%%%%%%%%%%%%%%%%%%%%%%%%%%%%%%%%%%%%%%%%%%%%%%%%%%
% Background Estimation
%%%%%%%%%%%%%%%%%%%%%%%%%%%%%%%%%%%%%%%%%%%%%%%%%%%%%%%%%%%
\subsection{Estimation of Remaining Backgrounds}\label{sec:effbg}
  \subsubsection{Beam Induced Dirt Background}\label{sec:dirt}
    The dirt background, desrcibed in Sec.~\ref{sec:beam}

    Reference MiniBooNE analysis with dirt scaling. \\
    Remaining events: 9.0 +/- 0.7 +/- 4.5 \\

    Discuss dirt neutrons, how they happen and estimated rates and energy
    distributions.  Show how well we can separate or understand them. Show any
    sort of data-driven correction we did to dirt neutron background and how it
    affects our uncertainty. Talk about how well we can tag cryostat neutrons
    with the PMTs.
  \subsubsection{Beam Induced TPC Background}
    Talk about neutral-current elastic neutrons that are produced in the TPC
    and how their distributions differ from NCEp ones. Also include BNB
    backgrounds (CCQE where muon wasn't reconstructed, NCpi0, etc.) Discuss how
    the optical signal would be different for each of these.
  \subsubsection{Cosmic Background}
    Discuss the difference between cosmic tracks and beam proton tracks. How do
    we separate them? What is the rate?


%%%%%%%%%%%%%%%%%%%%%%%%%%%%%%%%%%%%%%%%%%%%%%%%%%%%%%%%%%%
% Systematic Uncertainty Estimation
%%%%%%%%%%%%%%%%%%%%%%%%%%%%%%%%%%%%%%%%%%%%%%%%%%%%%%%%%%%
\subsection{Estimation of Systematic Uncertainty}\label{sec:systematics}
  Show how the ratio gets rid of a lot of measurement uncertainty like beam
  flux and efficiencies. Give exact equation that we will be using for
  analysis. Show how $\Delta s$ is still large at low $Q^2$.
  \subsubsection{Neutrino Beam Flux Uncertainty}\label{sec:fluxuncertainty}
    Reference public note, show plots
  \subsubsection{Measurement Uncertainty}\label{sec:detvar}
    Effects of detector variation MC
  \subsubsection{Nuclear Model Uncertainty}\label{sec:modeluncertainty}
    MEC based on CC change between Tunes 1 and 3 \\
    Separate CC MEC change from nuc. model based on NC MEC change \\
    Pauli blocking based on previous CCQE Genie MC reweighting \\
    Nuclear model from Tunes 1 and 3 differences \\
    FSI from nucleon MFP reweighting and Tunes 1 and 3 differences?
  \subsubsection{Form Factor Uncertainty}\label{sec:ffuncertainty}
    Describe z-exp uncertainty from z-exp papers \\
    Describe how it is implemented here



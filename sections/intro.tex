\section{Introduction} \label{sec:intro}
%The next line produces an indented paragraph to start the document
 %unit.  The LaTeX defaults start most units without indentations.
\hspace{\parindent}

%%%%%%%%%%%%%%%%%%%%%%%%%%%%%%%%%%%%%%%%%%%%%%%%%%%%%%%%%%%
% Standard Model
%%%%%%%%%%%%%%%%%%%%%%%%%%%%%%%%%%%%%%%%%%%%%%%%%%%%%%%%%%%
\subsection{The Standard Model}\label{sec:standardmodel}

Quarks, nucleons, neutrinos. Electromagnetic and weak forces.


%%%%%%%%%%%%%%%%%%%%%%%%%%%%%%%%%%%%%%%%%%%%%%%%%%%%%%%%%%%
% Spin Structure of Nucleons
%%%%%%%%%%%%%%%%%%%%%%%%%%%%%%%%%%%%%%%%%%%%%%%%%%%%%%%%%%%
\subsection{The Spin Structure of Nucleons} \label{sec:nuctheory}

  \subsubsection{Spin Structure Functions}
  In inclusive lepton-nucleon deep inelastic scattering (DIS), it is useful to
  parameterize the scattering cross section in terms of nucleon structure
  functions $F_1(x)$, $F_2(x)$, $g_1(x)$, and $g_2(x)$. In the QCD parton
  model~\cite{Feynman:1969wa}, $x$ is the fraction of the nucleon's momentum
  carried by the quarks, and $g_1$ and $g_2$, the spin structure functions,
  parameterize the polarized DIS cross section~\cite{Thomas:2001kw}.

  The $g_1$ spin structure function can be written as a combination of the spin
  contribution from each of the quark flavors~\cite{Bass:2007zzb},
  \begin{equation*}
    g_1(x) \frac{1}{2}\sum_q e_q^2 \Delta q(x) \,,
  \end{equation*}
  where $q$ is the quark flavor ($q = u,d,s$), $e_q$ is the electric charge of
  the quark, and $\Delta q$ is the contribution from the quark spin
  contribution to the nucleon spin,
  \begin{equation*}
    \Delta q(x) = \big(q^{\uparrow} + \bar{q}^{\uparrow}\big)(x) 
      - \big(q^{\downarrow} + \bar{q}^{\downarrow}\big)(x) \,.
  \end{equation*}
  Here $q^{\uparrow}(x)$ \big($q^{\downarrow}(x)$\big) is the probability of
  finding a quark with momentum fraction $x$ with its spin polarized in the
  (opposite) direction of the nucleon's spin, and $\bar{q}^{\uparrow}(x)$
  \big($\bar{q}^{\downarrow}(x)$\big) is the probability of finding an
  antiquark with momentum fraction $x$ with its spin polarized in the
  (opposite) direction of the nucleon's spin. Integrating over the quark spin
  structure gives the net quark spin contribution to the nucleon spin
  \begin{equation*}
    \Delta q = \int_0^1 \Big[\big(q^{\uparrow} + \bar{q}^{\uparrow}\big)(x)
      - \big(q^{\downarrow} + \bar{q}^{\downarrow}\big)(x) \Big] dx \,.
  \end{equation*}

  \subsubsection{The Ellis-Jaffe Sum Rule}

  The Ellis-Jaffe sum rule~\cite{Ellis:1973kp} relates the integral of the
  $g_1$ spin structure function to the axial charges~\cite{Thomas:2001kw},
  \begin{align*}
    g_A &= \Delta u - \Delta d \\
    g_A^{(8)} &= \Delta u + \Delta d - 2\Delta s \\
    g_A^{(0)} &= \Delta u + \Delta d + \Delta s \,.
  \end{align*}
  where $g_A$ is the isovector axial charge, $g_A^{(8)}$ is the $SU(3)$ octet
  axial charge, and $g_A^{(0)}$ is the flavor-singlet axial charge. For the
  proton, the integral of the $g_1$ spin structure function is
  \begin{align*}
    S_p &= \int_0^1 dx g_{1p}(x)  \\
        &= \int_0^1 dx \Big[\frac{4}{18}\Delta u(x) 
        + \frac{1}{18} \Delta d(x) + \frac{1}{18} \Delta s(x) \Big] \,, \\
        &= \frac{4}{18}\Delta u + \frac{1}{18}\Delta d + \frac{1}{18}\Delta s \,, \\
        &= \frac{g_A}{12} + \frac{g_A^{(8)}}{38} + \frac{g_A^{(0)}}{9} \,.
  \end{align*}
  The axial charges can be determined through experimental measurements.  The
  isovector axial charge, $g_A$, can be obtained in neutron
  $\beta$-decay~\cite{Dubbers:1991bh}. Assuming $SU(3)$ symmetry, $g_A^{(8)}$
  can be obtained though hyperon $\beta$-decay. If the net strange contribution
  to the nucleon spin is assumed to be negligible, the flavor-singlet charge is
  equal to the $SU(3)$ octet charge,
  \begin{equation*}
     \Delta s \sim 0 \Rightarrow g_A^{(0)} = g_A^{(8)} \,.
  \end{equation*}

  One of the first experiments to test the Ellis-Jaffe sum rule through
  inclusive DIS was the European Muon Collaboration (EMC) at CERN in
  1989~\cite{Ashman:1987hv,Ashman:1989ig}. The EMC scattered polarized muons
  off of a polarized proton target and detected the scattered muon in a forward
  muon spectrometer. Figure~\ref{fig:emcej} shows the EMC measurement of the
  integral of the $g_1$ spin structure function as a function of the lower
  bound on the integral, $x_m$, and the expected value of the integral at $x_m
  = 0$ from the Ellis-Jaffe sum rule. There is a significant discrepancy
  between the measured value and the value expected from theory. Assuming that
  the discrepancy comes from the assumption that $\Delta s = 0$, and not the
  $SU(3)$ symmetry assumption, the extracted non-zero value of $\Delta s$ to
  resolve the difference is
  \begin{equation*}
    \Delta s_{\textrm{EMC}} = -0.095 \pm 0.016 \pm 0.023 \,.
  \end{equation*}
  \begin{figure}[h]
    \centering
    \includegraphics[angle=0,width=5.5in]{figures/intro/strfunctions/EMC_Ellis-Jaffe.png}
    \caption{The integral of the $g_1(x)$ spin structure function measured by
      the EMC experiment~\cite{Ashman:1989ig}.}
    \label{fig:emcej}
  \end{figure}
  This implies not only that the overall spin polarization of the strange
  quarks and antiquarks in the nucleon sea is nonzero, but that they are
  polarized in the opposite direction of the proton.

  After the EMC result, many subsequent polarized target inclusive DIS
  experiments tested the Ellis-Jaffe sum rule over the next few decades.
  See~\cite{Aidala:2012mv}~and~\cite{Bass:2007zzb} for detailed reviews.
  Several inclusive DIS polarized target experiments were performed at
  SLAC~\cite{Baum:1983ha,Anthony:1996mw,Abe:1998wq} with a polarized electron
  beam, at CERN with the polarized muon beam by the Spin Muon Collaboration
  (SMC)~\cite{Adeva:1993km,Adeva:1998vv} and by
  COMPASS~\cite{Alexakhin:2006oza}, the polarized electron or positron HERA
  beam at DESY by the
  HERMES~\cite{Ackerstaff:1997ws,Ackerstaff:1999ey,Airapetian:2006vy}
  collaboration. A more recent measurements of the violation of the
  Ellis-Jaffe sum rule from polarized muon inclusive DIS off of a polarized
  target in the COMPASS experiment in 2007 gives~\cite{Alexakhin:2006oza}
  \begin{equation*}
    \Delta s_{\textrm{COMPASS}} = -0.08 \pm 0.01 \pm 0.02 \,,
  \end{equation*}
  and from the HERMES experiment in 2007 gives~\cite{Airapetian:2006vy}
  \begin{equation*}
    \Delta s_{\textrm{HERMES}} = -0.085 \pm 0.013 \pm 0.008 \pm 0.009 \,.
  \end{equation*}

  The nucleon spin structure can also be studied through semi-inclusive deep
  inelastic scattering (SIDIS). In SIDIS, in addition to detecting the
  scattered lepton, at least one of the final state pions or kaons is detected.
  If the detected hadron has a high enough energy fraction, it can be assumed
  that it contains the quark that was struck by the lepton~\cite{Bass:2007zzb}.
  A factor is included in the measured spin structure functions that describes
  the probability of a struck quark producing a pion or kaon with the measured
  energy fraction. This factor is called a fragmentation function, and it can
  be used to reconstruct individual quark flavor contributions to the nucleon
  spin. Several experiments has made measurements of the strange quark spin
  through SIDIS including COMPASS~\cite{Alekseev:2009ac,Alekseev:2010ub} at
  CERN and HERMES~\cite{Airapetian:2003ct,Airapetian:2004zf,Airapetian:2008qf}
  at DESY.  Measurements of the strange quark polarization in the nucleon
  through SIDIS tend to favor much smaller values of $\Delta s$ that are
  consistent with zero. While these results depend less on $SU(3)$ flavor
  symmetry than inclusive DIS results, they do depend strongly on the choice of
  fragmentation functions.
  
  \subsubsection{Theoretical Lattice QCD Calculations}


%%%%%%%%%%%%%%%%%%%%%%%%%%%%%%%%%%%%%%%%%%%%%%%%%%%%%%%%%%%
% Neutrinos as a Nucleon Probe
%%%%%%%%%%%%%%%%%%%%%%%%%%%%%%%%%%%%%%%%%%%%%%%%%%%%%%%%%%%

\subsection{Neutrino Measurements of the Strange Spin Structure}
\label{sec:neutrinos}
  Since neutrinos only interact via the weak force, neutrino-nucleon elastic
  scattering is sensitive to the weak currents and are great tools for
  measuring the axial form factor, $G_A(Q^2)$.
  See~\cite{Lyubushkin:2008pe}~and~\cite{Formaggio:2013kya} for detailed
  reviews of the many measurements of $G_A^s(Q^2)$ through charged current
  quasi-elastic (CCQE) scattering. Neutral current (NC) elastic
  neutrino-nucleon scattering ($\nu N \rightarrow \nu N$) specifically is
  sensitive to the NC form factor $G_A^{NC}(Q^2)$ which contains
  contributions from the up, down, and strange quarks to the spin structure
  of the nucleon ($G_A(Q^2)$ only contains contribution from the up and down
  quarks).

  At the limit where the negative four-momentum transfer squared, $Q^2$, goes
  to zero, the NC axial form factor becomes a combination of the net spin
  contribution from each of the quarks to the nucleon spin~\cite{Bass:2007zzb},
  \begin{equation*}
    G_A^{NC}(Q^2 = 0) = \frac{1}{2}(\Delta u - \Delta s - \Delta s)
  \end{equation*}

  The reconstructed four-momentum transfer is
  determined entirely from the nucleon kinetic energy using
  \begin{align*}
    Q^2_N &= -q^2 = -(\bf{p'}_N - \bf{p}_N)^2 \\
          &= -(E'_N - E_N)^2 + (\bar{p}'_N - \bar{p}_N)^2 \\
          &= 2 T_N M_N,
  \end{align*}
  where $\bf{p}$ is four-momentum, $E$ is energy, $\bar{p}$ is
  three-momentum, $M$ is mass, $T$ is kinetic energy determined by the length
  of the track, the $N$ subscript represents the nucleon in the
  neutrino-nucleon interaction, the prime represents the final state, and the
  nucleon momentum in the nucleus is assumed to be small compared to the
  final nucleon momentum. This means that the ability to measure the axial
  form factor at low $Q^2$ in NC elastic neutrino-nucleon scattering depends
  on the experimental nucleon energy threshold.

  Two previous neutrino experiments have performed a measurement of $\Delta
  s$ through neutral current elastic neutrino-nucleon scattering. The first
  was the E734 experiment~\cite{Ahrens:1986xe} at Brookhaven National Lab (BNL) in
  1987, and the second was the MiniBooNE experiment~\cite{Aguilar-Arevalo:2010cx} at
  Fermilab in 2010.

  \subsubsection{The BNL E734 Experiment}\label{sec:e734}
  The main target and detector of the E734 experiment was 170~tons and was
  made of a combination of liquid scintillator cells and proportional drift
  tubes (PDTs). The liquid scintillator composed 80\% of the target and was
  used for calorimetry and timing, while the PDTs were used for position
  information. Additionally, there was a electromagnetic shower counter and a
  muon spectrometer just downstream of the main detector. The full detector
  schematic is shown in Fig.~\ref{fig:e734detector}.
  \begin{figure}[h]
    \centering
    \includegraphics[angle=0,width=5.5in]{figures/intro/experiments/E734detector.pdf}
    \caption{Schematic of the BNL E734 detector~\cite{Ahrens:1986xe}.}
    \label{fig:e734detector}
  \end{figure}
  The E734 detector sat in a neutrino beam at BNL that could run in either
  neutrino of antineutrino mode with a mean energy of 1.3~GeV for neutrino
  and 1.2~GeV for antineutrinos.

  A simultaneous fit to the neutrino-proton and antineutrino-proton neutral
  current elastic cross sections in the negative four-momentum squared range
  between $Q^2 = 0.45$~GeV$^2$ and $Q^2 = 1.05$~GeV$^2$ was performed to
  extract the neutral current axial form factor.
  \begin{figure}[h]
    \centering
    \begin{subfigure}[t]{2.5in}
      \includegraphics[angle=0,width=2.5in]{figures/intro/experiments/E734flux.pdf}
      \caption{Measured neutrino-proton and antineutrino-proton cross sections.}
      \label{fig:e734xsec}
    \end{subfigure}
    \hspace{2pt}
    \begin{subfigure}[t]{2.5in}
      \includegraphics[angle=0,width=2.5in]{figures/intro/experiments/E734eta.pdf}
      \caption{Extracted neutral current axial form factor parameters.}
      \label{fig:e734eta}
    \end{subfigure}
    \caption{Results from the Brookhaven E734 measurement of the neutral
    current elastic cross section.}
    \label{fig:e734results}
  \end{figure}
  Figure~\ref{fig:e734xsec} shows the measured data and the cross section
  fits. Figure~\ref{fig:e734eta} shows the bounds on the axial mass $M_A$ and
  $\eta$ from the fit. In the parameter estimation, the NC axial form factor
  was assumed to have the form
  \begin{equation}\label{eq:axdipole}
    G_A^{NC}(Q^2) = \frac{1}{2}\frac{g_A}{(1+Q^2/M_A^2)^2}(1+\eta) \,,
  \end{equation}
  where $g_A$ is the weak coupling constant, $M_A$ is the axial mass, and
  $\eta$ is a factor that encodes the difference between the charged current
  axial form factor and the strange axial form factor. This form assumes that
  both parts of the form factor have the exact same shape. If the difference
  is only due to the net spin contribution of the strange quark, $\Delta s$,
  then $\Delta s = -\eta g_A$ which was found to be $-0.15 \pm 0.09$ in this
  analysis.

  A later analysis of the E734 NC elastic data was
  performed~\cite{Garvey:1992cg} in which the strange part of the electric
  and magnetic form factors was not assumed to be zero. Four fits to the
  neutrino-proton and antineutrino-proton cross section data were performed.
  In the first fit, only the axial mass was allowed to vary and the strange
  quark contribution to the electric, magnetic, and axial form factors were
  all held fixed at zero. In the second, the strange contribution to the
  electric and magnetic form factors were fixed at zero, but the axial mass
  and the strange axial form factor were allowed to vary. In the third fit,
  all three strange form factors and the axial mass were allowed to vary, and
  in the last fit, the strange form factors were all allowed to vary, but the
  axial mass was held to the world average at the time, $M_A =
  1.032\pm0.036$~GeV. The same form of the axial form factor in
  Eq.~\ref{eq:axdipole} was assumed and the strange electric and magnetic
  form factors were assumed to have the same shape as the charged current
  electric and magnetic form factors. The extracted value of $\Delta s$
  ranged from $\Delta s = -0.13 \pm 0.09$ in the second fit to $\Delta s =
  -0.21 \pm 0.10$ in the fourth fit.  Each of the extracted $\Delta s$ values
  is consistent with the original measurement and with a $\Delta s$ being
  negative.  Additionally, a strong correlation between $\Delta s$ and $M_A$
  was again observed. In the first fit with $\Delta s$ fixed at zero, a best
  fit to the data was found when $M_A = 1.086 \pm 0.015$~GeV which is very
  consistent with the original results in~\cite{Ahrens:1986xe} shown in
  Fig.~\ref{fig:e734eta}.

  An additional analysis of the E734 data considering ratios of neutral
  current elastic interactions to charged current elastic interactions was
  performed~\cite{Alberico:1998qw}. Specifically, they looked at the
  asymmetry
  \begin{equation*}
    \mathcal{A}_p(Q^2) = \frac{\Big(\frac{d\sigma}{dQ^2}\Big)_{\nu p \rightarrow \nu p} - \Big(\frac{d\sigma}{dQ^2}\Big)_{\bar{\nu} p \rightarrow \bar{\nu} p} }{\Big(\frac{d\sigma}{dQ^2}\Big)_{\nu n \rightarrow \mu^- p} - \Big(\frac{d\sigma}{dQ^2}\Big)_{\bar{\nu} p \rightarrow \mu^+ n}} \,.
  \end{equation*}
  This asymmetry has an enhancement in the strange axial and magnetic form
  factors. It was found that the experimental uncertainty was too large to
  determine $\Delta s$ and that a large factor of the uncertainty was due to
  the uncertainty on the axial mass. This analysis also assumed the dipole
  form of the axial form factor in Eq.~\ref{eq:axdipole}.

  \subsubsection{The MiniBooNE Experiment}\label{sec:miniboonence}
  The main target and detector of the MiniBooNE experiment was 800 tons of
  scintillator oil in a 12.2~m diameter spherical tank. Charged particles
  from the neutrino interactions in the mineral oil produced Cherenkov light
  which was collected by 1520 8-inch PMTs surrounding the oil. A schematic of
  the detector is shown in Fig.~\ref{fig:miniboonedetector}.
  \begin{figure}[h]
    \centering
    \includegraphics[angle=0,width=4in]{figures/intro/experiments/miniboone.png}
    \caption{Schematic of the MiniBooNE detector~\cite{Cheng:2012yy}.}
    \label{fig:miniboonedetector}
  \end{figure}
  MiniBooNE sat in the Booster Neutrino Beam (BNB) at Fermilab that can run
  in either neutrino or antineutrino mode with an average neutrino energy of
  $\sim$800~MeV~\cite{Aguilar-Arevalo:2008yp}.

  A $\Delta s$ fit to the ratio of the neutrino-proton to the
  neutrino-nucleon NC elastic cross section in the proton kinetic energy
  between $T = 350$~MeV and $T = 800$~MeV. This corresponds to a negative
  four-momentum transfer range of $Q^2 = 0.66$~GeV$^2$ to $Q^2 =
  1.5$~GeV$^2$. Figure~\ref{fig:miniboonedeltas} shows the measured ratio and
  the fits to the data.
  \begin{figure}[h]
    \centering
    \includegraphics[angle=0,width=4.5in]{figures/intro/experiments/miniboone_deltas.png}
    \caption{Ratio of the neutrino-proton NC elastic cross section to the
    neutrino-nucleon NC elastic cross section measured in
    MiniBooNE~\cite{Aguilar-Arevalo:2010cx}.}
    \label{fig:miniboonedeltas}
  \end{figure}
  In the analysis, the dipole shape in Eq.~\ref{eq:axdipole} was used for the
  axial form factor. When the value of the axial mass was held to $M_A =
  1.35$~GeV$^2$ a value of $\Delta s = 0.08 \pm 0.26$ was found, and when the
  axial mass was held to $M_A = 1.23$~GeV$^2$ a value of $\Delta s = 0.00 \pm
  0.30$. Both of these values are consistent with the E734 measurement and
  with zero.

  A later analysis of the MiniBooNE data was performed which included a
  two-body current contribution to the cross section~\cite{Golan:2013jtj}.
  The inclusion of the two-body current was done using the NuWro Monte Carlo
  neutrino event generator~\cite{Golan:2012wx}. The original MiniBooNE
  analysis used the NUANCE Monte Carlo neutrino event
  generator~\cite{Casper:2002sd}. A simultaneous extraction of $\Delta s$ and
  the axial mass from the assymetry $\mathcal{A}_p(Q^2)$ assuming the dipole
  form of the axial form factor in Eq.~\ref{eq:axdipole} and including
  two-body currents resulted in an axial mass value of $M_A =
  1.1^{+0.13}_{-0.15}$~GeV and $\Delta s = -0.4^{+0.5}_{-0.3}$. This result
  of $\Delta s$ is consistent with the original MiniBooNE result and with
  zero.

  Very recently, another re-analysis of the MiniBooNE result was performed
  using updated lattice QCD calculations of the strange electric and magnetic
  form factors and the measured MiniBooNE NC elastic nucleon cross section
  was performed~\cite{Sufian:2018qtw}. Notably, this analysis used
  z-expansion fit to the NC axial form factor (described in
  Sec.~\ref{sec:zexpansion}). A value of $\Delta s = -0.196\pm 0.127\pm
  0.041$ was found. The result was used to predict the BNL E734 NC elastic
  $\nu p$ and $\bar{\nu} p$ measurement and was found to be consistent. The
  NC elastic cross section calculated using the results are shown compared to
  MiniBooNE data in Fig.~\ref{fig:sufianuboone} and compared to E734 data in
  Fig.~\ref{fig:sufiane734}.
  \begin{figure}[h]
    \centering
    \begin{subfigure}[t]{2.5in}
      \includegraphics[angle=0,width=2.5in]{figures/intro/experiments/Sufian_MiniBooNE.png}
      \caption{Extracted NC elastic cross section compared to MiniBooNE data.}
      \label{fig:sufianuboone}
    \end{subfigure}
    \hspace{2pt}
    \begin{subfigure}[t]{2.5in}
      \includegraphics[angle=0,width=2.5in]{figures/intro/experiments/Sufian_BNL.png}
      \caption{Extracted NC elastic cross section compared to E734 data.}
      \label{fig:sufiane734}
    \end{subfigure}
    \caption{Neutral current elastic cross section results
    from~\cite{Sufian:2018qtw}}.
    \label{fig:sufiane734}
  \end{figure}

  \subsubsection{Global Fits of the Strange Axial Form Factor}


%This is the end of introduction

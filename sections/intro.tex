\section{Introduction} \label{sec:intro}
%The next line produces an indented paragraph to start the document
 %unit.  The LaTeX defaults start most units without indentations.
\hspace{\parindent}

%%%%%%%%%%%%%%%%%%%%%%%%%%%%%%%%%%%%%%%%%%%%%%%%%%%%%%%%%%%
% Standard Model
%%%%%%%%%%%%%%%%%%%%%%%%%%%%%%%%%%%%%%%%%%%%%%%%%%%%%%%%%%%
\subsection{The Standard Model}\label{sec:standardmodel}
  The standard model of particle physics describes all known fundamental
  particles and three of the four known forces forces in the universe. It also
  describes how each of the fundamental particles interacts with each of these
  three forces.

  The particles in the standard model behave differently based in part on their
  spin. Spin-$\frac{1}{2}$ particles are called fermions (the first three
  columns in Fig.~\ref{fig:standardmodel}), and they make up all known matter.
  Spin-$1$ particles are called gauge bosons (the fourth column in
  Fig.~\ref{fig:standardmodel}), and they carry the forces.  The only other
  particle is the Higgs boson with a spin of $0$. The fermions interact by
  exchanging gauge bosons with each other. They can also exchange bosons with
  bosons. What type of boson any particle can exchange depends on the what
  charges that particle carries. Quarks (the first two rows of fermions in
  Fig.~\ref{fig:standardmodel}) are the only fermions that carry the color
  charge, so they can interact via the strong force by exchanging gluons.
  Interestingly, gluons also carry the color charge, so they can also interact
  by exchanging gluons. Gluons are the only gauge bosons that carry their own
  charge. All of the quarks, half of the leptons, and the W boson carry
  electromagnetic charge and can interact electromagnetically via photon
  exchange. Lastly, all of the fermions and the Higgs boson carry the weak
  charge and can interact weakly via exchanging W$^{\pm}$ and Z bosons. An
  exchange of a W$^{\pm}$ boson is a charged-current weak interaction and an
  exchange of a Z$^0$ boson is a neutral-current weak interaction. In all
  interactions, charged must be conserved, so the fermions exchanging the
  W$^{\pm}$ boson will each change charge by $\pm 1$.

  \begin{figure}[ht]
    \centering
    \includegraphics[angle=0,width=4in]{figures/intro/Standard_Model_of_Elementary_Particles.png}
    \caption{The standard model of elementary particles~\cite{wikipedia}.}
    \label{fig:standardmodel}
  \end{figure}

  There are three generations of fermions (shown as columns in
  Fig.~\ref{fig:standardmodel}). The first generation is the lightest and
  therefore the most stable. The up and down quarks and the electron in the
  first generation make up all atoms and ordinary matter. The higher
  generations contain heavier relatives of each of the particles in the first
  generation that will decay into their first generation complement. Particle
  decays are a type of interaction and are subject to the same laws. The only
  interaction that allows quark flavor change is the charged-current weak
  interaction. Quark flavor comes in up and down for the first generation,
  charm and strange for the second generation, and top and bottom for the third
  generation.

  All other known particles are either the conjugate anti-particles of the
  fundamental standard model particles or composite particles made up of them.
  Nucleons (protons and neutrons) are composed of up and down quarks held
  together by the strong force. Protons consist of two up quarks and a down
  quark.

  Feynman diagrams and allowed vertices?

  General introduction to nucleons being made of quarks. Explain valance quarks
  and virtual sea quarks with gluons.  Talk about quark model and strong force.


%%%%%%%%%%%%%%%%%%%%%%%%%%%%%%%%%%%%%%%%%%%%%%%%%%%%%%%%%%%
% Neutrino Oscillations and Mass
%%%%%%%%%%%%%%%%%%%%%%%%%%%%%%%%%%%%%%%%%%%%%%%%%%%%%%%%%%%
\subsection{Neutrino oscillations and mass}\label{sec:bsm}
    Maybe a short introduction on discovery of neutrinos?
    Introduction to what neutrinos are. Talk generally about forces and the
    weak force.  Handedness and chirality. CKM Matrix.
  Not in the standard model. Quick summary and this is why MicroBooNE exists.

%%%%%%%%%%%%%%%%%%%%%%%%%%%%%%%%%%%%%%%%%%%%%%%%%%%%%%%%%%%
% Nucleon spin
%%%%%%%%%%%%%%%%%%%%%%%%%%%%%%%%%%%%%%%%%%%%%%%%%%%%%%%%%%%
\subsection{Nucleon spin structure}\label{sec:nucleonbsm}
  Talk about EMC and proton spin puzzle. The spins of the quarks don't add up
  to the total proton spin! Explain what $\Delta s$ is and previous
  measurements. Talk about structure functions, inclusive DIS, SIDIS...

  --- parton model \\
  --- spin structure functions (g1 and g2) \\
  --- spin sum rules: Bjorken and Ellis-Jaffe \\

  --- scaling and Regge theory?

%This is the end of introduction

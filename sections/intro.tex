\section{Introduction} \label{sec:intro}
%The next line produces an indented paragraph to start the document
 %unit.  The LaTeX defaults start most units without indentations.
\hspace{\parindent}

%%%%%%%%%%%%%%%%%%%%%%%%%%%%%%%%%%%%%%%%%%%%%%%%%%%%%%%%%%%
% Standard Model
%%%%%%%%%%%%%%%%%%%%%%%%%%%%%%%%%%%%%%%%%%%%%%%%%%%%%%%%%%%
\subsection{The Standard Model}\label{sec:standardmodel}

Quarks, nucleons, neutrinos. Electromagnetic and weak forces.


%%%%%%%%%%%%%%%%%%%%%%%%%%%%%%%%%%%%%%%%%%%%%%%%%%%%%%%%%%%
% Spin Structure of Nucleons
%%%%%%%%%%%%%%%%%%%%%%%%%%%%%%%%%%%%%%%%%%%%%%%%%%%%%%%%%%%
\subsection{The Spin Structure of Nucleons} \label{sec:nuctheory}
  Proton spin: \\
  --- spin vector $s_{\mu}$ from forward matrix element of axial vector current \\
  --- derive axial charges \\
  From Bass 1.1 (need to fill in with Peskin): \\
  Forward matrix element of the axial current vector (derive from peskin):
  \[
      2Ms_{\mu} = <p,s|\bar{\psi}\gamma_{\mu} \gamma_{5} \psi|p,s>
  \]
  where $s_{\mu}$ is the proton's spin vector, $\psi$ is the proton field
  vector and $M$ is the proton mass. The quark axial charges measure
  information about the quark ``spin content".
  \[
    2Ms_{\mu}\Delta q = <p,s| \bar{q}\gamma_{\mu}\gamma_{5}q|p,s> \,,
  \]
  where $q$ is the quark field operator and $\Delta q$ is the quark
  flavor-dependent axial charge, $\Delta u$, $\Delta d$, or $\Delta s$. The
  isovector, SU(3) octet, and flavor-singlet axial charges, $g_A^{(3)}$,
  $g_A^{(8)}$, and $g_A^{(0)}$, respectively, can be written as linear
  combinations of the quark axial charges (derive from Peskin)
  \begin{align}
      g_A^{(3)} &= \Delta u - \Delta d \\
      g_A^{(8)} &= \Delta u + \Delta d - 2\Delta s \\
      g_A^{(0)} &= \Delta u + \Delta d + \Delta s \,.
  \end{align}
  Can be interpreted semi-classically as amount of spin carried by quarks and
  antiquarks of flavor $q$.

  --- expectation of axial charges from non-relativistic quark model

  Talk about EMC and proton spin puzzle. The spins of the quarks don't add up
  to the total proton spin! Explain what $\Delta s$ is and previous
  measurements. Talk about structure functions, inclusive DIS, SIDIS...

  --- parton model \\
  --- spin structure functions (g1 and g2) \\
  --- spin sum rules: Bjorken and Ellis-Jaffe \\


%%%%%%%%%%%%%%%%%%%%%%%%%%%%%%%%%%%%%%%%%%%%%%%%%%%%%%%%%%%
% Neutrinos as a Nucleon Probe
%%%%%%%%%%%%%%%%%%%%%%%%%%%%%%%%%%%%%%%%%%%%%%%%%%%%%%%%%%%
\subsection{Neutrinos as a Nucleon Probe}\label{sec:neutrinos}

  Neutrinos only interact via the weak force, don't have electromagnetic
  interactions, axial form factor enhanced

  \subsection{Previous NC Elastic Neutrino-Nucleon Measurements}
  \label{sec:measurements}

    \subsubsection{The Brookhaven E734 Experiment}\label{sec:e734}
    E734 experimental setup
    analysis and results
    subsequent follow-ups

    \subsubsection{The MiniBooNE Experiment}\label{sec:miniboonence}
    MiniBooNE experimental setup
    analysis and results (all three)
    subsequent follow-ups (including last week)


%This is the end of introduction

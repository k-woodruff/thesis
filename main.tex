\documentclass[12pt]{nmsuth01}
%If your system does not have a latex package ``pkg'', for example,
 %attempting to run latex on the file will produce an error report
 %reading something like ``Latex error:  File pkg.sty not found.''
%
%Your system should have the latex symbol package.  If not comment out 
 %the next line by placing a percent sign % at the beginning of the line.
\usepackage{latexsym}
%Your system may not have the American Mathematical Society packages
 %for mathematics and theorem formatting.  If not, comment out the
 %next line.  If you comment out the next line, you should also
 %comment out all the theorem formatting commands, and you should not
 %try to process the part of the document called chp1a.tex.  Commment
 %out the line below calling for \input{chp1a}.
\usepackage{amsmath,amsthm,mathtools}
%For bra-ket notation
\usepackage{physics}
\DeclarePairedDelimiterX\phys[3]{\langle}{\rangle}{#1 \delimsize\vert\mathopen{} #2 \delimsize\vert\mathopen{} #3}
\newcommand*\Vector[1]{\boldsymbol{#1}}
%For slash notation
\usepackage{slashed}
%For tables
\usepackage{tabularx}
%Your system may not have the xypic drawing package.  If not, comment
 %out the next line.  If you comment out the next line, you should
 %also not try to process chp1a.tex.  
\usepackage[all]{xy}
%Your system should have the graphicx package for included graphics.
 %If not, or, if you have no graphics to include, comment out the next 
 %line.  If you commment out the next line, then you should not try to 
 %process the part of the document called graphics.tex.  Comment out
 %the line below calling for \section{SAMPLE TEXT WITH GRAPHICS} \label{graphics}
%This file illustrates how include in your thesis graphical
 %output.
%The next line produces an indented paragraph to start the document
 %unit.  The LaTeX defaults start most units without indentations.
\hspace{\parindent}
This is sample text with graphics.
\begin{figure}[h]
  \centering
  \includegraphics[angle=-90,width=4in]{figures/bz.pdf}
  \caption{This is an inserted EPS graphic}
  \label{fig:mygraph1}
\end{figure}
Sample ref of \ref{fig:mygraph1} 

\begin{figure}[t]
 \centering
  \includegraphics[angle=-90, width=4in]{figures/E734eta.pdf}  
    \caption{This is another inserted PDF graphic}
    \label{fig:mygraph2}
\end{figure}
%This is the end of the file graphics.tex


\usepackage{graphicx}
\usepackage{subcaption}
\usepackage{float}
\usepackage{placeins}
%The next settings give the correct margins for an NMSU Ph.D. thesis.
\setlength{\evensidemargin}{0.5in}
\setlength{\oddsidemargin}{0.5in}
\setlength{\textwidth}{5.75in}
\setlength{\topmargin}{-0.25in}
\setlength{\textheight}{8.25in}
%
%
% Theorem Formatting Commands.
\theoremstyle{plain}
\newtheorem{lemma}{Lemma}[section]
\newtheorem{prop}{Proposition}[section]
\newtheorem{theorem}{Theorem}[section]
\newtheorem{cor}{Corollary}[section]
%\newtheorem*{nametheorem}{Theorem}[section]
\theoremstyle{definition}
\newtheorem{defn}{Definition}[section]
\newtheorem{example}{Example}[section]
\theoremstyle{remark}
\newtheorem*{rem}{Remark}
\newtheorem*{convention*}{Convention}
%
% Miscellaneous Special Capitals
\newcommand{\bZ}{{\mathbf Z}}
\newcommand{\bQ}{{\mathbf Q}}
\newcommand{\bR}{{\mathbf R}}
\newcommand{\cC}{{\mathcal C}}
% 
%Miscellaneous symbols
%\newcommand{\op}{{\rm \oplus}}
\newcommand{\id}{\rm id}
%
%Miscellaneous Greek letters
\newcommand{\eps}{\ensuremath{\epsilon}}
\newcommand{\ga}{\ensuremath{\gamma}}
%
%Miscellaneous operators
\DeclareMathOperator{\Coker}{Coker}
\DeclareMathOperator{\Tor}{Tor}
\DeclareMathOperator{\Ext}{Ext}
\DeclareMathOperator{\Maps}{Maps}
\DeclareMathOperator{\Hom}{Hom}
\DeclareMathOperator{\Aut}{Aut}
\DeclareMathOperator{\Sin}{Sin}
\DeclareMathOperator{\Tot}{Tot}
\DeclareMathOperator{\im}{im}
%
%Prepare for  double spacing
%\renewcommand{\baselinestretch}{1.5}
\newlength{\singlespace}
\setlength{\singlespace}{\baselineskip}
\newlength{\doublespace}
\setlength{\doublespace}{2.0\baselineskip}
%
\begin{document}
%Use the next line to obtain double spacing for the main body of the document
\setlength{\baselineskip}{\doublespace}
%Comment out the preceding line and uncomment the next line containing
 %a \setlength command to obtain single spacing for the main body of 
 %the document.
%\setlength{\baselineskip}{\singlespace}
%
%Why do you care about this option?
%There are two answers.  One is that it takes less paper to print a
 %preliminary draft of the thesis.  The second is that singlespacing
 %makes the structural units of the document more apparent.  Single
 %spacing makes it easy to identify definitions, lemmas, theorems, and 
 %so on, and it makes apparent such problems as a paragraph that is
 %running too long. On the other hand, the double spacing is good for 
 %checking for typographical errors in a manuscript.  Note that a 
 %document printed in singlespacing and doublespacing formats will 
 %have its problems with page and line breaks in different locations.  
%
%The title page, approval page, dedication page, acknowledgment page,
 %vita page, abstract page, and pages listing tables and figures (if
 %there are any) all carry roman numerals.
\pagenumbering{roman}
\pagestyle{empty}
%This file makes a title page for a Ph.D. thesis.
\thispagestyle{empty}
\renewcommand{\baselinestretch}{2}
\begin{center}
NEUTRAL CURRENT ELASTIC SCATTERING \\ AND THE STRANGE SPIN STRUCTURE OF THE PROTON
\vspace{0.1in}
BY\\
\vspace{0.1in}
KATHERINE WOODRUFF
\end{center}
\vspace{1.0in}
\begin{center}
A dissertation submitted to the Graduate School\\
\vspace{0.1in}
in partial fulfillment of the requirements\\
\vspace{0.1in}
for the degree \\
\vspace{0.1in}
Doctor of Philosophy
\end{center}
\vfill
\begin{center}
Major Subject: Physics
\end{center}
\vspace{1.0in}
\begin{center}
New Mexico State University\\
\vspace{0.1in}
Las Cruces New Mexico\\
\vspace{0.1in}
November 2018
\end{center}
\newpage
%This is the end of the title page.

\pagestyle{plain}
%This file makes the approval page for a Ph.D. thesis at NMSU.  It
 %must be replaced when a permanent Dean of the Graduate School is named.
\noindent
``Neutral Current Elastic Scattering and the Strange Spin Structure of the Proton,'' 
a dissertation prepared by
Katherine Woodruff 
in partial fulfillment of the requirements for the degree, 
Doctor of Philosophy,
has been approved and accepted by the following:

\setlength{\baselineskip}{\singlespace}
\vspace{0.1in}
\begin{flushleft}
\hrulefill
\newline
Dr. Luis Cifuentes
\newline
Dean of the Graduate School
\vspace{0.5in}

\hrulefill
\newline
Dr. Vassili Papavassiliou
\newline
Chair of the Examining Committee
\vspace{0.5in}

\hrulefill
\newline
Date
\vspace{0.5in}
\newline
Committee in charge:
\end{flushleft}

\setlength{\baselineskip}{\doublespace}
Dr. Vassili Papavassiliou, Chair

Dr. Stephen Pate

Dr. Igor Vasiliev

Dr. Steven Stochaj

\newpage
%This is the end of the approval page.

%
%This file makes a dedication page.
\begin{center}
DEDICATION
\end{center}
%The next blank line is needed to have the dedication text
 %appropriately indented.

I dedicate this work to my mom, dad, sister, bill.

\newpage
%This is the end of the dedication page.

%
%This file makes a page for acknowledgments.
\begin{center}
ACKNOWLEDGMENTS
\end{center}
%The next blank line is needed to have the dedication text
 %appropriately indented.

I would like to thank my advisor, ...

\newpage
%This is the end of the page for acknowledgments.

%
%This file makes a vita page in the required two column format.
\begin{center}
            VITA
\end{center}
\begin{flushleft}
\begin{tabular}{ll}
2006-2009        &  A.S., Linn-Benton Community College, Albany, Oregon
\\
& \\
2009-2012        &  B.S., University of Oregon, Eugene, Oregon
\\
& \\
2012-2015        &  M.S., New Mexico State University, Las Cruces, New Mexico
\end{tabular}
\end{flushleft}
\vspace{0.1in}
\begin{center}
PROFESSIONAL  AND HONORARY SOCIETIES
\end{center}
\begin{flushleft}
American Physical Society
\end{flushleft}
\vspace{0.1in}
\begin{center}PUBLICATIONS [or Papers Presented]
\end{center}
Use the format of your bibliography for these.
\begin{center}
FIELD OF STUDY
\end{center}
\begin{flushleft}
Major Field: Experimental Particle Physics
\end{flushleft}

\newpage
%This is the end of the vita page.

%
%This file makes an abstract for a Ph.D. thesis.
\begin{center}
ABSTRACT
\end{center}
\vspace{0.3in}
\begin{center}
THESIS\\ TITLE
\\
BY
\\
KATHERINE WOODRUFF
\end{center}
\vspace{0.3in}
\begin{center}
Doctor of Philosophy

New Mexico State University

Las Cruces, New Mexico, 2017

Dr. Vassili Papavassiliou, Chair
\end{center}
\vspace{0.3in}
%The next line produces an indented paragraph to start the document
 %unit.  The LaTeX defaults start most units without indentations.
\hspace{\parindent}
This is the abstract.

\newpage
%This is the end of the abstract.

\tableofcontents
%If you have tables you will use the next three lines to create a list 
 %of tables following the table of contents page.  If you have no
 %tables in your document, then you comment out the next three lines
 %by placing a percent sign (%) at the beginning of each line.  You
 %may also delete the next three lines if they are not needed.
\newpage
\listoftables
\addcontentsline{toc}{section}{LIST OF TABLES}
%If you have figures you will use the next three lines to create a
 %list of figures following the table of contents page (and the list
 %of tables, if there is one).  If you have no list of figures, then
 %you comment out the next three lines by placing a percent sign (%)
 %at the beginning of the line.  You may also delete the next three
 %lines if they are not needed.
\newpage
\listoffigures
\addcontentsline{toc}{section}{LIST OF FIGURES}
%The next two lines are essentially the start of the main body of the
 %thesis.  You go to a new page, start numbering in arabic numerals,
 %and input the introduction. 
\newpage
\pagenumbering{arabic}
%
\section{Introduction} \label{intro}
%The next line produces an indented paragraph to start the document
 %unit.  The LaTeX defaults start most units without indentations.
\hspace{\parindent}

%%%%%%%%%%%%%%%%%%%%%%%%%%%%%%%%%%%%%%%%%%%%%%%%%%%%%%%%%%%
% Neutrinos
%%%%%%%%%%%%%%%%%%%%%%%%%%%%%%%%%%%%%%%%%%%%%%%%%%%%%%%%%%%
\subsection{Neutrinos}\label{nutheory}
  Maybe a short introduction on discovery of neutrinos?
  \subsubsection{Neutrinos in the Standard Model}
    Introduction to what neutrinos are. Talk about standard model and leptons.
    Maybe talk about mass and oscillations? Talk generally about forces and the
    weak force.
  \subsubsection{Electroweak Interactions}
    Show electroweak Langrangean. Handedness and chirality. CKM Matrix.

%%%%%%%%%%%%%%%%%%%%%%%%%%%%%%%%%%%%%%%%%%%%%%%%%%%%%%%%%%%
% Nucleon Structure
%%%%%%%%%%%%%%%%%%%%%%%%%%%%%%%%%%%%%%%%%%%%%%%%%%%%%%%%%%%
\subsection{Nucleon Structure}\label{nucleon}
  General introduction to nucleons being made of quarks. Explain valance quarks
  and virtual sea quarks with gluons.
  \subsubsection{Nucleons in the Standard Model}
    Talk about quark model and strong force.
  \subsubsection{Proton Spin Crisis}
    Talk about EMC and proton spin crisis. 
  \subsubsection{$\Delta s$}
    Explain what $\Delta s$ is and previous measurements.

%This is the end of introduction

\newpage
\section{Neutrino-Nucleon interactions} \label{sec:theory}
%The next line produces an indented paragraph to start the document
 %unit.  The LaTeX defaults start most units without indentations.
\hspace{\parindent}

%%%%%%%%%%%%%%%%%%%%%%%%%%%%%%%%%%%%%%%%%%%%%%%%%%%%%%%%%%%
% Particle interactions
%%%%%%%%%%%%%%%%%%%%%%%%%%%%%%%%%%%%%%%%%%%%%%%%%%%%%%%%%%%
\subsection{Two-particle interactions}
  The differential cross section for two-particle scattering is given by
  \begin{equation}\label{eq:twobodyxsec}
    d\sigma = \frac{(2\pi)^4 |\mathcal{M}|^2}{4 \sqrt{(p_1\cdot p_2)^2 - m_1^2m_2^2}}
      \cross d\Phi_2(p_1+p_2;p_3,p_4) \,,
  \end{equation}
  where $d\Phi_2(p_1+p_2;p_3,p_4)$ is an element of two-body phase space given by
  \begin{equation}\label{eq:twobodyphase}
      d\Phi_2(p_1+p_2;p_3,p_3) = \delta^4(p_1+p_2 - p_3-p_4)
        \frac{d^3\mathbf{p}_3}{(2\pi)^3 2E_3}\frac{d^3\mathbf{p}_4}{(2\pi)^3 2E_4} \,,
  \end{equation}
  and $\mathcal{M}$ is the scattering amplitude.
  Combining Eqns.~\ref{eq:twobodyxsec}~and~\ref{eq:twobodyphase} gives
  \begin{equation}
      d\sigma = \frac{|\mathcal{M}|^2}{64\pi^2}
        \frac{\delta^4(p_1+p_2-p_3-p_4)}{E_3E_4\sqrt{(p_1\cdot p_2)^2 - m_1^2m_2^2}}
        \, d\mathbf{p}_3d\mathbf{p}_4 \,.
  \end{equation}

  The scattering amplitude is given by the matrix element of the scattering
  matrix, $S$, between the final and initial states ($\mathcal{M} =
  \mel{f}{S}{i}$).  The general form of $S$ is
  \begin{equation}
    S = \sum_{n=0}^{\inf} \frac{(-i)^n}{n!}\int\dots\int d^4x_1 \, d^4x_2 \dots d^4x_n\,
      T\{\hat{\mathcal{H}}'_I(x_1)\hat{\mathcal{H}}'_I(x_2)\dots\hat{\mathcal{H}}'_I(x_n)\} \,,
  \end{equation}
  where $\hat{\mathcal{H}}'_I(x_i)$ is the interaction Hamiltonian density. The
  matrix element can more easily be determined using Feynman calculus.

  \begin{figure}[ht]
    \centering
    \includegraphics[angle=0,width=4in]{figures/theory/bz.pdf}
    \caption{Feynman diagram of two-fermion scattering.}
    \label{fig:feynmantwofermion}
  \end{figure}

  Figure~\ref{fig:feynmantwofermion} shows the Feynman diagram for two-fermion
  scattering. For a massive, vector-boson propagator the matrix element for
  this interaction is given by
  \begin{equation}\label{eq:genmatel}
    \mathcal{M} = \mel{f}{S}{i} = \mel{k'}{J^{\mu}(0)}{k} \frac{i}{q^2 - M_V^2}(-g_{\mu\nu} 
            + q_{\mu}q_{\nu}/M_V^2) \mel{p'}{J^{\mu}(0)}{p} \,,
  \end{equation}
  where $k$ and $k'$ initial and final four-momenta of the first fermion
  ($f_1$), $p$ and $p'$ are the initial and final four-momenta of the second
  fermion ($f_2$), $q$ is the four-momenta carried by the vector-boson
  propagator, and $M_V$ is the mass of the propagator, and $J^{\mu}$ is the
  probability current operator.


%%%%%%%%%%%%%%%%%%%%%%%%%%%%%%%%%%%%%%%%%%%%%%%%%%%%%%%%%%%
% Electroweak Interactions
%%%%%%%%%%%%%%%%%%%%%%%%%%%%%%%%%%%%%%%%%%%%%%%%%%%%%%%%%%%
\subsection{Electroweak interactions}

  The charged current, $j^{\mu}_{CC}$, which corresponds to the exchange of a
  $W^{\pm}$ boson, and the neutral current, $j^{\mu}_{NC}$, which corresponds
  to the exchange of the $Z^0$ boson are given by
  \begin{align}\label{eq:ccurrent}
      j^{\mu}_{CC} &= \sum_f \bar{\psi}_f \gamma^{\mu} (1-\gamma_5) \frac{1}{2}(\tau_1 + i\tau_2) \psi_f \\
        \label{eq:ncurrent}
      j^{\mu}_{NC} &= \sum_f \bar{\psi}_f \gamma^{\mu} (1-\gamma_5) \frac{1}{2}(\tau_3) \psi_f 
       - 2\sin^2(\theta_W) j^{\mu}_{em}
  \end{align}
  where $j^{\mu}_{em}$ is the electromagnetic current, $\psi_{f}$ are the weak
  isospin doublets, and $\tau_i$ are the Pauli matrices
  \begin{equation}
      \tau_1 = 
      \begin{pmatrix}
        0 & 1 \\
        1 & 0
      \end{pmatrix} \,,
      \hspace{5mm}
      \tau_2 = 
      \begin{pmatrix}
        0 & -i \\
        i & 0
      \end{pmatrix} \,,
      \hspace{5mm}
      \tau_3 = 
      \begin{pmatrix}
        1 & 0 \\
        0 & -1
      \end{pmatrix} \,.
  \end{equation}
  The lepton weak isospin doublets are
  \begin{equation}
    \psi_{e} = 
    \begin{pmatrix}
        \hat{\nu}_{e} \\
        \hat{e}^-
    \end{pmatrix} \,,
      \hspace{5mm}
    \psi_{\mu} = 
    \begin{pmatrix}
        \hat{\nu}_{\mu} \\
        \hat{\mu}^-
    \end{pmatrix} \,,
      \hspace{5mm}
    \psi_{\tau} = 
    \begin{pmatrix}
        \hat{\nu}_{\tau} \\
        \hat{\tau}^-
    \end{pmatrix} \,,
  \end{equation}
  and the quark weak isospin doublets are
  \begin{equation}
    \psi_{1} = 
    \begin{pmatrix}
        \hat{u} \\
        \hat{d'}
    \end{pmatrix} \,,
      \hspace{5mm}
    \psi_{2} = 
    \begin{pmatrix}
        \hat{c} \\
        \hat{s'}
    \end{pmatrix} \,,
      \hspace{5mm}
    \psi_{3} = 
    \begin{pmatrix}
        \hat{t} \\
        \hat{b'}
    \end{pmatrix} \,,
  \end{equation}
  where $d'$, $s'$, and $b'$ represent the ``mixed" states
  \begin{equation}
      \begin{pmatrix}
        \hat{d}' \\
        \hat{s}' \\
        \hat{b}'
      \end{pmatrix}
      =
      \begin{pmatrix}
          V_{ud} & V_{us} & V_{ub} \\
          V_{cd} & V_{cs} & V_{cb} \\
          V_{td} & V_{ts} & V_{tb}
      \end{pmatrix}
      \begin{pmatrix}
        \hat{d} \\
        \hat{s} \\
        \hat{b}
      \end{pmatrix}
  \end{equation}
  where $V$ is the Cabibbo-Kobayashi-Maskawa matrix. These doublets contain the
  allowed weak transitions.

  \subsubsection{The charged current}
  The combination of Pauli matrices in the charged current
  \begin{equation}
      \frac{1}{2}\tau_+ = \frac{1}{2}(\tau_1 + \tau_2) \,,
  \end{equation}
  acts as an ``isospin raising matrix" and corresponds to the exchange of a
  $W^+$ boson.
  For the leptons, this gives
  \begin{equation}
    \begin{aligned}
      j^{\mu}_{CC}(\textrm{leptons}) &= \sum_{l=e,\mu,\tau}
      \begin{pmatrix}
          \bar{\hat{\nu}}_l \\
          \bar{\hat{l}}
      \end{pmatrix}
      \gamma^{\mu}(1-\gamma_5)\frac{1}{2}
      \begin{pmatrix}
        0 & 1 \\
        0 & 0
      \end{pmatrix}
      \begin{pmatrix}
        \hat{\nu}_l \\
          \hat{l}
      \end{pmatrix} \\
      &= \sum_{l=e,\mu,\tau} \bar{\hat{\nu}}_{l} \gamma^{\mu}(1-\gamma_5)\frac{1}{2}\, \hat{l} \,,
    \end{aligned}
  \end{equation}
  and, similarly, for the quarks we get
  \begin{equation}
      j^{\mu}_{CC}(\textrm{quarks}) =
       \bar{\hat{u}}\gamma^{\mu}(1-\gamma_5)\frac{1}{2}\,\hat{d}'
       +\bar{\hat{c}}\gamma^{\mu}(1-\gamma_5)\frac{1}{2}\,\hat{s}'
       +\bar{\hat{t}}\gamma^{\mu}(1-\gamma_5)\frac{1}{2}\,\hat{b}' \,,
  \end{equation}
  with the total charged current being $j^{\mu}_{CC} =
  j^{\mu}_{CC}(\textrm{leptons}) + j^{\mu}_{CC}(\textrm{quarks})$.

  \subsubsection{The neutral current}
  In neutral current scattering, $\frac{1}{2}\tau_3$ gives the weak isospin
  which acts as a ``weak charge". The electromagnetic current is given by
  \begin{equation}
      j^{\mu}_{em} = \sum_f Q_f \bar{\hat{f}} \gamma^{\mu} \hat{f}
  \end{equation}
  where $f$ is the fermions, and $Q_f$ is the electric charge of $f$.
  So, the total neutral current is
  \begin{equation}
    \begin{aligned}
        j^{\mu}_{NC} &= \sum_{l=e,\mu,\tau} \left(\bar{\hat{\nu}}_{l}
        \gamma^{\mu}(1-\gamma_5) \frac{1}{2}\, \hat{\nu}_{l} - \bar{\hat{l}}
        \gamma^{\mu}(1-\gamma_5) \frac{1}{2}\, \hat{l} 
        +\sin^2\theta_W \bar{\hat{l}}\gamma^{\mu}\hat{l} \right) \\
        &+ \sum_{q=u,c,t} \left(\bar{\hat{q}} \gamma^{\mu}(1-\gamma_5)\frac{1}{2}\hat{q} 
        - \sin^2\theta_W \frac{2}{3} \bar{\hat{q}}\gamma^{\mu}\hat{q} \right) \\
        &+ \sum_{q=d,s,b} \left(- \bar{\hat{q}} \gamma^{\mu}(1-\gamma_5)\frac{1}{2}\hat{q} 
        + \sin^2\theta_W \frac{1}{3} \bar{\hat{q}}\gamma^{\mu}\hat{q} \right) \,.
     \end{aligned}
  \end{equation}
 
  \subsubsection{V$-$A structure}

  We can separate the currents into their vector and pseudovector, or axial
  vector, components. The terms that contain just $\gamma^{\mu}$ behave like
  vectors under a parity transformation
  \begin{equation}
    \hat{\textrm{\textbf{P}}}\hat{\psi}(\textbf{x},t)\hat{\textrm{\textbf{P}}}^{-1} \, 
      \gamma^{\mu} \, \hat{\textrm{\textbf{P}}}\hat{\psi(\textbf{x},t)}\hat{\textrm{\textbf{P}}}^{-1} 
      = - \hat{\textrm{\textbf{P}}}\hat{\psi}(-\textbf{x},t)\hat{\textrm{\textbf{P}}}^{-1} \, 
      \gamma^{\mu} \, \hat{\textrm{\textbf{P}}}\hat{\psi(-\textbf{x},t)}\hat{\textrm{\textbf{P}}}^{-1}
  \end{equation}
  where $\hat{\textrm{\textbf{P}}}$ is the parity operator
  \begin{equation}
    \textrm{\textbf{P}}: \textbf{x} \rightarrow -\textbf{x}, t \rightarrow t \,.
  \end{equation}
  The terms that contain $\gamma^{\mu}\gamma_{5}$ behave like axial vectors
  under a parity transformation
  \begin{equation}
    \hat{\textrm{\textbf{P}}}\hat{\psi}(\textbf{x},t)\hat{\textrm{\textbf{P}}}^{-1} \, 
      \gamma^{\mu}\gamma_5 \, \hat{\textrm{\textbf{P}}}\hat{\psi(\textbf{x},t)}\hat{\textrm{\textbf{P}}}^{-1} 
      = + \hat{\textrm{\textbf{P}}}\hat{\psi}(-\textbf{x},t)\hat{\textrm{\textbf{P}}}^{-1} \, 
      \gamma^{\mu}\gamma_5 \, \hat{\textrm{\textbf{P}}}\hat{\psi(-\textbf{x},t)}\hat{\textrm{\textbf{P}}}^{-1}
      \,.
  \end{equation}
  The charged current is simple
  \begin{equation}
    \begin{aligned}
    j^{\mu}_{CC} = \frac{g}{\sqrt{2}}\Bigg[&\sum_{l=e,\mu,\tau} \bar{\hat{\nu}}_l (\gamma^{\mu} 
          - \gamma^{\mu}\gamma_5)\frac{1}{2}\,\hat{l} \\
      &+ \bar{\hat{u}}(\gamma^{\mu} - \gamma^{\mu}\gamma_5)\frac{1}{2}\, \hat{d}'
      + \bar{\hat{c}}(\gamma^{\mu} - \gamma^{\mu}\gamma_5)\frac{1}{2}\, \hat{s}'
      + \bar{\hat{t}}(\gamma^{\mu} - \gamma^{\mu}\gamma_5)\frac{1}{2}\, \hat{b}' \Bigg] \,,
    \end{aligned}
  \end{equation}
  and the neutral current becomes
  \begin{equation}
    \begin{aligned}
        j^{\mu}_{NC} = \frac{g}{2\cos\theta_W} \Bigg[&\sum_{l=e,\mu,\tau} 
        \left(\bar{\hat{\nu}}_{l}(g_V^{l} \gamma^{\mu}- g_A^{l} \gamma^{\mu}\gamma_5) \hat{\nu}_{l} 
         + \bar{\hat{l}}(g_V^{l} \gamma^{\mu}- g_A^{l} \gamma^{\mu}\gamma_5)\, \hat{l} \right) \\
        &+ \sum_{q=u,d,c,s,t,b} 
         \left(- \bar{\hat{q}} (g_V^{q} \gamma^{\mu}-g_A^{q} \gamma^{\mu}\gamma_5)\hat{q} \right)\Bigg] \,.
     \end{aligned}
  \end{equation}
  where 
  \begin{equation}
    g_V^{f} = \frac{1}{2}\tau_3^{f} - 2\sin^2\theta_W Q_f \,,
    \hspace{3mm}
    g_A^{f} = \frac{1}{2}\tau_3^{f} \,.
  \end{equation}

%%%%%%%%%%%%%%%%%%%%%%%%%%%%%%%%%%%%%%%%%%%%%%%%%%%%%%%%%%%
% Nucleon Form Factors
%%%%%%%%%%%%%%%%%%%%%%%%%%%%%%%%%%%%%%%%%%%%%%%%%%%%%%%%%%%
\subsection{Nucleon form factors}

  \begin{figure}[h]
    \centering
    \begin{subfigure}{2.5in}
      \includegraphics[angle=0,width=2.5in]{figures/theory/bz.pdf}
      \caption{Feynman diagram of neutral-current elastic lepton-nucleon
      scattering.}
      \label{fig:ncefeynman}
    \end{subfigure}
    \hspace{2pt}
    \begin{subfigure}{2.5in}
      \includegraphics[angle=0,width=2.5in]{figures/theory/bz.pdf}
      \caption{Feynman diagram of charged-current elastic lepton-nucleon
      scattering.}
      \label{fig:ccqefeynman}
    \end{subfigure}
  \end{figure}

  To calculate the charged-current neutrino-nucleon scattering matrix element,
  as shown in Fig.~\ref{fig:ccqefeynman}, we need to determine the lepton CC
  matrix element, $\sideset{_{l}_}{{\nu_l}}{\mel{k'}{J^{\mu}_{CC}(0)}{k}}$, and the nucleon
  CC matrix element, $\sideset{_{\textrm{p}}}{_{\textrm{n}}}{\mel{p'}{J^{\mu}_{CC}(0)}{p}}$,
  and for neutral-current, as shown in Fig.~\ref{fig:ncefeynman}, we need the
  lepton NC matrix element, $\sideset{_{\nu_l}}{_{\nu_l}}{\mel{k'}{J^{\mu}_{NC}(0)}{k}}$, and
  the nucleon NC matrix element,
  $\sideset{_{\textrm{p}}}{_{\textrm{p}}}{\mel{p'}{J^{\mu}_{NC}(0)}{p}}$. Leptons are
  point-like particles, so their matrix elements are straight-forward
  \begin{equation}
    \begin{aligned}
      \sideset{_l}{_{\nu_l}}{\mel{k'}{J^{\mu}_{CC}(0)}{k}}
        &= -i\frac{G_F}{\sqrt{2}}u(k')(\gamma^{\mu} - \gamma^{\mu}\gamma_5)u(k) \,, \\
        \sideset{_{\nu_l}}{_{\nu_l}}{\mel{k'}{J^{\mu}_{NC}(0)}{k}}  
        &= -\frac{G_F}{\sqrt{2}}u(k')(\gamma^{\mu} - \gamma^{\mu}\gamma_5)u(k) \,.
    \end{aligned}
  \end{equation}

  \subsubsection{Nucleon matrix elements}
  Since nucleons have a finite structure, the nucleon matrix elements have a
  more complicated form. The current is corrected by form factors which
  represent the internal nucleon structure.

  If we define the vector and axial parts of the nucleon currents by
  \begin{equation}
    \begin{aligned}
    v^{\mu}_i &= \bar{\hat{\psi}}_f\, \gamma^{\mu}\frac{1}{2}\tau_i\, \hat{\psi}_f \,, \\
    a^{\mu}_i &= \bar{\hat{\psi}}_f\, \gamma^{\mu}\gamma_5\frac{1}{2}\tau_i\, \hat{\psi}_f \,,
    \end{aligned}
  \end{equation}
  where $\hat{\psi}$ are the quark doublets and $\tau_i$ are the Pauli matrices
  still, then the nucleon charged and neutral current become
  \begin{equation}
    \begin{aligned}
      j^{\mu}_{CC} &= v^{\mu}_+ - a^{\mu}_+ \,, \\
      j^{\mu}_{NC} &= v^{\mu}_3 - a^{\mu}_3 - 2\sin^2\theta_W j^{\mu}_{em;q} \,,
    \end{aligned}
  \end{equation}
  where $v^{\mu}_+ = v^{\mu}_1 + iv^{\mu}_2$ and $a^{\mu}_+ = a^{\mu}_1 +
  ia^{\mu}_2$. The nucleon matrix elements become
  \begin{equation}
    \begin{aligned}
      \sideset{_{\textrm{p}}}{_{\textrm{n}}}{\mel{p'}{J^{\mu}_{CC}}{p}} 
          &= \sideset{_{\textrm{p}}}{_{\textrm{n}}}{\mel{p'}{V^{\mu}_{+}}{p}} 
            - \sideset{_{\textrm{p}}}{_{\textrm{n}}}{\mel{p'}{A^{\mu}_{+}}{p}} \,, \\
      \sideset{_{\textrm{p}}}{_{\textrm{p}}}{\mel{p'}{J^{\mu}_{NC}}{p}} 
          &= \sideset{_{\textrm{p}}}{_{\textrm{p}}}{\mel{p'}{V^{\mu}_{3}}{p}} 
            - \sideset{_{\textrm{p}}}{_{\textrm{p}}}{\mel{p'}{A^{\mu}_{3}}{p}} 
            - \sideset{_{\textrm{p}}}{_{\textrm{p}}}{\mel{p'}{J^{\mu}_{em}}{p}} \,.
    \end{aligned}
  \end{equation}

  \subsubsection{Electric and magnetic form factors}

  It is easiest to first find an equation for $j^{\mu}_{em;q}$ in terms of the
  electric and magnetic form factors. The nucleon electromagnetic current
  should have the same vector form as the electromagnetic current for
  point-like particles. The available physical variables to construct the
  nucleon EM current are $p$, $p'$, $\gamma^{\mu}$. The most general form is
  \begin{equation}
    \Gamma^{\mu} = \gamma^{\mu}\cdot A + (p'^{\mu} + p^{\mu})\cdot B + (p'^{\mu} - p^{\mu})\cdot C \,,
  \end{equation}
  where $\Gamma^{\mu}$ is given by the equation
  $\sideset{_{\textrm{p}}}{_{\textrm{p}}}{\mel{p'}{J^{\mu}_{em}}{p}} =
  \bar{u}(p')\Gamma^{\mu}u(p)$, and $A$, $B$, and $C$ are arbitrary form factors.
  $\Gamma^{\mu}$ if constrained further by the Ward identity,
  $q_{\mu}\Gamma^{\mu} = 0$, which guarantees current conservation. The
  $\gamma^{\mu}$ and $(p'^{\mu} + p^{\mu})$ terms in $q_{\mu}\Gamma^{\mu}$ go
  to zero, but the $(p'^{\mu} - p^{\mu})$ term does not, so $C=0$. Using the Gordon
  identity~\cite{Gordon}, the general form for the nucleon EM current matrix
  element is
  \begin{equation}\label{eq:FFem}
    \sideset{_{\textrm{p}}}{_{\textrm{p}}}{\mel{p'}{J^{\mu}_{em}}{p}} =
      \bar{u}(p')\left[ \gamma^{\mu}F_1(Q^2) + \frac{i\sigma^{\mu\nu}q_{\mu}}{2M}F_2(Q^2)  \right]u(p) \,.
  \end{equation}
  The form factors $F_1$ and $F_2$ are the Dirac and Pauli form factors,
  respectively, and they are functions of $Q^2 = -q^2$. They can be transformed
  into the Sachs form factors, $G_E$ and $G_M$ by the relationships
  \begin{equation}
    G_E(Q^2) = F_1(Q^2) - \frac{Q^2}{4M^2}F_2(Q^2)\,, \hspace{5mm} G_M(Q^2) = F_1(Q^2) + F_2(Q^2) \,,
  \end{equation}
  where $Q^2 = -q^2$.  The electric form factor, $G_E$ represents the electric
  charge structure of the nucleon and the magnetic form factor, $G_M$,
  represents the magnetic structure. At the limit when $Q^2$ goes to zero, the
  Sachs form factors become the net charge and magnetic moment of the nucleon.
  \begin{equation}
    \begin{aligned}
      G_{E;p}(Q^2=0) = 1 \,,& \hspace{5mm} G_{M;p}(Q^2=0) = \mu_p \,, \\
      G_{E;n}(Q^2=0) = 0 \,,& \hspace{5mm} G_{M;n}(Q^2=0) = \mu_n \,,
    \end{aligned}
  \end{equation}
  where $\mu_p$ and $\mu_n$ are the proton and neutron magnetic moments.
 
  \subsubsection{Vector current form factors}

  The quark part of the electromagnetic current, $j^{\mu}_{em;q}$, can be
  written as
  \begin{equation}
    \begin{aligned}
      j^{\mu}_{em;q} &= \bar{\hat{\psi}}_f \, Q_f \gamma^{\mu}\, \hat{\psi}_f \\
                     &= \bar{\hat{\psi}}_f \, (\tau_3 + \frac{1}{6})\gamma^{\mu} \, \hat{\psi}_f \\
                     &= v^{\mu}_3 + v^{\mu}_0 \,,
    \end{aligned}
  \end{equation}
  where $v^{\mu}_0 = \frac{1}{6}\bar{\hat{\psi}}_f \, \gamma^{\mu}
  \hat{\psi}_f$. Now we can write the nucleon electromagnetic current matrix
  element in terms of the vector and isoscalar parts
  \begin{equation}
    \begin{aligned}
      \sideset{_{\textrm{p(n)}}}{_{\textrm{p(n)}}}{\mel{p'}{J^{\mu}_{em}}{p}} 
          &= \sideset{_{\textrm{p(n)}}}{_{\textrm{p(n)}}}{\mel{p'}{V^{\mu}_3 + V^{\mu}_0)}{p}} \\
          &= \sideset{_{\textrm{p(n)}}}{_{\textrm{p(n)}}}{\mel{p'}{V^{\mu}_3}{p}} 
           + \sideset{_{\textrm{p(n)}}}{_{\textrm{p(n)}}}{\mel{p'}{V^{\mu}_0}{p}} \,,
    \end{aligned}
  \end{equation}
  Since $V^{\mu}_3$ behaves like a vector under the charge symmetry operator and
  $V^{\mu}_0$ behaves as a scalar, the following equations are true
  \begin{equation}
    \begin{aligned}
      \sideset{_{\textrm{p}}}{_{\textrm{p}}}{\mel{p'}{V^{\mu}_3}{p}} 
        &= - \sideset{_{\textrm{n}}}{_{\textrm{n}}}{\mel{p'}{V^{\mu}_3}{p}} \,, \\
      \sideset{_{\textrm{p}}}{_{\textrm{p}}}{\mel{p'}{V^{\mu}_0}{p}} 
        &= + \sideset{_{\textrm{n}}}{_{\textrm{n}}}{\mel{p'}{V^{\mu}_0}{p}} \,. \\
    \end{aligned}
  \end{equation}
  Then
  \begin{align}\label{eq:V3em}
    \sideset{_{\textrm{p}}}{_{\textrm{p}}}{\mel{p'}{V^{\mu}_3}{p}} 
      &= \frac{1}{2} \left[\sideset{_{\textrm{p}}}{_{\textrm{p}}}{\mel{p'}{J^{\mu}_{em}}{p}} 
       - \sideset{_{\textrm{n}}}{_{\textrm{n}}}{\mel{p'}{J^{\mu}_{em}}{p}} \right] \,, \\
    \sideset{_{\textrm{p}}}{_{\textrm{p}}}{\mel{p'}{V^{\mu}_0}{p}} 
      &= \frac{1}{2} \left[\sideset{_{\textrm{p}}}{_{\textrm{p}}}{\mel{p'}{J^{\mu}_{em}}{p}} 
       + \sideset{_{\textrm{n}}}{_{\textrm{n}}}{\mel{p'}{J^{\mu}_{em}}{p}} \right] \,.
  \end{align}
  Combining equations~\ref{eq:FFem}~and~\ref{eq:V3em} gives
  \begin{equation}
    \sideset{_{\textrm{p}}}{_{\textrm{p}}}{\mel{p'}{V^{\mu}_3}{p}} 
      = \bar{u}(p') \left[\gamma^{\mu}F_1^V(Q^2) 
        + \frac{i\sigma^{\mu\nu}q_{\mu}}{2M}F_2^V(Q^2)\right] u(p) \,,
  \end{equation}
  where the vector form factors, $F_1^V$ and $F_2^V$ are defined as
  \begin{equation}
    \begin{aligned}
      F_1^V(Q^2) &= \frac{1}{2}\left( F_{1,\textrm{p}}(Q^2) - F_{1,\textrm{n}}(Q^2)\right) \,, \\
      F_2^V(Q^2) &= \frac{1}{2}\left( F_{2,\textrm{p}}(Q^2) - F_{2,\textrm{n}}(Q^2)\right)
    \end{aligned}
  \end{equation}
  The final vector currents are the ``raising" and ``lowering" vector currents,
  $V^{\mu}_{\pm}$, in the charged current.

  



  \subsubsection{Axial form factors}
    Derive the axial form factor. Discuss different axial form factor models
    (dipole...). Show that you can determine $\Delta s$. Make clear that
    this is the same $\Delta s$ from first subsubsection.

%%%%%%%%%%%%%%%%%%%%%%%%%%%%%%%%%%%%%%%%%%%%%%%%%%%%%%%%%%%
% Strangeness in the Nucleon
%%%%%%%%%%%%%%%%%%%%%%%%%%%%%%%%%%%%%%%%%%%%%%%%%%%%%%%%%%%
\subsection{Strangeness in the Nucleon} \label{sec:strangeness}

  If contributions to the nucleon from quarks heavier than the strange are
  neglected, the quark part of the charged and neutral currents can be
  separated between the light quarks and the strange quark.
  \begin{align}
    j^{\mu}_{CC}(\textrm{quarks}) &= \bar{\hat{N}}(\gamma^{\mu} 
                - \gamma^{\mu}\gamma_5)\frac{1}{2}\tau_{\pm}\hat{N} \,, \\
    j^{\mu}_{NC}(\textrm{quarks}) &= \bar{\hat{N}}(\gamma^{\mu} - \gamma^{\mu}\gamma_5)\frac{1}{2}\tau_3\hat{N} 
                - \bar{\hat{s}}(\gamma^{\mu} - \gamma^{\mu}\gamma_5)\frac{1}{2}\tau_3\hat{s} \,.
  \end{align}

  %%% starting text %%%
  From Alberico, starting with 2.3: \\
  It's convenient to separate the contributions from the light ($u,d$) and
  heavy ($s,c,...$) quarks. We get
  \[
    \j_{\alpha}^{NC;q} = v_{\alpha}^3 - a_{\alpha}^3 - \frac{1}{2}(v_{\alpha}^s - a_{\alpha}^s) - 2\mathrm{sin}^2\theta_W j_{\alpha}^{em} \,,
  \]
  where we define
  \begin{align*}
      v_{\alpha}^3 &= \bar{u}\gamma_{\alpha}\frac{1}{2}u - \bar{d}\gamma_{\alpha}\frac{1}{2}d \equiv \bar{N} \gamma_{\alpha}\frac{1}{2}\tau_3 N \\
      a_{\alpha}^3 &= \bar{u}\gamma_{\alpha}\gamma_5\frac{1}{2}u - \bar{d}\gamma_{\alpha}\gamma_5\frac{1}{2}d \equiv \bar{N} \gamma_{\alpha}\gamma_5\frac{1}{2}\tau_3 N \,.
  \end{align*}
  where $N = \left(\begin{matrix}{u}\\{d}\end{matrix} \right)$ is the isotopic
  SU(2) group doublet and the currents $v_{\alpha}^3$ and $a_{\alpha}^3$
  are the third components of the isovectors
  \begin{align*}
      v_{\alpha}^i &= \bar{N} \gamma_{\alpha}\frac{1}{2}\tau^i N \\
      a_{\alpha}^i &= \bar{N} \gamma_{\alpha}\gamma_5\frac{1}{2}\tau^i N \,.
  \end{align*}
  On the other hand, $v_{\alpha}^s$ and $a_{\alpha}^s$ are isoscalars. They
  represent the heavier quark contirbution to $j_{\alpha}^{NC;q}$. If we only
  include the strange quark, we have
  \begin{align*}
      v_{\alpha}^s &= \bar{s}\gamma_{\alpha}s \\
      a_{\alpha}^s &= \bar{s}\gamma_{\alpha}\gamma_5 s \,.
  \end{align*}
  The quark part of the electromagnetic current is
  \[
    j_{\alpha}^{em;q} = \sum_{q=u,d,...} e_q\bar{q} \gamma_{\alpha}q \,,
  \]
  which we can also separate between light and heavy contributions
  \[
    j_{\alpha}^{em;q} = v_{\alpha}^3 + v_{\alpha}^0 \,.
  \]
  where $v_{\alpha}^0$ is the isoscalar current which is given by
  \[
    v_{\alpha}^0 = \frac{1}{6}\bar{N}\gamma_{\alpha}N + \left(-\frac{1}{3}\right)\bar{s}\gamma_{\alpha}s \,,
  \]
  if we ignore charm and up.

  %%% ending text %%%

%%%%%%%%%%%%%%%%%%%%%%%%%%%%%%%%%%%%%%%%%%%%%%%%%%%%%%%%%%%
% Spin Structure of Nucleons
%%%%%%%%%%%%%%%%%%%%%%%%%%%%%%%%%%%%%%%%%%%%%%%%%%%%%%%%%%%
\subsection{The Spin Structure of Nucleons} \label{sec:nuctheory}
  Proton spin: \\
  --- spin vector $s_{\mu}$ from forward matrix element of axial vector current \\
  --- derive axial charges \\
  From Bass 1.1 (need to fill in with Peskin): \\
  Forward matrix element of the axial current vector (derive from peskin):
  \[
      2Ms_{\mu} = <p,s|\bar{\psi}\gamma_{\mu} \gamma_{5} \psi|p,s>
  \]
  where $s_{\mu}$ is the proton's spin vector, $\psi$ is the proton field
  vector and $M$ is the proton mass. The quark axial charges measure
  information about the quark ``spin content".
  \[
    2Ms_{\mu}\Delta q = <p,s| \bar{q}\gamma_{\mu}\gamma_{5}q|p,s> \,,
  \]
  where $q$ is the quark field operator and $\Delta q$ is the quark
  flavor-dependent axial charge, $\Delta u$, $\Delta d$, or $\Delta s$. The
  isovector, SU(3) octet, and flavor-singlet axial charges, $g_A^{(3)}$,
  $g_A^{(8)}$, and $g_A^{(0)}$, respectively, can be written as linear
  combinations of the quark axial charges (derive from Peskin)
  \begin{align}
      g_A^{(3)} &= \Delta u - \Delta d \\
      g_A^{(8)} &= \Delta u + \Delta d - 2\Delta s \\
      g_A^{(0)} &= \Delta u + \Delta d + \Delta s \,.
  \end{align}
  Can be interpreted semi-classically as amount of spin carried by quarks and
  antiquarks of flavor $q$.

  --- expectation of axial charges from non-relativistic quark model

%%%%%%%%%%%%%%%%%%%%%%%%%%%%%%%%%%%%%%%%%%%%%%%%%%%%%%%%%%%
% Neutrino-proton elastic cross section
%%%%%%%%%%%%%%%%%%%%%%%%%%%%%%%%%%%%%%%%%%%%%%%%%%%%%%%%%%%
\subsection{Neutrino-proton elastic cross section}\label{sec:probe}
  \subsubsection{Cross section model}

  \subsubsection{Previous measurements}
    E734

%This is the end of nu-N cross sections section

\newpage
\section{The MicroBooNE experiment}\label{microboone}

%%%%%%%%%%%%%%%%%%%%%%%%%%%%%%%%%%%%%%%%%%%%%%%%%%%%%%%%%%%
% MicroBooNE and Neutrino Beam
%%%%%%%%%%%%%%%%%%%%%%%%%%%%%%%%%%%%%%%%%%%%%%%%%%%%%%%%%%%
\subsection{The Booster Neutrino Beam and the MicroBooNE detector}\label{beam}
  \subsubsection{The neutrino beam}
    Where do the proton come from, how do they get accelerated, what is their final energy distribution.
    Bunches and spills.
    The target/horn, it's shape and materials.
    The decay chain of particles coming out of the target (mention dirt).
    The final neutrino flux at MicroBooNE (also mention uncertainty ==> hence ratio).
    Expected number of neutrino interactions, expected NCE interactions.
  \subsubsection{Dirt neutrons}
  \subsubsection{MicroBooNE LArTPC}
    Liquid argon target and TPC.
    LAr properties (density, ionization, scintillation).
    MicroBooNE electric field, drift time/distance.
    MicroBooNE specs (dimensions, wire counts).
  \subsubsection{MicroBooNE PMT system}
    PMT layout, timing, TPB.

%%%%%%%%%%%%%%%%%%%%%%%%%%%%%%%%%%%%%%%%%%%%%%%%%%%%%%%%%%%
% DAQ and Trigger
%%%%%%%%%%%%%%%%%%%%%%%%%%%%%%%%%%%%%%%%%%%%%%%%%%%%%%%%%%%
\subsection{Data Acquisition and Trigger}\label{daq}
  \subsubsection{PMT Readout Electronics and Trigger}
    Consist of signal shaper boards, an FEM modified from TPC version, PMT
    feedthrough, HV/signal splitters, and a trigger board. Write about optical
    flash reconstruction.
  \subsubsection{TPC Readout Electronics}
    Describe cold electronics (front end ASIC preamplifies and shapes), warm
    interface electronics (intermediate amplifier for transmission over 20 m.
    long cable), and digitizing electronics (TPC readout board in crate
    continuously samples received signals and passing from ADC to FPGA for
    processing, reducing, and storage). Maybe also talk about cabling and
    signal feedthrough. Also discuss specs like sampling rate.
  \subsubsection{DAQ System}
    Receives and buffers data from TPC and PMT readout system, then builds and
    records event based on trigger decision. Write about SEBs and EVB and
    passing data between. Should go into detail here about trigger algorithm.

%%%%%%%%%%%%%%%%%%%%%%%%%%%%%%%%%%%%%%%%%%%%%%%%%%%%%%%%%%%
% Simulation and Reconstruction
%%%%%%%%%%%%%%%%%%%%%%%%%%%%%%%%%%%%%%%%%%%%%%%%%%%%%%%%%%%
\subsection{Simulation and Reconstruction}\label{reco}
  The entire experimental process from the neutrino interactions in and around
  the detector to electronic signal readout to particle identification is
  simulated in software. To interface the software packages needed to simulate
  each step, a liquid argon software framework (LArSoft) was developed at
  Fermilab.
  \subsubsection{Simulation}
    The initial neutrino interactions are simulated using the GENIE Neutrino
    Monte Carlo Generator~\cite{Andreopoulos:2009rq,Andreopoulos:2015wxa}.
    Describe all 3 stages of simulations: Genie (generation), Geant4
    (propagation), and detector simulation.
  \subsubsection{Flash reconstruction}
    Describe simpleFlash algorithm.
  \subsubsection{Noise Filters and Hits}
    Write about filtering raw signal and hit finding algorithm used.
    I think give efficiencies at this stage.
  \subsubsection{Clusters and Tracks}
    Explain how hits are clustered and then combined into tracks or showers.
    Explain algorithm used and give efficiencies at this stage. Also include
    cosmic tagging and calorimetry.

%%%%%%%%%%%%%%%%%%%%%%%%%%%%%%%%%%%%%%%%%%%%%%%%%%%%%%%%%%%
% Particle Identification
%%%%%%%%%%%%%%%%%%%%%%%%%%%%%%%%%%%%%%%%%%%%%%%%%%%%%%%%%%%
\subsection{Particle Identification}
  Classification goals (cosmic rejection and neutrino-induced particle type).
  \subsubsection{Reconstructed track features}
    Itemize and discuss every feature used as inuot to classifier.
    Separate into cosmic rejection and particle ID.
  \subsubsection{Boosted decision trees}
    Decision trees, tree boosting, xgboost.
  \subsubsection{Performance}
    Show efficiency, accuracy, itemize backgrounds.
    Discuss reasons for different backgrounds.


%This is the end of detector section

\newpage
\section{Simulation and reconstruction}\label{sc:simreco}
In order to understand how different physics models would affect what we would
see in the detector, large and complex Monte Carlo simulations of the beam and
the detector are created.  These simulations also allow us to develop and test
algorithms that reconstruct the underlying neutrino interaction based on what
particles are seen in the detector. The simulation and reconstruction
algorithms used by MicroBooNE are described in this section.


%%%%%%%%%%%%%%%%%%%%%%%%%%%%%%%%%%%%%%%%%%%%%%%%%%%%%%%%%%%
% Monte Carlo Simulation Simulation
%%%%%%%%%%%%%%%%%%%%%%%%%%%%%%%%%%%%%%%%%%%%%%%%%%%%%%%%%%%
\subsection{Monte Carlo simulation}\label{sec:simulation}
  The entire experimental process from the neutrino interactions in and around
  the detector to electronic signal readout to particle identification is
  simulated in software. To interface the different software packages needed to
  simulate each step, a liquid argon software framework
  (LArSoft)~\cite{larsoft} was developed at Fermilab. Within the LArSoft
  framework, the simulation is divided into three steps: generation,
  propagation, and detector simulation.  The output from the detector
  simulation stage is designed to match the real data output from the detector
  as closely as possible. Event reconstruction is also handled within LArSoft,
  and the same algorithms can be applied to both real data and simulated data
  in an identical way. This section describes all three stages of simulation
  and all of the reconstruction stages, including TPC particle track and
  optical flash reconstruction.

  \subsubsection{Cross section model}\label{sec:geniexsec}
    The initial neutrino interactions are simulated using the GENIE Neutrino
    Monte Carlo Generator~\cite{Andreopoulos:2009rq,Andreopoulos:2015wxa}.
    Assuming a given neutrino flux, GENIE simulates the interaction of the
    neutrino with the nucleons inside of the atoms in and around the detector.
    It also simulates the interactions that occur while the nucleons and pions
    from the initial neutrino-nucleon interaction traverse and exit the
    nucleus. Within GENIE the nuclear models and neutrino-nucleon cross
    sections are configurable by the user. This analysis uses the GENIE version
    v2.12.2 default settings with the addition of the empirical MEC cross
    section model. The details of the physics models for each of the four
    processes (the nuclear physics model, the cross section model, the
    neutrino-induced hadron production model, and the intranuclear hadron
    transport model) within GENIE as well as the simulated cosmic ray
    generation are described in this section. The uncertainties in the analysis
    due to the models in this section are explored in
    Sec.~\ref{sec:modeluncertainty}.

    First, the relativistic Fermi gas (RFG) nuclear model~\cite{RFGmodel} is
    used for all processes. It has been modified in GENIE to incorporate
    short-range nucleon-nucleon correlations using the model by Bodek and
    Ritchie~\cite{BodekrRitchie}. The mass density for the argon nucleus is
    taken from review articles~\cite{nucdensity} and the two-parameter
    Woods-Saxon density function is used~\cite{WoodsSaxon}
    \begin{equation}\label{eq:woodssaxon}
      \rho(r) = N_0\frac{1}{1+e^{(r-c)/z}} \,,
    \end{equation}
    where $\rho$ is the density, $r$ is the distance from the center of the
    nucleus, $c$ describes the size of the nucleus, and $z$ describes the width
    of the surface. For argon, GENIE uses $c=3.53$ and $z=0.54$.  For elastic
    nucleon scattering in the nucleus, Pauli blocking is
    applied~\cite{PauliBlock}.
    
    Next, in the case of elastic (and quasi-elastic) neutrino-nucleon
    scattering, the free-nucleon cross section is calculated. The
    neutrino-nucleon cross section and form factor models used are described in
    Secs.~\ref{sec:formfactorforms}~and~\ref{sec:probe}. The methods used to
    evaluate the Monte Carlo given a chosen cross section and form factor
    model are described in section~\ref{sec:reweighting}.

    The hadrons produced in neutrino-nucleon interactions in a nucleus are
    modeled separately from the nuclear model and the free-nucleon cross
    section model. This is because hadron production in empirical neutrino
    scattering measurements does not match nuclear and neutrino-nucleon cross
    section models. The hadron discrepancy is corrected for in GENIE using the
    AGKY model~\cite{AGKY} that was developed to account for the data seen in
    the MINOS~\cite{MINOS} neutrino scattering experiment and tuned using
    bubble chamber experimental data.

    Finally, the transport of the final state hadrons through the argon atom is
    modeled. The hadrons produced in the neutrino-nucleon interactions can
    rescatter as the exit the nucleus which changes the observable final state
    particle distributions. The intranuclear transport model is implemented by
    the GENIE subpackage called INTRANUKE. INTRANUKE uses a cascade model in
    which the hadron sees a nucleus of isolated nucleons. The interaction
    probability is calculated based on the free nucleon cross section and the
    nucleon density~\cite{Andreopoulos:2015wxa}
    \begin{equation}
      \lambda(E,r) = \frac{1}{\sigma_{hN,tot} \rho(r)} \,,
    \end{equation}
    where $\lambda$ is the hadron interaction probability, $E$ is the hadron
    energy, $r$ is the distance from the center of the nucleus,
    $\sigma_{hN,tot}$ is the total \textit{free} nucleon cross section, and
    $\rho$ is the density of the argon nucleus. The free nucleon cross section
    is different for protons, neutrons, and pions. The density of the argon
    nucleus is again determined by equation~\ref{eq:woodssaxon}.

  \subsubsection{Event generation}
    The GENIE neutrino events are generated in LArSoft based on the neutrino
    flux described in Sec.~\ref{sec:beam}. The 1.6~$\mu$s long spill containing
    $5\times 10^{12}$ protons on target is simulated.  In addition to neutrino
    events, cosmic ray events are also simulated with the
    CORSIKA~\cite{corsika} cosmic ray generator. CORSIKA uses the FLUKA cosmic
    ray model and models the flux of protons, and helium, nitrogen, magnesium,
    and iron atoms on the atmosphere. To simulate real detector events both
    neutrino and cosmic ray interactions can be generated in the same
    simulation. We can also combine real cosmic data taken when the neutrino
    beam was turned off with simulated neutrino interactions in the same events
    to closely model the real detector data.

    In the standard simulation neutrino interactions are only generated within
    the liquid argon filled cryostat Additional special samples are made to
    study how secondary particles from neutrino interactions outside the
    cryostat contribute to our signal background. One special sample that we
    generate is a large \textit{dirt} sample in which neutrino interactions
    that happen in the dirt and detector hall outside of the cryostat are
    generated.  The neutrino interactions are allowed to happen anywhere in
    fifty feet of simulated dirt outside of the detector hall or anywhere in
    the detector hall outside of the cryostat. These are known as dirt events.

    All of the Monte Carlo samples that are generated and used in this analysis
    are listed here.
    \begin{enumerate}
      \item Simulated neutrino interactions with simulated cosmic ray
      backgrounds. This sample is used to develop the particle identification
      and event selection algorithms.
      \item Simulated neutrino interactions with real cosmic ray data backgrounds overlaid.
      This sample very closely represents the actual detector conditions when
      there is a neutrino interaction in MicroBooNE and is refered to as the
      overlay sample.
      \item Simulated dirt interactions plus simulated cosmic interactions.
      This sample contains all known backgrounds to neutrino events in the TPC
      and is refered to as the dirt sample.
    \end{enumerate}

  \subsubsection{Detector simulation}

    The final state particles from GENIE are passed to the GEANT4~\cite{geant4}
    software package to be propagated through the simulated geometry.  The
    entire MicroBooNE detector system, the detector hall, and fifty feet of
    dirt surrounding the detector hall are all included in the GEANT4
    simulation. Figure~\ref{fig:uboonegdml} shows the entire simulated geometry
    including the surroundnig dirt. This includes the electric field and the
    detector electronics.  The particles are stepped through the geometry and
    undergo a possible physics process at each step with a given probability.
    The particles are allowed to interact electromagnetically and hadronically
    with other particles and the detector system or decay through one of the
    physically possible decay modes.  Additionally, the energy loss through
    ionization and scintillation is simulated for all particles traversing the
    detector geometry. In the case of the ionization of the liquid argon, the
    resulting electrons are propagated through the electric field to the wire
    readouts.  For the scintillation of the argon, a photon library was
    generated for each position in the liquid cryostat. At each step particle
    takes through the liquid argon, the resulting photons that would interact
    in the PMTs are determined from a look-up table that was generated in a
    previous full optical simulation. 

    \begin{figure}[h]
      \centering
      \includegraphics[angle=0,width=5in]{figures/detector/simreco/uboone_gdml.png}
      \caption{Rendering of the simulated MicroBooNE geometry.}
      \label{fig:uboonegpml}
    \end{figure}

    After the simulated particles interact with the TPC or PMT system, the
    detector response is simulated. The detector-simulation stage includes the
    electronic responses of the sensitive detectors and reproduces the
    electronic signals from the TPC and PMT systems. First, the PMT signal is
    digitized and the PMT software trigger described in
    Sec.~\ref{sec:swtrigger} is fully simulated. Events that do not pass the
    PMT trigger are dropped. The TPC electronics, including the electronic
    noise on the wires and unresponsive wires, are also included at this stage.
    At this point, the simulated data resembles the actual raw detector data as
    closely as possible.

%%%%%%%%%%%%%%%%%%%%%%%%%%%%%%%%%%%%%%%%%%%%%%%%%%%%%%%%%%%
% Event Reconstruction
%%%%%%%%%%%%%%%%%%%%%%%%%%%%%%%%%%%%%%%%%%%%%%%%%%%%%%%%%%%
\subsection{Event reconstruction}\label{sec:reco}
Both the simulated waveforms and the raw detector waveforms need to be
reconstructed into the initial neutrino interactions. A series of
reconstruction algorithms for both PMT and TPC information exist in LArSoft.
These algorithms start by identifying peaks in the waveforms and combine these
peaks in stages to get to a 3-dimensional representation of the physics
interaction.

  \subsubsection{Flash reconstruction}\label{sec:flashreco}
    The optical flash reconstruction algorithm is applied identically to
    detector data and the simulated data that is output from the detector
    simulation stage.  The first step is to find pulses on the electronic
    signals read out from each of the 32 PMTs. This is done using a peak
    finding algorithm on the digitized signal. The time, amplitude, width, and
    area under the pulses are stored per pulse. Next, the flash reconstruction
    algorithm looks for coincident pulses across PMTs. The individual pulses
    are sorted by size, and all of the pulses that are within 8~$\mu$s of the
    largest pulse are collected. If there are at least three pulses in that
    time window and the sum of the pulse areas is at least 6~PE above the noise
    background, it is considered a flash and saved. The peak time, width,
    position, and size are reconstructed and saved along with information about
    the pulses in the individual PMTs that contributed to the flash. This
    process is repeated starting with the next largest remaining pulse until
    there are none left. An individual pulse can only contribute to one flash
    in an event.

  \subsubsection{TPC event reconstruction}\label{sec:tpcreco}
    Reconstructing TPC events involves more steps than the PMTs since there are
    thousands of wires being read out for milliseconds resulting in
    approximately 30~MB of raw data per event. The reconstruction algorithms
    are again applied identically to detector data and the simulated data from
    the detector simulation stage. First, a noise-deconvolution filter is
    passed over each of the digitized wire signals. Then, similar to the
    optical reconstruction, pulses, or \textit{hits}, are found on individual
    wires which are used as base building blocks for reconstructing 3D particle
    tracks across wires and wire planes.

    The 1D hit finding algorithm starts by walking along a wire signal until
    the value is above a given threshold. The point where the signal goes above
    the threshold is considered the start of the pulse, and the end of the
    pulse is defined as the point where the signal goes back down below the
    threshold.  Then, local minima and maxima are found between the start and
    end of the pulse which are used to determined where there are peaks within
    the pulse.  Adjacent pulses are merged if they are close enough in time.
    Once the pulse and the number of peaks is established, the algorithm
    attempts to fit Gaussian peaks to the pulse. The hypothesis signal is
    composed of one Gaussian per peak from the previous step. The mean and
    amplitude of each Gaussian is initially centered at the existing peaks and
    is allowed to float. If the residuals of the fit are sufficiently small,
    each Gaussian peak is saved as a 1D \textit{hit} with an amplitude, width,
    and time given by the fit. The hit finding is repeating along the length of
    the wire for each wire on all three planes.

    The 3D track finding algorithm is separated into two distinct parts. The
    first part attempts to reconstruct and tag as many cosmic-induced tracks.
    The hits associated with these tracks are then removed from the set of hits
    that are available to reconstruct neutrino-induced tracks in the second
    part. 

    The algorithm used to preferentially reconstruct cosmic tracks is called
    \textit{PandoraCosmic}. It is implemented in the Pandora Software
    Development Kit~\cite{Pandora} used by MicroBooNE. In PandoraCosmic, 1D
    hits are first clustered in 2D per wire readout plane. All of the
    reconstructed hits output by the hit finding algorithm on a given wire
    plane are used as input. The hits are clustered into unambiguous,
    continuous lines of hits. These initial clusters are meant to have a high
    purity, meaning all of the hits in the cluster were induced by the same
    true particle. The clusters in a 2D plane are then merged pairwise in an
    attempt to improve the completeness of the cluster, meaning most of the
    hits on that wire plane induced by a true particle are included in a single
    merged cluster. The 2D clusters are then matched across the three wire
    planes. The 3D track reconstruction algorithm attempts to find a cluster on
    each plane that each corresponds to the same true particle. All possible
    combinations of 2D cluster matching are considered by the algorithm. The
    most suitable set of cluster combinations are projected into 3D and saved
    as reconstructed track objects.

    Next, the reconstructed tracks from PandoraCosmic are passed to a cosmic
    tagging stage. There are two algorithms used to identify cosmic-induced
    tracks. The first is a geometry tagger that looks for tracks that are not
    fully contained in the TPC during the event, and the second is a
    flash-matching algorithm that looks for tracks that are inconsistent with
    any flashes in the beam spill window. Both of the cosmic tagging algorithms
    try to remove as few neutrino-induced tracks as possible and only remove
    tracks that are very likely cosmic-induced.
    
    The geometry tagger starts by locating any TPC wire hits that are
    reconstructed before of after the 1.6 millisecond readout frame. Any tracks
    that contain these hits are tagged as cosmic. Any tracks whose
    reconstructed start or end points are located within a given distance from
    the TPC boundary are also tagged. If both of the start and end points are
    near a TPC boundary, the track is given a cosmic score of 1. If only one of
    the start or end points is near a boundary, the track is given a cosmic
    score of 0.5.
    
    The flash-matching algorithm creates a hypothesis flash for each
    reconstructed track based on its position, size, and energy deposited. It
    then compares the hypothesis flash to each of the reconstructed flashes in
    the beam spill window. If a hypothesis flash is sufficiently incompatible
    with all true flashes in the window, the track is tagged as cosmic. After
    the cosmic tagging, all reconstructed hits associated with any of the
    tracks tagged as cosmic are removed from the set of possible hits that are
    used to reconstruct neutrino-induced tracks. This is refered to as the
    cosmic hit removal stage.

    The remaining set of hits is used to reconstruct neutrino-induced tracks.
    First, two dimensional clusters of hits are formed on each of the three
    wire planes using the TrajCluster~\cite{trajcluster} algorithm in
    LArSoft. TrajCluster creates line-like collections of hits and adds new
    hits to the cluster based on the 2D trajectory of the current set of hits.
    The algorithm stops when there are no additional hits along the trajectory
    of the cluster and is followed by an additional stage the combines cluster
    which start and end near eachother. Next, 3D tracks are formed using the
    projection matching algorithm (PMA)~\cite{PMAIcarus} which was developed
    for the ICARUS experiment and implemented in LArSoft. PMA works by
    proposing nodes and lines connecting the nodes in 3D, projecting the 3D
    lines onto the 2D planes, and determining the most likely positions of the
    nodes in 3D based on the fit of the 2D projections to the existing 2D
    reconstructed clusters from the previous stage. The algorithm start with a
    two-node hypothesis and adds nodes to the 3D line until a maximum number of
    nodes, which is based on the number of hits on the wires, has been reached.

    Calorimetric information is extracted when the reconstructed track objects
    are created. The calorimetric information that we calculate and use is
    related to the energy loss of the particle that created the track along its
    trajectory. At each point along the track, the difference in the total
    charge between the current point and the previous trajectory point is
    calculated. This gives us the change in charge as a function of distance
    along the track, which we label dQ/dx. The change in charge is determined
    by adding the total charge of each of the reconstructed wire hits between
    the two points on the track. The charge difference and the dQ/dx values are
    found for each of the three wire planes. The dQ/dx values can be converted
    to energy loss per unit distance, dE/dx, by multiplying the dQ/dx by a
    measured conversion factor. This conversion factor depends on the strength
    of the electric field and the gain of the readout electronics and is
    determined empirically.


%This is the end of detector section

\newpage
\section{Particle Identification and Event Selection}\label{protonid}
%The next line produces an indented paragraph to start the document
 %unit.  The LaTeX defaults start most units without indentations.
\hspace{\parindent}
Proton ID and event selection.

%%%%%%%%%%%%%%%%%%%%%%%%%%%%%%%%%%%%%%%%%%%%%%%%%%%%%%%%%%%
% Particle Identification and Event Selection
%%%%%%%%%%%%%%%%%%%%%%%%%%%%%%%%%%%%%%%%%%%%%%%%%%%%%%%%%%%
\subsection{Particle Identification}
  After the particle tracks are reconstructed, we use a predictive model to
  classify proton tracks. The inputs to the model are the reconstructed
  physical variables, and the output is the probability that the track is from
  a proton vs. some other particle. There are many predictive models that we
  can use, each with advantages and disadvantages. We chose gradient-boosted
  decision trees for a few main reasons: they are easily interpretable, the
  inputs can be a mix of numeric and categorical variables, and boosted
  decision trees perform well at identifying a small signal in a large
  background.  Each tree is essentially a series of cuts based on physical
  variables which have been fine-tuned to increase the efficiency and purity of
  the final selected sample.
  \subsubsection{Reconstructed track features}\label{sec:features}
    The reconstructed features that are used as input to the classifier are
    listed below. Most of the features come directly from the track object, but
    some are created for this classifier. Each of the features used to identify
    protons either helps to separate neutrino-induced tracks from
    cosmic-induced tracks or to separate neutrino-induced proton tracks from
    other neutrino-induced particle types. For colorimetric information we only
    use information from the collection plane.

    Below is a list and description of the features designed to separate
    neutrino-induced protons from other neutrino-induced particle types.
    \begin{itemize}
      \item \textbf{Number of hits:} This is the total number of hits on the
      collection plane that are associated with track. When used in combination
      with track length and average energy deposited, this feature can be used
      to determine the hit and energy density of the track.
      \item \textbf{Straightness:} This is the ratio of distance between
      reconstructed end points (displacement) to reconstructed path length. It
      represents the amount of scattering a track undergoes. The value is
      always between zero and one with one being perfectly straight.
      \item \textbf{Cosmic score:} This is the geometry tagging cosmic score
      from Sec.~\ref{sec:tpcreco}. Tracks with a cosmic score of 1 have already
      been removed in the cosmic hit removal stage. So, this value is either 0
      (fully contained within the TPC) or 0.5 (entering or exiting the TPC).
      \item \textbf{Length:} This is the reconstructed 3D track length found by
      stepping along the trajectory points.
      \item \textbf{Start dE/dx:} This is the total energy deposited on the
      collection plane in the first six non-zero hits along the track divided
      by the distance between hits to account for the angle with respect to the
      wire plane.
      \item \textbf{End to start dE/dx ratio:} This is the ratio of the total
      $dE/dx$ from the last six non-zero hits along the track on the collection
      plane to the total $dE/dx$ from the first six non-zero hits along the
      track on the collection plane.
      \item \textbf{Truncated total dE/dx:} This is the sum of the $dE/dx$ of a
      truncated set of hits on the collection plane associated with track. The
      truncated set includes all hits along the track with a $dE/dx$ value
      within one standard deviation of the median $dE/dx$ value of all hits
      along the track on the collection plane.
      \item \textbf{Truncated average dE/dx:} This is the truncated total
      $dE/dx$ divided by the number of hits in the truncated hit set associated
      with track.
    \end{itemize}

    Next is the list and description of the features designed to separate
    neutrino-induced tracks from cosmic-induced tracks.
    \begin{itemize}
      \item \textbf{Start and end positions:} These are the reconstructed x, y,
      and z positions of start and end of the track. Tracks that start closer
      to a TPC boundary are more likely to be cosmic-induced.
      \item \textbf{$\theta$ and $\phi$:} These are the reconstructed polar and
      azimuthal angles with respect to the beam direction. Vertical tracks are
      much more likely to be cosmic-induced, while forward-going tracks are
      more likely to be from the neutrino beam.
    \end{itemize}
   
    Determining which end of a track is the beginning is difficult when a
    vertex is not observable. Since we are particularly interested in
    neutral-current elastic events with only a single proton, the direction of
    the track is a concern. A proton will deposit much more energy at the end
    of its track than at the beginning which can be used to determine the true
    direction. Since this correction is not currently implemented within the
    reconstruction, we take all reconstructed tracks that have a higher
    deposited energy at the beginning of the track than at the end of the track
    and flip them. The deposited energy at the beginning (ending) of the track
    is defined as the total $dE/dx$ of the first (last) six non-zero hits along
    the track on the collection plane.  This includes changing the saved start
    positions, end positions, $\theta$, $\phi$, start $dE/dx$, end $dE/dx$, and
    the end to start $dE/dx$ ratio.
    
  \subsubsection{Boosted decision trees}\label{sec:decisiontrees}
    A decision tree can be thought of as a series of if/else statements that
    separate a data set into two or more classes as illustrated in
    Fig.~\ref{fig:dtree}. At each node of the tree, a split is chosen to
    maximize information gain until a set level of separation is reached.  At
    the terminus of the series of splits, called a leaf, a class is assigned.
    The usual parameters that can be set when creating a decision tree are: the
    maximum depth of the tree (how many layers of nodes you will allow), the
    minimum split size (how many data points do you require to keep splitting),
    and minimum leaf size (how small does a leaf have to be before you stop). 
    \begin{figure}[ht]
      \centering
      \includegraphics[angle=0,width=4in]{figures/analysis/protonid/trees_diagram.pdf}
      \caption{Graphical example of a decision tree.}
      \label{fig:dtree}
    \end{figure}
    
    A single tree can easily overfit a data set if it is at all complex, and
    its output is just a class label. Gradient-boosting addresses both of these
    issues by combining many weak classifiers into a strong one. Each weak
    classifier is built based on the error of the previous one. For a given
    training set, whenever a sample is classified incorrectly by a tree, that
    sample is given a higher importance when the next tree is being created.
    Mathematically, each tree is training on the gradient of the loss function.
    After all of the trees have been created, each tree is given a weight based
    on its ability to classify the training set, and the output of the
    gradient-boosted decision tree classifier is the probability that a sample
    is in a given class.
    
    The gradient-boosted decision tree software package we use is
    XGBoost~\cite{Chen:2016btl}. There are two types of classifiers we can use
    to separate protons from other tracks: binary and multiclass. Both
    classifiers are trained on all types of reconstructed tracks. A binary
    classifier classifies each track as either a proton or not a proton, and a
    multiclass classifier classifies a track as one of many types including a
    proton. We choose to use multiclass because the information about
    non-proton tracks is useful for selecting neutral current events. The five
    classes that we train the decision trees to classify are protons (both
    neutrino-induced and cosmic), neutrino-induced muons, neutrino-induced
    pions, neutrino-induced electrons/photons, and all non-proton cosmics.
    
  \subsubsection{Training}
    \begin{table}[ht]
      \caption{Breakdown by simulated particle type reconstructed tracks in the
        training set.
      \label{tab:mctrain}}
      \begin{tabularx}{\textwidth}{ l r r r r r }
        \hline
        & Protons & Muons & Pions & EM Showers & Non-proton Cosmics \\
        \hline
        No. of tracks  & 90,922 & 57,583 & 12,848 & 473,323 & 2,586,527 \\
        Fraction of set & 0.028 & 0.018 & 0.004 & 0.147 & 0.803 \\
        Class weight  & 0.141 & 0.223 & 1.000 & 0.027 & 0.005 \\
        \hline
      \end{tabularx}
    \end{table}

    The gradient-boosted decision tree model was trained on 95,600 events with
    both simulated GENIE neutrino interactions and simulated CORSIKA cosmic
    interactions. Each track in every event was treated as a separate training
    sample. Table~\ref{tab:mctrain} shows the number of each type of track that
    was used for training. There are were a total of 3,221,203 simulated
    training tracks.

    Because the training set has unbalanced classes (there are different
    numbers of each particle type) each training sample is initially weighted
    so that the sum of weights is equal to the size of the smallest class, in
    this case pions.
    \begin{equation*}
      N_s = \sum_{i=1}^{N_{n}}w^n_i \,,
    \end{equation*}
    where $N_s$ is the number of samples in the smallest class, $N_{n}$ is the
    number samples in the $n^{th}$ class, and $w^n_i$ is the weight given to
    the $ith$ sample in that class. The same weight is used for each sample in
    a class, so the value of each positive weight is $w^n=\frac{N_s}{N_{n}}$.
    Balancing the training set prevents the classifier from only learning the
    most frequent classes. In our case, the classifier could achieve a high
    accuracy by classifying everything as a cosmic in the unbalanced set
    because over 80\% of the tracks are cosmic-induced. One of our main goals
    is to have a proton ID efficiency, and since protons only make up 3\% of
    the training set, giving them a higher weight makes it more important to
    the classifier that they are correctly classified.

    The parameters used for training were chosen to both maximize
    classification accuracy and minimize overfitting to the training set.
    Overfitting occurs when the performance on the training set is more
    accurate than the performance on an external test set. The final training
    parameter settings are:
    \begin{itemize}
      \item \textbf{Objective: multiclass: softprob} \\
      The learning objective. We want to classify five different track types
      and get a probability of each class.  
      \item \textbf{Learning rate: 0.045}  \\
      The factor each incorrectly classified sample gets re-weighted by for the
      next tree.  A smaller learning rate requires more trees but prevents
      overfitting.
      \item \textbf{Number of trees: 500} \\
      The total number of trees in classifier.
      \item \textbf{Maximum depth: 10} \\ 
      The maximum number of layers of nodes each tree can have.
      \item \textbf{Maximum sampled features: 0.8} \\
      The fraction of total features that each tree can use to train. These are
      randomly sampled.
      \item \textbf{Maximum sampled observations: 0.85} \\
      The fraction of total samples that each tree can use to train. These are
      randomly sampled.
    \end{itemize}

  \subsubsection{Performance on a Test Set}
    The performance of the gradient-boosted decision tree model was tested on a
    set of 3,200,000 reconstructed tracks from 96,200 events with simulated
    GENIE neutrino interactions and simulated CORSIKA cosmic interactions. This
    set of tracks was generated in the exact same way as the training set.

    \begin{figure}[ht]
      \centering
      \includegraphics[angle=0,width=5in]{figures/analysis/protonid/protonid_mc_output_norm.pdf}
      \caption{Area-normalized histograms of decision tree proton
      identification scores for simulated protons and other simulated proton
      tracks.}
      \label{fig:pidmcout}
    \end{figure}
    \begin{figure}[ht]
      \centering
      \includegraphics[angle=0,width=5in]{figures/analysis/protonid/protonid_mc_output_norm_ncelastic.pdf}
      \caption{Area-normalized histogram of decision tree proton identification
      scores for simulated proton tracks from NC elastic proton interactions.}
      \label{fig:pidmcoutNCE}
    \end{figure}
    Figure~\ref{fig:pidmcout} shows normalized histograms of the output proton
    score for every track in the test set. The proton score ranges from zero to
    one with zero being the least proton-like and one being the most. The blue
    histogram shows all simulated neutrino-induced and cosmic induced proton
    tracks normalized so that the area under the histogram is one. The orange
    histogram shows every other simulated track type, also normalized so that
    the area under it is equal to one. Figure~\ref{fig:pidmcoutNCE} shows the
    area-normalized histogram of proton scores for simulated proton tracks that
    were produced in neutral current elastic proton events.

    \begin{figure}[ht]
      \centering
      \includegraphics[angle=0,width=5.5in]{figures/analysis/protonid/heatmap_mcc87_pmtrack_final.pdf}
      \caption{Heatmap showing the fraction of each class that is made up of a
      given particle type.}
      \label{fig:heatmap}
    \end{figure}
    Figure~\ref{fig:heatmap} shows the overall classification performance of
    the gradient-boosted decision tree model on the test set for each class.
    The x axis shows the true particle type and the y axis shows the particle
    classes. The numbers in the boxes are the fraction of the class that is
    made up of the given true particle type. The fraction of true protons in
    the set of tracks classified as protons is 0.71, the fraction of true muons
    in that set is 0.04, the fraction of true pions is 0.09, the fraction of
    electromagnetic shower particles is 0.04, and the fraction of non-proton
    cosmics in the proton-classified set is 0.05. A track is labelled as a
    given class type in this plot if the particle's decision tree score for
    that class is higher than its score for any of the other four classes.  The
    numbers in this plot were calculated using equal numbers of each true
    particle type. In reality, there are far more non-proton cosmic tracks than
    there are true protons, and the fraction of true protons in the set
    classified as protons will be smaller.

    \begin{figure}[h]
      \centering
      \begin{subfigure}[t]{2.5in}
        \includegraphics[angle=0,width=2.5in]{figures/analysis/protonid/PID_efficiency_allmcke.pdf}
        \caption{The full simulated kinetic energy range.}
        \label{fig:pideffkeall}
      \end{subfigure}
      \hspace{2pt}
      \begin{subfigure}[t]{2.5in}
        \includegraphics[angle=0,width=2.5in]{figures/analysis/protonid/PID_efficiency_mcke.pdf}
        \caption{The kinetic energy range used in this analysis.}
        \label{fig:pideffkerng}
      \end{subfigure}
      \caption{The efficiency of simulated neutrino-induced proton tracks
        correctly classified as protons as a function of true proton kinetic energy.}
      \label{fig:pideffke}
    \end{figure}
    Figure~\ref{fig:pideffke} shows the efficiency of the decision tree proton
    identification on simulated neutrino-induced protons as a function of true
    proton kinetic energy. The left plot (\ref{fig:pideffkeall}) shows the full
    simulated range of true proton kinetic energy, and the right
    (\ref{fig:pideffkerng}) shows the range of interest to this analysis. In
    the interesting range of kinetic energies, the proton identification
    efficiency stays relatively flat between 0.8 and 0.8 efficiency, with an
    average efficiency of 0.71. A track is considered positively identified as
    a proton in these plots if its decision tree proton score is higher than
    0.5, meaning it is more likely than not to be a proton.

    \begin{figure}[h]
      \centering
      \begin{subfigure}[t]{2.5in}
        \includegraphics[angle=0,width=2.5in]{figures/analysis/protonid/PID_efficiency_mctheta_kerange.pdf}
        \caption{Efficiency as a function of the cosine of the true proton
          angle from the beam direction.}
        \label{fig:pideffangletheta}
      \end{subfigure}
      \hspace{2pt}
      \begin{subfigure}[t]{2.5in}
        \includegraphics[angle=0,width=2.5in]{figures/analysis/protonid/PID_efficiency_mcphi_kerange.pdf}
        \caption{Efficiency as a function of the true proton angle around the beam direction.}
        \label{fig:pideffanglephi}
      \end{subfigure}
      \caption{The efficiency of simulated neutrino-induced proton tracks
        correctly classified as protons as a function of true proton angle.}
      \label{fig:pideffangle}
    \end{figure}
    Figure~\ref{fig:pideffangle} shows the efficiency of the decision tree proton
    identification on simulated neutrino-induced protons as a function of true
    proton angle. The efficiencies in these plots are calculated using only the
    simulated protons within the kinetic energy range of interest ($0.05$ GeV
    $\le T_p \le 0.5$ GeV) used in Figure~\ref{fig:pideffkerng}. The left plot
    (\ref{fig:pideffangletheta}) shows the efficiency as a function of
    $\cos(\theta_p)$, where $\theta_p$ is the angle of the proton from the
    neutrino beam direction. At $\cos(theta_p) = 1$ the proton is parallel to
    the beam, at $\cos(\theta_p) = -1$ the proton is anti-parallel to the beam,
    and at $\cos(\theta_p) = 0$ the proton is perpendicular to the beam. When
    the proton is perpendicular to the beam, it is aligned with the anode
    collection plane, and will not traverse more than one collection plane
    wire. A large contribution to the decrease in efficiency at $\cos(\theta_p)
    = 0$ is the fact that the decision tree classifier only uses calorimetry
    information from the collection plane. The right
    plot(\ref{fig:pideffanglephi}) shows the efficiency as a function of
    $\phi_p$ which is the angle around the neutrino beam direction. The flat
    efficiency is due to the fact that the neutrino-induced proton angle should
    be isotropic in $\phi_p$, and the angle around the beam direction has no
    effect on the angle with respect to the angle of the anode wires. Again, a
    track is considered positively identified as a proton in these plots if its
    decision tree proton score is higher than 0.5.
    \begin{figure}[ht]
      \centering
      \includegraphics[angle=0,width=5.0in]{figures/analysis/protonid/PID_efficiency_mcthetavmcke.pdf}
      \caption{Two-dimensional efficiency for true proton $\cos(\theta_p)$
      versus true proton kinetic energy.}
      \label{fig:pideffthetake}
    \end{figure}
    Figure~\ref{fig:pideffthetake} shows the two-dimensional efficiency for
    true proton $\cos(\theta_p)$ versus true proton kinetic energy. The kinetic
    energy range of interest to this analysis goes up to $0.5$ GeV (the bottom
    half of the plot).

  \subsubsection{Performance on a Neutrino Data Subset}
    The gradient-boosted decision tree classifier was tested on a subset of
    MicroBooNE neutrino data corresponding to 5e19 POT ($< 5\%$ of the full
    MicroBooNE approved POT). The data subset comes entirely from MicroBooNE's
    first year of running (Run I).
    
    Figures~\ref{fig:pidnhits}-\ref{fig:pidphi} show comparisons between the
    subset of MicroBooNE neutrino data and the MicroBooNE simulation for each
    of the input variables being used in the decision tree classifier. A
    description of each of these reconstructed track features is given in
    Sec.~\ref{sec:features}. The left plots show the histogram of the given
    variable for all tracks being input to the classifier, and the right plots
    show the histograms of the given variable for the tracks that were
    classified as protons. A track is considered classified as a proton if the
    decision tree proton score is greater than 0.5 for that track.
    
    In all of the figures, the black points in the top plots show the subset of
    neutrino data.  The horizontal bars represent the bin width, and the
    vertical bars represent the statistical uncertainty. The light gray filled
    histogram includes tracks from the off-beam data. These tracks represent
    the background of events where a cosmic interaction in the detector
    coincident with the beam time window triggered the event, and there was no
    actual neutrino interaction. The dark gray filled histograms include cosmic
    tracks that are in the background of events with actual neutrino
    interactions that triggered the event. For simulated neutrino interactions
    inside the detector, real data cosmic tracks are overlaid on the simulated
    event, and for simulated neutrino interaction outside the detector, the
    background cosmic tracks are from simulation. The color filled histograms
    include tracks from simulated neutrino interactions. The peach colored
    histograms include simulated neutrino-induced proton tracks, the dark green
    includes simulated neutrino-induced pion tracks, the light green includes
    simulated neutrino-induced muon tracks, and the purple includes simulated
    neutrino-induced electromagnetic shower tracks. The fraction of proton
    tracks in the right plots (the tracks classified as protons) is much larger
    than in the left plots (all tracks), which is the goal of the classifier.
    The bottom plots in all of the figures show the ratio of the neutrino data
    points to the sum of all of the stacked, filled histograms. A ratio of one
    means perfect data to simulation agreement.

    \begin{figure}[h]
      \centering
      \begin{subfigure}[t]{2.5in}
        \includegraphics[angle=0,width=2.5in]{figures/analysis/protonid/datamcpid/pid_nhits_all_truncated.pdf}
        \includegraphics[angle=0,width=2.5in]{figures/analysis/protonid/datamcpid/pid_nhits_all_truncated-ratio.pdf}
        \caption{All reconstructed tracks.}
      \end{subfigure}
      \hspace{2pt}
      \begin{subfigure}[t]{2.5in}
        \includegraphics[angle=0,width=2.5in]{figures/analysis/protonid/datamcpid/pid_nhits_pass_truncated.pdf}
        \includegraphics[angle=0,width=2.5in]{figures/analysis/protonid/datamcpid/pid_nhits_pass_truncated-ratio.pdf}
        \caption{All reconstructed tracks that are classified as protons.}
      \end{subfigure}
      \caption{Breakdown of the different particle track types in neutrino data
      and simulation as a function of the number of hits on the collection
      plane.}
      \label{fig:pidnhits}
    \end{figure}
    \begin{figure}[h]
      \centering
      \begin{subfigure}[t]{2.5in}
        \includegraphics[angle=0,width=2.5in]{figures/analysis/protonid/datamcpid/pid_distlenratio_all_truncated.pdf}
        \includegraphics[angle=0,width=2.5in]{figures/analysis/protonid/datamcpid/pid_distlenratio_all_truncated-ratio.pdf}
        \caption{All reconstructed tracks.}
      \end{subfigure}
      \hspace{2pt}
      \begin{subfigure}[t]{2.5in}
        \includegraphics[angle=0,width=2.5in]{figures/analysis/protonid/datamcpid/pid_distlenratio_pass_truncated.pdf}
        \includegraphics[angle=0,width=2.5in]{figures/analysis/protonid/datamcpid/pid_distlenratio_pass_truncated-ratio.pdf}
        \caption{All reconstructed tracks that are classified as protons.}
      \end{subfigure}
      \caption{Breakdown of the different particle track types in neutrino data
      and simulation as a function of the track straightness.}
      \label{fig:piddistlenratio}
    \end{figure}
    \begin{figure}[h]
      \centering
      \begin{subfigure}[t]{2.5in}
        \includegraphics[angle=0,width=2.5in]{figures/analysis/protonid/datamcpid/pid_length_all_truncated.pdf}
        \includegraphics[angle=0,width=2.5in]{figures/analysis/protonid/datamcpid/pid_length_all_truncated-ratio.pdf}
        \caption{All reconstructed tracks.}
      \end{subfigure}
      \hspace{2pt}
      \begin{subfigure}[t]{2.5in}
        \includegraphics[angle=0,width=2.5in]{figures/analysis/protonid/datamcpid/pid_length_pass_truncated.pdf}
        \includegraphics[angle=0,width=2.5in]{figures/analysis/protonid/datamcpid/pid_length_pass_truncated-ratio.pdf}
        \caption{All reconstructed tracks that are classified as protons.}
      \end{subfigure}
      \caption{Breakdown of the different particle track types in neutrino data
      and simulation as a function of the track length.}
      \label{fig:pidlength}
    \end{figure}
    \begin{figure}[h]
      \centering
      \begin{subfigure}[t]{2.5in}
        \includegraphics[angle=0,width=2.5in]{figures/analysis/protonid/datamcpid/pid_startdedx_all_truncated.pdf}
        \includegraphics[angle=0,width=2.5in]{figures/analysis/protonid/datamcpid/pid_startdedx_all_truncated-ratio.pdf}
        \caption{All reconstructed tracks.}
      \end{subfigure}
      \hspace{2pt}
      \begin{subfigure}[t]{2.5in}
        \includegraphics[angle=0,width=2.5in]{figures/analysis/protonid/datamcpid/pid_startdedx_pass_truncated.pdf}
        \includegraphics[angle=0,width=2.5in]{figures/analysis/protonid/datamcpid/pid_startdedx_pass_truncated-ratio.pdf}
        \caption{All reconstructed tracks that are classified as protons.}
      \end{subfigure}
      \caption{Breakdown of the different particle track types in neutrino data
      and simulation as a function of the track start $dE/dx$.}
      \label{fig:piddedx}
    \end{figure}
    \begin{figure}[h]
      \centering
      \begin{subfigure}[t]{2.5in}
        \includegraphics[angle=0,width=2.5in]{figures/analysis/protonid/datamcpid/pid_dedxratio_all_truncated.pdf}
        \includegraphics[angle=0,width=2.5in]{figures/analysis/protonid/datamcpid/pid_dedxratio_all_truncated-ratio.pdf}
        \caption{All reconstructed tracks.}
      \end{subfigure}
      \hspace{2pt}
      \begin{subfigure}[t]{2.5in}
        \includegraphics[angle=0,width=2.5in]{figures/analysis/protonid/datamcpid/pid_dedxratio_pass_truncated.pdf}
        \includegraphics[angle=0,width=2.5in]{figures/analysis/protonid/datamcpid/pid_dedxratio_pass_truncated-ratio.pdf}
        \caption{All reconstructed tracks that are classified as protons.}
      \end{subfigure}
      \caption{Breakdown of the different particle track types in neutrino data
      and simulation as a function of the end to start $dE/dx$ ratio.}
      \label{fig:piddedxratio}
    \end{figure}
    \begin{figure}[h]
      \centering
      \begin{subfigure}[t]{2.5in}
        \includegraphics[angle=0,width=2.5in]{figures/analysis/protonid/datamcpid/pid_trtotaldedx_all_truncated.pdf}
        \includegraphics[angle=0,width=2.5in]{figures/analysis/protonid/datamcpid/pid_trtotaldedx_all_truncated-ratio.pdf}
        \caption{All reconstructed tracks.}
      \end{subfigure}
      \hspace{2pt}
      \begin{subfigure}[t]{2.5in}
        \includegraphics[angle=0,width=2.5in]{figures/analysis/protonid/datamcpid/pid_trtotaldedx_pass_truncated.pdf}
        \includegraphics[angle=0,width=2.5in]{figures/analysis/protonid/datamcpid/pid_trtotaldedx_pass_truncated-ratio.pdf}
        \caption{All reconstructed tracks that are classified as protons.}
      \end{subfigure}
      \caption{Breakdown of the different particle track types in neutrino data
      and simulation as a function of the track truncated total $dE/dx$.}
      \label{fig:pidtrtotaldedx}
    \end{figure}
    \begin{figure}[h]
      \centering
      \begin{subfigure}[t]{2.5in}
        \includegraphics[angle=0,width=2.5in]{figures/analysis/protonid/datamcpid/pid_traveragededx_all_truncated.pdf}
        \includegraphics[angle=0,width=2.5in]{figures/analysis/protonid/datamcpid/pid_traveragededx_all_truncated-ratio.pdf}
        \caption{All reconstructed tracks.}
      \end{subfigure}
      \hspace{2pt}
      \begin{subfigure}[t]{2.5in}
        \includegraphics[angle=0,width=2.5in]{figures/analysis/protonid/datamcpid/pid_traveragededx_pass_truncated.pdf}
        \includegraphics[angle=0,width=2.5in]{figures/analysis/protonid/datamcpid/pid_traveragededx_pass_truncated-ratio.pdf}
        \caption{All reconstructed tracks that are classified as protons.}
      \end{subfigure}
      \caption{Breakdown of the different particle track types in neutrino data
      and simulation as a function of the truncated average $dE/dx$.}
      \label{fig:pidtraveragededx}
    \end{figure}
    \begin{figure}[h]
      \centering
      \begin{subfigure}[t]{2.5in}
        \includegraphics[angle=0,width=2.5in]{figures/analysis/protonid/datamcpid/pid_starty_all_truncated.pdf}
        \includegraphics[angle=0,width=2.5in]{figures/analysis/protonid/datamcpid/pid_starty_all_truncated-ratio.pdf}
        \caption{All reconstructed tracks.}
      \end{subfigure}
      \hspace{2pt}
      \begin{subfigure}[t]{2.5in}
        \includegraphics[angle=0,width=2.5in]{figures/analysis/protonid/datamcpid/pid_starty_pass_truncated.pdf}
        \includegraphics[angle=0,width=2.5in]{figures/analysis/protonid/datamcpid/pid_starty_pass_truncated-ratio.pdf}
        \caption{All reconstructed tracks that are classified as protons.}
      \end{subfigure}
      \caption{Breakdown of the different particle track types in neutrino data
      and simulation as a function of the track starting $y$ position.}
      \label{fig:pidstarty}
    \end{figure}
    \begin{figure}[h]
      \centering
      \begin{subfigure}[t]{2.5in}
        \includegraphics[angle=0,width=2.5in]{figures/analysis/protonid/datamcpid/pid_endy_all_truncated.pdf}
        \includegraphics[angle=0,width=2.5in]{figures/analysis/protonid/datamcpid/pid_endy_all_truncated-ratio.pdf}
        \caption{All reconstructed tracks.}
      \end{subfigure}
      \hspace{2pt}
      \begin{subfigure}[t]{2.5in}
        \includegraphics[angle=0,width=2.5in]{figures/analysis/protonid/datamcpid/pid_endy_pass_truncated.pdf}
        \includegraphics[angle=0,width=2.5in]{figures/analysis/protonid/datamcpid/pid_endy_pass_truncated-ratio.pdf}
        \caption{All reconstructed tracks that are classified as protons.}
      \end{subfigure}
      \caption{Breakdown of the different particle track types in neutrino data
      and simulation as a function of the track ending $y$ position.}
      \label{fig:pidendy}
    \end{figure}
    \begin{figure}[h]
      \centering
      \begin{subfigure}[t]{2.5in}
        \includegraphics[angle=0,width=2.5in]{figures/analysis/protonid/datamcpid/pid_startz_all_truncated.pdf}
        \includegraphics[angle=0,width=2.5in]{figures/analysis/protonid/datamcpid/pid_startz_all_truncated-ratio.pdf}
        \caption{All reconstructed tracks.}
      \end{subfigure}
      \hspace{2pt}
      \begin{subfigure}[t]{2.5in}
        \includegraphics[angle=0,width=2.5in]{figures/analysis/protonid/datamcpid/pid_startz_pass_truncated.pdf}
        \includegraphics[angle=0,width=2.5in]{figures/analysis/protonid/datamcpid/pid_startz_pass_truncated-ratio.pdf}
        \caption{All reconstructed tracks that are classified as protons.}
      \end{subfigure}
      \caption{Breakdown of the different particle track types in neutrino data
      and simulation as a function of the track starting $z$ position.}
      \label{fig:pidstartz}
    \end{figure}
    \begin{figure}[h]
      \centering
      \begin{subfigure}[t]{2.5in}
        \includegraphics[angle=0,width=2.5in]{figures/analysis/protonid/datamcpid/pid_endz_all_truncated.pdf}
        \includegraphics[angle=0,width=2.5in]{figures/analysis/protonid/datamcpid/pid_endz_all_truncated-ratio.pdf}
        \caption{All reconstructed tracks.}
      \end{subfigure}
      \hspace{2pt}
      \begin{subfigure}[t]{2.5in}
        \includegraphics[angle=0,width=2.5in]{figures/analysis/protonid/datamcpid/pid_endz_pass_truncated.pdf}
        \includegraphics[angle=0,width=2.5in]{figures/analysis/protonid/datamcpid/pid_endz_pass_truncated-ratio.pdf}
        \caption{All reconstructed tracks that are classified as protons.}
      \end{subfigure}
      \caption{Breakdown of the different particle track types in neutrino data
      and simulation as a function of the track ending $z$ position.}
      \label{fig:pidendz}
    \end{figure}
    \begin{figure}[h]
      \centering
      \begin{subfigure}[t]{2.5in}
        \includegraphics[angle=0,width=2.5in]{figures/analysis/protonid/datamcpid/pid_costheta_all_truncated.pdf}
        \includegraphics[angle=0,width=2.5in]{figures/analysis/protonid/datamcpid/pid_costheta_all_truncated-ratio.pdf}
        \caption{All reconstructed tracks.}
      \end{subfigure}
      \hspace{2pt}
      \begin{subfigure}[t]{2.5in}
        \includegraphics[angle=0,width=2.5in]{figures/analysis/protonid/datamcpid/pid_costheta_pass_truncated.pdf}
        \includegraphics[angle=0,width=2.5in]{figures/analysis/protonid/datamcpid/pid_costheta_pass_truncated-ratio.pdf}
        \caption{All reconstructed tracks that are classified as protons.}
      \end{subfigure}
      \caption{Breakdown of the different particle track types in neutrino data
      and simulation as a function of the track $\cos(\theta)$ angle.}
      \label{fig:pidcostheta}
    \end{figure}
    \begin{figure}[h]
      \centering
      \begin{subfigure}[t]{2.5in}
        \includegraphics[angle=0,width=2.5in]{figures/analysis/protonid/datamcpid/pid_phi_all_truncated.pdf}
        \includegraphics[angle=0,width=2.5in]{figures/analysis/protonid/datamcpid/pid_phi_all_truncated-ratio.pdf}
        \caption{All reconstructed tracks.}
      \end{subfigure}
      \hspace{2pt}
      \begin{subfigure}[t]{2.5in}
        \includegraphics[angle=0,width=2.5in]{figures/analysis/protonid/datamcpid/pid_phi_pass_truncated.pdf}
        \includegraphics[angle=0,width=2.5in]{figures/analysis/protonid/datamcpid/pid_phi_pass_truncated-ratio.pdf}
        \caption{All reconstructed tracks that are classified as protons.}
      \end{subfigure}
      \caption{Breakdown of the different particle track types in neutrino data
      and simulation as a function of the track $\phi$ angle.}
      \label{fig:pidphi}
    \end{figure}

    \FloatBarrier


%%%%%%%%%%%%%%%%%%%%%%%%%%%%%%%%%%%%%%%%%%%%%%%%%%%%%%%%%%%
% Neutral Current Elastic Proton Event Selection
%%%%%%%%%%%%%%%%%%%%%%%%%%%%%%%%%%%%%%%%%%%%%%%%%%%%%%%%%%%
\subsection{Event Selection}\label{sec:selection}


%%%%%%%%%%%%%%%%%%%%%%%%%%%%%%%%%%%%%%%%%%%%%%%%%%%%%%%%%%%
% Efficiency and Background Estimation
%%%%%%%%%%%%%%%%%%%%%%%%%%%%%%%%%%%%%%%%%%%%%%%%%%%%%%%%%%%
\subsection{Efficiency and Background Estimation}\label{sec:effbg}
  \subsubsection{Event selection efficiency}
    Efficiency due to TPC, PMT software trigger, reconstruction, proton ID,
    event selection, etc. This includes efficiency due to proton reinteracting
    in the nucleus and other nuclear effects.
  \subsubsection{Beam Induced Dirt Background}
    Discuss dirt neutrons, how they happen and estimated rates and energy
    distributions.  Show how well we can seperate or understand them. Show any
    sort of data-driven correction we did to dirt neutron background and how it
    affects our uncertainty. Talk about how well we can tag cryostat neutrons
    with the PMTs.
  \subsubsection{Beam Induced TPC Background}
    Talk about neutral-current elastic neutrons that are produced in the TPC
    and how their distributions differ from NCEp ones. Also include BNB
    backgounds (CCQE where muon wasn't reconstructed, NCpi0, etc.) Discuss how
    the optical signal would be different for each of these.
  \subsubsection{Cosmic Background}
    Discuss the difference between cosmic tracks and beam proton tracks. How do
    we separate them? What is the rate?


%%%%%%%%%%%%%%%%%%%%%%%%%%%%%%%%%%%%%%%%%%%%%%%%%%%%%%%%%%%
% Ratio of Cross Sections
%%%%%%%%%%%%%%%%%%%%%%%%%%%%%%%%%%%%%%%%%%%%%%%%%%%%%%%%%%%
\subsection{Ratio of NCEp to CCQEn Cross Sections}\label{sec:ratios}
  Show how the ratio gets rid of a lot of measurement uncertainty like beam
  flux and efficiencies. Give exact equation that we will be using for
  analysis. Show how $\Delta s$ is still large at low $Q^2$.
  \subsubsection{Sources of Measurement Uncertainty}
    TPC efficiency: If ionization electrons actually reach the
    TPC and leave a signal.
    PMT trigger efficiency: Refer back to PMT trigger studies. Give uncertainty
    due to on signal.
    Reconstruction efficiency: Refer back. Give uncertainty on signal.
  \subsubsection{Quantifying Uncertainty on Ratio}\label{errorcalc}
    Calculate exact uncertainty and show it here.
  \subsubsection{Model uncertainty}\label{sec:modeluncertainty}
    Nuclear effects and FSI. Discuss. Discuss which values are varied and by
    how much? Refer to reweighting section. 


\newpage
\section{Analysis}\label{analysis}
%The next line produces an indented paragraph to start the document
 %unit.  The LaTeX defaults start most units without indentations.
\hspace{\parindent}
Outline the analysis section.

%%%%%%%%%%%%%%%%%%%%%%%%%%%%%%%%%%%%%%%%%%%%%%%%%%%%%%%%%%%
% Particle Identification and Event Selection
%%%%%%%%%%%%%%%%%%%%%%%%%%%%%%%%%%%%%%%%%%%%%%%%%%%%%%%%%%%
\subsection{Particle Identification and Event Selection}
  After the particle tracks are reconstructed, we use a predictive model to
  classify proton tracks. The inputs to the model are the reconstructed
  physical variables, and the output is the probability that the track is from
  a proton vs. some other particle. There are many predictive models that we
  can use, each with advantages and disadvantages. We chose gradient-boosted
  decision trees for a few main reasons: they are easily interpretable, the
  inputs can be a mix of numeric and categorical variables, and boosted
  decision trees perform well at identifying a small signal in a large
  background.  Each tree is essentially a series of cuts based on physical
  variables which have been fine-tuned to increase the efficiency and purity of
  the final selected sample.
  \subsubsection{Reconstructed track features}
    The reconstructed features that are used as input to the classifier are
    listed below. Most of the features come directly from the track object, but
    some are created for this classifier. Each of the features used to identify
    protons either helps to separate neutrino-induced tracks from
    cosmic-induced tracks or to separate neutrino-induced proton tracks from
    other neutrino-induced particle types.

    First is a list and description of the features designed to separate
    neutrino-induced protons from other neutrino-induced particle types.
    \begin{itemize}
      \item \textbf{Number of hits:} This is the total number of hits on all
      three wire planes associated with track. When used in combination with
      track length and average energy deposited, this feature can be used to
      determine the hit and energy density of the track.
      \item \textbf{Straightness:} This is the ratio of distance between
      reconstructed end points (displacement) to reconstructed path length. It
      represents the amount of scattering a track undergoes. The value is
      always betwen zero and one with one being perfectly straight.
      \item \textbf{Cosmic score:} This is the geometry tagging cosmic score
      from Sec.~\ref{sec:tpcreco}. Tracks with a cosmic score of 1 have already
      been removed in the cosmic hit removal stage. So, this value is either 0
      (fully contained within the TPC) or 0.5 (entering or exiting the TPC).
      \item \textbf{Particle ID from energy deposited per unit length (dE/dx):}
      The PIDA and $\chi^2$ PID algorithms values described in
      Sec.~\ref{sec:tpcreco} are used.
      \item \textbf{Length:} This is the reconstructed 3D track length found by
      stepping along the trajectory points.
      \item \textbf{Start and end dQ/dx:} These are the total charge deposited
      at start and end points of track divided by the distance between hits to
      account for the angle with respect to the wires. The total charge of the
      first (or last) six track hits on each plane is summed.
      \item \textbf{End to start dQ/dx ratio and difference:} These are the
      ratio and difference of the end dQ/dx divided by (or subtracted by) the
      start dQ/dx described in the previous item.
      \item \textbf{Total dQ/dx:} This is the sum of the dQ/dx of all hits on
      all planes associated with track.
      \item \textbf{Average dQ/dx:} This is the total dQ/dx divided by the
      number of hits associated with track.
    \end{itemize}

    Next is the list and description of the features designed to separate
    neutrino-induced tracks from cosmic-induced tracks.
    \begin{itemize}
      \item \textbf{Start and end positions:} These are the reconstructed x, y,
      and z positions of start and end of the track. Tracks that start closer
      to a TPC boundary are more likely to be cosmic-induced.
      \item \textbf{$\theta$ and $\phi$:} These are the reconstructed polar and
      azimuthal angles with respect to the beam direction. Vertical tracks are
      much more likely to be cosmic-induced, while forward-going tracks are
      more likely to be from the neutrino beam.
    \end{itemize}
   
    Determining which end of a track is the beginning is difficult when a
    vertex is not observable. Since we are particularly interested in
    neutral-current elastic events with only a single proton, the direction of
    the track is a concern. A proton will deposit much more energy at the end
    of its track than at the beginning which can be used to determine the true
    direction. Since this correction is not currently implemented in pandoraNu,
    we take all reconstructed tracks that have a higher start charge than end
    charge and flip them. This means changing the saved start positions, end
    positions, $\theta$, $\phi$, start dQ/dx, end dQ/dx, and the end to start
    dQ/dx ratio and difference.
    
  \subsubsection{Boosted decision trees}\label{sec:decisiontrees}
    A decision tree can be thought of as a series of if/else statements that
    separate a data set into two or more classes as illustrated in
    Fig.~\ref{fig:dtree}. At each node of the tree, a split is chosen to
    maximize information gain until a set level of separation is reached.  At
    the terminus of the series of splits, called a leaf, a class is assigned.
    The usual parameters that can be set when creating a decision tree are: the
    maximum depth of the tree (how many layers of nodes you will allow), the
    minimum split size (how many data points do you require to keep splitting),
    and minimum leaf size (how small does a leaf have to be before you stop). 
    \begin{figure}[h]
      \centering
      \includegraphics[angle=0,width=4in]{figures/bz.pdf}
      \caption{Graphical example of a decision tree.}
      \label{fig:dtree}
    \end{figure}
    
    A single tree can easily overfit a data set if it is at all complex, and
    its output is just a class label. Gradient-boosting addresses both of these
    issues by combining many weak classifiers into a strong one. Each weak
    classifier is built based on the error of the previous one. For a given
    training set, whenever a sample is classified incorrectly by a tree, that
    sample is given a higher importance when the next tree is being created.
    Mathematically, each tree is training on the gradient of the loss function.
    After all of the trees have been created, each tree is given a weight based
    on its ability to classify the training set, and the output of the
    gradient-boosted decision tree classifier is the probability that a sample
    is in a given class.
    
    The gradient-boosted decision tree software package we use is
    XGBoost~\cite{xgboost}. There are two types of classifiers we can use to
    separate protons from other tracks: binary and multiclass. Both classifiers
    are trained on all types of reconstructed tracks. A binary classifier
    classifies each track as either a proton or not a proton, and a multiclass
    classifier classifies a track as one of many types including a proton. We
    choose to use multiclass because the information about non-proton tracks is
    useful for event identification. The five classes that we train the
    decision trees to classify are protons (both BNB and cosmic), BNB muons,
    BNB pions, BNB electrons/photons, and all non-proton cosmics.
    
    The decision trees were trained to select protons in general. The target
    class contains all protons from BNB interactions, as well as cosmic
    protons. However, the choice of track features and most of the improvements
    that were made were done with the goal of a high efficiency for isolated
    protons and cosmic rejection.

  \subsubsection{Training}
    Describe the training sets. What samples were used, how many of everything,
    and all hyperparamter settings. Describe hyperparameter optimization. 
  \subsubsection{Performance}
    Show efficiency, accuracy, itemize backgrounds.
    Discuss reasons for different backgrounds.
  \subsubsection{Event selection}
    Need to select both NCE and CCQE events.
    Use particle ID plus reconstructed flashes.
    Exactly how we select NCE events 

%%%%%%%%%%%%%%%%%%%%%%%%%%%%%%%%%%%%%%%%%%%%%%%%%%%%%%%%%%%
% Efficiency and Background Estimation
%%%%%%%%%%%%%%%%%%%%%%%%%%%%%%%%%%%%%%%%%%%%%%%%%%%%%%%%%%%
\subsection{Efficiency and Background Estimation}\label{background}
  \subsubsection{Event selection efficiency}
    Efficiency due to TPC, PMT software trigger, reconstruction, proton ID,
    event selection, etc. This includes efficiency due to proton reinteracting
    in the nucleus and other nuclear effects.
  \subsubsection{Beam Induced Dirt Background}
    Discuss dirt neutrons, how they happen and estimated rates and energy
    distributions.  Show how well we can seperate or understand them. Show any
    sort of data-driven correction we did to dirt neutron background and how it
    affects our uncertainty. Talk about how well we can tag cryostat neutrons
    with the PMTs.
  \subsubsection{Beam Induced TPC Background}
    Talk about neutral-current elastic neutrons that are produced in the TPC
    and how their distributions differ from NCEp ones. Also include BNB
    backgounds (CCQE where muon wasn't reconstructed, NCpi0, etc.) Discuss how
    the optical signal would be different for each of these.
  \subsubsection{Cosmic Background}
    Discuss the difference between cosmic tracks and beam proton tracks. How do
    we separate them? What is the rate?

%%%%%%%%%%%%%%%%%%%%%%%%%%%%%%%%%%%%%%%%%%%%%%%%%%%%%%%%%%%
% Ratio of Cross Sections
%%%%%%%%%%%%%%%%%%%%%%%%%%%%%%%%%%%%%%%%%%%%%%%%%%%%%%%%%%%
\subsection{Ratio of NCEp to CCQEn Cross Sections}\label{ratios}
  Show how the ratio gets rid of a lot of measurement uncertainty like beam
  flux and efficiencies. Give exact equation that we will be using for
  analysis. Show how $\Delta s$ is still large at low $Q^2$.
  \subsubsection{Sources of Measurement Uncertainty}
    TPC efficiency: If ionization electrons actually reach the
    TPC and leave a signal.
    PMT trigger efficiency: Refer back to PMT trigger studies. Give uncertainty
    due to on signal.
    Reconstruction efficiency: Refer back. Give uncertainty on signal.
  \subsubsection{Quantifying Uncertainty on Ratio}\label{errorcalc}
    Calculate exact uncertainty and show it here.
  \subsubsection{Model uncertainty}
    Nuclear effects and FSI. Discuss. Discuss which values are varied and by
    how much? Refer to reweighting section. 

%%%%%%%%%%%%%%%%%%%%%%%%%%%%%%%%%%%%%%%%%%%%%%%%%%%%%%%%%%%
% Comparison to Simulation
%%%%%%%%%%%%%%%%%%%%%%%%%%%%%%%%%%%%%%%%%%%%%%%%%%%%%%%%%%%
\subsection{Comparison of data to simulation}
  To determine what true physics values would have caused the data that we
  measure, we need to compare the data to Monte Carlo simulation. The physics
  model is a combination of many different physics models including nuclear
  physics models, neutrino cross section models, nucleon structure models,
  cosmic ray models, and others. It is too complicated to calculate the
  parameters directly. Instead, we simulate what data we would see in the
  detector given a set of physical parameters and models, and compare that
  directly to the actual data we detected. We do this for many possible values
  of the parameters and calculate the likelihood for each set of values. In
  addition to varying the parameters that we want to measure in the simulation,
  we vary the physical parameters whose true values aren't well constrained
  which might have a large effect on the final data. This allows us to quantify
  the uncertainty due to unknown quantities.
  \subsubsection{Reweighting}
    Full Monte Carlo simulations are both time and computing intensive. Instead
    of re-running the simulation for each possible parameter value that we are
    interested in, we can calculate the ratio of the probabilities of each
    interaction occurring given the new values to the probabilities of each
    interaction occurring for the original simulated parameter values.

    Need to add: how to calculate the probabilities. Summing the weights into
    histograms (measure thousands of interactions, want to compare to at least
    that many, reweight each one and combine for the ratio). Generating
    systematic weights in GENIE. Refer to model uncertainty section and say
    which parameters we vary. Calculating axial FF weights from cross section.
    Which model is numerator. Show plots of ratio with parameters the same but
    models different.  Multiply weights. 
  \subsubsection{Likelihood calculation}

%%%%%%%%%%%%%%%%%%%%%%%%%%%%%%%%%%%%%%%%%%%%%%%%%%%%%%%%%%%
% Joint Estimation
%%%%%%%%%%%%%%%%%%%%%%%%%%%%%%%%%%%%%%%%%%%%%%%%%%%%%%%%%%%
\subsection{Strange axial form factor parameter estimation}\label{deltas}
  Here is where we do the actual MCMC.
  \subsubsection{Bayesian inference}
    Walk through the Bayesian math. Start with wanting the probability of
    $\theta$ given the data and the model and get final equation. Then show how
    to combine with previous experiments. Should give detail for each factor in
    equation (prior, likelihood, posterior, evidence). Should especially give
    detail on how we choose priors. Maybe talk about model selection when
    talking about the evidence.
  \subsubsection{Markov Chain Monte Carlo}
    Go through detail on our implementation of MCMC. Discuss how Markov Chains
    work in general. Describe Metropolis, Gibbs, and our combination of both.
  \subsubsection{Results}
    Lots of plots! $\Delta s$!


%This is the end of analysis section

\newpage
%\section{SAMPLE TEXT WITH GRAPHICS} \label{graphics}
%This file illustrates how include in your thesis graphical
 %output.
%The next line produces an indented paragraph to start the document
 %unit.  The LaTeX defaults start most units without indentations.
\hspace{\parindent}
This is sample text with graphics.
\begin{figure}[h]
  \centering
  \includegraphics[angle=-90,width=4in]{figures/bz.pdf}
  \caption{This is an inserted EPS graphic}
  \label{fig:mygraph1}
\end{figure}
Sample ref of \ref{fig:mygraph1} 

\begin{figure}[t]
 \centering
  \includegraphics[angle=-90, width=4in]{figures/E734eta.pdf}  
    \caption{This is another inserted PDF graphic}
    \label{fig:mygraph2}
\end{figure}
%This is the end of the file graphics.tex


%\newpage
%The next four lines create a list of references for the 
 %paper from a .bbl file created by BibTeX from a .bib bibliographic
 %database file.  
 %The MathSciNet clipboard was used to prepare 
 %the .bib file.  See Lamport's book mentioned in the references for
 %basic information about BibTeX.  If you don't use BibTeX, you need
 %to create and format the list of references yourself.
\addcontentsline{toc}{section}{REFERENCES}
\setlength{\baselineskip}{\singlespace}
\bibliographystyle{ieeetr}
\bibliography{main}
%These next three command lines create a list of references for the paper from
 %a biblio.tex file you create yourself.
%\addcontentsline{toc}{section}{REFERENCES}
%\setlength{\baselineskip}{\singlespace}
%%This file is a list of references prepared by hand.  If you are going 
 %to prepare your list of references by hand, you need to look
 %carefully at a journal and try  %to follow consistently the
 %journal's style.  
\begin{thebibliography}{99}

\bibitem{loday2} Fiedorowicz, Zbigniew  and Loday, Jean-Louis.
{\em Crossed Simplicial Groups and Their Associated Homology.}
Transactions of the American Mathematical Society, {\bf 36}(1991), p 57-87.

\bibitem{Higham}
Higham, Nicholas, J.  
{\em Handbook of writing for the mathematical sciences}, second edition.
Society for Industrial and Applied Mathematics, 1998.

\bibitem{Lamport} Lamport, Leslie.
{\em LaTeX:  A Document Preparation System}, Second Edition.
Addison-Wesley, (1994).

\bibitem{Lodaybook}Loday, Jean-Louis.
{\em Cyclic Homology.}
Springer Verlag, (1992).

\bibitem{L-Q} Loday, Jean-Louis and Quillen, Daniel.
{\em Cyclic Homology and the Lie Algebra Homology of Matrices.}
Comment. Math. Helv. {\bf 59}(1984), p 565-591.

\bibitem{Reckdahl}  Reckdahl, Keith.
{\em Using EPS Graphics in \LaTeX\ $2_{\varepsilon}$ Documents}.
Preprint, 1997.
 
\bibitem{Swanson}  Swanson, Ellen.
{\em Mathematics into type,} revised edition.
American Mathematical Society, 1979.

\end{thebibliography}
%This is the end of the file biblio.tex














%
\end{document}
